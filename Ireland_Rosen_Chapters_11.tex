%&LaTeX
\documentclass[11pt,a4paper]{article}
\usepackage[frenchb,english]{babel}
\usepackage[applemac]{inputenc}
\usepackage[OT1]{fontenc}
\usepackage[]{graphicx}
\usepackage{amsmath}
\usepackage{amsfonts}
\usepackage{amsthm}
\usepackage{amssymb}
\usepackage{yfonts}
\usepackage{mathrsfs}
%\input{8bitdefs}

% marges
\topmargin 10pt
\headsep 10pt
\headheight 10pt
\marginparwidth 30pt
\oddsidemargin 40pt
\evensidemargin 40pt
\footskip 30pt
\textheight 670pt
\textwidth 420pt

\def\imp{\Rightarrow}
\def\gcro{\mbox{[\hspace{-.15em}[}}% intervalles d'entiers 
\def\dcro{\mbox{]\hspace{-.15em}]}}

\newcommand{\D}{\mathrm{d}}
\newcommand{\Q}{\mathbb{Q}}
\newcommand{\Z}{\mathbb{Z}}
\newcommand{\N}{\mathbb{N}}
\newcommand{\R}{\mathbb{R}}
\newcommand{\C}{\mathbb{C}}
\newcommand{\F}{\mathbb{F}}
\newcommand{\re}{\,\mathrm{Re}\,}
\newcommand{\ord}{\mathrm{ord}}
\newcommand{\n}{\mathrm{N}}
\newcommand{\legendre}[2]{\genfrac{(}{)}{}{}{#1}{#2}}



\title{Solutions to Ireland, Rosen ``A Classical Introduction to Modern Number Theory''}
\author{Richard Ganaye}

\begin{document}

{ \Large \bf Chapter 11} 

\paragraph{Ex. 11.1} Suppose that we may write the power series $1+a_1u +a_2u^2+\cdots$ as the quotient of two polynomials $P(u)/Q(u)$. Show that we may assume that $P(0) = Q(0) = 1$.
{\it 

}
\begin{proof}
Here $f(u) = 1+a_1u +a_2u^2+\cdots \in F[[u]]$ is a formal series in the variable $u$.

We suppose that $f(u) = P(u)/Q(u)$, where we may assume, after simplification, that the two polynomials are relatively prime. Then $P(1)/Q(1) =1$. Write $c = P(1) = Q(1) \in F$.

If $c = 0$, then $u\mid P(u)$ and $u \mid Q(u)$. This is impossible since $P \wedge Q = 1$. So $c \ne 0$.

Define $P_1(u) = (1/c) P(u), Q_1(u) = (1/c) Q(u)$. Then $f(u) = P_1(u)/Q_1(u)$ and $P_1(0) = Q_1(0) = 1$. If we replace $P,Q$ by $P_1,Q_1$, then the pair $(P_1,Q_1)$ has the required properties.
\end{proof}

\paragraph{Ex. 11.2}{\it Prove the converse to Proposition 11.1.1.
}

\begin{proof}
If $N_s = \sum_{j=1}^e \beta_j^s - \sum_{i=1}^d \alpha_i^s,$ where $\alpha_i,\beta_j$ are complex numbers,  then
\begin{align*}
\sum_{s=1}^\infty \frac{N_s u^s}{s} &=  \sum_{j=1}^e \left( \sum_{s=1}^\infty \frac{(\beta_j u)^s}{s} \right) - \sum_{i=1}^d \left( \sum_{s=1}^\infty \frac{(\alpha_i u)^s}{s} \right)\\
&=-\sum_{j=1}^e \ln(1 - \beta_ju) + \sum_{i=1}^d \ln(1 - \alpha_i u).
\end{align*}
Here $u$ is a variable, and both members are formal polynomials in $\C[[u]]$, so we don't study convergence. Nevertheless, the left member has a radius of convergence at least $q^{-n}$, and the right member $\min_{i,j}(1/\beta_j|,1/|\alpha_i|)$.

Therefore,
$$Z_f(u) = \exp\left(\sum_{s=1}^\infty \frac{N_s u^s}{s} \right) = \prod_{j=1}^e(1 - \beta_j u)^{-1} \prod_{i=1}^d (1-\alpha_i u) = \frac{\prod_{i=1}^d (1 - \alpha_i u)}{\prod_{j=1}^e (1 - \beta_j u)}$$
is a rational fraction.
\end{proof}


\paragraph{Ex. 11.3}{\it Give the details of the proof that $N_s$ is independant of the field $F_s$ (see the concluding paragraph to section 1).
}
\begin{proof}
Suppose that $E$ and $E'$ are two fields containing $F$ both with $q^s$ elements. We first show that there is a isomorphism $\sigma : E \to E'$ which fixes the elements of $F$, by showing that that both $E$ and $E'$ are isomorphic over $F$ to $F[x]/(f(x))$ for some irreducible polynomial $f(x) \in F(x)$.

There is a primitive element $\alpha' \in E'$, i.e. such that $E' = F(\alpha')$. For example, take $\alpha'$ to be a primitive $q^s - 1$ root of unity : since $\alpha$ is a generator of $E'^*$, every element $\gamma \in E'^*$ is equal to $\alpha'^k$ for some integer $k$, thus $\gamma \in F(\alpha')$ (and $0 \in F(\alpha')$). This proves $E' \subset F(\alpha')$, and since $\alpha' \in E'$ and $F \subset E'$, $F(\alpha') \subset E'$, so $E' = F(\alpha')$.

Let $f(x) \in F[x]$ be the minimal polynomial of $\alpha'$ over $F$. Then
$$E' = F(\alpha') \simeq F(x)/(f(x)),$$
where the isomorphism $\sigma_1 : F(\alpha') \to F(x)/(f(x))$ maps $\alpha'$ to $\overline{x} = x + (f(x))$, and maps $a \in F$ on $\overline{a} = a + (f(x))$.
Since $\alpha'$ is a root of $x^{q^s} -x$, $f(x) \mid x^{q^s} - x$. 

 $E$ is a field with $q^s$ elements, so we have $x^{q^s} - x = \prod_{\alpha \in E} (x-\alpha)$. Thus $f(x) \mid \prod_{\alpha \in E} (x-\alpha)$, where $\deg(f(x)) = s \geq 1$, so $f(\alpha) = 0$ for some $\alpha \in E$. The polynomial $f$ being irreducible over $F$, $f$ is the minimal polynomial of $\alpha$ over $F$, thus $F(\alpha) \simeq F[x]/(f(x))$ is a field with $q^s$ elements. 
 Since $F(\alpha) \subset E$, and $|F(\alpha)| = |E|$, we conclude $E = F(\alpha)$, therefore
$$E = F(\alpha) \simeq F(x)/(f(x)),$$
where the isomorphism $\sigma_2 : F(\alpha) \to F(x)/(f(x))$ maps $\alpha$ to $\overline{x} = x + (f(x))$, and maps $a \in F$ on $\overline{a} = a + (f(x))$.

Then  $\sigma = \sigma_1^{-1} \circ \sigma_2 : E \to E'$ is an isomorphism, and $\sigma(a) = a$ for all $a \in F$.
 
 \bigskip
 
 We can now use the isomorphism $\sigma$ to induce a map 
 $$\overline{\sigma}
 \left\{
 \begin{array}{ccl}
 P^n(E)& \to &P^n(E')\\
 {[}\alpha_0,\ldots,\alpha_n{]} &\mapsto & {[}\sigma(\alpha_0),\ldots,\sigma(\alpha_n){]}.
 \end{array}
 \right.
 $$
 Then $\overline{\sigma}$ is injective: if $[\sigma(\alpha_0),\ldots,\sigma(\alpha_n)] = [\sigma(\beta_0),\ldots,\sigma(\beta_n)]$, then there is $\lambda \in F^*$ such that $\beta_i = \lambda \sigma(\alpha_i) = \sigma(\lambda)\sigma(\alpha_i) = \sigma(\lambda \alpha_i,\ i = 0,\ldots,n$, thus $\beta_i = \lambda \alpha_i$, which proves $[\alpha_0,\ldots,\alpha_n] = [\beta_0,\ldots,\beta_n]$.
 
 If $[\gamma_0,\ldots,\gamma_n] $ is any projective point of $P^n(E')$, then 
 $$[\gamma_0,\ldots,\gamma_n] = \overline{\sigma}([\sigma^{-1}(\gamma_0),\ldots,\sigma^{-1}(\gamma_n)]).$$ This proves that $\overline{\sigma}$ is surjective. So $\overline{\sigma}$ is a bijection.
 
 Now take $f(y_0,\ldots,y_n)\in F[y_0,\ldots,y_n]$ an homogeneous polynomial, $\overline{H}_f(E)$ the corresponding projective hypersurface in $P^n(E)$, and $\overline{H}_f(E')$  the corresponding projective hypersurface in $P^n(E')$. We show that $\overline{\sigma}(\overline{H}_f(E)) = \overline{H}_f(E')$.
 
 Since $\sigma$ is a $F$-isomorphism, $\sigma(f(\alpha_0,\ldots,\alpha_n)) = f(\sigma(\alpha_0),\ldots,\sigma(\alpha_n))\quad (\alpha_i \in E)$, and similarly $\sigma^{-1}(f(\beta_0,\ldots,\beta_n)) = f(\sigma^{-1} (\beta_0),\ldots, \sigma^{-1}(\beta_n))\quad  (\beta_i \in E')$, thus
 \begin{align*}
 {[}\alpha_0,\ldots,\alpha_n{]} \in  \overline{H}_f(E)
 &\Rightarrow f(\alpha_0,\ldots,\alpha_n) = 0 \\
 &\Rightarrow \sigma(f(\alpha_0,\ldots,\alpha_n)) = \sigma(0) = 0\\
 & \Rightarrow f(\sigma(\alpha_0),\ldots,\sigma(\alpha_0)) = 0\\
 &\Rightarrow \overline{\sigma}([\alpha_0,\ldots,\alpha_n]) = [\sigma(\alpha_0),\ldots,\sigma(\alpha_0)] \in \overline{H}_f(E').
\end{align*}
This shows $\overline{\sigma}(\overline{H}_f(E)) \subset \overline{H}_f(E')$.

Conversely, 
 \begin{align*}
 {[}\beta_0,\ldots,\beta_n{]} \in  \overline{H}_f(E')
 &\Rightarrow f(\beta_0,\ldots,\beta_n) = 0 \\
 &\Rightarrow \sigma^{-1}(f(\beta_0,\ldots,\beta_n)) = \sigma(0) = 0\\
 & \Rightarrow f(\sigma^{-1}(\beta_0),\ldots,\sigma^{-1}(\beta_0)) = 0\\
 &\Rightarrow \overline{\sigma}^{-1}([\beta_0,\ldots,\beta_n]) = [\sigma^{-1}(\beta_0),\ldots,\sigma^{-1}(\beta_0)] \in \overline{H}_f(E).
\end{align*}
If we define $\alpha_i = \sigma^{-1}(\beta_i),\ i=0,\ldots,n$, then $[\alpha_0,\ldots,\alpha_n] \in \overline{H}_f(E)$, and $[\beta_0,\ldots,\beta_n] = \overline{\sigma}([\alpha_0,\ldots,\alpha_n]) \in \overline{\sigma}(\overline{H}_f(E))$. This shows $ \overline{H}_f(E') \subset \overline{\sigma}(\overline{H}_f(E))$, and so
$$\overline{\sigma}(\overline{H}_f(E)) = \overline{H}_f(E').$$
Since $\overline{\sigma}$ is a bijection,
$$N_s = |\overline{H}_f(E)| = |\overline{H}_f(E') = N'_s.$$
So $N_s$ is independent of the choice of the extension $F_s = \F_{q^s}$ of $F = \F_q$.
\end{proof}
\end{document}