%&LaTeX
\documentclass[11pt,a4paper]{article}
\usepackage[frenchb,english]{babel}
\usepackage[applemac]{inputenc}
\usepackage[OT1]{fontenc}
\usepackage[]{graphicx}
\usepackage{amsmath}
\usepackage{amsfonts}
\usepackage{amsthm}
\usepackage{amssymb}
\usepackage{yfonts}
\usepackage{mathrsfs}
%\input{8bitdefs}

% marges
\topmargin 10pt
\headsep 10pt
\headheight 10pt
\marginparwidth 30pt
\oddsidemargin 40pt
\evensidemargin 40pt
\footskip 30pt
\textheight 670pt
\textwidth 420pt

\def\imp{\Rightarrow}
\def\gcro{\mbox{[\hspace{-.15em}[}}% intervalles d'entiers 
\def\dcro{\mbox{]\hspace{-.15em}]}}

\newcommand{\D}{\mathrm{d}}
\newcommand{\Q}{\mathbb{Q}}
\newcommand{\Z}{\mathbb{Z}}
\newcommand{\N}{\mathbb{N}}
\newcommand{\R}{\mathbb{R}}
\newcommand{\C}{\mathbb{C}}
\newcommand{\F}{\mathbb{F}}
\newcommand{\re}{\,\mathrm{Re}\,}
\newcommand{\ord}{\mathrm{ord}}                                                                              \newcommand{\n}{\mathrm{N}}
\newcommand{\legendre}[2]{\genfrac{(}{)}{}{}{#1}{#2}}



\title{Solutions to Ireland, Rosen ``A Classical Introduction to Modern Number Theory''}
\author{Richard Ganaye}

\begin{document}

{ \Large \bf Chapter 11} 

\paragraph{Ex. 11.1}{\it  Suppose that we may write the power series $1+a_1u +a_2u^2+\cdots$ as the quotient of two polynomials $P(u)/Q(u)$. Show that we may assume that $P(0) = Q(0) = 1$.
}

\begin{proof}
Here $f(u) = 1+a_1u +a_2u^2+\cdots \in \C[[u]]$ is a formal series in the variable $u$.

We suppose that $f(u) = P(u)/Q(u)$, where we may assume, after simplification, that the two polynomials are relatively prime. Then $P(1)/Q(1) =1$. Write $c = P(1) = Q(1) \in F$.

If $c = 0$, then $u\mid P(u)$ and $u \mid Q(u)$. This is impossible since $P \wedge Q = 1$. So $c \ne 0$.

Define $P_1(u) = (1/c) P(u), Q_1(u) = (1/c) Q(u)$. Then $f(u) = P_1(u)/Q_1(u)$ and $P_1(0) = Q_1(0) = 1$. If we replace $P,Q$ by $P_1,Q_1$, then the pair $(P_1,Q_1)$ has the required properties.
\end{proof}

\paragraph{Ex. 11.2}{\it Prove the converse to Proposition 11.1.1.
}

\begin{proof}
If $N_s = \sum_{j=1}^e \beta_j^s - \sum_{i=1}^d \alpha_i^s,$ where $\alpha_i,\beta_j$ are complex numbers,  then
\begin{align*}
\sum_{s=1}^\infty \frac{N_s u^s}{s} &=  \sum_{j=1}^e \left( \sum_{s=1}^\infty \frac{(\beta_j u)^s}{s} \right) - \sum_{i=1}^d \left( \sum_{s=1}^\infty \frac{(\alpha_i u)^s}{s} \right)\\
&=-\sum_{j=1}^e \ln(1 - \beta_ju) + \sum_{i=1}^d \ln(1 - \alpha_i u).
\end{align*}
Here $u$ is a variable, and both members are formal polynomials in $\C[[u]]$, so we don't study convergence. Nevertheless, the left member has a radius of convergence at least $q^{-n}$, and the right member $\min_{i,j}(1/|\beta_j|,1/|\alpha_i|)$.

Therefore,
$$Z_f(u) = \exp\left(\sum_{s=1}^\infty \frac{N_s u^s}{s} \right) = \prod_{j=1}^e(1 - \beta_j u)^{-1} \prod_{i=1}^d (1-\alpha_i u) = \frac{\prod_{i=1}^d (1 - \alpha_i u)}{\prod_{j=1}^e (1 - \beta_j u)}$$
is a rational fraction.
\end{proof}


\paragraph{Ex. 11.3}{\it Give the details of the proof that $N_s$ is independent of the field $F_s$ (see the concluding paragraph to section 1).
}
\begin{proof}
Suppose that $E$ and $E'$ are two fields containing $F$ both with $q^s$ elements. We first show that there is a isomorphism $\sigma : E \to E'$ which fixes the elements of $F$, by showing that that both $E$ and $E'$ are isomorphic over $F$ to $F[x]/(f(x))$ for some irreducible polynomial $f(x) \in F(x)$.

There is a primitive element $\alpha' \in E'$, i.e. such that $E' = F(\alpha')$. For example, take $\alpha'$ to be a primitive $q^s - 1$ root of unity : since $\alpha$ is a generator of $E'^*$, every element $\gamma \in E'^*$ is equal to $\alpha'^k$ for some integer $k$, thus $\gamma \in F(\alpha')$ (and $0 \in F(\alpha')$). This proves $E' \subset F(\alpha')$, and since $\alpha' \in E'$ and $F \subset E'$, $F(\alpha') \subset E'$, so $E' = F(\alpha')$.

Let $f(x) \in F[x]$ be the minimal polynomial of $\alpha'$ over $F$. Then
$$E' = F(\alpha') \simeq F(x)/(f(x)),$$
where the isomorphism $\sigma_1 : F(\alpha') \to F(x)/(f(x))$ maps $\alpha'$ to $\overline{x} = x + (f(x))$, and maps $a \in F$ on $\overline{a} = a + (f(x))$.
Since $\alpha'$ is a root of $x^{q^s} -x$, $f(x) \mid x^{q^s} - x$. 

 $E$ is a field with $q^s$ elements, so we have $x^{q^s} - x = \prod_{\alpha \in E} (x-\alpha)$. Thus $f(x) \mid \prod_{\alpha \in E} (x-\alpha)$, where $\deg(f(x)) = s \geq 1$, so $f(\alpha) = 0$ for some $\alpha \in E$. The polynomial $f$ being irreducible over $F$, $f$ is the minimal polynomial of $\alpha$ over $F$, thus $F(\alpha) \simeq F[x]/(f(x))$ is a field with $q^s$ elements. 
 Since $F(\alpha) \subset E$, and $|F(\alpha)| = |E|$, we conclude $E = F(\alpha)$, therefore
$$E = F(\alpha) \simeq F(x)/(f(x)),$$
where the isomorphism $\sigma_2 : F(\alpha) \to F(x)/(f(x))$ maps $\alpha$ to $\overline{x} = x + (f(x))$, and maps $a \in F$ on $\overline{a} = a + (f(x))$.

Then  $\sigma = \sigma_1^{-1} \circ \sigma_2 : E \to E'$ is an isomorphism, and $\sigma(a) = a$ for all $a \in F$.
 
 \bigskip
 
 We can now use the isomorphism $\sigma$ to induce a map 
 $$\overline{\sigma}
 \left\{
 \begin{array}{ccl}
 P^n(E)& \to &P^n(E')\\
 {[}\alpha_0,\ldots,\alpha_n{]} &\mapsto & {[}\sigma(\alpha_0),\ldots,\sigma(\alpha_n){]}.
 \end{array}
 \right.
 $$
 Then $\overline{\sigma}$ is injective: if $[\sigma(\alpha_0),\ldots,\sigma(\alpha_n)] = [\sigma(\beta_0),\ldots,\sigma(\beta_n)]$, then there is $\lambda \in F^*$ such that $\beta_i = \lambda \sigma(\alpha_i) = \sigma(\lambda)\sigma(\alpha_i) = \sigma(\lambda \alpha_i,\ i = 0,\ldots,n$, thus $\beta_i = \lambda \alpha_i$, which proves $[\alpha_0,\ldots,\alpha_n] = [\beta_0,\ldots,\beta_n]$.
 
 If $[\gamma_0,\ldots,\gamma_n] $ is any projective point of $P^n(E')$, then 
 $$[\gamma_0,\ldots,\gamma_n] = \overline{\sigma}([\sigma^{-1}(\gamma_0),\ldots,\sigma^{-1}(\gamma_n)]).$$ This proves that $\overline{\sigma}$ is surjective. So $\overline{\sigma}$ is a bijection.
 
 Now take $f(y_0,\ldots,y_n)\in F[y_0,\ldots,y_n]$ an homogeneous polynomial, $\overline{H}_f(E)$ the corresponding projective hypersurface in $P^n(E)$, and $\overline{H}_f(E')$  the corresponding projective hypersurface in $P^n(E')$. We show that $\overline{\sigma}(\overline{H}_f(E)) = \overline{H}_f(E')$.
 
 Since $\sigma$ is a $F$-isomorphism, $\sigma(f(\alpha_0,\ldots,\alpha_n)) = f(\sigma(\alpha_0),\ldots,\sigma(\alpha_n))\quad (\alpha_i \in E)$, and similarly $\sigma^{-1}(f(\beta_0,\ldots,\beta_n)) = f(\sigma^{-1} (\beta_0),\ldots, \sigma^{-1}(\beta_n))\quad  (\beta_i \in E')$, thus
 \begin{align*}
 {[}\alpha_0,\ldots,\alpha_n{]} \in  \overline{H}_f(E)
 &\Rightarrow f(\alpha_0,\ldots,\alpha_n) = 0 \\
 &\Rightarrow \sigma(f(\alpha_0,\ldots,\alpha_n)) = \sigma(0) = 0\\
 & \Rightarrow f(\sigma(\alpha_0),\ldots,\sigma(\alpha_0)) = 0\\
 &\Rightarrow \overline{\sigma}([\alpha_0,\ldots,\alpha_n]) = [\sigma(\alpha_0),\ldots,\sigma(\alpha_0)] \in \overline{H}_f(E').
\end{align*}
This shows $\overline{\sigma}(\overline{H}_f(E)) \subset \overline{H}_f(E')$.

Conversely, 
 \begin{align*}
 {[}\beta_0,\ldots,\beta_n{]} \in  \overline{H}_f(E')
 &\Rightarrow f(\beta_0,\ldots,\beta_n) = 0 \\
 &\Rightarrow \sigma^{-1}(f(\beta_0,\ldots,\beta_n)) = \sigma(0) = 0\\
 & \Rightarrow f(\sigma^{-1}(\beta_0),\ldots,\sigma^{-1}(\beta_0)) = 0\\
 &\Rightarrow \overline{\sigma}^{-1}([\beta_0,\ldots,\beta_n]) = [\sigma^{-1}(\beta_0),\ldots,\sigma^{-1}(\beta_0)] \in \overline{H}_f(E).
\end{align*}
If we define $\alpha_i = \sigma^{-1}(\beta_i),\ i=0,\ldots,n$, then $[\alpha_0,\ldots,\alpha_n] \in \overline{H}_f(E)$, and $[\beta_0,\ldots,\beta_n] = \overline{\sigma}([\alpha_0,\ldots,\alpha_n]) \in \overline{\sigma}(\overline{H}_f(E))$. This shows $ \overline{H}_f(E') \subset \overline{\sigma}(\overline{H}_f(E))$, and so
$$\overline{\sigma}(\overline{H}_f(E)) = \overline{H}_f(E').$$
Since $\overline{\sigma}$ is a bijection,
$$N_s = |\overline{H}_f(E)| = |\overline{H}_f(E') = N'_s.$$
So $N_s$ is independent of the choice of the extension $F_s = \F_{q^s}$ of $F = \F_q$.
\end{proof}


\paragraph{Ex. 11.4}{\it Calculate the zeta function of $x_0 x_1 - x_2 x_3 = 0$ over $\F_p$.
}
\begin{proof}
Here $F = \F_p$, and $F_s = \F_{p^s}$.

To calculate $N_s$, we calculate the number of points at infinity (such that $x_0 = 0$), and the numbers of affine points of the curve $\overline{H}_f(\F_{p^s})$ associate to $$f(x_0,x_1,x_2,x_3) = x_0 x_1 - x_2 x_3.$$
\begin{enumerate}
\item[$\bullet$] To estimate le number of points at infinity, we calculate first the cardinality of the set
$$U = \{(\alpha_0,\alpha_1,\alpha_2,\alpha_3) \in F_s^4 \mid \alpha_0 \alpha_1 - \alpha_2  \alpha_3 =0,\ \alpha_0 = 0\}.$$ 

Then $\alpha_1$ takes an arbitrary value $a \in F_s$. Write
$$U_a =\{(\alpha_0,\alpha_1,\alpha_2,\alpha_3) \in U \mid  \alpha_1 = a\}.$$
Then $U_a =\{(\alpha_0,\alpha_1,\alpha_2,\alpha_3) \in  F_s^4\mid  \alpha_0 = 0,\ \alpha_1 = a,\ \alpha_2\alpha_3 = 0\}$ , thus $U_a = A \cup B$, where 
\begin{align*}
A &= \{(\alpha_0,\alpha_1,\alpha_2,\alpha_3) \in U_a \mid \alpha_2 = 0\},\\
B &= \{(\alpha_0,\alpha_1,\alpha_2,\alpha_3) \in U_a \mid \alpha_3 = 0\}.
\end{align*}
Since $\alpha_0,\alpha_1,\alpha_3$ are fixed in $A$, the map $A \to F_s$ defined by $(\alpha_0,\alpha_1,\alpha_2,\alpha_3) \mapsto \alpha_3$ is a bijection, therefore $|A| = p^s$, and similarly $|B| = p^s$. But $A\cap B = \{(0,0,0,0)\}$, thus 
$$|U_a| = |A| + |B| - |A\cap B| = 2p^s - 1.$$
Since $U$ is the disjoint union of the $U_a$, thus
$$|U| = \sum_{a\in F_s} |U_a| = \sum_{a\in F_s} (2p^s - 1) = 2p^{2s} - p^s .$$
Therefore the number of projective points $[\alpha_0,\alpha_1,\alpha_2,\alpha_3] \in P^3(F_s)$ at infinity (such that $\alpha_0 = 0$) is
$$N_\infty = \frac{|U|-1}{p^s-1} =\frac{ 2p^{2s} - p^s -1}{p^s-1} = 2p^s+1.$$

\item[$\bullet$] Now we calculate the number of points of the affine surface $H_f(\F_s)$ associate to the equation $y_1 = y_2 y_3$ (where $y_i = x_i/x_0$).

The maps 
$$
u
\left\{
\begin{array}{ccl}
F_s^2 & \to & H_f(F_s)\\
(\beta, \gamma) & \mapsto &(\beta \gamma, \beta, \gamma)
\end{array}
\right.
\qquad
\left\{
\begin{array}{ccl}
H_f(F_s) & \to & F_s^2\\
(\alpha,\beta, \gamma) & \mapsto & ( \beta, \gamma)
\end{array}
\right.
$$
satisfy $u \circ v = \mathrm{id}, v\circ u = \mathrm{id}$, so $u$ is a bijection. With more informal words, the arbitrary choice of $\beta, \gamma \in F_s$ gives the affine point $(\alpha,\beta,\gamma)$, where $\alpha = \beta \gamma$.

This gives $|H_f(F_s)| = p^{2s}$.
\end{enumerate}
Therefore
$$N_s = |\overline{H}_f(F_s)| = p^{2s} + 2p^s + 1.$$
We obtain in $\C[[u]]$
\begin{align*}
\sum_{s=1}^\infty \frac{N_s u^s}{s} &= \sum_{s=1}^\infty \frac{(p^2u)^{s}}{s} + 2 \sum_{s=1}^\infty \frac{(pu)^{s}}{s} + \sum_{s=1}^\infty \frac{u^{s}}{s}\\
&= -\ln(1- p^2 u) - 2\ln(1-pu) - \ln(1-u).
\end{align*}
This gives
$$Z_f(u) = (1-p^2u)^{-1}(1-pu)^{-2}(1-u)^{-1}.$$

\bigskip

Note: The result for $N_s$ is verified with the naive and very slow following code in Sage:
\begin{verbatim}

def N(p,s):
    Fs = GF(p^s)
    counter = 0
    for x in Fs:
        for y in Fs:
            for z in Fs:
                for t in Fs:
                    if x*y == z*t:
                        counter += 1
    return (counter - 1)//(p^s - 1)

p, s = 5, 3
print N(p,s), p^(2*s) + 2*p^s +1
\end{verbatim}
\begin{center}
15876 15876
\end{center}
There is a misprint in the ``Selected Hints for the Exercises'' in Ireland-Rosen p.371.
\end{proof}

\paragraph{Ex. 11.5}{\it Calculate as explicitly as possible the zeta function of $a_0x_0^2+a_1x_1^2+\cdots+a_nx_n^2$ over $\F_q$, where $q$ is odd. The answer will depend on wether $n$ is odd or even and whether $q\equiv 1 \pmod 4$ or $q \equiv 3 \pmod 4$.
}

\begin{proof}
Since $q$ is odd, there is a unique character $\chi$ of order 2 over $F = \F_q$, and a unique character of order 2 over $F_s = \F_{q^s}$ . We first compute the number in $\F_q^{n+1}$ of solutions of the equation $f(x_0,\ldots,x_n) = 0$, where $f(x_0,\ldots,x_n) = a_0x_0^2+\cdots+a_n x_n^2\in F[x_0,\ldots,x_n]$.
\begin{align*}
N(a_0x_0^2+\cdots+a_n x_n^2=0)&=\sum\limits_{a_0u_0+\cdots+a_n u_n=0}N(x_0^2=u_0)\cdots N(x_n^2=u_n)\\
&=\sum\limits_{a_0u_0+\cdots+a_n u_n=0}(1+\chi(u_0))\cdots(1+\chi(u_n)) \\
&=\sum\limits_{v_0+\cdots+v_n=0}(1+\chi(a_0)^{-1}\chi(v_0))\cdots(1+\chi(a_n^{-1})\chi(v_n)) \hspace{0.5cm}(v_i = a_i u_i)\\
&=q^{n} + \chi(a_0^{-1})\cdots\chi(a_n^{-1})J_0(\chi,\chi,\cdots,\chi),
\end{align*}
Indeed $J_0(\varepsilon,\ldots,\varepsilon) = q^{l-1}$, and $J_0(\chi_0,\ldots,\chi_n)= 0$ if some but not all of the $\chi_i$ are trivial (generalization of Proposition 8.5.1).

We estimate $J_0(\chi,\ldots,\chi)$, where there are $n+1$ entries of $\chi$.

\begin{enumerate}
\item[$\bullet$] If $n$ is even, then $\chi^{n+1} = \chi \ne \varepsilon$, thus $J_0(\chi,\ldots,\chi) = 0$ (Proposition 8.5.1(d)), and so
$$N(a_0x_0^2+\cdots+a_n x_n^2=0) = q^{n},$$
and the number of projective points on the hypersurface is given by 
$$N_1 = \frac{q^n -1}{q-1} = q^{n-1} + \cdots + q + 1.$$
\item[$\bullet$] If $n$ is odd, then $\chi^{n+1} = \varepsilon$, thus $J_0(\chi,\ldots,\chi) = \chi(-1)(q-1) J(\chi,\ldots,\chi)$, with $n$ entries of $\chi$ (same Proposition).

By Theorem 3 of chapter 8,
$$J(\chi,\ldots,\chi) = \frac{g(\chi)^n}{g(\chi)} = g(\chi)^{n-1}.$$
Since $g(\chi)^2 = g(\chi) g(\chi)^{-1}= \chi(-1)q$ (Exercise 10.22), 
\begin{align*}
\frac{1}{q-1} \, J_0(\chi,\ldots,\chi) &= \chi(-1)g(\chi)^{n-1}\\
&= \chi(-1) g(\chi)^{n-1}\\
&=\frac{\chi(-1) g(\chi)^{n+1}}{g(\chi)^2}\\
&= \frac{1}{q}\,  g(\chi)^{n+1}.
 \end{align*}
 Therefore
 $$N(a_0x_0^2+\cdots+a_n x_n^2=0)  = q^n + \chi(a_0)^{-1}\cdots \chi(a_n)^{-1} \frac{q-1}{q} g(\chi)^{n_1},$$
 and 
 $$N_1 = q^{n-1} + \cdots + q + 1 + \frac{1}{q} \chi(a_0)^{-1}\cdots \chi(a_n)^{-1}g(\chi)^{n+1}.$$
\end{enumerate}
To conclude this first part,
$$
\begin{array}{ll}
N_1 = q^{n-1} + \cdots + q + 1 &\text{if  $n$ is even},\\
N_1 = q^{n-1} + \cdots + q + 1 + \frac{1}{q} \chi(a_0)^{-1}\cdots \chi(a_n)^{-1}g(\chi)^{n+1}& \text{if $n$ is odd}.
\end{array}
$$
To compute $N_s$, we must replace $q$ by $q^s$ and $\chi$ by $\chi_s$, the character of order $2$ on $F_s$. Then
$$
\begin{array}{ll}
N_s = q^{s(n-1)} + \cdots + q^s + 1 &\text{if  $n$ is even},\\
N_s = q^{s(n-1)} + \cdots + q^s + 1 + \frac{1}{q^s} \chi_s(a_0)^{-1}\cdots \chi_s(a_n)^{-1}g(\chi_s)^{n+1}& \text{if $n$ is odd}.
\end{array}
$$
(These two results can also be obtained by using the equations (1) and (2) in Theorem 2 of Chapter 10.)
\bigskip

It remains to study $\chi_s$ in the odd case. 

Since $\chi_s^2 = \varepsilon$, for all $\alpha \in F_s$, $\chi_s(\alpha)^{-1} = \chi_s(\alpha)$, and $\chi_s(\alpha) = -1 \in \C$ if $\alpha^\frac{q^s - 1}{2} = -1 \in F_s, \chi_s(\alpha) = 1$ otherwise.

If $a \in F$, $a^\frac{q-1}{2} = \pm 1 = \varepsilon$. Since $q$ is odd, $1+q+\cdots+q^{s-1} \equiv s \pmod 2$, thus
$$a^{\frac{q^s-1}{2}} = a^{\frac{q-1}{2} (1+q+\cdots + q^{s-1})} =\varepsilon^{1+q+\cdots + q^{s-1}} = \varepsilon^{s},$$
so
$$\chi_s(a) = \chi(a)^s \qquad (a \in F).$$
We know that $g(\chi_s)^2 = \chi_s(-1) q^s$ (Ex. 10.22), thus, as $n$ is odd,
\begin{align*}
g(\chi_s)^{n+1} &= \left[g(\chi_s)^2 \right]^{\frac{n+1}{2}}\\
&=\chi_s(-1)^\frac{n+1}{2} q^{s \frac{n+1}{2}}.
\end{align*}

If $q\equiv 1 \pmod 4$, then $(-1)^{\frac{q-1}{2}} = 1$, so $-1$ is a square in $\F_q$. In this case, $-1$ is a square in $\F_{q^s}$, and $\chi_s(-1) = 1$ for all $s\geq 1$.
In this case, using $a_i \in F$,
\begin{align*}
N_s &= q^{s(n-1)} + \cdots + q^s + 1 + \chi_s(a_0)\cdots \chi_s(a_n)q^{s\frac{n-1}{2}}\\
&= q^{s(n-1)} + \cdots + q^s + 1 + [\chi(a_0)\cdots \chi(a_n)]^s q^{s\frac{n-1}{2}}
\end{align*}

If $q\equiv -1 \pmod 4$, then $\chi(-1) = (-1)^{\frac{q-1}{2}} = -1$, and
\begin{align*}
\chi_s(-1) &=\chi(-1)^s = (-1)^s,
\end{align*}
thus 
$$\frac{1}{q^s} g(\chi_s)^{n+1} = (-1)^{s \frac{n+1}{2}}q^{s \frac{n-1}{2}}.$$
This gives for odd integers $n$, and $q \equiv -1 \pmod 4$,
\begin{align*}
N_s &= q^{s(n-1)} + \cdots + q^s + 1 +  (-1)^{s \frac{n+1}{2}} \chi_s(a_0)\cdots \chi_s(a_n) q^{s \frac{n-1}{2}}\\
&=  q^{s(n-1)} + \cdots + q^s + 1 +  [(-1)^{ \frac{n+1}{2}} \chi(a_0)\cdots \chi(a_n)]^s q^{s \frac{n-1}{2}}.
\end{align*}


To collect all these cases, we have proved
$$
\begin{array}{ll}
N_s = q^{s(n-1)} + \cdots + q^s + 1 &\text{if  } n\equiv 0 \pod 2,\\
N_s = q^{s(n-1)} + \cdots + q^s + 1 + \phantom{(-1)^{ \frac{n+1}{2}}}[\chi(a_0)\cdots \chi(a_n)]^s\, q^{s\frac{n-1}{2}} &\text{if } n \equiv 1 \pod 2, q \equiv +1 \pod 4,\\
N_s = q^{s(n-1)} + \cdots + q^s + 1 +[(-1)^{ \frac{n+1}{2}} \chi(a_0)\cdots \chi(a_n)]^s\, q^{s \frac{n-1}{2}}& \text{if }n \equiv 1 \pod 2, q \equiv -1 \pod 4.
\end{array}
$$
If $n$ is even this gives, as in paragraph 1, 
$$Z_f(u)  = (1-q^{n-1}u)^{-1}\cdots (1-qu)^{-1}(1-u)^{-1}.$$


In the case $n \equiv 1 \pod 2, q \equiv +1 \pod 4$, we write for simplicity $\varepsilon = \chi(a_0)\cdots \chi(a_n) =\pm 1$.
Then 
\begin{align*}
\sum_{s=1}^\infty \frac{N_su^s}{s} &= \sum_{m=0}^{n-1}\left( \sum_{s=1}^\infty \frac{(q^m u)^s}{s} \right) + \sum_{s=1}^\infty \frac{ (\varepsilon q^\frac{n-1}{2} u)^s}{s}\\
&=  - \sum_{m=0}^{n-1} \ln(1-q^m u) -  \ln(1- \varepsilon q^\frac{n-1}{2}u).
\end{align*}
Therefore
$$Z_f(u) = \left[\prod_{m=0}^{n-1} (1 -q^mu)^{-1}\right] (1 - \chi(a_0)\cdots \chi(a_n) q^\frac{n-1}{2}u)^{-1}.$$
(Same calculation in the last case, with $\varepsilon = (-1)^{ \frac{n+1}{2}} \chi(a_0)\cdots \chi(a_n)$.)

We obtain
$$
\begin{array}{ll}
Z_f(u) = P(u)&\text{if  } n\equiv 0 \pod 2,\\
Z_f(u) = P(u)(1 - \phantom{ (-1)^\frac{n+1}{2}}\chi(a_0)\cdots \chi(a_n) q^\frac{n-1}{2}u)^{-1}&\text{if } n \equiv 1 \pod 2, q \equiv +1 \pod 4,\\
Z_f(u) = P(u)(1 - (-1)^\frac{n+1}{2}\chi(a_0)\cdots \chi(a_n) q^\frac{n-1}{2}u)^{-1}& \text{if }n \equiv 1 \pod 2, q \equiv -1 \pod 4,
\end{array}
$$
where $P(u) =  (1-q^{n-1}u)^{-1}\cdots (1-qu)^{-1}(1-u)^{-1}.$

(These results are consistent with the example $N_s = q^{2s} + q^s + 1 +\chi_s(-1) q^s$ given in paragraph 1 for the surface defined by $-y_0^2 + y_1^2 + y_2^2 + y_3^2 =0$, where $n=3$ is odd.
\begin{align*}
Z_f(u) &= (1-q^2u)^{-1}(1-qu)^{-1}(1-u)^{-1} (1 - \chi(-1) qu)^{-1}\\
&=
\left\{
\begin{array}{ll}
(1-q^2u)^{-1}(1-qu)^{-2}(1-u)^{-1}  & \text{if } q \equiv 1 \pmod 4,\\
(1-q^2u)^{-1}(1-qu)^{-1}(1-u)^{-1}(1+qu)^{-1} & \text{if } q \equiv -1 \pmod 4.)\\
\end{array}
\right.
\end{align*}
\end{proof}


\paragraph{Ex. 11.6}{\it Consider $x_0^3 + x_1^3 + x_2^3 = 0$ as an equation over $F_4$, the field with four elements. Show that there are nine points on the curve in $P^2(F_4)$. Calculate the zeta function. [Answer: $(1+2u)^2/((1-u)(1-4u))$.]
}

\begin{proof}
Since $q=4 \equiv 1 \pmod 3$, we can apply Theorem 2 of Chapter 10. Let $\chi$ be a character of order 3 over $F=\F_4$. The only other character of order $3$ is then $\chi^2$. Thus

\begin{align*}
N_1 &= q+1 + \frac{1}{q-1}  \sum_{i,j,k} J_0(\chi^i,\chi^j,\chi^k),
\end{align*}
where the sum is over all $(i,j,k) \in \{1,2\}^3$ such that $i+j+k \equiv 0 \pmod 3$, that is $(1,1,1)$ and $(2,2,2)$. Thus
\begin{align*}
N_1& = q +1 + \frac{1}{q-1} \left(J_0(\chi,\chi,\chi) + J_0(\chi^2,\chi^2,\chi^2)\right).
\end{align*}
Using $\frac{1}{q-1}\,  J_0(\chi^k,\chi^k,\chi^k) = \frac{1}{q}\, g(\chi^k)^3$ for $k=1,2$, we obtain
\begin{align*}
N_1&=q +1 + \frac{1}{q}\left ( g(\chi)^3 + g(\chi^2)^3\right).
\end{align*}
Consider $\F_4 = \F_2[x]/(x^2+x+1)$, where $a = \overline{x} = x + (x^2+x+1)$ is a generator of $\F_4^*$. Then  $\F_4 = \{0,1,a,a^2 = a+1\}$.
We compute $g(\chi)$ for the character $\chi$ of order $3$ defined by
$$
\begin{array}{c|cccc|}
t & 0 & 1 & a & a^2\\
\hline
\chi(t) & 0 & 1 & \omega & \omega^2
\end{array}
$$
where $\omega = e^\frac{2i\pi}{3}$. 

for each $t\in \F_4$, $\mathrm{tr}(a) = a+ a^2 \in \F_2$, so the traces are $\mathrm{tr}(1) = 1 + 1 = 0, \mathrm{tr}(a) = a+a^2 = 1, \mathrm{tr}(a^2) = a^2 + a^4 = a^2 + a = 1$. Therefore
\begin{align*}
g(\chi) &= \sum_{t \in \F_4} \chi(t) \zeta_2^{\mathrm{tr}(t)}\\
&= \sum_{t \in \F_4} \chi(t) (-1)^{\mathrm{tr}(t)}\\
&= 1 - \omega - \omega^2\\
&= 2.
\end{align*}
(This is in accordance with $|g(\chi)| = q^{1/2} = 2$.)
Then $g(\chi^2) = g(\chi^{-1}) = \chi(-1) \overline{g(\chi)} = g(\chi) = 2$. Therefore
\begin{align*}
N_1 &= q +1 + \frac{1}{q} g(\chi)^3 + \frac{1}{q} g(\chi^2)^3\\
&= 5 + \frac{1}{4}( 8 + 8)\\
&= 9.
\end{align*} 
There are nine points on the curve with equation $x_0^3 + x_1^3 + x_2^3 = 0$ in $P^2(F_4)$ (this is verified with a naive program in Sage).

Now we compute $N_s$. We must replace $q =4$ by $q^s = 4^s$, and $\chi$ by $\chi_s$, a character with order 3 on $F_s = \F_{4^s}$.

We obtain
$$N_s = q^s + 1 + \frac{1}{q^s} \left (g(\chi_s)^3 + g(\chi_s^2)^3 \right).$$

Now we compute $g(\chi_s)^3$. By the generalization of Corollary of Proposition 8.3.3., 
$$g(\chi_s)^3 = q^s J(\chi_s,\chi_s),$$
thus
$$N_s = q^s + 1 +  J(\chi_s,\chi_s) + J(\chi_s^2,\chi_s^2) .
$$
We know that $|J(\chi_s,\chi_s)|^2 = q^s = 4^s$ (generalization of Corollary of Theorem 1). Writing $J(\chi_s,\chi_s) = a+ b \omega,\ a,b\in \Z$, we search the solutions of
$$|a+ b \omega|^2 = a^2 - ab + b^2 = 4^s.$$
Since $\Z[\omega]$ is a PID, the factorization in primes is unique. Here $2$ is a prime element of $\Z[\omega]$, and $(a+b \omega)(a+b \omega^2) = 2^{2s}$, therefore $a+b \omega =\varepsilon 2^k, a+b \omega^2 = \zeta 2^l$, where $l,k \in \N$ and $\varepsilon, \zeta$ are units. Moreover $2^k = |a+ b \omega| = |a+b \omega^2| = 2^l$, so $k = l = s$. This shows that every solution $a+ b \omega$ of $|a+ b \omega|^2  =4^s$ is associated to $2^s$:
$$|a+ b \omega|^2  =4^s \iff a+b \omega \in \{-2^s, -1 - 2^s \omega, -2^s \omega, 2^s, 1 + 2^s \omega, 2^s \omega\}.$$

Moreover, we know that $a \equiv -1 \pmod 3,\ b \equiv 0 \pmod 3$ (generalization of Proposition 8.3.4.). Therefore $$J(\chi_s,\chi_s) = a + b \omega = -(-2)^s,$$
and similarly $J(\chi_s^2,\chi_s^2) = -(-2)^s$ (this proves particular cases of the Hasse-Davenport relation, which we have not used here).  This gives
$$N_s = 4^s + 1 -2 (-2)^{s}.$$
For $s = 1$, we find anew $N_1 = 9$.

Then 
\begin{align*}
\sum_{s=1}^\infty \frac{N_s u^s}{s} &= \sum_{s=1}^\infty \frac{(4u)^s}{s}  + \sum_{s=1}^\infty \frac{u^s}{s} - 2 \sum_{s=1}^\infty \frac{(-2u)^s}{s}\\
&= - \ln(1-4u) - \ln(1-u) + 2 \ln(1+2u).\\
\end{align*}
This gives
$$Z_f(u) = \frac{(1+2u)^2}{(1-4u)(1-u)}.$$
This is the first example where $Z_f$ has a zero, which satisfies the Riemann hypothesis for curves.
\end{proof}

\paragraph{Ex. 11.7}{\it Try this exercise if you know a little projective geometry. Let $N_s$ be the number of lines in $P_n(F_{p^s})$. Find $N_s$ and calculate $\sum_{s=1}^\infty N_su^s/s$. (The set of lines in projective space form an algebraic variety calles a Grassmannian variety. So do the set of planes three-dimensinal linear subspaces, etc.)
}
\begin{proof}
Write $q = p^s$. The set of lines in $P_n(F_{q})$ is in bijective correspondence with the set of planes of the vector space $F_q^{n+1}$. To count these planes, consider the set $A$ of  linearly independent pairs$(u,v)$ of the space $F_q^{n+1}$, and $B$ the set of planes of $F_q^{n+1}$, and
$$
f 
\left\{
\begin{array}{ccl}
A & \to B \\
(u,v) & \mapsto \langle u, v \rangle.
\end{array}
\right.
$$
 The set of pre-images of a fixed plane $P$ in $B$ is the set of basis of this plane $P$. Thus, to obtain $N_s$, we divides the number of linearly independent pairs$(u,v)$ of the space by the number of basis of a fixed plane. To build such a pair, we choose first a nonzero vector $u$, and then a vector $v$ not on the line generated by $u$. Therefore
\begin{align*}
N_s &= \frac{(q^{n+1} - 1)(q^{n+1}- q)}{(q^2 - 1)(q^2 -q)}\\
&= \frac{(q^{n+1} - 1)(q^{n}- 1)}{(q^2 - 1)(q -1)}.
\end{align*}
\end{proof}

$\bullet$ If $n = 2m+1$ is odd, then
\begin{align*}
N_s &= \frac{q^{2m+2}-1}{q^2-1} \cdot \frac{q^{2m+1}-1}{q-1}\\
&=\sum_{k=0}^m q^{2k} \sum_{l = 0}^2 q^l\\
&=\sum_{k=0}^m \sum_{l = 0} ^{2m} q^{2k+l}\\
&=\sum_{r=0}^{4m} a_r q^r \qquad (r= 2k+l),
\end{align*}
where $a_r$ is the cardinality of the set
$$A_r = \{(k,l) \in \gcro 0, m \dcro \times \gcro 0, 2m \dcro \mid 2k+l = r\}.$$
We note that $0 \leq l = r-2k \leq 2m$ gives 
$$
\left\{
\begin{array}{rcl}
\frac{r}{2} - m \leq &k& \leq \frac{r}{2},\\
0 \leq &k& \leq m,
\end{array}
\right.
$$
that is
\begin{align}
\max\left( 0,  \frac{r}{2}  - m \right) \leq k \leq \min \left(\frac{r}{2} ,m \right),
\end{align}
 and each such $k$ gives a unique pair $(k,l) = (k,r-2k)$ in $A_r$.

\begin{enumerate}
\item[$-$] If $0 \leq r \leq 2m$, then $(1) \iff 0 \leq k \leq \frac{r}{2}$, thus $a_r = \left \lfloor \frac{r}{2} \right \rfloor + 1$.
\item[$-$] If $2m < r \leq 4m$, then $(1) \iff \frac{r}{2} -m  \leq k \leq m$, thus $a_r = 2m - \left \lceil \frac{r}{2} \right \rceil + 1$.
\end{enumerate}
If $n$ is odd, we have proved that
\begin{align*}
N_s &= \sum_{r=0}^{2m} \left (\left \lfloor \frac{r}{2} \right \rfloor + 1 \right) q^r + \sum_{r=2m+1}^{4m} \left(2m+1 - \left \lceil \frac{r}{2} \right \rceil \right) q^r\\
&=\sum_{r=0}^{n-1} \left (\left \lfloor \frac{r}{2} \right \rfloor + 1 \right) p^{sr} + \sum_{r=n}^{2n-2} \left (n- \left \lceil \frac{r}{2} \right \rceil  \right) p^{sr}.
\end{align*}

\bigskip

$\bullet$ If $n = 2m$ is even, then
\begin{align*}
N_s &=\frac{q^{2m}-1}{q^2-1}\cdot  \frac{q^{2m+1}-1}{q-1} \\
&= \sum_{k=0}^{m-1} q^{2k} \sum_{l = 0}^{2m} q^l\\
&=\sum_{k=0}^{m-1} \sum_{l = 0} ^{2m} q^{2k+l}\\
&=\sum_{r=0}^{4m-2} b_r q^r \qquad (r= 2k+l),
\end{align*}
where $b_r$ is the cardinality of the set
$$B_r = \{(k,l) \in \gcro 0, m-1 \dcro \times \gcro 0, 2m \dcro \mid 2k+l = r\}.$$
Here $0 \leq l = r-2k \leq 2m$ gives 
$$
\left\{
\begin{array}{rcl}
\frac{r}{2} - m \leq &k& \leq \frac{r}{2},\\
0 \leq &k& \leq m -1,
\end{array}
\right.
$$
that is
\begin{align}
\max\left( 0,  \frac{r}{2}  - m \right) \leq k \leq \min \left(\frac{r}{2} ,m-1 \right),
\end{align}
 and each such $k$ gives a unique pair $(k,l) = (k,r-2k)$ in $B_r$.
\begin{enumerate}
\item[$-$] If $0 \leq r \leq 2m-1$, then $(2) \iff 0 \leq k \leq \frac{r}{2}$, thus $b_r = \left \lfloor \frac{r}{2} \right \rfloor + 1$.
\item[$-$] If $2m \leq r \leq 4m-2$, then $(2) \iff \frac{r}{2} -m  \leq k \leq m-1$, thus $b_r = 2m - \left \lceil \frac{r}{2} \right \rceil$.
\end{enumerate}
If $n$ is odd, we have proved that
\begin{align*}
N_s &= \sum_{r = 0}^{2m-1}  \left (\left \lfloor \frac{r}{2} \right \rfloor + 1\right) q^r + \sum_{r=2m}^{4m-2} \left ( 2m - \left \lceil \frac{r}{2} \right \rceil \right) q^r\\
&= \sum_{r = 0}^{n-1}  \left (\left \lfloor \frac{r}{2} \right \rfloor + 1\right) p^{sr} + \sum_{r=n}^{2n-2} \left ( n - \left \lceil \frac{r}{2} \right \rceil \right) p^{sr}.
\end{align*}
This is the same formula as in the odd case !
To conclude, for all dimension $n$,
$$N_s = \sum_{r = 0}^{n-1}  \left (\left \lfloor \frac{r}{2} \right \rfloor + 1\right) p^{sr} + \sum_{r=n}^{2n-2} \left ( n - \left \lceil \frac{r}{2} \right \rceil \right) p^{sr},$$
therefore
$$\sum_{s=1}^\infty \frac{N_s u^s}{s} = - \sum_{r=0}^{n-1}  \left (\left \lfloor \frac{r}{2} \right \rfloor + 1 \right) \ln(1-p^ru) - \sum_{r=n}^{2n-2}  \left (n - \left \lceil \frac{r}{2} \right \rceil  \right) \ln(1 - p^r u)$$
This gives the order of the poles $p^{-r}$ of $Z(u) = \exp\left( \sum_{s=1}^\infty \frac{N_s u^s}{s}  \right)$.

To verify the equality between the two formulas giving $N_s$, we test this equality with a Sage program.
\begin{verbatim}
def N(n,p,s):
     q = p^s
     num = (q^(n+1) - 1)*(q^(n+1) - q)
     den = (q^2 - 1)*(q^2-q)
     return num // den
     
 def M(n,p,s):
    q = p^s
    a = sum((floor(r/2) +1)*q^r for r in range(n))
    b = sum((n - ceil(r/2))*q^r for r in range(n,2*n-1))
    return a+b
    
 N(4,5,3),M(4,5,3)
\end{verbatim}
\begin{center}
(3845707062626, 3845707062626)
\end{center}


\paragraph{Ex. 11.8}{\it If $f$ is a nonhomogeneous polynomial, we can consider the zeta function of the projective closure of the hypersurface defined by $f$ (see Chapter 10). One way to calculate this is to count the number of points on $H_f(F_q)$ and then add to it the number of points at infinity. For example, consider $y^2 = x^3$ over $F_{p^s}$. Show that there is one point at infinity. The origin $(0,0)$ is clearly on this curve. If $x \ne 0$, write $(y/x)^2 = x$ and show that there are $p^s$ more points on this curve. Altogether we have $p^s$ points and the zeta function over $F_p$ is $(1-pu)^{-1}$.
}
\begin{proof}
Consider the polynomial $f(x,y) = y^2 - x^3$ and $g(x,z) = y^2 - x$, and
\begin{align*}
\Gamma =H_f(F_q) &= \{(x,y) \in F_p^2 \mid y^2 = x^3\},\\
\Gamma_1 = H_g(F_q) &= \{(x,y)\in F_q^2 \mid y^2 = x\}.
\end{align*}
Then 
$$
\varphi
\left\{
\begin{array}{ccl}
\Gamma \setminus \{(0,0)\}  & \to &\Gamma_1\setminus \{(0,0)\} \\
(x,y) & \mapsto & \left(x,\frac{y}{x} \right)
\end{array}
\right.
$$
is defined, since $\left(\frac{y}{x} \right) ^2 = x$ for $(x,y) \in \Gamma \setminus \{(0,0)\}$, thus $ \left(x,\frac{y}{x} \right) \in \Gamma_1$.
Moreover
$$
\psi
\left\{
\begin{array}{ccl}
 \Gamma_1\setminus \{(0,0)\} & \to & \Gamma \setminus \{(0,0)\} \\
(x,y) & \mapsto & \left(x,xy \right)
\end{array}
\right.
$$
is correctly defined, since for each $(x,y) \in \Gamma_1\setminus \{(0,0)\}, y^2 = x$, then $x \ne 0$, thus $(xy)^2 =x^3$, and $(x,xy) \in \Gamma$, where $(x,xy) \ne (0,0)$.

Moreover $\psi$ satisfies $\psi \circ \varphi = \mathrm{id}, \varphi \circ \psi = \mathrm{id}$:
\begin{align*}
(\psi \circ \varphi)(x,y) &= \psi\left(x,\frac{y}{x} \right) =\left (x, x \frac{y}{x}\right) = (x,y) \qquad ((x,y) \in \Gamma \setminus \{(0,0)\}),\\
(\varphi \circ \psi)(x,y) &= \varphi(x,xy) = \left( x, \frac{xy}{x} \right) = (x,y) \qquad ((x,y) \in \Gamma_1 \setminus \{(0,0)\}).
\end{align*}
So $\varphi$ is a bijection. This shows that $| \Gamma \setminus \{(0,0)\} | =  |\Gamma_1\setminus \{(0,0)\}|$, where $(0,0) \in \Gamma$ and $(0,0) \in \Gamma_1$, thus
$$|\Gamma_1| = |\Gamma|.$$ 
To count the points on $\Gamma_1$, we consider
$$\lambda
\left\{
\begin{array}{ccl}
F_q & \to & \Gamma_1\\
y & \mapsto & (y^2,y).
\end{array}
\right.
$$
Then $\lambda$ is bijective, with inverse $\mu : (x,y) \mapsto y$. This show that $$|\Gamma| = |\Gamma_1| = q = p^s.$$

Therefore the zeta function of the affine curve $y^2 = x^3$ over $F_p$ is 
$$Z_f(u) = (1-pu)^{-1}.$$

But the projective closure $H_{\overline{f}}(F_q)$ of this curve has $p^s + 1$ points, with only one point at infinity, since $ty^2 = x^3$ has only one point $[t,x,y]$ satisfying $t= 0$, the point $[0,0,1]$.

The zeta function of the curve with homogeneous equation $\overline{f}(t,x,y) = ty^2 - x^3$ over $F_p$ is 
$$Z_{\overline{f}}(u) = (1-u)^{-1} (1-pu)^{-1}.$$
\end{proof}


\paragraph{Ex. 11.9}{\it Calculate the zeta function of $y^2 = x^3 + x^2$ over $F_p$.
}
\begin{proof}
The curve $\Gamma$ defined by the  equation $y^2 = x^3 + x^2$ has a singularity at the origine, as in the previous exercise. The same method applies here: if we use $z= y/x$, then $z^2 = x+1$.

Watch out! Here there are two points $(x,z) \in \Gamma_1$ such that $x= 0$, the points $(0,1)$ and $(0,-1)$ (here we assume that $p\ne 2$).
The curve $\Gamma_1$ defined by the equation $z^2 = x+ 1$ is such that 
$$
\varphi
\left\{
\begin{array}{ccl}
\Gamma \setminus \{(0,0)\}  & \to &\Gamma_1\setminus \{(0,1),(0,-1)\} \\
(x,y) & \mapsto & \left(x,\frac{y}{x} \right)
\end{array}
\right.
$$
is bijective, thus $|\Gamma| = |\Gamma_1| - 1$. Since each point of $\Gamma_1$ is determined by its coordinate $z$, $|\Gamma_1| = q = p^s$, and $|\Gamma| = p^s - 1$.

Therefore the zeta function of the affine curve $y^2 = x^3 + x^2$ over $F_p$ is 
$$Z_f(u) = (1-u)(1-pu)^{-1},$$
There is only one point $p$ at infinity, given by $y^2 t = x^3 + x^2 t, t=0$, i.e. $p = [0,0,1]$. Thus $N_s = p^s$, and the zeta function of the projective closure of $\Gamma$ is
$$Z_{\overline{f}}(u) = (1-pu)^{-1}.$$
\end{proof}
The results of Ex.8 and Ex. 9 concern only singular cubics.


\paragraph{Ex. 11.10}{\it  If $A \ne 0$ in $F_q$ and $q \equiv 1 \pmod 3$, show that the zeta function of $y^2 = x^3 + A$ over $F_q$ has the form $Z(u) =(1+au+qu^2)/((1-u)(1-qu))$, where $a \in \Z$ and $|a| \leq 2q^{1/2}$.
}
\begin{proof} 

Here we compute the zeta function of the projective closure $\overline{H}_f(F_q)$, with equation $f(x,y,t) =y^2 t = x^3 + At^3$. If $t= 0$, then $x=0$, thus there is only one point $[0,1,0]$ at infinity (over $F_q$ or over $F_{q^s}$).

We assume that the characteristic is not 2. Then $q$ is odd, and so $q\equiv 1 \pmod 6$. Therefore, there are characters of order $2$ and $3$ on $F_q$. Write $\rho$ the unique character of order $2$, and write $\chi$ a character of order $3$. As $\chi$ is a character of order 3, the characters whose order divides 3 are $\varepsilon,\chi,\chi^2$.

We compute first $N_1$. We write $N(y^2=x^3+A)$ for the number of points of the affine cubic over $F_q$, and $N_1$ for the number of points of the projective cubic, so that $N_1 = N(y^2=x^3+A) + 1$. We recall the results obtained in Ex. 8.15.

The map $x \mapsto -x$ is a bijection between the set of roots of $x^3 = b$ and the set of roots of $(-x)^3 = b$, so $N(x^3 = b) = N((-x)^3 = b) = N(x^3 = -b)$.

 Using Prop. 8.1.5, we obtain, since $A\ne 0$,
\begin{align*}
N(y^2=x^3+A)  &= \sum\limits_{a+b=A} N(y^2=a) N(x^3 =-b)\\
&= \sum\limits_{a+b=A} N(y^2=a) N(x^3=b)\\
&= \sum\limits_{a+b=A}(1+\rho(a))(1+\chi(b)+\chi^2(b))\\
&=\sum\limits_{i=0}^1\sum\limits_{j=0}^2\sum\limits_{a+b=A} \rho^i(a) \chi^j(b)\\
&=\sum\limits_{i=0}^1\sum\limits_{j=0}^2\rho(A)^i \chi(A)^j\sum\limits_{a'+b'=1} \rho^i(a') \chi^j(b')\qquad (a = Aa', b = A b')\\
&=\sum\limits_{i=0}^1\sum\limits_{j=0}^2\rho(A)^i \chi(A)^j J(\chi^j,\rho^i).\\
\end{align*}
We know (generalization of Theorem 1, Chapter 8) that  $J(\chi,\varepsilon)=J(\chi^2,\varepsilon)=J(\varepsilon,\rho)=0,$ and $J(\varepsilon,\varepsilon)=q$, so
$$N(y^2=x^3+A) = q+ \rho(A)\chi(A) J(\chi,\rho) + \rho(A) \chi^2(A) J(\chi^2,\rho).$$
As $\chi^2(A) = \chi^{-1}(A) = \overline{\chi(A)}$, and as $\overline{\rho(A)} = \rho(A)$, then $J(\chi^2,\rho) =  J(\overline{\chi},\overline{\rho}) = \overline{J(\chi,\rho)}$, and
$$N(y^2 = x^3+A) = q +\pi + \bar{\pi},\ \mathrm{where}\ \pi = \rho(A) \chi(A) J(\chi,\rho),$$
therefore
$$N_1 = q + 1 + \pi + \bar{\pi},\ \mathrm{where}\ \pi = \rho(A) \chi(A) J(\chi,\rho).$$
Since the orders of $\chi, \rho$, and $\chi \rho$ are $3,2$ and $6$, $\chi \ne \varepsilon, \rho \ne \varepsilon, \chi \rho \ne \varepsilon$, thus Theorem 1 of Chapter 6 gives
$$J(\chi,\rho) = \frac{g(\chi) g(\rho)}{g(\chi \rho)}, \qquad \pi = \rho(A) \chi(A) \frac{g(\chi) g(\rho)}{g(\chi \rho)}.$$
Write $\chi' = \chi \circ \mathrm{N}_{F_{q^s}/F_q}, \rho' = \rho \circ \mathrm{N}_{F_{q^s}/F_q}$. Then $\chi', \rho'$ are characters on $F_{q^s}$, and the orders of $\chi', \rho'$ are $3$ and $2$ (by properties (a), (b) of �3). The same reasoning in $F_{q^s}$ gives
$$N_s = q^s + 1 + \pi' + \overline{\pi'}, \qquad \pi' = \rho'(A) \chi'(A) \frac{g(\chi') g(\rho')}{g(\chi' \rho')}.$$
Since $A \in F_q$, the property (c) of �3 gives $\chi'(A) = \chi(A)^s, \rho'(A) = \rho(A)^s$. Using the Hasse-Davenport  Relation, and $(\chi \rho)' = \chi' \rho'$, we obtain
\begin{align*}
\pi' &= \rho'(A) \chi'(A) \frac{g(\chi') g(\rho')}{g(\chi' \rho')}\\
&=-\rho(A)^s \chi(A)^s \frac{(-g(\chi))^s (-g(\rho))^s}{(-g(\chi \rho))^s}\\
&=(-1)^{s+1} \rho(A)^s \chi(A)^s \left[ \frac{g(\chi) g(\rho)}{g(\chi \rho)} \right]^s\\
&= - \left[ - \rho(A) \chi(A) \frac{g(\chi) g(\rho)}{g(\chi \rho)}\right]^s\\
&=-(-\pi)^s.
\end{align*}
This gives $N_s$ in the appropriate form:
$$N_s = q^s + 1- (-\pi)^s - (-\overline{\pi})^s, \qquad \pi = \rho(A) \chi(A) J(\chi,\rho) = \rho(A) \chi(A) \frac{g(\chi) g(\rho)}{g(\chi \rho)}.$$
Using the converse to Proposition 11.1.1 given in Exercise 2, we obtain
$$Z_f(u) =  \frac{(1 + \pi u)(1 + \overline{\pi} u)}{(1 -u)(1-qu)}.$$
Note that $\pi \overline{\pi}= |\pi|^2 = q$ (by Exercise 10.22). Expanding the numerator, this gives
$$Z_f(u) =  \frac{1 + au + qu^2}{(1 -u)(1-qu)},$$
where $a = \pi +\overline{\pi}$.

 For all $t \in F_q^*$,  $\chi^3(t) = 1$, thus $\chi(t) \in \{1,\omega, \omega^2\} \subset \Z[\omega]$, and $\rho(t) = \pm 1$, therefore $\pi = \rho(A) \chi(A) \sum_{t\in F_q} \chi(t) \rho(t) \in \Z[\omega]$. Writing $\pi = u + v \omega,\ u,v \in \Z$, we obtain $a = \pi + \overline{\pi} = 2u -v \in \Z$.
 
 Moreover,
 $$|a|  \leq |\pi| + |\overline{\pi}| = 2 |\pi|  = 2 q^{1/2}.$$
 To conclude,
 $$Z_f(u) =  \frac{1 + au + qu^2}{(1 -u)(1-qu)}, \qquad a \in \Z,\  |a| \leq  2 q^{1/2}.$$
\end{proof}


\paragraph{Ex. 11.11}{\it Consider the curve $y^2 = x^3 -Dx$ over $F_p$, where $D \ne 0$. Call this curve $C_1$. Show that the substitution $x = \frac{1}{2}(u+v^2)$ and $y = \frac{1}{2} v(u + v^2)$ transforms $C_1$ into the curve $C_2$ given by $u^2 - v^4 = 4D$. Show that in any given finite field the number of finite points on $C_1$ is one more than the number of finite points on $C_2$.
}

\begin{proof} Let $F$ be a finite field such that the characteristic of $F$ is not $2$. Here
\begin{align*}
C_1 &= \{(x,y) \in F^2 \mid y^2 = x^3 -Dx\},\\
C_2 &=\{(u,v) \in F^2 \mid u^2 - v^4 = 4D\}.
\end{align*}
Consider the maps
$$
\varphi
\left\{
\begin{array}{ccl}
C_1 \setminus \{(0,0)\} & \to & C_2\\
(x,y) & \mapsto &\left(2x - \left(\frac{y}{x} \right)^2, \frac{y}{x}\right),\\
\end{array}
\right.
\qquad
\psi
\left\{
\begin{array}{ccl}
C_2  & \to & C_1 \setminus \{(0,0)\}\\
(u,v) & \mapsto & \left(\frac{1}{2}(u+v^2), \frac{1}{2} v (u+v^2)\right).\\
\end{array}
\right.
$$

$\bullet$ The map $\varphi$ is  well defined: If $(x,y) \in C_1 \setminus \{(0,0)\}$, then 
$y^2 = x^3 - Dx$, and $x \ne 0$, otherwise $y^2 = x^3 - Dx = 0$, and then $(x,y) = (0,0)$. 

Write $(u,v) = \left(2x - \left(\frac{y}{x} \right)^2, \frac{y}{x}\right)$, then
 $x = \frac{1}{2}(u+v^2)$ and $y = \frac{1}{2} v(u+v^2)$. The equality $y^2 = x^3 - Dx$ gives 
\begin{align*}
\frac{1}{2} v^2(u+v^2) &= \frac{1}{4} (u + v^2)^2 - D,\\
4D &= (u+v^2)^2 - 2 v^2(u+v^2),\\
4D &= u^2 - v^4, 
\end{align*}
so that $(u,v) = \left(2x - \left(\frac{y}{x} \right), \frac{y}{x}\right) \in C_2$.

$\bullet$ The map $\psi$ is well defined: if $(u,v) \in C_2$, then $u^2 - v^4 = 4D$. Then $u + v^2 \ne 0$, otherwise $4D = u^2 - v^4 = (u-v^2)(u+v^2) =0$, where $4D \ne 0$ ($D \ne 0$ , and the characteristic is not $2$ by hypothesis).

 Write $(x,y) = \left(\frac{1}{2}(u+v^2), \frac{1}{2} v (u+v^2)\right)$. Then $x =\frac{1}{2}(u+v^2) \ne 0$, and $(u,v) = \left(2x - \left(\frac{y}{x} \right)^2, \frac{y}{x}\right)$.
The equality $u^2 - v^4 = 4D$ gives
\begin{align*}
\left(2x - \left(\frac{y}{x} \right)^2\right)^2 -  \left(\frac{y}{x}\right)^4&= 4D,\\
 4x^2 - 4 \frac{y^2}{x}&= 4D,\\
x^3 - Dx &= y^2,
\end{align*}
so that $(x,y) = \left(\frac{1}{2}(u+v^2), \frac{1}{2} v (u+v^2)\right) \in C_1$, and $(x,y) \ne (0,0)$.

Take any point $(x,y) \in C_1 \setminus \{(0,0)\} $, then $x \ne 0$. Write $(u,v) = \varphi(x,y) = \left(2x - \left(\frac{y}{x} \right), \frac{y}{x}\right)$. Then $(x,y) = \left(\frac{1}{2}(u+v^2), \frac{1}{2} v (u+v^2)\right)= \psi(u,v) = (\psi \circ \varphi)(x,y)$. Thus $\psi \circ \varphi = 1_{C_1 \setminus \{(0,0)\}}$.
Similarly, take any point $(u,v) \in C_2$. Write $(x,y) = \psi(u,v) = \left(\frac{1}{2}(u+v^2), \frac{1}{2} v (u+v^2)\right)$. Then $(u,v) = \left(2x - \left(\frac{y}{x} \right)^2, \frac{y}{x}\right) = \varphi(x,y) = (\varphi \circ \psi)(u,v)$. Thus $\varphi \circ \psi = 1_{C_2}$.

This proves that $\varphi$ and $\psi$ are bijections.

Therefore $|C_2| = |C_1 \setminus \{(0,0)\}| = |C_1| - 1$, and $|C_1| = |C_2| +1$. 

To conclude, in any given finite field whose characteristic is not $2$, the number of finite points on $C_1$ is one more than the number of finite points on $C_2$.
\end{proof}

\paragraph{Ex. 11.12}{\it (continuation) 

 If $p\equiv 3 \pmod 4$, show that the number of projective points on $C_1$ is just $p+1$.

 If $p\equiv 1 \pmod 4$, show that the answer is $p+1 + \overline{\chi(D)} J(\chi,\chi^2) + \chi(D) J(\chi,\chi^2)$, where $\chi$ is a character of order $4$ on $F_p$.
}

\medskip

Note: There is an obvious misprint. We must read $p+1 + \overline{\chi(D)} J(\chi,\chi^2) + \chi(D) \overline{J(\chi,\chi^2)}$
\begin{proof} 

$\bullet$ Assume first that $p\equiv 3 \pmod 4$. First, we count the number of affine points on $C_2$. 

In this case, there is no character of order $4$, and the only characters whose order divides $4$ are $\varepsilon$ and $\rho$, where $\rho$ is the Legendre's character. Then Exercises 8.1, 8.2,  with $d = 4 \wedge (p-1) = 2$, and Proposition 8.1.5 show that $N(x^4 = a) = N(y^2 =a) = 1 + \rho(a).$ Therefore


\begin{align*}
N(u^2 - v^4 = 4D) &= \sum_{a-b = 4D} N(u^2=a)N(v^4=b)\\
&=\sum_{a-b = 4D} (1+\rho(a))(1+\rho(b))\\
&=\sum_{a \in F} (1+ \rho(a))(1+ \rho(a-4D))\\
&= \sum_{a \in F} 1 + \sum_{a \in F} \rho(a) + \sum_{a \in F} \rho(a-4D) + \sum_{a\in F}\rho(a) \rho(a-4D)\\
&= p + \sum_{a\in F}\rho(a) \rho(a-4D).
\end{align*}
We compute this last sum.
\begin{align*}
\sum_{a\in F}\rho(a) \rho(a-4D) &= \rho(-1) \sum_{a \in F} \rho(a) \rho(c)\\
&= \rho(-1) \sum_{a+ c = 4D} \rho(a) \rho(c)\\
&= \rho(-1) \sum_{a'+c' = 1} \rho(4D)^2 \rho(a') \rho(b')\qquad (a = 4Da', c = 4Db')\\
&= \rho(-1) J(\rho,\rho).
\end{align*}
Moreover, by Theorem 1(c), Chapter 8, since $\rho^2 = \varepsilon$,
$$J(\rho,\rho) = J(\rho, \rho^{-1}) = -\rho(-1).$$
Putting all together, we obtain
$$N(u^2 - v^4 = 4D) = p-1.$$
Then Exercise 11 gives
$$N(y^2 = x^3 -Dx) = p.$$
The projective closure of $C_1$ has equation $y^2 t =x^3 - Dxt^2$. For $t=0$, $x=0$, thus $[0,1,0]$ is the only point at infinity. The number of projective points on $C_1$ is
$$N_1 = p+1.$$

\bigskip

$\bullet$ Now we assume that $p\equiv 1 \pmod 4$. Then there is a character $\chi$ of order $4$ on $F_p$.
\begin{align*}
N(u^2 - v^4 = 4D) &= \sum_{a-b = 4D} N(u^2=a)N(v^4=b)\\
&=\sum_{a-b = 4D} (1+\rho(a))(1+\chi(b) + \chi^2(b) + \chi^3(b))\\
&=\sum_{i=0}^1 \sum_{j=0}^3 \sum_{a-b = 4D} \rho^i(a) \chi^j(b).\\
\end{align*}
The inner sum for each fixed pair $(i,j)$ is
\begin{align*}
\sum_{a-b = 4D} \rho^i(a) \chi^j(b) &= \sum_{a \in F_p} \rho^i(a) \chi^j(a - 4D)\\
&=\chi^j(-1)\sum_{a \in F_p} \rho^i(a) \chi^j(4D-a)\\
&= \chi^j(-1)\sum_{a + c = 4D}\rho^i(a) \chi^j(c)\\
&= \chi^j(-1)\sum_{a' + c' = 1}\rho^i(a') \chi^j(c') \qquad (a = 4Da', c = 4Db')\\
&=\chi^j(-1) \rho^i(4D) \chi^j(4D) J(\rho^i, \chi^j).
\end{align*}
Since $\chi^2$ is of order 2, $\rho = \chi^2$, thus
$$\sum_{a-b = 4D} \rho^i(a) \chi^j(b)  = \chi^j(-1)  \chi^{2i+j}(4D) J(\chi^{2i},\chi^j),$$
and, using $J(\varepsilon,\varepsilon) = p , J(\varepsilon,\chi^j) =0$ if $j\ne 0$,
\begin{align*}
N(u^2 - v^4 = 4D)  &= \sum_{i=0}^1 \sum_{j=0}^3 \chi^j(-1)  \chi^{2i+j}(4D) J(\chi^{2i},\chi^j)\\
&=p + \chi(-1) \chi^3(4D) J(\chi^2,\chi) \\
&\phantom{=p\, } + \chi^2(-1) \chi^4(4D) J(\chi^2,\chi^2) \\
&\phantom{=p\, }+ \chi^3(-1) \chi^5(4D) J(\chi^2,\chi^3).
\end{align*}
Since $J(\chi^2,\chi^2) = J(\chi^2,\chi^{-2}) = -\chi^2(-1) = -1$, and $\chi^3 = \overline{\chi}$, we obtain
$$N(u^2 - v^4 = 4D) = p-1 + \chi(-1) [\overline{\chi(4D)} J(\chi,\chi^2) +\chi(4D) \overline{J(\chi,\chi^2)}].$$
Comme $\chi(4)^2 = \chi(2^4) = \chi^4(2) = 1$,  $\chi(4) = \pm 1$ is real. Therefore
$$N(u^2 - v^4 = 4D) = p-1 + \chi(-4) \left [\overline{\chi(D)} J(\chi,\chi^2) +\chi(D) \overline{J(\chi,\chi^2)}\right].$$
We must add one to obtain the number of affine points of $C_1$, and one more to the point at infinity. Thus the number of projective points on $C_1$ is
$$N_1 = p+1 + \chi(-4)[\overline{\chi(D)} J(\chi,\chi^2) +\chi(D) \overline{J(\chi,\chi^2)}].$$
But $\chi(-1) = (-1)^{\frac{p-1}{4}}$. To prove this equality, take $g$ a generator of $F_p^*$ such that $\chi(g) = i$ (such a generator exists, since $\chi(g) = \pm i$: if $\chi(g) = -i$, replace $g$ by $g^{-1}$). Since $g^{p-1} = 1$, and $g^{(p-1)/2} \ne 1$, we obtain $g^{(p-1)/2} = -1$, thus $\chi(-1) = \chi(g)^{(p-1)/2} = i ^{(p-1)/2} = (-1)^{(p-1)/4}$.
Moreover $\chi(4) = \chi^2(2) = \rho(2) = (-1)^{(p^2-1)/8}$. Thus, for $p = 4k +1$,
$$\chi(-4) = \chi(-1) \chi(4) = (-1)^\frac{p-1}{4} (-1)^\frac{p^2 - 1}{8} = (-1)^k (-1)^{2k^2 + k} = 1.$$
Alleluia! We conclude
$$N_1 = p+1 + \overline{\chi(D)} J(\chi,\chi^2) +\chi(D) \overline{J(\chi,\chi^2)}.$$
\end{proof}

\paragraph{Ex. 11.13}{\it (continuation) If $p\equiv 1 \pmod 4$, calculate the zeta function of $y^2 = x^3 - Dx$ over $F$ in terms of $\pi$ and $\chi(D)$, where $\pi = -J(\chi,\chi^2)$. This calculation in somewhat sharpened form is contained in [23]. The result has played a key role in recent empirical work of B.J.Birch and H.P.F. Swinnerton-Dyer on elliptic curves.
}

\begin{proof}
Here $p\equiv1 \pmod 4$, thus $p^s \equiv 1 \pmod 4$. We consider here the two fields $F = \F_p$ and $F_s = \F_{p^s}$, where $|F| = p$ and $F_s = p^s$.

Let $\rho' = \rho \circ \mathrm{N_{F_s/F}}$, and $\chi' = \chi \circ  \mathrm{N_{F_s/F}}$. The results of �3 show that the map $\xi \mapsto \xi' = \xi \circ\mathrm{N_{F_s/F}}$ induces a group isomorphism between the group cyclic $C_n$ of characters on $F$ whose order divides $n$ on the group cyclic $C'_n$ of characters  on $F_s$ whose order divides $n$ (see Exercise 16). Thus the order of $\rho'$ is $2$ and the order of $\chi'$ is $4$, and $\chi'^2 = \rho'$.

Replacing $\chi, rho$ by $\chi',\rho'$, and $p$ by $p^s$, we obtain by the same reasoning that the number of projective point of $C_1$ in $\overline{H}_f(F_s)$ is
$$N_s = p^s+1 + \chi'(-4)\left[\overline{\chi'(D)} J(\chi',\chi'^2) +\chi'(D) \overline{J(\chi',\chi'^2)}\right].$$
To compute $\chi'(-4)$ and $\chi'(D)$ we use the property (c) of �3. Since $-4$ and $D$ are in $F$,
$$\chi'(-4) = \chi(-4) ^s = 1,\qquad \chi'(D) = \chi(D)^s.$$
Therefore
$$N_s = p^s+1 + \overline{\chi(D)}^s J(\chi',\chi'^2) +\chi(D)^s \overline{J(\chi',\chi'^2)}.$$
It remains to compute $J(\chi',\chi'^2)$. Since $\chi' \ne \varepsilon, \chi'^2 \ne \varepsilon, \chi'^3 \ne \varepsilon$,
$$J(\chi',\chi'^2) = \frac{g(\chi') g(\chi'^2)}{g(\chi'^3)}.$$
The Hasse-Davenport relation gives $g(\chi'^k) = - (- g(\chi^k))^s$, thus
\begin{align*}
J(\chi',\chi'^2) &= - \left[- \frac{ g(\chi) g(\chi^2)}{g(\chi^3)}\right]^s\\
&= - (-J(\chi,\chi^2))^s\\
&= -\pi^s,
\end{align*}
where $\pi =-J(\chi,\chi^2) \in \Z[i]$. To conclude,
$$N_s = p^s + 1 - \overline{\chi(D)}^s \pi^s - \chi(D)^s \overline{\pi}^s,\qquad \pi = -J(\chi,\chi^2).$$
Then Exercise 2 gives
$$Z_f(u) = \frac{(1- \overline{\chi(D)} \pi u)(1 - \chi(D) \overline{\pi}u)}{(1-u)(1-pu)},\qquad \pi = -J(\chi,\chi^2).$$
Since $|\pi|^2 = |J(\chi,\chi^2)|^2 = p$ (corollary of Theorem 1, chapter 8), expanding the numerator, we obtain
$$Z_f(u) = \frac{1  + a u + pu^2}{(1-u)(1-pu)},\qquad a = -\mathrm{tr}\left(\overline{\chi(D)} \, \pi \right) \in \Z, \quad \pi = - J(\chi,\chi^2) \in \Z[i].$$
\bigskip
Note: Since $Z_f(u) =\exp(N_1u+ \cdots) = 1 + N_1 u + \cdots$, and
\begin{align*}
Z_f(u) &= (1 +au + pu^2)(1 + u + u^2+\cdots)(1+pu+p^2 u^2 + \cdots)\\
&= 1 + (a+p+1) u + \cdots,
\end{align*}
the comparison of the coefficient of $u$ in the two power series gives
$$a = N_1 - p - 1, \qquad \text{where } N_1 = p+1 - \overline{\chi(D)} \pi - \chi(D) \overline{\pi}, \quad \pi = - J(\chi,\chi^2).$$
This gives anew $a=  -\mathrm{tr}\left(\overline{\chi(D)}\pi\right)$.


\end{proof}

\paragraph{Ex. 11.14}{\it Suppose that $p\equiv 1 \pmod 4$ and consider the curve $x^4 + y^4 = 1$ over $F_p$. Let $\chi$ be a character of order $4$ and $\pi = -J(\chi,\chi^2)$. Give a formula for the number of projective points over $F_p$ and calculate the zeta function. Both answers should depend only on $\pi$. (Hint: See Exercises 7 and 16 of Chapter 8, but be careful since there were counting only finite points.)
}

\begin{proof}
We count the number of points at infinity of the curve $C : x^4 + y^4 = 1$ over a finite field $F$. The projective closure of $C$ has equation $x^4 + y^4 = t^4$. The projective points $[t,x,y]$ such that $t= 0$ satisfy the equation  $x^4 + y^4 = 0$. Note that $y = 0$ is impossible since $[0,0,0]$ is not a projective point. Thus the points at infinity of the curve $C$ are the points $[0,x,y]$ such that $(0,x,y) = y (0,a,1)$, where $a^4 = -1$, so that the points at infinity are
$$[0,a,1], \qquad \text{where } a^4 = -1.$$
Since $(0,a,1) = \lambda (0,b,1)$ for some $\lambda \in F$ implies $a= b$, their number is $N(a^4 = -1)$.

Write, as in Chapter 8 and Exercise 8.16, for $a \in F$,
$$
\left\{\begin{array}{rl}
\delta_4(a) &= 1 \text{ if $a$ is a fourth power in $F$,}\\
&= 0 \text{ if not}.
\end{array}
\right.
$$
If $\delta_4(-1) = 0$, then $N(a^4 = -1) = 0$, and if  $\delta_4(-1) = 1$, then $N(a^4 = -1) = 4 \wedge (p-1) = 4$ because $p \equiv 1 \pmod 4$. In both cases $N(a^4 = -1) = 4 \delta_4(-1)$.

 To conclude, the number of points at infinity of the curve $C : x^4 + y^4 = 1$ over a finite field $F$ is $4 \delta_4(-1)$.
 
 \bigskip
 
 In Exercise 8.16, we show that the number of affine points of $C$ is
 $$N(x^4 + y^4 = 1) = p+1 -4\delta_4(-1) + 2 \mathrm{Re}(J(\chi,\chi)) + 4 \mathrm{Re}(J(\chi,\chi^2)).$$
 Therefore the number of  points of the projective closure of $C$ in $\overline{H}_f(F_p)$ is
 $$N_1 = p+1 +  2 \mathrm{Re}(J(\chi,\chi)) + 4 \mathrm{Re}(J(\chi,\chi^2)).$$
With the same calculation as in Exercise 16 and above, we obtain similarly in the field $F_{p^s}$,
$$N_s = p^s + 1 +  2 \mathrm{Re}(J(\chi',\chi')) + 4 \mathrm{Re}(J(\chi',\chi'^2)),$$
where $\chi' =\chi \circ \mathrm{N}_{F_{p^s}/F_p}$ is a character of order $4$ on $F_{p^s}$.

The generalization of Exercise 8.7 gives
$$J(\chi',\chi') = \chi'(-1) J(\chi',\chi'^2),$$
where $\chi'(-1) = \chi(-1)^s =((-1)^{\frac{p-1}{4}})^s$.

As in exercise 13, the Hasse-Davenport relation shows that
\begin{align*}
J(\chi',\chi'^2) &= \frac{g(\chi') g(\chi'^2)}{g(\chi'^3)}\\
&= - \left[- \frac{ g(\chi) g(\chi^2)}{g(\chi^3)}\right]^s\\
&= - (-J(\chi,\chi^2))^s\\
&= -\pi^s.
\end{align*}
Putting all together, we obtain
$$N_s = p^s +1 - (((-1)^{\frac{p-1}{4}})^s+ 2) (\pi^s + \overline{\pi}^s),\qquad \pi = -J(\chi,\chi^2),$$
that is
$$N_s = p^s + 1  - ((-1)^{\frac{p-1}{4}} \pi)^s -((-1)^{\frac{p-1}{4}} \overline{\pi})^s -2 \pi^s - 2 \overline{\pi}^s.$$
Then Exercise 2 gives
$$Z_f(u) = \frac{(1 - (-1)^\frac{p-1}{4} \pi u)(1 - (-1)^\frac{p-1}{4} \overline{\pi} u)(1 - \pi u)^2(1- \overline{\pi}u)^2}{(1-u)(1-pu)}.$$
Using $|\pi|^2 = p$, we conclude
$$Z_f(u) = \frac{(1 - 2(-1)^\frac{p-1}{4} a u + p u^2)(1 -2au + pu^2)^2}{(1-u)(1-pu)}, \qquad a = \mathrm{Re}(\pi) \in \Z,\ \pi = -J(\chi,\chi^2) \in \Z[i].$$

Note: By �5  (or Ex. 8.18), we know that $a$ is the unique integer such that $p=a^2+b^2$ where $a+bi \equiv 1 \pmod{2+2i}$. With a simpler formulation $p = a^2 + b^2$, and $a \equiv 1 \pmod 4$ if $4 \mid b$, $a \equiv -1 \pmod 4$ if $4 \nmid b$. So we can verify these results for small primes $p$.
\end{proof}


\paragraph{Ex. 11.15}{\it Find the number of points on $x^2 + y^2 + x^2y^2 = 1$ for $p=13$ and $p=17$. Do it both by means of the formula in section 5 and by direct calculation.
}
\begin{proof}
$\bullet$ If $p=13$, the only finite points on the curve are the $4$ points $(0,1)(0,-1),(1,0),(-1,0)$. We must add the $2$ points  at infinity to obtain the $6$ points $[t,x,y]$ 
$$[0,1,0],[0,0,1], [1,0,1],[1,0,-1],[1,1,0],[1,-1,0].$$

Since $p = 13 = 3^2 + 2^2$, where $4 \nmid 2$ and $3 \equiv -1 \pmod 4$, here $a = 3$, thus the formula of �5 gives
$$N_1 = p-1 - 2a = 6.$$

$\bullet$ If $p = 17$, the finite points on the curve, given by the following naive program, are the 12 points
       	$$(0, 1), (0, 16), (1, 0), (2, 8), (2, 9), (8, 2), (8, 15), (9, 2), (9,15), (15, 8), (15, 9), (16, 0).$$
With the two points at infinity, we obtain $14$ projective points.

Here $p = 1^2 + 4^2$, and $p \mid b=4$, $a = 1 \equiv 1 \pmod 4$, thus $a = 1$, and the formula of �5 gives
$$N_1 = p-1-2a =14.$$
The formula is verified in both cases.

\bigskip

Program Sage to obtain the finite points on the curve $x^2 + y^2 + x^2y^2 =1$:
\begin{verbatim}
def N(p):
    Fp = GF(p)
    l = []
    for x in Fp:
        for y in Fp:
            if x^2 + y^2 + x^2*y^2 == 1:
                l.append((x,y))
    return l
\end{verbatim}
\end{proof}

\paragraph{Ex. 11.16}{\it Let $F$ be a field with $q$ elements and $F_s$ an extension of degree $s$. If $\chi$ is a character of $F$, let $\chi' = \chi \circ \mathrm{N}_{F_s/F}$. Show that
\begin{enumerate}
\item[(a)] $\chi'$ is a character of $F_s$.
\item[(b)] $\chi \ne \rho$ implies that $\chi' \ne \rho'$.
\item[(c)] $\chi^m = \varepsilon$ implies that $\chi'^m = \varepsilon$.
\item[(d)] $\chi'(a) = \chi(a)^s$ for $a \in F$.
\item[(e)] As $\chi$ varies over all characters of $F$ with order dividing $m$, $\chi'$ varies over all characters of $F_s$ with order dividing $m$. Here we are assuming that $q \equiv 1 \pmod m$.
\end{enumerate}
}
\begin{proof}
\item[(a)] If $\alpha,\beta \ F_s$, we know that $\mathrm{N}_{F_s/F}(\alpha \beta) = \mathrm{N}_{F_s/F}(\alpha) \mathrm{N}_{F_s/F}(\beta)$ (Proposition 11.2.2). Therefore
$$\chi'(\alpha \beta) =\chi(\mathrm{N}_{F_s/F}(\alpha \beta)) =  \chi(\mathrm{N}_{F_s/F}(\alpha) \mathrm{N}_{F_s/F}(\beta)) = \chi(\mathrm{N}_{F_s/F}(\alpha))\chi(\mathrm{N}_{F_s/F}(\beta)) = \chi'(\alpha) \chi'(\beta).$$
This shows that $\chi'$ is a character.

\item[(b)] Assume that $\chi' = \rho'$. Then for all $\alpha \in K^*$, $\chi(\mathrm{N}_{F_s/F}(\alpha)) =\rho(\mathrm{N}_{F_s/F}(\alpha))$.
By Proposition 11.2.2 (d), the map
$$\varphi
\left\{
\begin{array}{ccl}
K^* & \to & F^*\\
\alpha & \mapsto & \mathrm{N}_{K/F}(\alpha)
\end{array}
\right.
$$
is surjective. Let $a$ be any element of $F^*$. Since $\varphi$ is surjective, there is some $\alpha \in F_s^*$ such that $\mathrm{\alpha} = a$.
Then $\chi(a) = \chi(\mathrm{N}_{F_s/F}(\alpha)) =\rho(\mathrm{N}_{F_s/F}(\alpha)) = \rho(a)$. Since this is true for every $a \in F^*$, and $\chi(0) = 0 = \rho(0)$, this shows that $\chi = \rho$. 

To conclude, $\chi' = \rho'$ implies $\chi = \rho$, thus $\chi \ne \rho$ implies $\chi' \ne \rho'$.

\item[(c)] If $\chi^m = \varepsilon$, then for all $\alpha \in K$, 
$(\chi')^m(\alpha) = \chi^m(\mathrm{N}_{F_s/F}(\alpha)) = 1,$
thus $\chi'^m = \varepsilon$.

\item[(d)] Si $a \in F$, by Proposition 11.2.2(c), $\mathrm{N}_{F_s/F}(a) = a^s$, therefore
$$
\chi'(a) = \chi(N_{K/F}(a))= \chi(a^s)= \chi(a)^s.
$$

\item[(e)] Assume that $q \equiv 1 \pmod m$. Write $C$ the group of character on $F$, $C'$ the group of characters on $F_s$, $C_m$ the group of character on $F$ with order dividing $m$, and $C'_m$ the group of character on $F_s$ with order dividing $m$. By the generalization of Proposition 8.1.3, $C$ is a cyclic group of order $q-1$, and $C'$ a cyclic group of order $q^s -1$. 

We know that if $m \mid q-1 = |C|$, the subgroup $C_ m = \{\chi \in C \mid \chi^m = \varepsilon\}$ of the cyclic group $C$ is cyclic of order $m$. Since $m \mid q- 1 \mid q^s - 1$, it is the same for $C'_m$:
$$|C_m |= |C'_m| = m.$$
Let $\psi$ be the map
$$
\psi
\left\{
\begin{array}{ccl}
C_m & \to & C'_m\\
\chi & \mapsto & \chi' =  \chi \circ \mathrm{N}_{F_s/F}.
\end{array}
\right.
$$
Part (b) shows that $\psi$ is injective, and $|C_m| = |C'_m| = m$, therefore $\psi$ is bijective. In other words, as $\chi$ varies over all characters of $F$ with order dividing $m$, $\chi'$ varies over all characters of $F_s$ with order dividing $m$. 
\end{proof}


\paragraph{Ex. 11.17}{\it In Theorem 2 show that the order of the numerator of the zeta function, $P(u)$ has degree $m^{-1}((m-1)^{n+1} + (-1)^{n+1}(m-1))$.
}

\begin{proof}
In Theorem 2,
$$P(u) = \prod_{(\chi_0,\ldots,\chi_n) \in A } \left( 1 - (-1)^{n+1} \frac{1}{q}\, \chi_0(a_0)^{-1} \cdots \chi_n(a_n^{-1}) g(\chi_0)\cdots g(\chi_n)u \right),
$$
where $A$ is the set of $(n+ 1)$-tuples  $(\chi_0,\ldots,\chi_n)$ of characters on $F$ such that $\chi_i^m = \varepsilon, \chi_i \ne \varepsilon \ (i = 0,\ldots,n)$ and $\chi_0 \cdots \chi_n = \varepsilon$. In each factor, the coefficient of $u$ is not zero, thus each factor has degree $1$. Therefore the degree $d$ of $P$ is $d = \deg(P) = |A|$. Write $C_m$ the subgroup of characters on F such that $\chi^m = \varepsilon$. 

Since $q \equiv 1 \pmod m$ is an hypothesis of Theorem 2, $C_m$ is a subgroup of order $m$:  $|C_m| = m$ and $|C_m - \{\varepsilon\} |= m-1$. We count the number $d$ of $(n+1)$-tuples $(\chi_0,\ldots,\chi_n) \in (C_m - \{\varepsilon\})^{n+1}$ such that $\chi_0 \cdots \chi_n = \varepsilon$, that is $\chi_n = \chi_0^{-1}\cdots\chi_{n-1}^{-1}$, and $\chi_n \ne \varepsilon$. Let $\chi$ be a character of order $m$ (such a character exists because $C_m$ is cyclic). Write $\chi_i = \chi^{k_i}$, where $1 \leq k_i \leq m-1$. Then $d$ is the number of $n$-tuples $(k_0,\ldots,k_{n-1}) \in \gcro 1, m-1 \dcro^n$ such that $$k_0 + k_1 + \cdots +k_{n-1} \not \equiv 0 \pmod m.$$

In other words, $d$ is the number of $n$-tuples $(a_0,\ldots,a_{n-1}) \in ((\Z/m\Z)^*)^n$ such that $$a_0 + a_1 + \cdots +a_{n-1}\ne 0.$$

To begin an induction, fix the integer $m \in \N^*$, and write
$$d_n = \mathrm{Card} \{(a_0,\ldots,a_{n-1}) \in ((\Z/m\Z)^*)^n \mid a_0 + a_1 + \cdots +a_{n-1}\ne 0\}.$$
For $n \geq 2$, if $(a_0,\ldots,a_{n-2}) \in ((\Z/m\Z)^*)^{n-1}$ is given,  we count the number of $a_{n-1} \in (\Z/m\Z)^*$ such that  $a_{n-1} \ne -a_0-a_1 - \cdots -a_{n-2}$.

There are two cases.

If $a_0+ \cdots + a_{n-2} \ne 0$, there are $m-2$ choices for $a_{n-1} \in \Z/m\Z \setminus \{0, -a_0-\cdots -a_{n-2}\}$, and if $a_0+ \cdots + a_{n-2} = 0$, there are $m-1$ choices for $a_{n-1} \in \Z/m\Z \setminus \{0\}$. This gives the relation
\begin{align*}
d_n &= (m-2) d_{n-1} + (m-1) ((m-1)^{n-1} -d_{n-1}), \\
&=  (m-1)^n -d_{n-1}.
\end{align*}
Since $d_1 = \mathrm{Card}\{a \in (\Z/m\Z)^* \mid a \ne 0\} = m-1$, we obtain by immediate induction
$$d_n = (m-1)^n - (m-1)^{n-1} + \cdots + (-1)^{n-1}(m-1)\qquad (n\geq 1).$$
Then 
\begin{align*}
d_n &= (m-1)^n - (m-1)^{n-1} + \cdots + (-1)^{n-1}(m-1)\\
&=(-1)^{n-1} (m-1)\left\{ [-(m-1)]^{n-1} + [-(m-1)]^{n-2} + \cdots + 1\right\}\\
&=(-1)^{n-1} (m-1) \frac{[-(m-1)]^{n} - 1}{-(m-1) - 1}\\
&=\frac{(-1)^n (m-1) \left\{[-(m-1)]^{n} - 1\right\}}{m}\\
&=\frac{(m-1)^{n+1} + (-1)^{n+1}(m-1)}{m}.
\end{align*}
This is the waited answer,
$$\deg(P(u)) = \frac{(m-1)^{n+1} + (-1)^{n+1}(m-1)}{m}.$$
\end{proof}


\paragraph{Ex. 11.18}{\it Let the notation be as in Exercise 16. Use the Hasse-Davenport relation to show that $J(\chi'_1,\chi'_2,\ldots,\chi'_n) = (-1)^{(s-1)(n-1)} J(\chi_1,\chi_2,\ldots,\chi_n)^s$, where the $\chi_i$ are non trivial characters of $F$ and $\chi_1\chi_2\cdots \chi_n \ne \varepsilon$.
}

\begin{proof} Note that $(\chi \rho)' = \chi' \rho'$, thus $(\chi_1 \ldots \chi_n)'= \chi'_1 \chi'_2\cdots \chi'_n$.

The conditions on the characters, and Exercise 16,  show that $\chi'_i \ne \varepsilon$ and $\chi'_1\chi'_2\cdots \chi'_n \ne \varepsilon$. By Theorem 3 of Chapter 8,
$$J(\chi'_1,\ldots,\chi'_n) = \frac{g(\chi'_1)g(\chi'_2) \cdots g(\chi'_n)}{g(\chi'_1 \chi'_2\cdots \chi'_n)}.$$
Then the Hasse-Davenport relation gives
\begin{align*}
J(\chi'_1,\ldots,\chi'_n) &= \frac{[-(-g(\chi_1))^s] [-(-g(\chi_2))^s] \cdots[- (-g(\chi_n))]^s]}{-(-g(\chi_1 \chi_2 \cdots \chi_n))^s}\\
&=(-1)^{n-1}(-1)^{s(n-1)} \left(\frac{g(\chi_1)g(\chi_2) \cdots g(\chi_n)}{g(\chi_1 \chi_2\cdots \chi_n)}\right)^s\\
&=(-1)^{(s+1)(n-1)}  J(\chi_1,\ldots,\chi_n)^s\\
&=(-1)^{(s-1)(n-1)}  J(\chi_1,\ldots,\chi_n)^s.\\
\end{align*}
\end{proof}


\paragraph{Ex. 11.19}{\it Prove the identity $\sum \lambda(f) t^{\deg(f)} = \prod (1 - \lambda(f) t^{\deg(f)})^{-1}$, where the sum is over all monic polynomials in $F[t]$ and the product is over all monic irreducible in $F[t]$. $\lambda$ is defined in Section 4.
}

\bigskip

(The solution of this exercise requires external knowledge on formal power series. To learn more about formal power series, see [Niven, Formal power series],  [Bourbaki, Algebra IV, �4], and [Wikipedia, Formal power series]. About summable families, see [Bourbaki, General Topology, III �5]. )

\bigskip

\begin{proof}

For each monic polynomial $f(x) = x^n - c_1x^{n-1} + \cdots + (-1)^n c_n \in F[x]$, $\lambda(f)$ is defined by
$$\lambda(f) = \psi(c_1) \chi(c_n).$$
To complete this definition, we define $\lambda(1) = 1$. By Lemma 1, $\lambda(fg) = \lambda(f) \lambda(g)$ for all monic polynomials $f,g \in F[x]$.

We must prove the following equality in the ring of formal power series $\C[[t]]$:
\begin{align}
\sum_{f \in M} \lambda(f) t^{\deg(f)} = \prod_{f \in I} \left(1 - \lambda(f) t^{\deg(f)}\right)^{-1},
\end{align}
where $M$ is the set of monic polynomials of $F[x]$, and $I$ is the set of monic polynomials of  $F[x]$ which are irreducible over $F$. 

Since $I$ and $M$ are infinite sets, we must give a sense at this formula. This implies to introduce a topology on the algebra $\C[[t]]$, which is given by the distance $d$ defined by
$$d(\alpha, \beta) = 2^{-\nu(\alpha - \beta)}, \qquad \alpha, \beta \in \C[[t]],$$
where $\nu : \C[[t]] \to \N \cup \{\infty\}$ is the valuation on $\C[[t]]$: if $\alpha = \sum_{k=0}^\infty a_k t^k$, then $\nu(0) = \infty$, and $\nu(\alpha) = \min \{k \in \N \mid a_k \ne 0\}$.
This distance is associated to the norm $|| \cdot ||$, given by $||\gamma|| = 2^{-\nu(\gamma)}, \gamma \in \C[[t]]$, so that $E = \C[[t]]]$ is a normed vector space. %Moreover the metric space $\C[[t]]$ is complete.


As in Bourbaki, a family $(u_i)_{i \in I}$ of vectors of a normed vector space is summable, if there is some $S \in E$ such that
$$\forall \varepsilon,\ \exists J_\varepsilon \in {\cal F}(I), \ \forall J \in {\cal F}(I),\  J \supset J_\varepsilon \Rightarrow \left |\left | \sum_{i \in J} u_i - S \right |\right | < \varepsilon,$$
 where ${\cal F}(I)$ is the set of finite subsets of $I$. Then we write $S = \sum\limits_{i \in J} u_i$. There is a similar definition for multipliable families.
 
In the algebra $\C[[t]]$, this is equivalent to $\lim_k u_k = 0$ under the filter of the complementaries of finite sets: ${\{A \in {\cal P}(I) \mid I \setminus A \in {\cal F}(I)\}}$ (Bourbaki, IV, 4, Lemma 1), which means, for $(u_i)_{i\in I} \in \C[[t]]^I$,
$$\forall \varepsilon,\ \exists J_\varepsilon \in {\cal F}(I), \ \forall i \in I \setminus J,\  \left |\left | u_i \right |\right | < \varepsilon.$$

Moreover, if the family $(u_i)_{i \in I}$ is summable, then $(1 + u_i)_{i \in I}$ is multipliable (Bourbaki, Algebra IV, 4, Proposition 2).

A summable family $(u_i)_{i \in I}$, where $I$ is a countable set, can be summed in any order (Bourbaki, General Topology, III,7 Proposition 9). If $\varphi: \N \to I$ is a bijection, then
\begin{align}
\sum_{i \in I} u_i = \sum_{j = 0}^\infty u_{\varphi(j)}.
\end{align}
\bigskip

After these preliminaries, we can show that the family $\left(\lambda(f) t^{\deg(f)} \right)_{f \in M}$ is summable.

If $\varepsilon >0$, let $N$ be an integer such that $2^{-N} < \varepsilon$, and consider the set $J_\varepsilon$ of monic polynomials $f$ such that $\deg(f) \leq N$. Then $J_\varepsilon$ is a finite set, and for all $f \in I \setminus J$, $\deg(f) >N$,  so that $ ||\lambda(f) t^{\deg(f)}|| = 2^{-\deg(f)} \leq 2^{-N} < \varepsilon$.

This proves that the family $\left(\lambda(f) t^{\deg(f)} \right)_{f \in M}$ is summable, and $\sum_{f \in M} \lambda(f) t^{\deg(f)}$ makes sense.

Then the sub-family $\left(\lambda(f) t^{\deg(f)} \right)_{f \in I}$ is also summable. This proves that $  (1 - \lambda(f) t^{\deg(f)})_{f\in I}$ is multiplicable, and $\prod\limits_{f \in I}(1 - \lambda(f) t^{\deg(f)})^{-1}$ makes sense.



To prove (3), we use first geometric power series. For all $f \in I$, 
$$(1 - \lambda(f) t^{\deg(f)})^{-1} = \sum_{k=0}^\infty \lambda(f)^k t^{k \deg(f)}.$$

The set $I$ is a countable set (countable union of finite sets). We use an arbitrary numbering of $I$, $I = \{f_1,f_2,\ldots,f_n,\ldots\}$, obtained by a bijection $\varphi : \N^* \to I, \varphi(n) =f_n$. Write $I_m$ the finite set $I_m = \{f_1,\ldots,f_m\}$, et $M_m$ the set of monic polynomials whose irreducible factors are in $I_m$, so that every $f \in M$  uniquely decomposes under the form
$$f = f_1^{a_1}\cdots f_m^{a_m}, \qquad a_1,\ldots,a_m \in \N.$$
Write $d_i = \deg(f_i)$. Then (see Bourbaki, Algebra IV, 4, Proposition 2)
\begin{align*}
\sum_{f \in M_m} \lambda(f) t^{\deg(f)} &= \sum_{(a_1,\ldots,a_m) \in \N^m} \lambda(f_1)^{a_1} \cdots \lambda(f_m)^{a_m} t^{a_1d_1+ \cdots+a_md_m}\\
&= \left(\sum_{a_1 = 0}^\infty \lambda(f_1)^{a_1} t^{a_1d_1} \right) \cdots \left(\sum_{a_m = 0}^\infty \lambda(f_m)^{a_m} t^{a_md_m} \right)\\
&= \left(1 - \lambda(f_1)t^{a_1} \right)^{-1} \cdots  \left(1 - \lambda(f_m)t^{a_m} \right)^{-1}\\
&=\prod_{i=1}^m \left( 1 - \lambda(f_i) t^{\deg(f_i)} \right)^{-1}.
\end{align*}
Then, using (4),
\begin{align*}
\lim_{m \to \infty} \prod_{i=1}^m \left( 1 - \lambda(f_i) t^{\deg(f_i)} \right)^{-1} &=\prod_{i=1}^\infty \left( 1 - \lambda(f_i) t^{\deg(f_i)} \right)^{-1} \\
&=\prod_{f \in I} (1 - \lambda(f) t^{\deg(f)})^{-1},
\end{align*}
the limit being in the metric space $\C[[t]]$ with the distance $d$.

Since $M$ is the increasing union of the $M_m$, 
\begin{align*}
\lim_{m\to \infty} \sum_{f \in M_m} \lambda(f) t^{\deg(f)}  &=  \sum_{f \in M} \lambda(f) t^{\deg(f)}.
\end{align*}
We justify this statement.

 If $\varepsilon >0$, let $N$ be an integer such that $2^{-N} < \varepsilon$. The set of monic irreducible polynomials $f_i \in I$ such that $\deg(f) \leq N$ is finite, thus there is some integer $M$ such that, for all integers $i$, $i \geq M$ implies $\deg(f_i) > N$.
 
 For every $m\geq M$, if $f \in M \setminus M_m$, there is some irreducible monic factor $f_i$ of $f$ such that  $i \geq m$, therefore $\deg(f) \geq \deg(f_i) > N$. Then
 $$\nu  \left(\sum_{f \in M\setminus M_m} \lambda(f) t^{\deg(f)} \right) \geq N,$$
so
 \begin{align*}
 \left | \left |  \sum_{f \in M} \lambda(f) t^{\deg(f)} - \sum_{f \in M_m} \lambda(f) t^{\deg(f)} \right | \right |  &=  \left | \left |  \sum_{f \in M\setminus M_m} \lambda(f) t^{\deg(f)}  \right | \right | \leq 2^{-N} < \varepsilon.
 \end{align*}
 This shows the statement.
 
 Since
 $$
 \left\{
 \begin{array}{lcl}
\lim\limits_{m\to \infty} \sum\limits_{f \in M_m} \lambda(f) t^{\deg(f)}  &= & \sum\limits_{f \in M} \lambda(f) t^{\deg(f)},\\
\lim\limits_{m \to \infty} \prod\limits_{i=1}^m \left( 1 - \lambda(f_i) t^{\deg(f_i)} \right)^{-1} &=& \prod\limits_{f \in I} (1 - \lambda(f) t^{\deg(f)})^{-1},
\end{array}
\right.
$$
where 
$$\sum_{f \in M_m} \lambda(f) t^{\deg(f)} = \prod_{i=1}^m \left( 1 - \lambda(f_i) t^{\deg(f_i)} \right)^{-1},$$
 the unicity of the limit shows that
\begin{align*}
\sum_{f \in M} \lambda(f) t^{\deg(f)} = \prod_{f \in I} (1 - \lambda(f) t^{\deg(f)})^{-1}.\\
\end{align*}

\end{proof}


\paragraph{Ex. 11.20}{\it If in Theorem 2 we consider the base field to be $F_s$ instead of $F$, we get a different zeta function, $Z_f^{(s)}(u)$. Show that $Z_f^{(s)}(u)$ and $Z_f(u)$ are related by the equation $Z_f^{(s)}(u^s) = Z_f(u)Z_f(\rho u)\cdots Z_f(\rho^{s-1} u)$, where $\rho = e^{2i\pi/s}$.
}

\begin{proof}
Let $\Omega$ be an algebraic closure of $\F_p$ and write $\F_q$ for the unique subfield of $\Omega$ with cardinality $q$, if $q$ is a power of $p$. Here $F = \F_q$, and $F_s = \F_{q^s}$. Recall that the function zeta only depends on the cardinality of the finite field, not on the choice of this field (see Exercise 3).

Then 
$$Z_f^{(s)}(u) = \exp \left( \sum_{t=1}^\infty \frac{N_t^{(s)} u^t}{t} \right),$$
where $N_t^{(s)}$ is the number of points of $\overline{H}_f(\F_{q^{st}})$, because the degree of $\F_{q^{st}}$ over $\F_{q^s}$ is
$$[\F_{q^{st}} : \F_{q^s}] = \frac{[\F_{q^{st}} : \F_q]}{[\F_{q^s} : \F_q]} = \frac{st}{s} = t.$$
 Therefore $N_t^{(s)} = N_{st}$,  where as usual $N_s$ is the number of points of $\overline{H}_f(\F_q)$. This gives
$$Z_f^{(s)}(u) = \exp \left( \sum_{t=1}^\infty \frac{N_{st} u^t}{t} \right).$$

Now, since $\ln(Z_f(u)) = \sum\limits_{k=0}^\infty N_k \frac{u^k}{k}$, we obtain
$$\ln(Z_f(\rho^j u)) = \sum_{k=0}^\infty N_k\, \rho^j \, \frac{u^k}{k},\qquad j = 0,1,\ldots,s-1.$$
The sum of these $s$ equalities gives
$$\sum_{j=0}^{s-1} \ln(Z_f(\rho^j u)) = \sum_{k=0}^\infty N_k \left( \sum_{j=0}^{s-1} \rho^{kj} \right) \frac{u^k}{k}.$$
Moreover,
$$
 \sum_{j=0}^{s-1} \rho^{kj} =
\left\{
\begin{array}{ll}
\frac{1 - \rho^{ks}}{1-\rho^k} = 0 & \text{if } s \nmid k,\\
 s & \text{otherwise.}
\end{array}
\right.
$$
Therefore
\begin{align*}
\sum_{j=0}^{s-1} \ln(Z_f(\rho^j u)) &= \sum_{s \mid k} N_k \, s \frac{u^k}{k}\\
&= \sum_{t = 1}^\infty  N_{st} \frac{u^{st}}{t}\qquad (k = st)\\
&=\ln(Z_f^{(s)}(u^s)).
\end{align*}
To conclude,
$$Z_f(u^s) = Z_f(u) Z_f(\rho u) \cdots Z_f(\rho^{s-1} u),\qquad (\rho = e^{2i\pi/s}).$$
\end{proof}


\paragraph{Ex. 11.21}{\it In Exercise 6 we considered the equation $x_0^3 + x_1^3 + x_2^3 = 0$ over the field with four elements. Consider the same equation over the field with two elements. The trouble here is that $2 \not \equiv 1 \pmod 3$ and so our usual calculations do not work. Prove that in every extension of $\Z/2\Z$ of odd degree every element is a cube and that every extension of even degree, $3$ divides the order of the multiplicative group. Use this information to calculate the zeta function over $\Z/2\Z$. [Answer: $(1+ 2u^2)/(1-u)(1-2u)$.]
}

\begin{proof} Consider the extension $\F_{2^s}$ of degree $s$ over $\F_2$.
\item[$\bullet$] If $s = 2k+1$ is odd, then $2^s - 1 = 2^{2k+1} - 1 \equiv 1 \pmod 3$, thus $d = (2^s-1) \wedge 3 = 1$. An element $a \in \F_{2^s}^*$ is a cube if and only if $a^{(2^s-1)/d} = 1$, that is $a^{2^s-1} = 1$, which is true for all elements $a \in \F_{2^s}^*$ (and $0 = 0^3$). So every element is a cube. The number of solutions of $a^3 = 1$ is $N(a^3 = 1) =d = 1$, thus every element of $\F_{2^s}$ is the cube of a unique element.

\item[$\bullet$] If $s = 2k$ is even, then $2^s - 1 = 2^{2k}- 1 \equiv 0 \pmod 3$, thus $d = (2^s-1) \wedge 3 = 3$. So $3 \mid 2^s-1 = |\F_{2^s}^*|$. Therefore there exists a character $\chi_s$ of order $3$ in $\F_{2^s}$.

\bigskip

We can now compute $N_s$.

\item[$\bullet$] If $s = 2k+1$ is odd, in the field $\F_{2^s}$,
\begin{align*}
N(x_0^3 + x_1^3 + x_2^3 = 0) &= \sum_{a+b+c = 0} N(x_0^3 = a) N(x_1^3 = b) N(x_2^3 = c)\\
&= \sum_{a+b+c = 0} 1\\
&= 2^{2s}.
\end{align*}
Thus the number of projective points of $\overline{H}_f(\F_{2^s})$ is
$$N_s = \frac{2^{2s}- 1}{2^s -1} = 2^s + 1\qquad (s \text{ odd}).$$
(Alternatively, we can compute the number of affine points, which is $N(y_0^3 +y_1^3 = -1) = N(a+b = -1) = 2^s$, and add a unique point $[0,-1,1]$ at infinity, since $a^3 = -1$ has exactly one solution $-1$. We obtain anew $N_s = 2^s +1$.)

\item[$\bullet$]  If $s = 2k$ is even, in the field $\F_{2^s}$,
\begin{align*}
N(x_0^3 + x_1^3 + x_2^3 = 0) &= \sum_{a+b+c = 0} N(x_0^3 = a) N(x_1^3 = b) N(x_2^3 = c)\\
&= \sum_{a+b+c = 0} \sum_{i=0}^2 \chi_s^i (a) \sum_{j=0}^2 \chi_s^j (b) \sum_{k=0}^2 \chi_s^k(c)\\
&=\sum_{(i,j,k) \in \gcro 0, 2 \dcro^3} \ \sum_{a+b+c = 0} \chi_s^i (a) \chi_s^j (b)\chi_s^k(c)\\
&=\sum_{(i,j,k) \in \gcro 0, 2 \dcro^3} J_0(\chi_s^i, \chi_s^j, \chi_s^k).
\end{align*}
Using the generalization of Proposition 8.5.1, with $J_0(\varepsilon,\varepsilon,\varepsilon) = 2^{2s}$, we obtain
\begin{align*}
N(x_0^3 + x_1^3 + x_2^3 = 0) &= 2^{2s} + \sum_{(i,j,k) \in A} J_0(\chi_s^i, \chi_s^j, \chi_s^k),
\end{align*}
where $A$ is the set of $(i,j,k) \in  \{1,2\}^3$ such that $i+j+k  \equiv 0 \pmod 3$, that is $(1,1,1)$ and $(2,2,2)$. Thus
$$N = N(x_0^3 + x_1^3 + x_2^3 = 0) = 2^{2s} + J_0(\chi_s,\chi_s,\chi_s) + J_0(\chi_s^2,\chi_s^2,\chi_s^2).$$
Thus the number of projective points is
$$N_s = \frac{N-1}{2^s - 1} = 2^s + 1 +\frac{1}{2^s - 1}(J_0(\chi_s,\chi_s,\chi_s) + J_0(\chi_s^2,\chi_s^2,\chi_s^2)).$$
Moreover, the same Proposition 8.5.2 gives, using $\chi_s(-1) = \chi_s(1) = 1$,
\begin{align*}
J_0(\chi_s,\chi_s,\chi_s)  &= (2^s-1) J(\chi_s,\chi_s)\\
&= (2^s -1) \frac{g(\chi_s)^2}{g(\chi_s^2)}\\
&= (2^s -1) \frac{g(\chi_s)^3}{g(\chi_s) g(\chi_s^{-1})}\\
&= (2^s - 1) \frac{g(\chi_s)^3}{2^s}.\\
\end{align*}
This gives
$$\frac{1}{2^s - 1} J_0(\chi_s,\chi_s,\chi_s)  = \frac{1}{2^s} g(\chi_s)^3.$$
(This is also formula (2) in Theorem 2 of Chapter 10). This is the same for $\chi_s^2$, thus
$$N_s = 2^s +1 + \frac{1}{2^s} ( g(\chi_s)^3 + g(\chi_s^2)^3).$$
We choose a character $\chi$ of order $3$ on $\F_4 = \F_{2^2}$, given by
$$
\begin{array}{c|cccc|}
t & 0 & 1 & a & a^2\\
\hline
\chi(t) & 0 & 1 & \omega & \omega^2
\end{array}
$$
where $a$ is a generator of $\F_4$.
We can take $\chi_s = \chi \circ \mathrm{N}_{\F_{2^s}/ \F_{2^2}}$ (this makes sense since $2 \mid s$, so that $\F_{2^2}$ is a subfield of $\F_{2^s} = \F_{2^{2k}}$).

Since $\F_{2^s}$ is an extension of degree $s/2$ of $\F_4$, the Hasse-Davenport relation shows that 
$$g(\chi_s) = -(-g(\chi))^{s/2}.$$
The computations of $g(\chi)$ and $g(\chi^2)$ are given in Exercise 6. We obtained 
$$g(\chi) = g(\chi^2) = 2.$$
Then
$$N_s = 2^s +1  - 2(-2)^{\frac{s}{2}}.$$

These two results can be written under the form
$$
\left\{
\begin{array}{lll}
N_{2k+1} &= 2^{2k+1} +1& (k \geq 0),\\
N_{2k} &= 2^{2k}+1 - 2 (-2)^k& (k \geq 1).
\end{array}
\right.
$$
We can compute $Z_f(u)$.
\begin{align*}
\ln(Z_f(u)) &= \sum_{k=1}^\infty N_{2k} \frac{u^{2k}}{2k} + \sum_{k=0}^\infty N_{2k+1} \frac{u^{2k+1}}{2k+1}\\
&= \sum_{k=1}^\infty \left(2^{2k} + 1 - 2(-2)^k \right) \frac{u^{2k}}{2k} + \sum_{k=0}^\infty \left( 2^{2k+1} + 1\right) \frac{u^{2k+1}}{2k+1}\\
&= \left(  \sum_{k=1}^\infty 2^{2k}  \frac{u^{2k}}{2k}  +  \sum_{k=0}^\infty 2^{2k+1} \frac{u^{2k+1}}{2k+1} \right) +  \left(  \sum_{k=1}^\infty \frac{u^{2k}}{2k}  +  \sum_{k=0}^\infty\frac{u^{2k+1}}{2k+1} \right)  -2  \sum_{k=1}^\infty(-2)^k \frac{u^{2k}}{2k} \\
&=  \sum_{l=1}^\infty 2^l \frac{u^l}{l} +  \sum_{l=1}^\infty  \frac{u^l}{l} - \sum_{k=1}^\infty \frac{(-2u^2)^k}{k}\\
&= - \ln(1-2u) - \ln(1-u) + \ln( 1 + 2u^2).
\end{align*}
Therefore
$$Z_f(u) = \frac{1 + 2u^2}{(1-u)(1-2u)}.$$
\end{proof}
Note: Using Exercise 20, we obtain anew the result of Exercise 6. Here $s = 2$, and $\rho = e^{2\pi i/s} = e^{i \pi} = -1$. This gives
\begin{align*}
Z_f^{(2)}(u^2) &= Z_f(u) Z_f(-u)\\
&= \frac{1+2u^2}{(1-u)(1-4u)} \frac{1+2u^2}{(1+u)(1+4u)}\\
&= \frac{(1 + 2u^2)^2}{(1 - u^2)(1 - 4u^2)}.
\end{align*}
Therefore the function zeta of $f (x_0,x_1,x_2) = x_0^3+x_1^3+x_2^3$ with base field $\F_4$ is $$Z_f^{(2)}(u) = \frac{(1 + 2u)^2}{(1-u)(1-4u)}.$$



\paragraph{Ex. 11.22}{\it Use the ideas developed in Exercise 21 to show that Theorem 2 continues to hold (in a suitable sense) even when the hypothesis $q \equiv 1 \pmod m$ is removed.
}

\begin{proof}
?????
\end{proof}



\paragraph{Ex. 11.23}{\it Let $p_1<p_2<p_3<\cdots$ denote the positive prime numbers arranged in order. Let $N_m = p_1^mp_2^m\cdots p_m^m$ and let $E_m$ denote the field with $q^{N_m}$ elements. Show that $E_m$ can be considered as a subfield of $E_{m+1}$ and that $E = \bigcup E_m$ is an extension of $E_0 = F$, a finite field with $q$ elements, with the following property; for every positive integer $n$, $E$ contains one and only one subfield $F_n$ with $q^n$ elements.
}

\begin{proof}
Here $q = p^a$ is a power of $p$.

We build the family $E_m$ by induction. 

For $m = 0$, $N_0 = 1$. Take $E_0 = F$, a finite field with $q$ elements, whose existence is proved in Theorem 3, Chapter 7. Then $|E_0| = q = q^{N_0}$.

Suppose that we know an extension $E_m$ of $F$ with $q^{N_m}$ elements, so that $[E_m : F] = N_m$.

Write $s = N_m$ and $t = N_{m+1}$. Then $t= \left (p_1\cdots p_m p_{m+1}^{m+1}\right) s$, so $s \mid t$, and $k = t/s$ is an integer. 
By Exercise 7.14, there exists a polynomial $p(x) \in E_m[x]$ of degree $k$, irreducible over $E_m$. Then $K = E_m[x]/(p(x))$ is a field, and the map $j : E_m \to K$ defined by $j(\alpha) = \overline{\alpha} = \alpha + (p(x))$ is injective. This allows us to ``identify'' $E_m$ and $j(E_m)$. 

More explicitly, if we define $E_{m+1} = (K \setminus j(E_m)) \cup E_m$ (that is, we replace the elements of $j(E_m)$ by the corresponding elements in $E_m$), then $E_m \subset E_{m+1}$ absolutely, and
$$
\varphi
\left\{
\begin{array}{ccl}
K &\to& E_{m+1}\\
\alpha & \mapsto & 
	\left\{
	\begin{array}{ll}
	\beta &\text{if } \alpha = j(\beta) \in j(E_m)\\
	\alpha & \text{if } \alpha \not \in j(E_m)
	\end{array}
	\right.
\end{array}
\right.
$$
is a bijection. This bijection allows us to define a structure of field over $E_{m+1}$ by transport of structure, i.e. the laws $+, \times$ on $E_{m+1}$ are given by
$$u + v = \varphi(\varphi^{-1}(u) +\varphi^{-1}(v)),\quad u \times v = \varphi(\varphi^{-1}(u) \times \varphi^{-1}(v)), u,v \in E_{m+1}.$$
Then $E_{m+1}$ is a field for these laws, $\varphi$ is a field isomorphism, and $E_m$ is a subfield of $E_{m+1}$. Since the degree of $E_{m+1} \simeq K = E_m[x]/(p(x))$ is $k = \deg(p)$, $[E_{m+1} : E_m] = k = N_{m+1}/N_m$, therefore $[E_{m+1} : F ] = [E_{m+1} : E_m] [E_m:F] = k N_m = N_{m+1}$. Thus $|E_{m+1}| = q^{N_{m+1}}$.

 To conclude this part, the sequence $(E_m)_{m\in \N}$ is an increasing sequence for inclusion, and for each $m\in N$, $|E_m| = q^{N_m}$.

\bigskip
Now consider the set union
$$E = \bigcup_{m \in \N} E_m.$$
We can define additive and multiplicative laws on $E$. If $\alpha, \beta \in E$, then $\alpha \in E_r, \beta \in E_s$. If $m = \max(r,s)$, then $\alpha, \beta \in E_m$, so that $\alpha + \beta$ is defined in $E_m$. Moreover, assume that $\alpha, \beta \in E_{m'}$ for another index $m'$. Then $E_m \subset E_{m'}$, or $E_{m'} \subset E_m$. If we suppose $E_m \subset E_{m'}$ (the other case is similar), $E_m$ is a subfield of $E_m'$, so that $\alpha + \beta$ is the same in $E_m$ or $E_{m'}$. This allows us to define $\alpha + \beta$ in $E$ as the sum of $\alpha , \beta$ in any field $E_m$ such that $\alpha, \beta$ are both in $E_m$. Similarly, we define the law $\times$. 

Then the axioms of a field are verified. For instance, if $\alpha, \beta, \gamma \in E$, there is some $m\in \N$ such that $\alpha, \beta,\gamma \in E_m$, where $\alpha(\beta + \gamma) = \alpha \beta + \alpha \gamma$. Thus this equality is true on $E$.

This shows that $(E,+,\times)$ is a field, and $E_m$ is a subfield of $E$ for every $m\in \N$. In particular, $E$ is an extension of $F = E_0$.

Now we verify that $E$ has the expected property. Let $n \in \N^*$ be a positive integer. Consider
$$F_n = \{\alpha \in E \mid \alpha^{q^n} = \alpha\}.$$
Then $F_n$ is a subfield of $E$. Indeed, $1 \in F_n$, and if $\alpha, \beta \in F_n$, and $\gamma \in F_n^*$, then
\begin{align*}
(\alpha + \beta)^{q^n} &= \alpha^{q^n} + \beta^{q^n} = \alpha + \beta,\\
(\alpha  \beta)^{q^n} &= \alpha^{q^n}  \beta^{q^n} = \alpha  \beta,\\
(\gamma^{-1})^{q^n} &=( \gamma^{q^n})^{-1} = \gamma^{-1}
\end{align*}
thus $\alpha + \beta, \alpha \beta, \gamma^{-1} \in F_n$.        
Let $n = p_1^{a_1}\cdots p_k^{a_k}$ be the decomposition of $n$ in prime factors, for some $k\in \N$,  and $a_i \geq 0, \ i= 1,2,\ldots,k$. If $m = \max \{a_1,\ldots,a_k\}$, then $n \mid N_m$. By Lemma 2 and 3 of Chapter 7,  this shows that $q^n - 1 \mid q^{N_m} - 1$, thus $x^{q^n - 1} - 1 \mid x^{q^{N_m} - 1} - 1$, thus $x^{q^n} - x \mid x^{q^{N_m}} - x$. Ny proposition 7.1.1, since $E_m$ is a field with $q^{N_m}$ elements,
$$x^{q^{N_m}} - x =\prod_{\alpha \in E_m} (x- \alpha).$$
therefore the factor $x^{q^n} - x$ of $x^{q^{N_m}} - x$ splits completely over $E_m$, a fortiori over $E$, and Corollary 2 shows that all the roots of $x^{q^n} - x$, which are in $E_m$, are simple roots.

This prove that
$$ x^{q^n} -  x =\prod_{\alpha \in A} (x- \alpha),$$
where $A \subset E_m \subset E$.

By definition of $F_n$, for all $\alpha \in E$, $\alpha$ is a root of $x^{q^n}- 1$ if and only if $\alpha \in F_n$, thus $A = F_n$, and
$$ x^{q^n} -  x =\prod_{\alpha \in F_n} (x- \alpha),$$
The comparison of the degrees gives
$$q^n = |F_n|.$$
This proves that $E$ contains a field $F_n$ with $q^n$ elements.

Suppose that $E$ contains another field $F'_n$ with $q^n$ elements, then the preceding argument shows that
$$x^{q^n} - x =\prod_{\alpha \in F_n} (x - \alpha) =\prod_{\alpha \in F'_n} (x - \alpha),$$
therefore $F_n = F'_n$.

For every positive integer $n$, $E$ contains one and only one subfield $F_n$ with $q^n$ elements.
\end{proof}

\bigskip
Note: We can show a little more, that $E$ is an algebraic closure of $F$.

First, $E$ is algebraic over $F$, since every $\alpha \in E$ is in some $E_m$, which is a finite extension of $F$, thus $\alpha$ is algebraic over $F*$.

Next, we show that $E$ is algebraically closed. Let $p(x) = \sum_{k=0}^l a_k x^k \in E[x]$ be any non constant polynomial with coefficients in $E$. There is a $m\in \N$ such that all the coefficients $a_i$ are in $E_m$, so that $p(x) \in E_m[x]$. Let $f(x)$ be an irreducible factor of $p(x)$ over $E_m$, with $\deg(f) = d \geq 1$.

%The generalization of Theorem 2, Chapter 7, shows that
%$$x^{q^{dN_m}} - x = \prod_{\delta \mid d N_m} F_\delta(x),$$
%o� $F_\delta(x)$ is the product of the monic irreducible polynomials in $E_m[x]$ of degree $\delta$. 
Then $f(x)$ has a root $\gamma$ in the field $K = E_m[x]/(f(x))$, where $|K| = q^{dN_m}$, so $\gamma$ is a root of $x^{q^{dN_m}} - x$. Since $f(x)$ is the minimal polynomial of $\gamma$ over $E_m$, this proves that $f(x) \mid x^{q^{dN_m}} - x$. If $n = dN_m$, we have seen that if $F_n$ is the subfield of $E$ with $q^n$ elements, then  $x^{q^n} - x =\prod_{\alpha \in F_n} (x - \alpha)$, so that $f(x)$ splits completely over $F_n = F_{dN_m} \subset E$. Therefore $f(x)$ has a root in $E$, and also $p(x)$.

We have proved that $E$ is an algebraic closure of $F$ (with a concrete construction, without the axiom of choice, used in the general proof of the existence of algebraic closure).
\end{document}





