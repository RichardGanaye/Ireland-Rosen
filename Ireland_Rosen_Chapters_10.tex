%&LaTeX
\documentclass[11pt,a4paper]{article}
\usepackage[frenchb,english]{babel}
\usepackage[applemac]{inputenc}
\usepackage[OT1]{fontenc}
\usepackage[]{graphicx}
\usepackage{amsmath}
\usepackage{amsfonts}
\usepackage{amsthm}
\usepackage{amssymb}
\usepackage{yfonts}
\usepackage{mathrsfs}
%\input{8bitdefs}

% marges
\topmargin 10pt
\headsep 10pt
\headheight 10pt
\marginparwidth 30pt
\oddsidemargin 40pt
\evensidemargin 40pt
\footskip 30pt
\textheight 670pt
\textwidth 420pt

\def\imp{\Rightarrow}
\def\gcro{\mbox{[\hspace{-.15em}[}}% intervalles d'entiers 
\def\dcro{\mbox{]\hspace{-.15em}]}}

\newcommand{\D}{\mathrm{d}}
\newcommand{\Q}{\mathbb{Q}}
\newcommand{\Z}{\mathbb{Z}}
\newcommand{\N}{\mathbb{N}}
\newcommand{\R}{\mathbb{R}}
\newcommand{\C}{\mathbb{C}}
\newcommand{\F}{\mathbb{F}}
\newcommand{\re}{\,\mathrm{Re}\,}
\newcommand{\ord}{\mathrm{ord}}
\newcommand{\n}{\mathrm{N}}
\newcommand{\legendre}[2]{\genfrac{(}{)}{}{}{#1}{#2}}



\title{Solutions to Ireland, Rosen ``A Classical Introduction to Modern Number Theory''}
\author{Richard Ganaye}

\begin{document}

{ \Large \bf Chapter 10} 

\paragraph{Ex. 10.1}

{\it If $K$ is an infinite field and $f(x_1,x_2,\ldots,x_n)$ is a non-zero polynomial with coefficients in $K$, show that $f$ is not identically zero on $A_n(K)$. (Hint: Imitate the proof of Lemma 1 in Section 2.)
}

\begin{proof}
Assume that $f$ vanishes on all of $A_n(K)$. We have to prove that $f$ is the zero polynomial.

The proof is by induction on $n$. If $n=1$, then $f$ is a polynomial with one variable, which vanishes on $A_1(K) = K$. Since $K$ is infinite, $f$ have more than $d$ roots, where $d = \deg(f)$, thus $f$ is the zero polynomial.

Suppose that we have proved the result for $n-1$ and write
$$f(x_1,\ldots,x_n) = \sum_{i=0}^{s-1} g_i(x_1,\ldots,x_{n-1}) x_n^i,$$
where the $x_i$ are variables, and $g_i$ are polynomials in $x_1,\ldots,x_{n-1}$.

For all $(a_1,\ldots,a_n) \in K^n$, 
$$0 = f(a_1,\ldots,a_n) = \sum_{i=0}^{s-1} g_i(a_1,\ldots,a_{n-1}) a_n^i.$$
From the result for $n=1$, we obtain that the polynomial $ \sum_{i=0}^{s-1} g_i(a_1,\ldots,a_{n-1}) x_n^i$ is null, thus for all $(a_1,\ldots,a_{n-1}) \in K^{n-1}$,
$$ g_i(x_1,\ldots,x_{n-1}) = 0.$$
The induction hypothesis shows that $ g_i(x_1,\ldots,x_{n-1}) = 0$, thus $f(x_1,\ldots,x_n) = 0$.
\end{proof}

\paragraph{Ex. 10.2}
{\it In section 1 it was asserted that $H$, the hyperplane at infinity in $P_n(F)$, has the structure of $P_{n-1}(F)$. Verify this by constructing a one-to-one, onto map from $P_{n-1}(F)$ to $H$.}

\begin{proof}
Note that if one representative $(x_0,\ldots,x_n)$ of a projective point satisfies $x_0 = 0$, then it is the same for all representatives of this point, so we can define
$$\overline{H} = \{[x_0,\ldots,x_n] \in P_n(F) \mid x_0 = 0\},$$
where we write for simplicity $[x_0,\ldots,x_n]$ for $[(x_0,\ldots,x_n)]$.

Consider
$$
\psi
\left\{
\begin{array}{ccl}
\overline{H} & \to &P_{n-1}(F)\\
{[}0,x_1,\ldots,x_n{]}& \mapsto & {[}x_1,\ldots,x_n{]}
\end{array}
\right.
$$
Then $\psi$ is well-defined. Indeed, if $(0,x_1,\ldots,x_n) \sim (0,y_1,\ldots,y_n)$, then there is some $\lambda \in F^*$ such that $(0,y_1,\ldots,y_n) = \lambda (0,x_1,\ldots,x_n)$, thus $(y_1,\ldots,y_n) = \lambda (x_1,\ldots,x_n)$, and $ {[}x_1,\ldots,x_n{]} =  {[}y_1,\ldots,y_n{]}$.

If $ \psi({[}0,x_1,\ldots,x_n{]}) =  \psi({[}0,y_1,\ldots,y_n{]})$, then ${[}(x_1,\ldots,x_n){]} = {[}(y_1,\ldots,y_n){]}$, so there is some $\lambda \in F^*$ such that $y_i = \lambda x_i, \ i=1,\ldots,n$. Since $0 = \lambda 0$, $(0,y_1,\ldots,y_n) \sim (0,x_1,\ldots,x_n)$, therefore ${[}0,x_1,\ldots,x_n{]} = {[}0,y_1,\ldots,y_n{]}$, so $\psi$ is injective.

Moreover if ${[}x_1,\ldots,x_n{]} $ is any projective point of $P_{n-1}(F)$, then ${[}x_1,\ldots,x_n{]}  =  \psi({[}0,x_1,\ldots,x_n{]})$ so $\psi$ is surjective. 

To conclude, $\psi$ is a bijection.
\end{proof}

\paragraph{Ex. 10.3} {\it Suppose that $F$ has $q$ elements. Use the decomposition of $P_n(F)$ into finite points and points at infinity to give another proof of the formula for the number of points in $P_n(F)$.
}
\begin{proof}

By exercise 2, the bijection $\psi$ shows that $|\overline{H}| = |P_{n-1}(F)|$. Therefore
$$|P_n(F)| = |P_n(F) \setminus \overline{H} | + |\overline{H} | = |A_n(F)| + |P_{n-1}(F)| = q^n +  |P_{n-1}(F)|.$$
Moreover $|P_0(F)| = 1$. Consequently,
$$|P_n(F)|  = |P_0(F)|  + \sum_{k=1}^n (|P_k(F)| - |P_{k-1}(F)| )= 1 + \sum_{k=1}^n q^k = q^n + q^{n-1} + \cdots + q + 1,$$
This gives another proof of the formula for the number of points in $P_n(F)$.
\end{proof}

\paragraph{Ex. 10.4} {\it The hypersurface defined by a homogeneous polynomial of degree 1, $a_0x_0+a_1x_1+\cdots +a_nx_n $ is called a hyperplane. Show that any hyperplane in $P_n(F)$ has the same number of elements as $P_{n-1}(F)$.
}
\begin{proof}
Define the hyperplane $\overline{K}$ by
$$\overline{K} = \{[x_0,\ldots,x_n] \in P_n(F) \mid a_0 x_0 + \cdots +a_n x_n = 0\},$$
where $(a_0,\ldots,a_n) \ne (0,\ldots,0)$ (if $(a_0,\ldots,a_n) = (0,\ldots,0)$, then $\overline{K} = P_n(F)$ is not a hyperplane). Note that, if $(x_0,\ldots,x_n) \sim (y_0,\ldots,y_n)$, there is $\lambda \in F^*$ such that $y_i = \lambda x_i,\ i=0,\ldots,n$, thus $a_0 x_0 + \cdots +a_n x_n \iff 0 = a_0 y_0 + \cdots +a_n y_n = 0$, so that the condition doesn't depend on the choice of the projective point representative.

Since $(a_0,\ldots,a_n) \ne (0,\ldots,0)$, suppose, without loss of generality, that $a_0 \ne 0$. Consider
$$
\chi
\left\{
\begin{array}{ccl}
\overline{K} & \to & P_{n-1}(F)\\
{[}x_0,\ldots,x_n{]} & \mapsto & [x_1,\ldots, x_n]
\end{array}
\right.
$$
Then $\chi$ is well defined. Indeed, if $(x_0,\ldots,x_n) \sim (y_0,\ldots,y_n)$, there is some $\lambda \in F^*$ such that $(y_0,\ldots, y_n) = \lambda (x_0,\ldots,x_n)$. In particular, $(y_1,\ldots, y_n) = \lambda (x_1,\ldots,x_n)$, thus $[x_1,\ldots,x_n] = [y_1,\ldots,y_n]$.

If $\chi([x_0,\ldots,x_n]) =  \chi([y_0,\ldots,y_n])$, where $[x_0,\ldots, x_n]$ and $[y_0,\ldots, y_n]$ are in $\overline{K}$, then $[x_1,\ldots, x_n] = [y_1,\ldots, y_n]$, thus there is $\lambda \in F^*$ such that $(y_1,\ldots, y_n) = \lambda (x_1,\ldots, x_n)$. Since $a_0 \ne 0$,
$$y_0 = -\frac{1}{a_0}(a_1y_1 + \cdots + a_ny_n) = - \lambda \frac{1}{a_0}(a_1x_1 + \cdots + a_nx_n) = \lambda x_0,$$
therefore $[x_0,\ldots,x_n] = [y_0,\ldots,y_n]$. So $\varphi$ is injective.

At last, let $[x_1,\ldots, x_n]$ be any point of $P_{n-1}(F)$. Define $x_0 = - \frac{1}{a_0}(a_1x_1 + \cdots + a_nx_n)$. Then $a_0x_0 + \cdots +a_nx_n = 0$, so that $[x_0,\ldots,x_n] \in \overline{K}$, and $\chi([x_0,\ldots,x_n]) = [x_1,\ldots,x_n]$. This proves that $\chi$ is surjective.

To conclude, $\chi$ is a bijection, therefore $|\overline{K}| = |P_{n-1}(F)| = q^{n-1} + \cdots+ q+1$.
\end{proof}

\paragraph{Ex. 10.5} {\it Let $f(x_0,x_1,x_2)$ be a homogeneous polynomial of degree $n$ in $F(x_0,x_1,x_2]$. Suppose that not every zero of $a_0x_0+a_1x_1+a_2x_2$ is a zero of $f$. Prove that there are at most $n$ common zeros of $f$ and $a_0x_0+a_1x_1+a_2x_2$ in $P_2(F)$. In more geometric language this says that a curve of degree $n$ and a line have at most $n$ points in common unless the line is contained in the curve.
}
\begin{proof}
Let $\mathscr{C}$ be the curve with equation $f(x_0,x_1,x_2) = 0$.

Since $a_0 x_0+a_1x_1+a_2x_2 =0$ is the equation of a line $l$, $(a_0,a_1,a_2) \ne 0$, so that we can suppose without loss of generality that $a_0 \ne 0$. Then 
\begin{align*}
[u_0,u_1,u_2] \in &l \iff a_0 u_0+a_1u_1+a_2u_2 =0\\
& \iff u_0 = -\frac{a_1}{a_0} u_1 - \frac{a_2}{a_0} u_2\\
&\iff u_0 = \alpha u_1 + \beta u_2,
\end{align*}
where $\alpha = -\frac{a_1}{a_0},\ \beta = - \frac{a_2}{a_0}.$
Therefore
\begin{align*}
[u_0,u_1,u_2] \in \mathscr{C} \cap l &\iff 
\left\{
\begin{array}{ll}
  a_0 u_0+a_1u_1+a_2u_2 &=0,\\
  f(u_0,u_1,u_2) &= 0,
\end{array}
\right.\\
&\iff 
\left\{
\begin{array}{ll}
  u_0 = \alpha u_1+\beta u_2,\\
  f\left(\alpha u_1+ \beta u_2,u_1,u_2\right) &= 0.
\end{array}
\right.
\end{align*}\\
Let $[u_0,u_1,u_2] \in  \mathscr{C} \cap l $. 

We show that $u_1 \ne 0$. If $u_1 = 0$, then $u_0 = \beta u_2$, therefore $[u_0,u_1,u_2] = [\beta u_2, 0, u_2] = [\beta, 0 ,1]$, and $f\left (\beta u_2, 0,u_2\right) = 0$. Therefore $p = [\beta,0,1] \in \mathscr{C} \cap l$.

Since $[1,0,0]$ and $[\beta,0,1]$ are two distinct points of $l$, an equation of $l$ is
$$
\begin{vmatrix}
1 & 0 & 0\\
\beta & 0 & 1\\
x_0 & x_1 & x_2
\end{vmatrix}
= -x_1,
$$
thus an equation of $l$ is given by $x_1$, therefore no equation $a_0 x_0 + a_1 x_1 + a_2 x_2$ of $l$ satisfies $a_0 \ne 0$, and this is in contradiction with $a_0 \ne 0$. We have proved $u_1 \ne 0$.



Since $f$ is homogeneous of degree $n$,
$$0 = u_1^n f\left(\alpha + \beta \frac{u_2}{u_1},1,\frac{u_2}{u_1}\right),$$
and using $u_1 \ne 0$,
$$0 =  f\left(\alpha + \beta \frac{u_2}{u_1},1,\frac{u_2}{u_1}\right).$$
\end{proof}
Consider the formal polynomial $P(x) = f\left(\alpha + \beta x,1,x\right) \in F[x].$

Then $\deg(P) \leq n$. If $P \ne 0$, then $P$ has at most $n$ roots $\lambda_1,\ldots,\lambda_k$, where $k\leq n$. In this case, $u_2 = \lambda_i u_1$ and $u_0  =\alpha u_1 + \beta u_2 = u_1\left(\alpha + \beta \lambda_i\right)$, therefore
$$[u_0,u_1,u_2]  =\left [\alpha + \beta \lambda_i, 1 , \lambda_i \right],\ 1\leq i \leq k,$$ 
so that $\mathscr{C}$ and $l$ have at most $n$ points in common.

Therefore, if $|\mathscr {C} \cap l| >n$, then $P =  f\left(\alpha + \beta x,1,x\right) = 0$.

Similarly, by exchanging the roles of $u_1,u_2$, if $|\mathscr {C} \cap l| >n$, then $u_2 \ne 0$, and
$$0 = f\left (\alpha \frac{u_1}{u_2} + \beta, \frac{u_1}{u_2}, 1\right ),$$
so that the same reasoning gives $Q(x) = f(\alpha x + \beta, x ,1) = 0$.

 Let $[v_0,v_1,v_2]$ be any point on $l$. 
 
 If $v_1 \ne 0$,
$$f(v_0,v_1,v_2) = f(\alpha v_1 + \beta v_2, v_1,v_2) = v_1^n f\left(\alpha + \beta \frac{v_2}{v_1}, 1 ,\frac{v_2}{v_1}\right) = v_1^n P\left (\frac{v_2}{v_1} \right) = 0.$$

If $v_1 = 0$, then $[v_0,v_1,v_2] = [\beta, 0,1] = p$, thus
$$f(\beta,0,1) = Q(0) =0.$$
This proves that $l \subset \mathscr{C}$.

To conclude, if $l \not \subset \mathscr {C}$, then $|l \cap \mathscr{C}| \leq n$ : a curve of degree $n$ and a line have at most $n$ points in common unless the line is contained in the curve.

\paragraph{Ex. 10.6} {\it Let $F$ be a field with $q$ elements. Let $M_n(F)$ be the set of $n \times n$ matrices with coefficients in $F$. Let $\mathrm{SL}_n(F)$ be the subset of those matrices with determinant equal to one. Show that $\mathrm{SL}_n(F)$ can be considered as a hypersurface in $A^{n^2}(F)$. Find a formula for the number of points on this hypersurface. [Answer:$(q-1)^{-1}(q^n-1)(q^n-q)\cdots(q^n-q^{n-1})$.]
}

\begin{proof}
If $M = (a_{i,j})_{1\leq i \leq n, 1\leq j \leq n} \in M_n(F)$, 
$$M \in \mathrm{SL}_n(F) \iff \sum_{\sigma \in S_n} \mathrm{sgn}(\sigma) a_{\sigma(1)1}\cdots a_{\sigma(n)n}-1=0.$$
if $f(x_{1,1},\ldots,x_{n,n}) = \sum\limits_{\sigma \in S_n} \mathrm{sgn}(\sigma) x_{\sigma(1)1}\cdots x_{\sigma(n)n}-1$, then  $M \in \mathrm{SL}_n(F)$ if and only if ${f(a_{1,1}, \ldots,a_{n,n}) = 0}$, where $f$ is a non zero polynomial, since it contains the non zero term $x_{1,1}\cdots x_{n,n}$. Therefore $\mathrm{SL}_n(F)$ is an hypersurface of $M_n(F)$.

Since a matrix $M \in M_n(F)$ is inversible if and only if its columns $(C1,\ldots,C_n)$ is a basis of $F^n$, the number of matrices in $\mathrm{GL}_n(F)$ is $$(q^n-1)(q^n-q)\cdots(q^n-q^{n-1}).$$Indeed we choose $C_1$ between $(q^n-1)$ non zero scalars, then we choose $C_2$ between the $q^n-q$ vectors $v  \not \in \langle C_1 \rangle$. If $C_1,\ldots,C_k$ are chosen, we take $C_{k+1}$ between the $q^n -q^k$ vectors $v \not \in \langle C_1,\ldots,C_k \rangle$. At last, we choose $C_n \not \in \langle C_1,\ldots C_{n-1}\rangle$. This gives
$$|\mathrm{GL}_n(F)| = (q^n-1)(q^n-q)\cdots(q^n-q^{n-1}).$$
Moreover, $\mathrm{SL}_n(F)$ is the kernel of the group homomorphism
$$
\left\{
\begin{array}{ccl}
\mathrm{GL}_n(F) & \to & F^*\\
M & \mapsto &\det(M).
\end{array}
\right.
$$
Therefore $F^* \simeq \mathrm{GL}_n(F)/\mathrm{SL}_n(F)$. This gives
$$|\mathrm{SL}_n(F)| = |\mathrm{GL}_n(F)|/ |F^*| = (q-1)^{-1}(q^n-1)(q^n-q)\cdots(q^n-q^{n-1}).$$
\end{proof}

\paragraph{Ex. 10.7} {\it Let $f \in F[x_0,\ldots,x_n]$. One can define the partial derivatives $\partial f/\partial x_0, \ldots,\partial f/ \partial x_n$ in a formal way. Suppose that $f$ is homogeneous of degree $m$. Prove that $\sum_{i=0}^n x_i (\partial f/ \partial x_i) = mf$. This result is due to Euler. (Hint: Do it first for the case that $f$ is a monomial.)
}
\begin{proof}
For the case that $f = x_1^{a_1}\cdots x_n^{a_n}$ is a monomial, where $a_1+ \ldots + a_n = m = \deg(f)$, then
$$\frac{\partial f}{\partial x_i} = a_i x_1^{a_1}\cdots x_i^{a_i - 1}\cdots x_n^{a_n}, \qquad i=1,\ldots,n.$$
Therefore $x_i \partial f/ \partial x_i = a_i f$, and
$$\sum_{i=1}^n x_i \frac{\partial f}{\partial x_i}  =\left (\sum_{i=1}^n a_i \right ) f = m f.$$
Since the maps $f \mapsto \sum_{i=1}^n x_i \frac{\partial f}{\partial x_i} $ and $f \mapsto mf$ are $FG-$ linear, and since every homogeneous polynomial $f$ is a linear combination of monomial with degree $m$, the relation is true for all such polynomials.

To conclude, every homogeneous polynomial $f \in F[x_0,\ldots,x_n]$ of degree $m$ satisfies
$$\sum_{i=1}^n x_i \frac{\partial f}{\partial x_i}  = m f.$$
\end{proof}

\paragraph{Ex. 10.8} {\it (continuation) If $f$ is homogeneous, a point $\overline{a}$ on the hypersurface defined by $f$ is said singular if it is simultaneously a zero of all the partial derivatives of $f$. If the degree of $f$ is prime to the characteristic, show that a common zero of all the partial derivatives of $f$ is automatically a zero of $f$.
}
\begin{proof}
If $\frac{\partial f}{\partial x_i} (\overline{a}) = 0$ for all $i = 1,\ldots,n$, then $mf(\overline{a}) = \sum\limits_{i=1}^n x_i \frac{\partial f}{\partial x_i} (\overline{a}) = 0$. Since $m = \deg(f)$ is prime with the characteristic, then $m$ is non zero in the field $F$, thus $f(\overline{a}) = 0$.
\end{proof}

\paragraph{Ex. 10.9} {\it If $m$ is prime to the characteristic of $F$, show that the hypersurface defined by $a_0x_0^m+a_1x_1^m+ \cdots + a_n x_n^m$ has no singular points.
}

\medskip

Note: The sentence is not true if some coefficient $a_i$ is zero. To give an counterexample, the projective curve given by $f(x_0,x_1,x_2) = x_1^2 -x_2^2$ is the union of two lines, and the intersection point $a =[1,0,0]$ of these two lines is singular : $\partial f/ \partial x_0(a) = \partial f/ \partial x_1(a) = \partial f/ \partial x_2(a) = 0$. We must assume that $a_i \ne 0$ for every index $i$ (see the hint p. 371).


\begin{proof} Let $V$ be the projective hypersurface defined by $f(x_0,\ldots,x_n) = a_0x_0^m+a_1x_1^m+ \cdots + a_n x_n^m$. 

If $m=1$, $V$ is an hyperplane, without singularity since $\frac{\partial f}{\partial x_i} (a) = a_i \ne 0$ for some index $i$.

We assume now that $m>1$. If $a = [u_0,\dots,u_n] \in V$ is a singular point, $$\frac{\partial f}{\partial x_i}(a) = ma_i u_i^{m-1} = 0\qquad (i=1,\ldots,n).$$ 

Since $m$ is prime with the characteristic, $m \ne 0$ in $F$, and $a_i \ne 0$, thus $u_i = 0$ for all indices $i$. Then $[u_0,\ldots,u_n]$ is not a projective point. This prove that $V$ has no singular point.
\end{proof}


\paragraph{Ex. 10.10} {\it A point on an affine hypersurface is said to be singular if the corresponding point on the projective closure is singular. Show that this is equivalent to the following definition. Let $f \in F[x_1,x_2,\ldots,x_n]$, not necessarily homogeneous, and $a \in H_f(F)$. Then $a$ is singular if it is a common zero of $\partial f/\partial x_i$ for $i=1,2,\ldots,n$.
}

\begin{proof}
Let $H_f(F)$ an affine hypersurface defined by $f(x_1,\ldots,x_n)$, with $\deg(f) = d$, and $a = (u_1,\ldots,u_n) \in F$.
\begin{enumerate} 
\item[$\bullet$] Suppose that the corresponding point $\overline{a} = [1,u_1,\ldots,u_n] \in \overline{F}$ is singular, and let
$$\overline{f}(y_0,\ldots,y_n) = y_0^d\, f\left(\frac{y_1}{y_0},\ldots,\frac{y_i}{y_0},\ldots,\frac{y_n}{y_0}\right)$$
be the homogeneous polynomial defining $\overline{F}$. Then the chain rule gives
$$\frac{\partial \overline{f}}{\partial y_i}(x_0,\ldots,x_n) = x_0^{d-1} \frac{\partial f}{\partial x_i}\left(\frac{x_1}{x_0},\ldots,\frac{x_i}{x_0},\ldots,\frac{x_n}{x_0}\right).$$
Since $\overline{a}$ is singular,
$$0 =  \frac{\partial \overline{f}}{\partial y_i}(\overline{a}) = \frac{\partial \overline{f}}{\partial y_i}(1,u_1,\ldots,u_n) = \frac{\partial f}{\partial x_i} (u_1,\ldots,u_n) =\frac{\partial f}{\partial x_i}(a).$$
This proves that $a$ is a common zero of $\partial f/\partial x_i$ for $i=1,2,\ldots,n$
\item[$\bullet$] Conversely, suppose that $\partial f/\partial x_i (a) = 0$ for $i=1,\ldots,n$. Then 
$$\frac{\partial \overline{f}}{\partial y_i}(\overline{a}) = \frac{\partial \overline{f}}{\partial y_i}(1,u_1,\ldots,u_n) =\frac{\partial f}{\partial x_i} (u_1,\ldots,u_n) =0,$$
which proves that $\overline{a}$ is singular.
\end{enumerate}
\end{proof}

\paragraph{Ex. 10.11} {\it Show that the origin is a singular point on the curve defined by $y^2 - x^3 = 0$.
}
\begin{proof}
If $f(x,y) = y^2 - x^3$, then
$$\frac{\partial f}{\partial x}  = 3x^2,\qquad \frac{\partial f}{\partial y} = 2y,$$
thus $\partial f/\partial x (0,0) = \partial f/\partial y (0,0) = 0$. This proves that the origin is a singular point for the curve defined by $f$.
\end{proof}

\paragraph{Ex. 10.12} {\it Show that the affine curve defined by $x^2 + y^2 + x^2 y^2 = 0$ has two points at infinity and that both are singular.
}
\begin{proof}
The homogeneous equation of this curve is
$$\overline{f} (t,x, y) =  x^2 t^2 + y^2 t^2 + x^2 y^2, $$
where $t=0$ is the equation of the line at infinity.

The point $\overline{a} = [u_0, u_1,u_2]$ is a point at infinity if $u_0 = 0$. This gives the equation $$\overline{f} (0,u_1,u_2) = u_1^2u_2^2 = 0,$$
 where $u_1 \ne 0$ or $u_2\ne 0$ (otherwise $u_0 = u_1 = u_2 = 0$, and $[u_0, u_1,u_2]$ is not a projective point).
 
 If $u_1 \ne 0$, then $u_2 = 0$, and if $u_2 \ne 0$, then $u_1 = 0$.
 
 Therefore $\overline{a} = [0, u_1,0] = [0,1,0]$, or $\overline{a} = [0,0,u_2] = [0,0,1]$.
 
 $p = [0,1,0]$ and $q = [0,0,1]$ are the two points at infinity of the curve.
 
 $$\frac{\partial \overline{f}}{\partial t} =  2 t (x^2 + y^2), \qquad \frac{\partial \overline{f}}{\partial x} = 2x (t^2 +y^2),\qquad  \frac{\partial \overline{f}}{\partial y} = 2y (t^2 + x^2).$$
Therefore
$$\frac{\partial \overline{f}}{\partial t}(0,1,0) = \frac{\partial \overline{f}}{\partial x}(0,1,0)=\frac{\partial \overline{f}}{\partial y}(0,1,0) = 0,$$
and
$$\frac{\partial \overline{f}}{\partial t}(0,0,1) = \frac{\partial \overline{f}}{\partial x}(0,0,1)=\frac{\partial \overline{f}}{\partial y}(0,0,1) = 0.$$
This proves that the two points at infinity $p,q$ are singular.
\end{proof}

\paragraph{Ex. 10.13} {\it Suppose that the characteristic of $F$ is not $2$, and consider the curve defined by $ax^2+bxy+cy^2 = 1$, where $a,b,c \in F^*$. If $b^2 -4ac \in F^2$, show that there are one or two points at infinity depending on whether $b^2 -4ac$ is zero. If $b^2 -4ac = 0$, show that the point at infinity is singular.
}
\begin{proof}
Let $\mathscr{C}$ be the curve defined by $f(x,y) = ax^2+bxy+cy^2 -1$. The homogeneous equation of the projective closure $\overline{\mathscr{C}}$ of $\mathscr{C}$ is
$$\overline{f}(t,x,y) = ax^2+bxy+cy^2 -t^2.$$

The points $[0,u,v]$ at infinity are given by the equation
$$au^2 + buv + cv^2 = 0.$$

Assume that $\Delta = b^2 - 4ac = \delta^2 \in F^2$. Since $a \ne 0$,
\begin{align*}
au^2 + buv + c v^2 &= a \left[ \left(u+ \frac{b}{2a} v\right)^2 - \frac{b^2 - 4ac}{4a^2} v^2\right]\\
&= a \left[ \left(u+ \frac{b}{2a} v\right)^2 - \frac{\delta^2}{4a^2} v^2\right]\\
&= a \left( u - \frac{-b+\delta}{2a} v \right)  \left( u - \frac{-b-\delta}{2a} v \right) \\
&= a( u - \alpha v)(u - \beta v),
\end{align*}
where $\alpha = \frac{-b+\delta}{2a}, \beta = \frac{-b-\delta}{2a}$ are the two roots of $aX^2 + b X +c$.

Therefore the points at infinity are $p = [0,\alpha, 1]$ and $q = [0,\beta, 1]$.

\begin{enumerate}
\item[$\bullet$] If $b^2 - 4ac \ne 0$ (hyperbolic case), then $\alpha \ne \beta$ and $p \ne q$, so that $\mathscr{C}$ has two points at infinity.

\item[$\bullet$] If $b^2 - 4ac = 0$ (parabolic case), then  $\alpha = \beta$, and $\mathscr{C}$ has one  (double) point  at infinity $r = [0,\alpha, 1,]$, where $\alpha = -\frac{b}{2a}$ is the root of multiplicity $2$ of $aX^2 + b X +c$. Thus $r=[0, -b, 2a]$.

Since
$$\frac{\partial \overline{f}}{\partial t} (t,x,y)= -2t, \qquad \frac{\partial \overline{f}}{\partial x}(t,x,y) = 2ax + by,\qquad  \frac{\partial \overline{f}}{\partial y}(t,x,y) =  bx + 2c y,$$
then
$$\frac{\partial \overline{f}}{\partial t}(0,-b,2a) = 0, \qquad \frac{\partial \overline{f}}{\partial x}(0,-b,2a) = -2ab + 2ab =  0,\qquad  \frac{\partial \overline{f}}{\partial y}(0,-b,2a) = -(b^2 - 4ac) =  0.$$
This shows that the point at infinity $r = [0,-b,2a]$ is singular.
\end{enumerate}
\end{proof}

\paragraph{Ex. 10.14} {\it Consider the curve defined by $y^2 = x^3 + ax +b$. Show that it has no singular points (finite or infinite) if $4a^3 + 27 b^2 \ne 0$.
}

\begin{proof} Let $\mathscr{C}$ be  the curve defined by $f(x,y) = y^2 - x^3  - ax - b$. The homogeneous equation of the projective closure $\overline{\mathscr{C}}$ of $\mathscr{C}$ is
$$\overline{f}(t,x,y) = y^2 t - x^3 -a xt^2 - bt^3.$$
The only point at infinity is given by $t = 0, -x^3 = 0$, thus is the point $p = [0,0,1]$.
Since
$$\frac{\partial \overline{f}}{\partial t} (t,x,y)= y^2 -2axt -3bt^2, \qquad \frac{\partial \overline{f}}{\partial x}(t,x,y) = -3x^2 -at^2,\qquad  \frac{\partial \overline{f}}{\partial y}(t,x,y) = 2y t,$$
then $\frac{\partial \overline{f}}{\partial t} (0,0,1) = 1$, thus the point at infinity $p$ is not singular.

For some other points $a = (u,v)$ on $\overline{C}$ not at infinity, it is sufficient by Exercise 10 to verify $(\partial f/\partial x (u,v), \partial f/\partial y (u,v)) \ne (0,0)$. Since
$$\frac{\partial f}{\partial x}(u,v) = -3u^2 -a,\qquad \frac{\partial f}{\partial y}(u,v) = 2v,$$
if $a$ is singular, then
$$
\left \{
\begin{array}{ll}
v^2 &= u^3  + au +b,\\
-3u^2 - a &= 0,\\
2v &= 0.
\end{array}
\right.
$$
Therefore
$$
\left \{
\begin{array}{ll}
0&= u^3  + au +b,\\
-\frac{a}{3}&= u^2,\\
\end{array}
\right.
$$

If $a = 0$, then $u=v=0$, thus $b=0$, so that $4a^3 + 27b^2 = 0$.

If $a \ne 0$,  we eliminate $u$ between these two equations to obtain, 
$$0 = u(u^2 + a) + b = \frac{2}{3}a u +b,$$
thus
$u = -\frac{3b}{2a}$, and $u^2 = \frac{9b^2}{4a^2} = -\frac{a}{3}$, which gives $4a^3 + 27b^2 = 0$.
To conclude, if $4a^4 + 27b^2 \ne 0$, then the curve defined by $y^2 = x^3 + ax +b$ has no singular points, finite or infinite.
\end{proof}


\paragraph{Ex. 10.15} {\it Let $\Q$ be the field of rational numbers and $p$ a prime. Show that the form $x_0^{n+1} + p x_1^{n+1} + p^2 x_2^{n+1} + \cdots + p^n x_n^{n+1}$ has no zeros in $P^n(\Q)$. (Hint: If $\overline{a}$ is a zero, one can assume that the components of $a$ are integers and that they are not all divisible by $p$.)
}
\begin{proof}
Write $f(x_0,\ldots,x_n) = x_0^{n+1} + p x_1^{n+1} + p^2 x_2^{n+1} + \cdots + p^n x_n^{n+1}$.

Reasoning by contradiction, suppose that $\overline{a} = [\alpha_0,\ldots,\alpha_n]$ is a zero of $f$, where $\alpha_i \in \Q$ for $i=0,\ldots,n$. Using a common denominator $c$ for these rational numbers, we can write
\begin{align*}
\overline{a} &= [\alpha_0,\ldots,\alpha_n]\\
&= \left[\frac{b_0}{c},\ldots, \frac{b_n}{c}\right] \qquad (b_i \in \Z)\\
&= \left[d \frac{a_0}{c},\ldots,d \frac{a_n}{c}\right]\\
 &=  [a_0,\ldots,a_n],
\end{align*}
where $d =b_0\wedge \cdots \wedge b_n$ is the gcd of the $b_i$, so that the $a_i \in \Z$ satisfy $a_0 \wedge \cdots \wedge a_n = 1$.

Then 
$$a_0^{n+1} + p a_1^{n+1} + p^2 a_2 ^{n+1} + \cdots + p^n a_n^{n+1} = 0,$$
where the integers $a_i$ are not all divisible by $p$.

To obtain a contradiction, we will show that all the $a_i$ are divisible by $p$.

$p \mid - p a_1^{n+1} - p^2 a_2 ^{n+1} + \cdots - p^n a_n^{n+1} = a_0^{n+1}$, thus $p \mid a_0$.

Reasoning by induction, suppose that $p$ divides $a_0,\ldots,a_k$, where $k<n$. Then $p^{n+1} \mid a_0^{n+1} + p a_1^{n+1} + p^2 a_2 ^{n+1} + \cdots + p^k a_k^{n+1}$, therefore
$$p^{n+1} \mid  p^{k+1} a_{k+1} ^{n+1} + p^{k+2} a_{k+2}^{n+1} +\cdots + p^n a_n^{n+1} = p^{k+1}(a_{k+1}^{n+1} + p a_{k+2}^{n+1} + \cdots + p^{n-k-1}a_n^{n+1}).$$
Since $n>k$, $ p \mid a_{k+1}^{n+1} + p a_{k+2}^{n+1} + \cdots + p^{n-k-1}a_n^{n+1}$, therefore $p \mid a_{k+1}^{n+1}$, thus $p \mid a_{k+1}$.

The induction is done. This proves that $p\mid a_0, \ldots, p \mid a_n$. This is a contradiction, since the $a_i$ are not all divisible by $p$. So the form $x_0^{n+1} + p x_1^{n+1} + p^2 x_2^{n+1} + \cdots + p^n x_n^{n+1}$ has no zeros in $P_n(\Q)$.
\end{proof}


\paragraph{Ex. 10.16} {\it Show by explicit calculation that every cubic form in two variables over $\Z/2\Z$ has a non trivial zero.
}
\medskip

Note : this assertion seems false  (or I don't understood the sentence).

\begin{proof} 

We can write a cubic form on $P_1(\F_2)$ under the form
$$f(x_0,x_1) = a x_0^3 + b x_0^2 x_1 + c x_0 x_1^2 + d x_1^3,\qquad a,b,c,d \in \F_2.$$
Thus there are $15$ such cubic forms. 

This small Sage program computes the set of non trivial solutions for each of these forms

\begin{verbatim}
F2 = GF(2)
R.<x0,x1>= F2[]
l = [a*x0^3 + b * x0^2 * x1 + c * x0 * x1^2 + d * x1^3 
      for a in F2 for b in F2 for c in F2 for d in F2 
      if not [a,b,c,d] == [0,0,0,0]]
l
\end{verbatim}

$[x_{1}^{3}, x_{0} x_{1}^{2}, x_{0} x_{1}^{2} + x_{1}^{3}, x_{0}^{2} x_{1}, x_{0}^{2} x_{1} + x_{1}^{3}, x_{0}^{2} x_{1} + x_{0} x_{1}^{2},x_{0}^{2} x_{1} + x_{0} x_{1}^{2} + x_{1}^{3}, x_{0}^{3}, x_{0}^{3} +x_{1}^{3},\\
 x_{0}^{3} + x_{0} x_{1}^{2}, x_{0}^{3} + x_{0} x_{1}^{2} +x_{1}^{3}, x_{0}^{3} + x_{0}^{2} x_{1}, x_{0}^{3} + x_{0}^{2} x_{1} +x_{1}^{3}, x_{0}^{3} + x_{0}^{2} x_{1} + x_{0} x_{1}^{2}, x_{0}^{3} +x_{0}^{2} x_{1} + x_{0} x_{1}^{2} + x_{1}^{3}]$


\begin{verbatim}
for f in l:
    S = []
    for x in F2:
        for y in F2:
            if [x,y] != [0,0] and f.subs(x0=x,x1=y) == 0:
                S.append([x,y])
    print f,  ' : ', S
    
x1^3 			 :  [[1, 0]]
x0*x1^2 			 :  [[0, 1], [1, 0]]
x0*x1^2 + x1^3 			 :  [[1, 0], [1, 1]]
x0^2*x1 			 :  [[0, 1], [1, 0]]
x0^2*x1 + x1^3 			 :  [[1, 0], [1, 1]]
x0^2*x1 + x0*x1^2 			 :  [[0, 1], [1, 0], [1, 1]]
x0^2*x1 + x0*x1^2 + x1^3 			 :  [[1, 0]]
x0^3 			 :  [[0, 1]]
x0^3 + x1^3 			 :  [[1, 1]]
x0^3 + x0*x1^2 			 :  [[0, 1], [1, 1]]
x0^3 + x0*x1^2 + x1^3 			 :  []
x0^3 + x0^2*x1 			 :  [[0, 1], [1, 1]]
x0^3 + x0^2*x1 + x1^3 			 :  []
x0^3 + x0^2*x1 + x0*x1^2 			 :  [[0, 1]]
x0^3 + x0^2*x1 + x0*x1^2 + x1^3 			 :  [[1, 1]]

\end{verbatim}

This shows that two cubics forms have no non trivial solutions. We verify this for the form $x_0^3 + x_0x_1^2 + x_1^3$ :
$$
\begin{array}{|c|c|c|}
\hline
x_0 & x_1 & x_0^3 + x_0x_1^2 + x_1^3\\
\hline
0 & 1 & 1\\
1 & 0 & 1\\
1 & 1 & 1\\
\hline
\end{array}
$$
So the sentence is false.

With three variables $x_0,x_1,x_2$, there are $1023$ cubics forms. A similar program gives among them the form
$$f(x_0,x_1,x_2) = x_0^3 + x_0x_1^2 + x_1^3 + x_0x_1x_2 + x_0x_2^2 + x_1x_2^2 + x_2^2,$$
which has no non trivial zero:
$$
\begin{array}{|c|c|c|c|}
\hline
x_0 & x_1 & x_2 &f(x_0,x_1,x_2)\\
\hline
0 & 0 & 1& 1\\
0 & 1 &0 &1\\
0 & 1 & 1& 1\\
1 & 0 & 0 & 1\\
1 & 0 & 1 & 1\\
1 & 1 & 0 & 1\\
1 & 1 & 1 & 1\\
\hline
\end{array}
$$
The Chevalley's Theorem shows that with 4 (or more) variables $x_0,x_1,x_2, x_3$, every cubic form has non trivial solutions.
\end{proof}


\paragraph{Ex. 10.17} {\it Show that for each $m>0$ and finite field $F_q$ there is a form of degree $m$ in $m$ variables with no nontrivial zero. [Hint: Let $\omega_1,\omega_2,\ldots,\omega_m$ be a basis for $F_{q^m}$ over $F_q$ and show that $f(x_1,x_2,\ldots,x_m) = \prod_{i=0}^{m-1} (\omega_1^{q^i} x_1 + \cdots + \omega_m^{q^i} x_m)$ has the required properties.]
}

\begin{proof}
Let $\omega_1,\omega_2,\ldots,\omega_m$ be a basis for $F_{q^m}$ over $F_q$.

Consider $$f(x_1,\ldots,x_m) = \prod_{i=0}^{m-1} (\omega_1^{q^i} x_1 + \cdots + \omega_m^{q^i} x_m).$$
Then $f$ is a form of degree $m$ in $m$ variables.

By definition, $f \in \F_{q^m}(x_1,\ldots,x_m)$. We show first that $f \in \F_q[x_1,\ldots,x_m]$.

Let $f$ be the Frobenius automorphism on $\F_{q^m}$, defined by
$$F
\left\{
\begin{array}{ccl}
\F_{q^m} & \to &\F_{q^m}\\
\alpha & \mapsto & \alpha^q.
\end{array}
\right.
$$

By Corollary 1 of Proposition 7.1.1, for every $\alpha \in \F_{q^m}$, $\alpha \in \F_q$ if and only if $F(\alpha) = \alpha$. If $p = \sum_{i=0}^d a_{i_1,\ldots,i_m}  x_1^{i_1} \cdots x_m^{i_m} \in \F_{q^m}[x_1,\ldots,x_m]$, define $F\cdot p =\sum_{i=0}^d F(a_{i_1,\ldots,i_m})  x_1^{i_1} \cdots x_m^{i_m} $. Then $F \cdot p \in \F_{q^m}[x] \iff F \cdot p = p$ and $F\cdot (pq) = (F\cdot p)(F \cdot q)$ for all $p,q \in  \F_{q^m}[x_1,\ldots,x_m]$.

Then, using this last property, 
\begin{align*}
F \cdot f &= \prod_{i=0}^{m-1} F \cdot (\omega_1^{q^i} x_1 + \cdots + \omega_m^{q^i} x_m)\\
&=\prod_{i=0}^{m-1} (\omega_1^{q^{i+1}} x_1 + \cdots + \omega_m^{q^{i+1}} x_m)\\
&=\prod_{j=1}^{m} (\omega_1^{q^j} x_1 + \cdots + \omega_m^{q^j} x_m) \qquad (j=i+1)\\
&=\prod_{j=0}^{m-1} (\omega_1^{q^j} x_1 + \cdots + \omega_m^{q^j} x_m) \qquad (\text{since } \omega_k^{q^m} = \omega_k = \omega_k^{q^0},\ k=1,\ldots,m)\\
&= f.
\end{align*}
Therefore $f \in \F_q[x_1,\ldots,x_m]$.

Now we prove that $f$ has no non trivial zero $\overline{a} = (\alpha_1,\ldots,\alpha_m) \in \F_q^m \setminus \{(0,\ldots,0)\}$. If $f$ had such a zero $(\alpha_1,\ldots,\alpha_m) \in \F_q^m$, then 
$$\prod_{i=0}^{m-1} (\omega_1^{q^i} \alpha_1 + \cdots + \omega_m^{q^i} \alpha_m) = 0,\qquad \alpha_1,\ldots,\alpha_m \in \F_q.$$
Then for some $i \in \gcro 0,m-1 \dcro$,
$$\omega_1^{q^i} \alpha_1 + \cdots + \omega_m^{q^i} \alpha_m = 0.$$
Applying $F^{m-i}$ to this equality, and using $F(\alpha_i) = \alpha_i$, we obtain
$$\omega_1^{q^m} \alpha_1 + \cdots + \omega_m^{q^m} \alpha_m.$$
Since $\omega_i^{q^m} = \omega_i,\ i = 1,\ldots,m$, this gives
$$\omega_1 \alpha_1 + \cdots + \omega_m \alpha_m = 0.$$
Since $\omega_1,\omega_2,\ldots,\omega_m$ is a basis for $F_{q^m}$ over $F_q$, this proves
$$(\alpha_1,\ldots,\alpha_m) = (0,\ldots,0).$$
So $f$ has no non trivial zero.

Note : this proves that we cannot extend the Chevalley's Theorem to the forms of degree $m$ in $m$ varibles.
\end{proof}


\paragraph{Ex. 10.18} {\it Let $g_1,g_2,\ldots,g_m\in \F_q[x_1,x_2,\ldots,x_n]$ be homogeneous polynomials of degree $d$ and assume that $n>md$. Prove that there is nontrivial common zero. [Hint: Let $f$ be as in Exercise 17 and consider the polynomial $f(g_1(x_1,\ldots,x_n),\ldots,g_m(x_1,\ldots,x_m))$.]
}
\begin{proof}
Consider the polynomial $h = f(g_1(x_1,\ldots,x_n),\ldots,g_m(x_1,\ldots,x_m) \in \F_q[x_1,\ldots,x_m]$. Then $h$ is homogeneous of degree $md$. Since $n > md$, the Chevalley's Theorem (Corollary of Theorem 1) shows that there is a non trivial zero $\overline{a} = (\alpha_1,\ldots,\alpha_m) \in \F_q^m\setminus\{(0,\ldots,0)\}$ of $h$, so that
$$f(g_1(\alpha_1,\ldots,\alpha_n),\ldots,g_m(\alpha_1,\ldots,\alpha_m)) = 0,\qquad (\alpha_1,\ldots,\alpha_m) \in \F_q^m \setminus\{(0,\ldots,0)\}.$$
Then
$$f(\beta_1,\ldots,\beta_m) = 0,\qquad \text{where } \beta_i = g_1(\alpha_1,\ldots,\alpha_n) \in \F_q.$$
Since $f$ has no trivial zero by Exercise 17, we obtain $\beta_1 =\cdots = \beta_m = 0$, that is
$$g_1(\alpha_1,\ldots,\alpha_n)=\ldots =g_m(\alpha_1,\ldots,\alpha_m)=0,\qquad (\alpha_1,\ldots,\alpha_m) \in \F_q^m\setminus\{(0,\ldots,0)\}.$$
This proves that here is nontrivial common zero for $g_1, \ldots g_m$, if $n>md$.
\end{proof}

\paragraph{Ex. 10.19} {\it Characterize those extensions $\F_{p^n}$ of $\F_p$ that are such that the trace is identically zero on $\F_p$.
}
\begin{proof}
If $\alpha \in \F^p$, then $\alpha^p = \alpha$, thus $\alpha^{p^k} = \alpha$ for all exponents $k\geq 0$.

In the extension $\F_{p^n}$ of $\F_p$, for all $\alpha \in \F_p$,
\begin{align*}
\mathrm{tr}(\alpha) &= \alpha + \alpha^p + \alpha^{p^2}+ \cdots + \alpha^{p^{m-1}}\\
&= n \alpha.
\end{align*}
If the characteristic $p$ divides $n$, then $n = 0$ in $\F_{p^n}$, thus $\mathrm{tr}(\alpha) = 0$ for all $\alpha \in \F_p$.

Conversely, if $\mathrm{tr}(\alpha) = 0$ for all $\alpha \in \F_p$, then $\mathrm{tr}(1) = n\cdot 1 = 0$, thus the characteristic $p$ divides $n$.

The extensions $\F_{p^n}$ of $\F_p$ that are such that the trace is identically zero on $\F_p$ are those which satisfy $p \mid n$.


\end{proof}

\paragraph{Ex. 10.20} {\it Show that if $\alpha \in \F_q$ has trace zero, then $\alpha = \beta - \beta^p$ for some $\beta \in \F_q$.
}
\begin{proof}
Here $q = p^n$. Consider first the map
$$\mathrm{tr}
\left\{
\begin{array}{ccl}
\F_{p^n} & \to & \F_p\\
\alpha & \mapsto & \mathrm{tr}(\alpha) = \alpha + \alpha^p + \alpha^{p^2} + \cdots + \alpha^{p^{n-1}}
\end{array}
\right.
$$

This makes sense, since by Proposition 10.3.1(a), $\mathrm{tr}(\alpha) \in \F_p$ for all $\alpha \in \F_{p^n}$. Moreover, parts (b),(c) of this proposition show that $\mathrm{tr}$ is $\F_p$-linear, and by part (d) that $\mathrm{tr}$ is surjective (onto): $\mathrm{Im}(\mathrm{tr}) = \F_p$.

The rank theorem gives
$$\mathrm{dim}_{\F_p} \mathrm{Im}(\mathrm{tr}) = \mathrm{dim}_{\F_p} \F_{p^n} - \mathrm{dim}_{\F_p} \ker(\mathrm{tr}),$$
thus 
$$\mathrm{dim}_{\F_p} \ker(\mathrm{tr}) = n-1.$$

Consider now
$$
T 
\left\{
\begin{array}{ccl}
\F_{p^n} & \to& \F_{p^n}\\
\beta & \mapsto & \beta - \beta^p
\end{array}
\right.
$$
$T$ is a $\F_p$-linear map: for $a,b \in \F^p$, and $\alpha, \beta \in \F_{p^n}$, using $a^p = a, b^p = b$,
$$T(a \alpha + b \gamma) = a \alpha + b \gamma - (a^p \alpha^p +b^p \beta^p) = a (\alpha -\alpha^p) + b (\beta - \beta^p) = a T(\alpha) + b T(\beta).$$

If $\gamma = T(\beta) = \beta - \beta^p$ is in $\mathrm{Im}(T)$, then
\begin{align*}
\mathrm{tr}(\gamma) &= \mathrm{tr}\beta) - \mathrm{tr}(\beta^p)\\
&=\left(\beta + \beta^p + \beta^{p^2} + \cdots + \beta^{p^{n-1}}\right) - \left(\beta^p + \beta^{p^2} + \beta^{p^3} + \cdots + \beta^{p^{n}}\right)\\
&=\beta - \beta^{p^n}\\
&=0.
\end{align*}

This proves that $$\mathrm{Im}(T) \subset \ker(\mathrm{tr}).$$

Moreover, 
$$\beta \in \ker(T) \iff \beta = \beta^p \iff \beta \in \F_p,$$
so that $\ker(T) = \F_p$.

Using anew the rank theorem on $T$, we obtain
\begin{align*}
\mathrm{dim}_{\F_p} \mathrm{Im}(T) &= \mathrm{dim}_{\F_p} \F_{p^n} - \mathrm{dim}_{\F_p} \ker(T)\\
&=n-1.
\end{align*}

From $\mathrm{Im}(T) \subset \ker(\mathrm{tr})$, where $\mathrm{dim}_{\F_p} \mathrm{Im}(T) = \mathrm{dim}_{\F_p} \ker(\mathrm{tr}) = n-1$, we deduce 
$$\mathrm{Im}(T) = \ker(\mathrm{tr}).$$
To conclude, if $\alpha \in \F_q$ has trace zero, then $\alpha \in \mathrm{Im}(T)$, i.e. $\alpha = \beta - \beta^p$ for some $\beta \in \F_q$.
\end{proof}



\paragraph{Ex. 10.21} {\it Let $\psi$ be a map from $\F_q$ to $\C^*$ such that $\psi(\alpha + \beta) = \psi(\alpha)\psi(\beta)$ for all $\alpha, \beta \in \F_q$. Show that there is a $\gamma \in \F_q$ such that $\psi(x) = \zeta^{\mathrm{tr}(\gamma x)}$ for all $x \in \F_q$, where $\zeta = {2i\pi/p}$.
}

\begin{proof} Here $q = p^n$. 

The map $\psi$ is a group homomorphism, from $(\F_q,+)$ to $(\C^*,\times)$, thus $\psi(0) = 1$, and $\psi(a \alpha) = \psi(\alpha)^a$, where $\alpha \in \F_q$ and $a \in \Z$.

Let $(\omega_1,\ldots,\omega_n)$ be a basis for $\F_{p^n}$ over $\F_p$. For each $k \in \gcro 1,n\dcro$, since the characteristic of $\F_{q}$ is $p$,
$$\psi(\omega_k)^p = \psi( p \omega_i) = \psi(0) = 1.$$
Thus $\psi(\omega_k)$ is a $p$-th root of unity, of the form
$$\psi(\omega_k) = \zeta^{c_k}, \quad c_k \in \{0, \ldots, p-1\}.$$
Since $\zeta^{c_k} = \zeta^{c_k + lp} \ (l\in \Z)$, we can give a sense to $\psi(\omega_k) = \zeta^{c_k} = \zeta^{[c_k]}$, where $[c_k] \in \F_p$ is the class of $c_k$ modulo $p$.

Consider the map
$$
\varphi
\left\{
\begin{array}{ccl}
\F_q & \to & (\F_p)^n\\
\gamma & \mapsto & (\mathrm{tr}(\gamma \omega_1),\ldots, \mathrm{tr}(\gamma \omega_n)).
\end{array}
\right.
$$
We will show that the linear map $\varphi$ is bijective.

If $\gamma \in \ker(\varphi)$, then $\mathrm{tr}(\gamma \omega_1) = \ldots, \mathrm{tr}(\gamma \omega_n) = 0$. If $y$ is any element in $\F_q$, then $y = b_1 \omega_1+\cdots + b_n \omega_n$, where $b_1,\ldots, b_n \in \F_p$. Then  $\mathrm{tr}(\gamma y) = b_1 \mathrm{tr}(\gamma \omega_1) + b_n \mathrm{tr}(\omega_n) = 0$, which gives
$$\forall y \in \F_q,\ \mathrm{tr}(\gamma y ) = 0.$$ 
Reasoning by contradiction suppose that $\gamma \ne 0$. Since $\mathrm{tr}$ maps $\F_q$ onto $\F_p$( Proposition 10.3.1.(d)), there is some $\delta \in \F_q$ such that $\mathrm{tr}(\delta) = 1$. If $y = \delta \gamma^{-1}$, then $0 = \mathrm{tr}(\gamma y) = \mathrm{tr}(\delta) = 1$. This is a contradiction, so $y = 0$, and this proves $\ker(\varphi) = \{0\}$.

Moreover $\mathrm{dim}_{\F_p} (\F_q) = \mathrm{dim}_{\F_p}(\F_p)^n = n$, thus $\varphi$ is a bijection.

Thus there exists $\gamma \in \F_q$ such that $$\mathrm{tr}(\gamma \omega_k) = [c_k],\qquad k=1,\ldots,n.$$

Then, if $x$ is any element in $\F_q$, we can write $x = a_1 \omega_1 + \cdots + a_n \omega_n$, where $a_1,\ldots,a_k \in \F_p$. Since $\psi(\omega_k)$ is a $p$-th root of unity,
\begin{align*}
\psi(x) &= \psi( a_1 \omega_1 + \cdots + a_n \omega_n)\\
&=\psi(\omega_1)^{a_1} \cdots \psi(\omega_n)^{a_n}\\
&=\zeta^{a_1 \mathrm{tr}(\gamma \omega_1) + \cdots + a_n \mathrm{tr}(\gamma\omega_n)}\\
&=\zeta^{\mathrm{tr}(\gamma x)}.
\end{align*}
\end{proof}
If $\psi$ is a group homomorphism from $\F_q$ to $\C^*$, then there is a $\gamma \in \F_q$ such that $\psi(x) = \zeta^{\mathrm{tr}(\gamma x)}$ for all $x \in \F_q$.


\paragraph{Ex. 10.22} {\it If $g_\alpha(\chi)$ is a Gauss sum on $F$, defined in section 3, show that
\begin{enumerate}
\item[(a)] $g_\alpha(\chi) = \overline{\chi(\alpha)} g(\chi)$.
\item[(b)] $g(\chi^{-1}) = g(\overline{\chi}) = \chi(-1) \overline{g(\chi)}.$
\item[(c)] $|g_\alpha(\chi) | = q^{1/2}$.
\item[(d)] $g(\chi) g(\chi^{-1}) = \chi(-1) q$.
\end{enumerate}
}
\begin{proof}
Here $\psi : \F_q \to \C$ is defined by $\psi(\alpha) = \zeta_p^{\mathrm{tr}(\alpha)}$, and the Gauss sum for a character $\chi$ of $\F_q$ by
$$g_\alpha(\chi) = \sum_{t \in \F_q} \chi(t) \psi(\alpha t) = \sum_{t \in \F_q} \chi(t) \zeta_p^{\mathrm{tr}(\alpha t)}.$$

First we generalize Proposition 8.1.2, with the same proof.
If $\chi \ne \varepsilon$, there an $a \in \F_q^*$ such that $\chi(a) \ne 1$. Then, if $T = \sum_{t\in \F_q} \chi(t)$, then
$$\chi(a) T = \sum_{t\in \F_q} \chi(a) \chi(t) = \sum_{t\in \F_q} \chi(at) = \sum_{s \in \F_q} \chi(s) = T.$$
Since $\chi(a) T = T$ and $\chi(a) \ne 1$, it follows that $T=0$. This proves, for a non trivial character $\chi$,
$$g_0(\chi) =\sum_{t\in \F_q} \chi(t) = 0.$$
\begin{enumerate}
\item[(a)] 
If $\alpha \in \F_q^*$,
\begin{align*}
\chi(\alpha) g_\alpha(\chi) &=\sum_{t \in \F_q} \chi(\alpha)\chi(t) \psi(\alpha t)\\
&=\sum_{t \in \F_q} \chi(\alpha t) \psi(\alpha t)\\
&= \sum_{s \in \F_q} \chi(s) \psi( s)\qquad (s =\alpha t)\\
&= g(\chi).
\end{align*}
Since $|\chi(\alpha)| = 1$, $\chi(\alpha)^{-1} = \overline{\chi(\alpha)}$, thus
$$g_\alpha(\chi) = \overline{\chi(\alpha)} g(\chi).$$

\item[(b)] Since $(-1)^2 = 1$, $(\chi(-1))^2 = 1$, thus $\chi(-1) \pm 1$ is real, therefore $\overline{\chi(-1)} = \chi(-1)$. This gives
\begin{align*}
\overline{g(\chi)} &= \sum_{t \in \F_q} \overline{\chi(t)} \zeta_p^{-\mathrm{tr}( t)}\\
&=\sum_{t \in \F_q} \overline{\chi(-1) \chi(-t)}  \zeta_p^{-\mathrm{tr}( t)}\\
&=\chi(-1) \sum_{t \in \F_q} \overline{ \chi(-t)}  \zeta_p^{\mathrm{tr}( -t)}\\
&=\chi(-1) \sum_{s \in \F_q} \overline{ \chi(s)}  \zeta_p^{\mathrm{tr}( s)}\qquad (s = -t)\\
&=\chi(-1) g(\overline{\chi})\\
\end{align*}
We have seen in part (a) that $\chi^{-1} = \overline{\chi}$. This gives
$$g(\chi^{-1}) = g(\overline{\chi}) = \chi(-1) \overline{g(\chi)}.$$

\item[(c)]Here we assume that $\chi \ne \varepsilon$.  By part (a), $|g_\alpha(\chi)| = |g(\chi)|$, so it it sufficient to verify $|g(\chi) | = q^{1/2}$.

We evaluate the sum $S = \sum_{\alpha \in \F_q} g_a(\chi) \overline{g_a(\chi)}$ in two ways.

\begin{enumerate}
\item[$\bullet$]  We have proved in the introduction that $g_0(\chi) = 0$. If $a \in \F_q^*$, then $g_a(\chi) = \chi(a^{-1}) g(\chi)$, and $\overline{g_a(\chi)} = \overline{\chi(a^{-1})} \overline{g(\chi)} = \chi(a) \overline{g(\chi)}$. It follows that
\begin{align*}
S&= \sum_{a \in \F_q^*} \chi(a^{-1}) g(\chi)  \chi(a) \overline{g(\chi)}\\
&= \sum_{a \in \F_q^*} |g(\chi)|^2\\
&= (q-1) |g(\chi)|^2
\end{align*}
\item[$\bullet$] Furthermore $$g_a(\chi) \overline{g_a(\chi)} = \sum_{x \in \F_q} \sum_{y \in \F_q} \chi(x) \overline{\chi(y)} \psi(a(x-y)).$$
Therefore,
\begin{align*}
S &= \sum_{a\in \F_q}  \sum_{x \in \F_q} \sum_{y \in \F_q} \chi(x) \overline{\chi(y)} \psi(a(x-y))\\
&=  \sum_{x \in \F_q} \sum_{y \in \F_q}\chi(x) \overline{\chi(y)} \left(  \sum_{a\in \F_q}\psi(a(x-y))\right)
\end{align*}
By Proposition 10.3.3,
\begin{align*}
 \sum_{a\in \F_q} \psi(a(x-y)) &= q \delta(x,y)
  \end{align*}
Therefore,
 \begin{align*}
 S &=  q \sum_{x \in \F_q} \sum_{y \in \F_q}\chi(x) \overline{\chi(y)}  \delta(x,y) \\
 &= q \sum_{x \in \F_q} \chi(x) \overline{\chi(x)}\\
 \end{align*}
Since $\chi(x) \overline{\chi(x)} = 1$ if $x \ne 0$, et $\chi(x) \overline{\chi(x)} = 0$ if $x = 0$, we obtain
 $$S = q(q-1).$$
\end{enumerate}
The comparison of these two results gives
$$(q-1) |g(\chi)|^2 = (q-1)q,$$
thus $$ |g_\alpha(\chi)| = |g(\chi)| = \sqrt{q}.$$

\item[(d)] Here $\chi \ne \varepsilon$. Then, by parts (b) and (c),
\begin{align*}
g(\chi) g(\chi^{-1}) &= \chi(-1) g(\chi) \overline{g(\chi)}\\
&=\chi(-1) |g(\chi)|^2\\
&= \chi(-1) q.
\end{align*}
\end{enumerate}
\end{proof}

\paragraph{Ex. 10.23} {\it Suppose that $f$ is a function mapping $F$ to $\C$. Define $\hat{f}(s) = (1/q) \sum_t f(t) \overline{\psi(st)}$ and prove that $f(t) = \sum_s \hat{f}(s) \psi(st)$. The last sum is called the finite Fourier series expansion of $f$.
}
\begin{proof} Using the proposition 10.3.3, we obtain, for all $t \in \F_q$,
\begin{align*}
 \sum_{s \in \F_q} \hat{f}(s) \psi(st) &=  \frac{1}{q} \sum_{s \in \F_q} \left(\sum_{u \in \F_q} f(u) \overline{\psi(su)} \right ) \psi(st)\\
 &= \frac{1}{q}  \sum_{u \in \F_q} f(u)  \sum_{s \in \F_q} \psi(s(t-u))\\
 &= \frac{1}{q}  \sum_{u \in \F_q} f(u)  q \delta(t,u)\\
 &= f(t).
\end{align*}

\end{proof}

\paragraph{Ex. 10.24} {\it In Exercise 23 take $f$ to be a non trivial character $\chi$ and show that $\hat{\chi}(s) = (1/q) g_{-s}(\chi)$.
}
\begin{proof} By definition,
\begin{align*}
\hat{\chi}(s) &= \frac{1}{q} \sum_{t\in \F_q} \chi(t) \overline{\psi(st)}\\
&= \frac{1}{q} \sum_{t\in \F_q} \chi(t) \zeta_p(-\mathrm{tr}(st)\\
&= \frac{1}{q} \sum_{t\in \F_q} \chi(t) \zeta_p(\mathrm{tr}((-s)t)\\
&=  \frac{1}{q} \,  g_{-s}(\chi).
\end{align*}

\end{proof}
\end{document}
