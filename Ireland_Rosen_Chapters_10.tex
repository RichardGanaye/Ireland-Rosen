%&LaTeX
\documentclass[11pt,a4paper]{article}
\usepackage[frenchb,english]{babel}
\usepackage[applemac]{inputenc}
\usepackage[OT1]{fontenc}
\usepackage[]{graphicx}
\usepackage{amsmath}
\usepackage{amsfonts}
\usepackage{amsthm}
\usepackage{amssymb}
\usepackage{yfonts}
\usepackage{mathrsfs}
%\input{8bitdefs}

% marges
\topmargin 10pt
\headsep 10pt
\headheight 10pt
\marginparwidth 30pt
\oddsidemargin 40pt
\evensidemargin 40pt
\footskip 30pt
\textheight 670pt
\textwidth 420pt

\def\imp{\Rightarrow}
\def\gcro{\mbox{[\hspace{-.15em}[}}% intervalles d'entiers 
\def\dcro{\mbox{]\hspace{-.15em}]}}

\newcommand{\D}{\mathrm{d}}
\newcommand{\Q}{\mathbb{Q}}
\newcommand{\Z}{\mathbb{Z}}
\newcommand{\N}{\mathbb{N}}
\newcommand{\R}{\mathbb{R}}
\newcommand{\C}{\mathbb{C}}
\newcommand{\F}{\mathbb{F}}
\newcommand{\re}{\,\mathrm{Re}\,}
\newcommand{\ord}{\mathrm{ord}}
\newcommand{\n}{\mathrm{N}}
\newcommand{\legendre}[2]{\genfrac{(}{)}{}{}{#1}{#2}}



\title{Solutions to Ireland, Rosen ``A Classical Introduction to Modern Number Theory''}
\author{Richard Ganaye}

\begin{document}

{ \Large \bf Chapter 10} 

\paragraph{Ex. 10.1}

{\it If $K$ is an infinite field and $f(x_1,x_2,\ldots,x_n)$ is a non-zero polynomial with coefficients in $K$, show that $f$ is not identically zero on $A_n(K)$. (Hint: Imitate the proof of Lemma 1 in Section 2.)
}

\begin{proof}
Assume that $f$ vanishes on all of $A_n(K)$. We have to prove that $f$ is the zero polynomial.

The proof is by induction on $n$. If $n=1$, then $f$ is a polynomial with one variable, which vanishes on $A_1(K) = K$. Since $K$ is infinite, $f$ have more than $d$ roots, where $d = \deg(f)$, thus $f$ is the zero polynomial.

Suppose that we have proved the result for $n-1$ and write
$$f(x_1,\ldots,x_n) = \sum_{i=0}^{s-1} g_i(x_1,\ldots,x_{n-1}) x_n^i,$$
where the $x_i$ are variables, and $g_i$ are polynomials in $x_1,\ldots,x_{n-1}$.

For all $(a_1,\ldots,a_n) \in K^n$, 
$$0 = f(a_1,\ldots,a_n) = \sum_{i=0}^{s-1} g_i(a_1,\ldots,a_{n-1}) a_n^i.$$
From the result for $n=1$, we obtain that the polynomial $ \sum_{i=0}^{s-1} g_i(a_1,\ldots,x_{a-1}) x_n^i$ is null, thus for all $(a_1,\ldots,a_{n-1}) \in K^{n-1}$,
$$ g_i(x_1,\ldots,x_{n-1}) = 0.$$
The induction hypothesis shows that $ g_i(x_1,\ldots,x_{n-1}) = 0$, thus $f(x_1,\ldots,x_n) = 0$.
\end{proof}

\paragraph{Ex. 10.2}
{\it In section 1 it was asserted that $H$, the hyperplane at infinity in $P_n(F)$, has the structure of $P_{n-1}(F)$. Verify this by constructing a one-to-one, onto map from $P_{n-1}(F)$ to $H$.}

\begin{proof}
Note that if one representative $(x_0,\ldots,x_n)$ of a projective point satisfies $x_0 = 0$, then it is the same for all representatives of this point, so we can define
$$\overline{H} = \{[x_0,\ldots,x_n] \in P_n(F) \mid x_0 = 0\},$$
where we write for simplicity $[x_0,\ldots,x_n]$ for $[(x_0,\ldots,x_n)]$.

Consider
$$
\psi
\left\{
\begin{array}{ccl}
\overline{H} & \to &P_{n-1}(F)\\
{[}0,x_1,\ldots,x_n{]}& \mapsto & {[}x_1,\ldots,x_n{]}
\end{array}
\right.
$$
Then $\psi$ is well-defined. Indeed, if $(0,x_1,\ldots,x_n) \sim (0,y_1,\ldots,y_n)$, then there is some $\lambda \in F^*$ such that $(0,y_1,\ldots,y_n) = \lambda (0,x_1,\ldots,x_n)$, thus $(y_1,\ldots,y_n) = \lambda (x_1,\ldots,x_n)$, and $ {[}x_1,\ldots,x_n{]} =  {[}y_1,\ldots,y_n{]}$.

If $ \psi({[}0,x_1,\ldots,x_n{]}) =  \psi({[}0,y_1,\ldots,y_n{]})$, then ${[}(x_1,\ldots,x_n){]} = {[}(y_1,\ldots,y_n){]}$, so ther is some $\lambda \in F^*$ such that $y_i = \lambda x_i, \ i=1,\ldots,n$. Since $0 = \lambda 0$, $(0,y_1,\ldots,y_n) \sim (0,x_1,\ldots,x_n)$, therefore ${[}0,x_1,\ldots,x_n{]} = {[}0,y_1,\ldots,y_n{]}$, so $\psi$ is injective.

Moreover if ${[}x_1,\ldots,x_n{]} $ is any projective point of $P_{n-1}(F)$, then ${[}x_1,\ldots,x_n{]}  =  \psi({[}0,x_1,\ldots,x_n{]})$ so $\psi$ is surjective. 

To conclude, $\psi$ is a bijection.
\end{proof}

\paragraph{Ex. 10.3} {\it Suppose that $F$ has $q$ elements. Use the decomposition of $P_n(F)$ into finite points and points at infinity to give another proof of the formula for the number of points in $P_n(F)$.
}
\begin{proof}

By exercise 2, the bijection $\psi$ shows that $|\overline{H}| = |P_{n-1}(F)|$. Therefore
$$|P_n(F)| = |P_n(F) \setminus \overline{H} | + |\overline{H} | = |A_n(F)| + |P_{n-1}(F)| = q^n +  |P_{n-1}(F)|.$$
Moreover $|P_0(F)| = 1$. Consequently,
$$|P_n(F)|  = |P_0(F)|  + \sum_{k=1}^n (|P_k(F)| - |P_{k-1}(F)| )= 1 + \sum_{k=1}^n q^k = q^n + q^{n-1} + \cdots + q + 1,$$
This gives another proof of the formula for the number of points in $P_n(F)$.
\end{proof}

\paragraph{Ex. 10.4} {\it The hypersurface defined by a homogeneous polynomial of degree 1, $a_0x_0+a_1x_1+\cdots +a_nx_n $ is called a hyperplane. Show that any hyperplane in $P_n(F)$ has the same number of elements as $P_{n-1}(F)$.
}
\begin{proof}
Define the hyperplane $\overline{K}$ by
$$\overline{K} = \{[x_0,\ldots,x_n] \in P_n(F) \mid a_0 x_0 + \cdots +a_n x_n = 0\},$$
where $(a_0,\ldots,a_n) \ne (0,\ldots,0)$ (if $(a_0,\ldots,a_n) \ne (0,\ldots,0)$, then $\overline{K} = P_n(F)$ is not a hyperplane). Note that, if $(x_0,\ldots,x_n) \sim (y_0,\ldots,y_n)$, there is $\lambda \in F^*$ such that $y_i = \lambda x_i,\ i=0,\ldots,n$, thus $a_0 x_0 + \cdots +a_n x_n \iff 0 = a_0 y_0 + \cdots +a_n y_n = 0$, so that the condition does'nt depends of the choice of the representative of the projective point.

Since $(a_0,\ldots,a_n) \ne (0,\ldots,0)$, suppose, without loss of generality, that $a_0 \ne 0$. Consider
$$
\chi
\left\{
\begin{array}{ccl}
\overline{K} & \to & P_{n-1}(F)\\
{[}x_0,\ldots,x_n{]} & \mapsto & [x_1,\ldots, x_n]
\end{array}
\right.
$$
Then $\chi$ is well defined. Indeed, if $(x_0,\ldots,x_n) \sim (y_0,\ldots,y_n)$, there is some $\lambda \in F^*$ such that $(y_0,\ldots, y_n) = \lambda (x_0,\ldots,x_n)$. In particular, $(y_1,\ldots, y_n) = \lambda (x_1,\ldots,x_n)$, thus $[x_1,\ldots,x_n] = [y_1,\ldots,y_n]$.

If $\chi([x_0,\ldots,x_n]) =  \chi([y_0,\ldots,y_n])$, where $[x_0,\ldots, x_n]$ and $[y_0,\ldots, y_n]$ are in $\overline{K}$, then $[x_1,\ldots, x_n] = [y_1,\ldots, y_n]$, thus there is $\lambda \in F^*$ such that $(y_1,\ldots, y_n) = \lambda (x_1,\ldots, x_n)$. Since $a_0 \ne 0$,
$$y_0 = -\frac{1}{a_0}(a_1y_1 + \cdots + a_ny_n) = - \lambda \frac{1}{a_0}(a_1x_1 + \cdots + a_nx_n) = \lambda x_0,$$
therefore $[x_0,\ldots,x_n] = [y_0,\ldots,y_n]$. So $\varphi$ is injective.

At last, let $[x_1,\ldots, x_n]$ be any point of $P_{n-1}(F)$. Define $x_0 = - \frac{1}{a_0}(a_1x_1 + \cdots + a_nx_n)$. Then $a_0x_0 + \cdots +a_nx_n = 0$, so that $[x_0,\ldots,x_n] \in \overline{K}$, and $\chi([x_0,\ldots,x_n]) = [x_1,\ldots,x_n]$. This proves that $\chi$ is surjective.

To conclude, $\chi$ is a bijection, therefore $|\overline{K}| = |P_{n-1}(F)| = q^{n-1} + \cdots+ q+1$.
\end{proof}

\paragraph{Ex. 10.5} {\it Let $f(x_0,x_1,x_2)$ be a homogeneous polynomial of degree $n$ in $F(x_0,x_1,x_2]$. Suppose that not every zero of $a_0x_0+a_1x_1+a_2x_2$ is a zero of $f$. Prove that there are at most $n$ common zeros of $f$ and $a_0x_0+a_1x_1+a_2x_2$ in $P_2(F)$. In more geometric language this says that a curve of degree $n$ and a line have at most $n$ points in common unless the line is contained in the curve.
}
\begin{proof}
Let $\mathscr{C}$ be the curve with equation $f(x_0,x_1,x_2) = 0$.

Since $a_0 x_0+a_1x_1+a_2x_2 =0$ is the equation of a line $l$, $(a_0,a_1,a_2) \ne 0$, so that we can suppose without loss of generality that $a_0 \ne 0$. Then 
\begin{align*}
[u_0,u_1,u_2] \in &l \iff a_0 u_0+a_1u_1+a_2u_2 =0\\
& \iff u_0 = -\frac{a_1}{a_0} u_1 - \frac{a_2}{a_0} u_2\\
&\iff u_0 = \alpha u_1 + \beta u_2,
\end{align*}
where $\alpha = -\frac{a_1}{a_0},\ \beta = - \frac{a_2}{a_0}.$
Therefore
\begin{align*}
[u_0,u_1,u_2] \in \mathscr{C} \cap l &\iff 
\left\{
\begin{array}{ll}
  a_0 u_0+a_1u_1+a_2u_2 &=0,\\
  f(u_0,u_1,u_2) &= 0,
\end{array}
\right.\\
&\iff 
\left\{
\begin{array}{ll}
  u_0 = \alpha u_1+\beta u_2,\\
  f\left(\alpha u_1+ \beta u_2,u_1,u_2\right) &= 0.
\end{array}
\right.
\end{align*}\\
Let $[u_0,u_1,u_2] \in  \mathscr{C} \cap l $. 

We show that $u_1 \ne 0$. If $u_1 = 0$, then $u_0 = \beta u_2$, therefore $[u_0,u_1,u_2] = [\beta u_2, 0, u_2] = [\beta, 0 ,1]$, and $f\left (\beta u_2, 0,u_2\right) = 0$. Therefore $p = [\beta,0,1] \in \mathscr{C} \cap l$.

Since $[1,0,0]$ and $[\beta,0,1]$ are two distinct points of $l$, an equation of $l$ is
$$
\begin{vmatrix}
1 & 0 & 0\\
\beta & 0 & 1\\
x_0 & x_1 & x_2
\end{vmatrix}
= -x_1,
$$
thus an equation of $l$ is given by $x_1$, therefore no equation $a_0 x_0 + a_1 x_1 + a_2 x_2$ of $l$ satisfies $a_0 \ne 0$, and this is in contradiction with $a_0 \ne 0$. We have proved $u_1 \ne 0$.



Since $f$ is homogeneous of degree $n$,
$$0 = u_1^n f\left(\alpha + \beta \frac{u_2}{u_1},1,\frac{u_2}{u_1}\right),$$
and using $u_1 \ne 0$,
$$0 =  f\left(\alpha + \beta \frac{u_2}{u_1},1,\frac{u_2}{u_1}\right).$$
\end{proof}
Consider the formal polynomial $P(x) = f\left(\alpha + \beta x,1,x\right) \in F[x].$

Then $\deg(P) \leq n$. If $P \ne 0$, then $P$ has at most $n$ roots $\lambda_1,\ldots,\lambda_k$, where $k\leq n$. In this case, $u_2 = \lambda_i u_1$ and $u_0  =\alpha u_1 + \beta u_2 = u_1\left(1 +\alpha \lambda_i\right)$, therefore
$$[u_0,u_1,u_2]  =\left [1 +\alpha \lambda_i, 1 , \lambda_i \right],\ 1\leq i \leq k,$$ 
so that $\mathscr{C}$ and $l$ have at most $n$ points in common.

Therefore, if $|\mathscr {C} \cap l| >n$, then $P =  f\left(\alpha + \beta x,1,x\right) = 0$.

Similarly, by exchanging the roles of $u_1,u_2$, if $|\mathscr {C} \cap l| >n$, then $u_2 \ne 0$, and
$$0 = f\left (\alpha \frac{u_1}{u_2} + \beta, u\frac{u_1}{u_2}, 1\right ),$$
so that the same reasoning gives $Q(x) = f(\alpha x + \beta, x ,1) = 0$.

 Let $[v_0,v_1,v_2]$ be any point on $l$. 
 
 If $v_1 \ne 0$,
$$f(v_0,v_1,v_2) = f(\alpha v_1 + \beta v_2, v_1,v_2) = v_1^n f\left(\alpha + \beta \frac{v_2}{v_1}, 1 ,\frac{v_2}{v_1}\right) = v_1^n P\left (\frac{v_2}{v_1} \right) = 0.$$

If $v_1 = 0$, then $[v_0,v_1,v_2] = [\beta, 0,1] = p$, thus
$$f(\beta,0,1) = Q(0) =0.$$
This proves that $l \subset \mathscr{C}$.

To conclude, if $l \not \subset \mathscr {C}$, then $|l \cap \mathscr{C}| \leq n$ : a curve of degree $n$ and a line have at most $n$ points in common unless the line is contained in the curve.

\paragraph{Ex. 10.6} {\it Let $F$ be a field with $q$ elements. Let $M_n(F)$ be the set of $n \times n$ matrices with coefficients in $F$. Let $\mathrm{SL}_n(F)$ be the subset of those matrices with determinant equal to one. Show that $\mathrm{SL}_n(F)$ can be considered as a hypersurface in $A^{n^2}(F)$. Find a formula for the number of points on this hypersurface. [Answer:$(q-1)^{-1}(q^n-1)(q^n-q)\cdots(q^n-q^{n-1})$.]
}

\begin{proof}
If $M = (a_{i,j})_{1\leq i \leq n, 1\leq j \leq n} \in M_n(F)$, 
$$M \in \mathrm{SL}_n(F) \iff \sum_{\sigma \in S_n} \mathrm{sgn}(\sigma) a_{\sigma(1)1}\cdots a_{\sigma(n)n}.$$
if $f(x_{1,1},\ldots,x_{n,n}) = \sum\limits_{\sigma \in S_n} \mathrm{sgn}(\sigma) x_{\sigma(1)1}\cdots x_{\sigma(n)n}$, then  $M \in \mathrm{SL}_n(F)$ if and only if ${f(a_{1,1}, \ldots,a_{n,n}) = 0}$, where $f$ is a non zero polynomial, since it contains the non zero term $x_{1,1}\cdots x_{n,n}$. Therefore $\mathrm{SL}_n(F)$ is an hypersurface of $M_n(F)$.

Since a matrix $M \in M_n(F)$ is inversible if and only if its columns $(C1,\ldots,C_n)$ is a basis of $F^n$, the number of matrices in $\mathrm{GL}_n(F)$ is $$(q^n-1)(q^n-q)\cdots(q^n-q^{n-1}).$$Indeed we choose $C_1$ between $(q^n-1)$ non zero scalars, then we choose $C_2$ between the $q^n-q$ vectors $v  \not \in \langle C_1 \rangle$. If $C_1,\ldots,C_k$ are chosen, we take $C_{k+1}$ between the $q^n -q^k$ vectors $v \not \in \langle C_1,\ldots,C_k \rangle$. At last, we choose $C_n \not \in \langle C_1,\ldots C_{n-1}\rangle$. This gives
$$|\mathrm{GL}_n(F)| = (q^n-1)(q^n-q)\cdots(q^n-q^{n-1}).$$
Moreover, $\mathrm{SL}_n(F)$ is the kernel of the group homomorphism
$$
\left\{
\begin{array}{ccl}
\mathrm{GL}_n(F) & \to & F^*\\
M & \mapsto &\det(M).
\end{array}
\right.
$$
Therefore $F^* \simeq \mathrm{GL}_n(F)/\mathrm{SL}_n(F)$. This gives
$$|\mathrm{SL}_n(F)| = |\mathrm{GL}_n(F)|/ |F^*| = (q-1)^{-1}(q^n-1)(q^n-q)\cdots(q^n-q^{n-1}).$$
\end{proof}

\paragraph{Ex. 10.7} {\it Let $f \in F[x_0,\ldots,x_n]$. One can define the partial derivatives $\partial f/\partial x_0, \ldots,\partial f/ \partial x_n$ in a formal way. Suppose that $f$ is homogeneous of degree $m$. Prove that $\sum_{i=0}^n x_i (\partial f/ \partial x_i) = mf$. This result is due to Euler. (Hint: Do it first for the case that $f$ is a monomial.)
}
\begin{proof}
For the case that $f = x_1^{a_1}\cdots x_n^{a_n}$ is a monomial, where $a_1+ \ldots + a_n = m = \deg(f)$, then
$$\frac{\partial f}{\partial x_i} = a_i x_1^{a_1}\cdots x_i^{a_i - 1}\cdots x_n^{a_n}, \qquad i=1,\ldots,n.$$
Therefore $x_i \partial f/ \partial x_i = a_i f$, and
$$\sum_{i=1}^n x_i \frac{\partial f}{\partial x_i}  =\left (\sum_{i=1}^n a_i \right ) f = m f.$$
Since the maps $f \mapsto \sum_{i=1}^n x_i \frac{\partial f}{\partial x_i} $ and $f \mapsto mf$ are $FG-$ linear, and since every homogeneous polynomial $f$ is a linear combination of monomial with degree $m$, the relation is true for all such polynomials.

To conclude, every homogeneous polynomial $f \in F[x_0,\ldots,x_n]$ of degree $m$ satisfies
$$\sum_{i=1}^n x_i \frac{\partial f}{\partial x_i}  = m f.$$
\end{proof}

\paragraph{Ex. 10.8} {\it (continuation) If $f$ is homogeneous, a point $\overline{a}$ on the hypersurface defined by $f$ is said singular if it is simultaneously a zero of all the partial derivatives of $f$. If the degree of $f$ is prime to the characteristic, show that a common zero of all the partial derivatives of $f$ is automatically a zero of $f$.
}
\begin{proof}
If $\frac{\partial f}{\partial x_i} (\overline{a}) = 0$ for all $i = 1,\ldots,n$, then $mf(\overline{a}) = \sum\limits_{i=1}^n x_i \frac{\partial f}{\partial x_i} (\overline{a}) = 0$. Since $m = \deg(f)$ is prime with the characteristic, then $m$ is non zero in the field $F$, thus $f(\overline{a}) = 0$.
\end{proof}

\paragraph{Ex. 10.9} {\it If $m$ is prime to the characteristic of $F$, show that the hypersurface defined by $a_0x_0^m+a_1x_1^m+ \cdots + a_n x_n^m$ has no singular points.
}

\medskip

Note: The sentence is not true if some coefficient $a_i$ is zero. To give an counterexample, the projective curve given by $f(x_0,x_1,x_2) = x_1^2 -x_2^2$ is the union of two lines, and the intersection point $a =[1,0,0]$ of these two lines is singular : $\partial f/ \partial x_0(a) = \partial f/ \partial x_1(a) = \partial f/ \partial x_2(a) = 0$. We must assume that $a_i \ne 0$ for every index $i$ (see the hint p. 371).


\begin{proof} Let $V$ be the projective hypersurface defined by $f(x_0,\ldots,x_n) = a_0x_0^m+a_1x_1^m+ \cdots + a_n x_n^m$. 

If $m=1$, $V$ is an hyperplane, without singularity since $\frac{\partial f}{\partial x_i} (a) = a_i \ne 0$ for some index $i$.

We assume now that $m>1$. If $a = [u_0,\dots,u_n] \in V$ is a singular point, $$\frac{\partial f}{\partial x_i}(a) = ma_i u_i^{m-1} = 0\qquad (i=1,\ldots,n).$$ 

Since $m$ is prime with the characteristic, $m \ne 0$ in $F$, and $a_i \ne 0$, thus $u_i = 0$ for all indices $i$. Then $[u_0,\ldots,u_n]$ is not a projective point. This prove that $V$ has no singular point.
\end{proof}


\paragraph{Ex. 10.10} {\it A point on an affine hypersurface is said to be singular if the corresponding point on the projective closure is singular. Show that this is equivalent to the following definition. Let $f \in F[x_1,x_2,\ldots,x_n]$, not necessarily homogeneous, and $a \in H_f(F)$. Then $a$ is singular if it is a common zero of $\partial f/\partial x_i$ for $i=1,2,\ldots,n$.
}

\begin{proof}
Let $H_f(F)$ an affine hypersurface defined by $f(x_1,\ldots,x_n)$, with $\deg(f) = d$, and $a = (u_1,\ldots,u_n) \in F$.
\begin{enumerate} 
\item[$\bullet$] Suppose that the corresponding point $\overline{a} = [1,u_1,\ldots,u_n] \in \overline{F}$ is singular, and let
$$\overline{f}(y_0,\ldots,y_n) = y_0^d\, f\left(\frac{y_1}{y_0},\ldots,\frac{y_i}{y_0},\ldots,\frac{y_n}{y_0}\right)$$
be the homogeneous polynomial defining $\overline{F}$. Then the chain rule gives
$$\frac{\partial \overline{f}}{\partial y_i}(x_0,\ldots,x_n) = x_0^{d-1} \frac{\partial f}{\partial x_i}\left(\frac{x_1}{x_0},\ldots,\frac{x_i}{x_0},\ldots,\frac{x_n}{x_0}\right).$$
Since $\overline{a}$ is singular,
$$0 =  \frac{\partial \overline{f}}{\partial y_i}(\overline{a}) = \frac{\partial \overline{f}}{\partial y_i}(1,u_1,\ldots,u_n) = \frac{\partial f}{\partial x_i} (u_1,\ldots,u_n) =\frac{\partial f}{\partial x_i}(a).$$
This proves that $a$ is a common zero of $\partial f/\partial x_i$ for $i=1,2,\ldots,n$
\item[$\bullet$] Conversely, suppose that $\partial f/\partial x_i (a) = 0$ for $i=1,\ldots,n$. Then 
$$\frac{\partial \overline{f}}{\partial y_i}(\overline{a}) = \frac{\partial \overline{f}}{\partial y_i}(1,u_1,\ldots,u_n) =\frac{\partial f}{\partial x_i} (u_1,\ldots,u_n) =0,$$
which proves that $\overline{a}$ is singular.
\end{enumerate}
\end{proof}

\paragraph{Ex. 10.11} {\it Show that the origin is a singular point on the curve defined by $y^2 - x^3 = 0$.
}
\begin{proof}
If $f(x,y) = y^2 - x^3$, then
$$\frac{\partial f}{\partial x}  = 3x^2,\qquad \frac{\partial f}{\partial y} = 2y,$$
thus $\partial f/\partial x (0,0) = \partial f/\partial y (0,0) = 0$. This proves that the origin is a singular point for the curve defined by $f$.
\end{proof}

\paragraph{Ex. 10.12} {\it Show that the affine curve defined by $x^2 + y^2 + x^2 y^2 = 0$ has two points at infinity and that both are singular.
}
\begin{proof}
The homogeneous equation of this curve is
$$\overline{f} (t,x, y) =  x^2 t^2 + y^2 t^2 + x^2 y^2, $$
where $t=0$ is the equation of the line at infinity.

The point $\overline{a} = [u_0, u_1,u_2]$ is a point at infinity if $u_0 = 0$. This gives the equation $$\overline{f} (0,u_1,u_2) = u_1^2u_2^2 = 0,$$
 where $u_1 \ne 0$ or $u_2\ne 0$ (otherwise $u_0 = u_1 = u_2 = 0$, and $[u_0, u_1,u_2]$ is not a projective point).
 
 If $u_1 \ne 0$, then $u_2 = 0$, and if $u_2 \ne 0$, then $u_1 = 0$.
 
 Therefore $\overline{a} = [0, u_1,0] = [0,1,0]$, or $\overline{a} = [0,0,u_2] = [0,0,1]$.
 
 $p = [0,1,0]$ and $q = [0,0,1]$ are the two points at infinity of the curve.
 
 $$\frac{\partial \overline{f}}{\partial t} =  2 t (x^2 + y^2), \qquad \frac{\partial \overline{f}}{\partial x} = 2x (t^2 +y^2),\qquad  \frac{\partial \overline{f}}{\partial y} = 2y (t^2 + x^2).$$
Therefore
$$\frac{\partial \overline{f}}{\partial t}(0,1,0) = \frac{\partial \overline{f}}{\partial x}(0,1,0)=\frac{\partial \overline{f}}{\partial y}(0,1,0) = 0,$$
and
$$\frac{\partial \overline{f}}{\partial t}(0,0,1) = \frac{\partial \overline{f}}{\partial x}(0,0,1)=\frac{\partial \overline{f}}{\partial y}(0,0,1) = 0.$$
This proves that the two points at infinity $p,q$ are singular.
\end{proof}

\paragraph{Ex. 10.13} {\it Suppose that the characteristic of $F$ is not $2$, and consider the curve defined by $ax^2+bxy+cy^2 = 1$, where $a,b,c \in F^*$. If $b^2 -4ac \in F^2$, show that there are one or two points at infinity depending on whether $b^2 -4ac$ is zero. If $b^2 -4ac = 0$, show that the point at infinity is singular.
}
\begin{proof}
Let $\mathscr{C}$ be the curve defined by $f(x,y) = ax^2+bxy+cy^2 -1$. The homogeneous equation of the projective closure $\overline{\mathscr{C}}$ of $\mathscr{C}$ is
$$\overline{f}(t,x,y) = ax^2+bxy+cy^2 -t^2.$$

The points $[0,u,v]$ at infinity are given by the equation
$$au^2 + buv + cv^2 = 0.$$

Assume that $\Delta = b^2 - 4ac = \delta^2 \in F^2$. Since $a \ne 0$,
\begin{align*}
au^2 + buv + c v^2 &= a \left[ \left(u+ \frac{b}{2a} v\right)^2 - \frac{b^2 - 4ac}{4a^2} v^2\right]\\
&= a \left[ \left(u+ \frac{b}{2a} v\right)^2 - \frac{\delta^2}{4a^2} v^2\right]\\
&= a \left( u - \frac{-b+\delta}{2a} v \right)  \left( u - \frac{-b-\delta}{2a} v \right) \\
&= a( u - \alpha v)(u - \beta v),
\end{align*}
where $\alpha = \frac{-b+\delta}{2a}, \beta = \frac{-b-\delta}{2a}$ are the two roots of $aX^2 + b X +c$.

Therefore the points at infinity are $p = [0,\alpha, 1]$ and $q = [0,\beta, 1]$.

\begin{enumerate}
\item[$\bullet$] If $b^2 - 4ac \ne 0$ (hyperbolic case), then $\alpha \ne \beta$ and $p \ne q$, so that $\mathscr{C}$ has two points at infinity.

\item[$\bullet$] If $b^2 - 4ac = 0$ (parabolic case), then  $\alpha = \beta$, and $\mathscr{C}$ has one  (double) point  at infinity $r = [0,\alpha, 1,]$, where $\alpha = -\frac{b}{2a}$ is the root of multiplicity $2$ of $aX^2 + b X +c$. Thus $r=[0, -b, 2a]$.

Since
$$\frac{\partial \overline{f}}{\partial t} (t,x,y)= -2t, \qquad \frac{\partial \overline{f}}{\partial x}(t,x,y) = 2ax + by,\qquad  \frac{\partial \overline{f}}{\partial y}(t,x,y) =  bx + 2c y,$$
then
$$\frac{\partial \overline{f}}{\partial t}(0,-b,2a) = 0, \qquad \frac{\partial \overline{f}}{\partial x}(0,-b,2a) = -2ab + 2ab =  0,\qquad  \frac{\partial \overline{f}}{\partial y}(0,-b,2a) = -(b^2 - 4ac) =  0.$$
This shows that the point at infinity $r = [0,-b,2a]$ is singular.
\end{enumerate}
\end{proof}

\paragraph{Ex. 10.14} {\it Consider the curve defined by $y^2 = x^3 + ax +b$. Show that it has no singular points (finite or infinite) if $4a^3 + 27 b^2 \ne 0$.
}

\begin{proof} Let $\mathscr{C}$ be  the curve defined by $f(x,y) = y^2 - x^3  - ax - b$. The homogeneous equation of the projective closure $\overline{\mathscr{C}}$ of $\mathscr{C}$ is
$$\overline{f}(t,x,y) = y^2 t - x^3 -a xt^2 - bt^3.$$
The only point at infinity is given by $t = 0, -x^3 = 0$, thus is the point $p = [0,0,1]$.
Since
$$\frac{\partial \overline{f}}{\partial t} (t,x,y)= y^2 -2axt -3bt^2, \qquad \frac{\partial \overline{f}}{\partial x}(t,x,y) = -3x^2 -at^2,\qquad  \frac{\partial \overline{f}}{\partial y}(t,x,y) = 2y t,$$
then $\frac{\partial \overline{f}}{\partial t} (0,0,1) = 1$, thus the point at infinity $p$ is not singular.

For some other points $a = (u,v)$ on $\overline{C}$ not at infinity, it is sufficient by Exercise 10 to verify $(\partial f/\partial x (u,v), \partial f/\partial y (u,v)) \ne (0,0)$. Since
$$\frac{\partial f}{\partial x}(u,v) = -3u^2 -a,\qquad \frac{\partial f}{\partial y}(u,v) = 2v,$$
$a$ is singular if
$$
\left \{
\begin{array}{ll}
v^2 &= u^3  + au +b,\\
-3u^2 - a &= 0,\\
2v &= 0.
\end{array}
\right.
$$
Therefore
$$
\left \{
\begin{array}{ll}
0&= u^3  + au +b,\\
-\frac{a}{3}&= u^2,\\
\end{array}
\right.
$$

If $a = 0$, then $u=v=0$, thus $b=0$, so that $4a^3 + 27b^2 = 0$.

If $a \ne 0$,  we eliminate $u$ between these two equations to obtain, 
$$0 = u(u^2 + a) + b = \frac{2}{3}a u +b,$$
thus
$u = -\frac{3b}{2a}$, and $u^2 = \frac{9b^2}{4a^2} = -\frac{a}{3}$, which gives $4a^3 + 27b^2 = 0$.
To conclude, if $4a^4 + 27b^2 \ne 0$, then the curve defined by $y^2 = x^3 + ax +b$ has no singular points, finite or infinite.
\end{proof}
\end{document}