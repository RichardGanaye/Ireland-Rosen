%&LaTeX
\documentclass[11pt,a4paper]{article}
\usepackage[frenchb,english]{babel}
\usepackage[applemac]{inputenc}
\usepackage[OT1]{fontenc}
\usepackage[]{graphicx}
\usepackage{amsmath}
\usepackage{amsfonts}
\usepackage{amsthm}
\usepackage{amssymb}
\usepackage{yfonts}
%\input{8bitdefs}

% marges
\topmargin 10pt
\headsep 10pt
\headheight 10pt
\marginparwidth 30pt
\oddsidemargin 40pt
\evensidemargin 40pt
\footskip 30pt
\textheight 670pt
\textwidth 420pt

\def\imp{\Rightarrow}
\def\gcro{\mbox{[\hspace{-.15em}[}}% intervalles d'entiers 
\def\dcro{\mbox{]\hspace{-.15em}]}}

\newcommand{\D}{\mathrm{d}}
\newcommand{\Q}{\mathbb{Q}}
\newcommand{\Z}{\mathbb{Z}}
\newcommand{\N}{\mathbb{N}}
\newcommand{\R}{\mathbb{R}}
\newcommand{\C}{\mathbb{C}}
\newcommand{\F}{\mathbb{F}}
\newcommand{\ord}{\mathrm{ord}}
\newcommand{\legendre}[2]{\genfrac{(}{)}{}{}{#1}{#2}}



\title{Solutions to Ireland, Rosen ``A Classical Introduction to Modern Number Theory''}
\author{Richard Ganaye}

\begin{document}

\maketitle

{\large \bf Chapter 7}

\paragraph{Ex. 7.1}

{\it Use the method of Theorem 1 to show that a finite subgroup of the
multiplicative group of a field is cyclic.
}

\bigskip

A solution is already given in Ex. 4.15

\paragraph{Ex. 7.2}

{\it Find the finite subgroups of $\R^*$ and $\C^*$ and show directly that they are cyclic.
}

\begin{proof}
If $G$ is a finite subgroup of $\R$ or $\C$, and $n = \vert G \vert$, then from Lagrange's Theorem, $x^n = 1$ for all $x \in G$.

$\bullet$ If  $G$ is a finite subgroup of $\R^*$, then the solutions of $x^n=1$ are in $\{-1,1\}$, so $\{1\} \subset G\subset \{-1,1\}$ : 
$G = \{1\}$ or $G = \{-1,1\}$, both cyclic.

$\bullet$ If  $G$ is a finite subgroup of $\C^*$, then $G \subset \mathbb{U}_n = \{e^{2ik\pi/n}\ \vert \ 0 \leq k \leq n-1\}$. As $\vert G \vert = \vert \mathbb{U}_n \vert = n$, then  $G = \mathbb{U}_n \simeq \Z/n\Z$ is cyclic.
\end{proof}

\paragraph{Ex. 7.3}

{\it  Let $F$ a field with $q$ elements and suppose that $q\equiv 1 \pmod n$. Show that for $\alpha \in \F^*$, the equation $x^n = \alpha$ has either no solutions or $n$ solutions.
}

\begin{proof} This is a particular case of Prop. 7.1.2., where $d = n \wedge (q-1) = n$ : the equation $x^n =\alpha$ has solutions iff $\alpha^{(q-1)/n} = 1$. In this case, there are exactly $d = n$ solutions.

We give  here a direct proof.

Let $g$ a generator of $F^*$.  Write $x = g^y, \alpha = g^a$. Then
$$x^n = \alpha \iff g^{ny} = g^a\iff q-1 \mid ny -a.$$

Suppose that there exists $x\in F$ such that $x^n = \alpha$. Then there exists $y\in \Z$ such that $q-1 \mid ny -a$. Since $n \mid q-1$, then  $n \mid a$.
$$q-1 \mid ny -a \iff \frac{q-1}{n} \mid y - \frac{a}{n}\iff y= \frac{a}{n} + k \frac{q-1}{n}, k\in \mathbb{Z}.$$

As  $\frac{a}{n} + (k+n) \frac{q-1}{n} = \frac{a}{n} + k \frac{q-1}{n}, k\in \mathbb{Z}$, the values $k=0,1\cdots,n-1 $ are sufficient :
$$x^n=\alpha \iff y = \frac{a}{n} + k \frac{q-1}{n}, k\in \{0,1,\cdots,n-1\}.$$

Moreover, these solutions are all distinct : if $k,l\in  \{0,1,\cdots,n-1\}$,
\begin{align*}
g^{\frac{a}{n} + k \frac{q-1}{n}} = g^{\frac{a}{n} + l \frac{q-1}{n}} & \Rightarrow g^{(k-l) \frac{q-1}{n}} = 1\\
&\Rightarrow q-1 \mid (k-l) \frac{q-1}{n}\\
& \Rightarrow n \mid k-l \\
&\Rightarrow k\equiv l \ [n] \Rightarrow k=l.
\end{align*}

Conclusion : if  $F$ is a field with $q$ elements and $n \mid q-1$, the equation $x^n = \alpha$ has either no solutions or $n$ solutions in $F$.

Remark : $$\exists x \in F^*, x^n = \alpha \iff n \mid a \iff \alpha^{(q-1)/n} = 1.$$

Indeed, if $x^n = \alpha$ has a solution, we have proved that $n\mid a$, thus $\alpha^{(q-1)/n} = (g^{a/n})^{q-1} = 1$.

Reciprocally, if $\alpha^{(q-1)/n} = 1$, $g^{a.(q-1)/ n }=1$, thus $q-1 \mid a(q-1)/n$, so $ n \mid a$ :  $\alpha = x^n$, with $x = g^{n/a}$.
\end{proof}

\paragraph{Ex. 7.4}

{\it (continuation) Show that the set of $\alpha \in F^*$ such that $x^n = \alpha$ is solvable is a subgroup with $(q-1)/n$ elements.
}

\begin{proof}
Here $n \mid q-1$.

Let $\varphi = F^* \to F^*$ the application defined by $\varphi(x) = x^n$. $\varphi$ is a morphism of groups, and $\ker \varphi$ is the set of solutions of $x^n = 1$.
As $n \mid q-1$, $x^n = 1$ has exactly $n$ solutions (Prop 7.1.1, Corollary2, or Ex 7.3 with $\alpha = 1$). So $\vert \ker \varphi \vert = n$.

Thus $\mathrm{Im} \varphi \simeq F^*/\ker \varphi$ is a subgroup with cardinality $\vert F^* \vert / \vert \ker \varphi \vert = (q-1)/n$, and $\mathrm{Im} \varphi$ is the set of $\alpha$ such that $x^n = \alpha$ is solvable.

Conclusion : the set of $\alpha \in F^*$ such that $x^n = \alpha$ is solvable is a subgroup with $(q-1)/n$ elements.

\end{proof}

\paragraph{Ex. 7.5}

{\it (continuation) Let $K$ be a field containing $F$ such that $[K:F] = n$. For all $\alpha \in F^*$, show that the equation $x^n = \alpha$ has $n$ solutions in $K$. [Hint: Show that $q^n-1$ is divisible by $n(q-1)$ and use the fact that $\alpha^{q-1} = 1$.]

}

\begin{proof}
As $q\equiv 1 \ [n], \frac{q^n-1}{q-1} = 1+q+\cdots+q^{n-1} \equiv 0 \ [n]$, then $n \mid \frac{q^n-1}{q-1} : $
$$q^n-1 = k n (q-1), k \in \mathbb{N}.$$

Since $\alpha \in F^*$, $\alpha^{q-1} = 1$, so $$\alpha^{(q^n-1)/n} = (\alpha^{q-1})^k = 1.$$

As $\vert K \vert = q^n$, Prop. 7.1.2 (or the final remark in Ex.7.3) show that there exists $x \in K^*$ such that $x^n = \alpha$. Then, from Ex.7.3, we know that there exist $n$ solutions in $K$.

Conclusion : if $[K:F] = n$, the equation $x^n = \alpha$ has $n$ solutions in $K$.

\end{proof}

\paragraph{Ex. 7.6}

{\it Let $K \supset F$ be finite fields with $[K:F] = 3$. Show that if $\alpha \in F$ is not a square in $F$, it is not a square in $K$.

}

\begin{proof}
Let $q = \vert F \vert$. Then $\vert K \vert = q^3.$

If the characteristic of $F$ is 2, $q = 2^k$, and for all $x \in F$, $x = x^q = \left (x^{2^{k-1}}\right)^2$. So all elements in $F$ or $K$ are squares.
We can now suppose that the characteristic of $F$ is not 2, and consequently $1 \neq -1$ in $F$.

As $\alpha$ is not a square in $F$, $\alpha^{(q-1)/2} \ne 1$ (Prop. 7.1.2). From  $0 = \alpha^{q-1} - 1 = (\alpha^{(q-1)/2}-1)(\alpha^{(q-1)/2}+1)$, we deduce $\alpha^{(q-1)/2} = -1$. Then 
$$\alpha^{(q^3-1)/2} = (\alpha^{(q-1)/2})^{q^2+q+1} = (-1)^{q^2+q+1} = -1,$$
since $q^2+q+1$ is always odd.

$\alpha^{(q^3-1)/2} \neq 1$ : this implies (Prop. 7.1.2) that $\alpha$ is not a square in $K$.
\end{proof}

\paragraph{Ex. 7.7}

{\it  Generalize Exercise 6 by showing that if $\alpha$ is not a square in $F$, it is not a square in any extension of odd degree and is a square in every extension of even degree.

}

\begin{proof}
Write $q = [K:F]$, and $q = \mathrm{Card}\ F$.

As $\alpha$ is not a square in $F$, the characteristic of $F$ is not 2 (see Ex.7.6), and $\alpha^{(q-1)/2} \ne 1$. Since $\alpha^{q-1} = 1$, $\alpha^{(q-1)/2}= -1$.
$$\alpha^{(q^n-1)/2} =( \alpha^{(q-1)/2})^{1+q+\cdots+q^{n-1}} = (-1)^{1+q+\cdots+q^{n-1}} .$$

$\bullet$ If $n$ is odd, $1 + q +\cdots+q^{n-1} \equiv 1 \pmod 2$, thus $\alpha^{(q^n-1)/2} = -1\ne 1$, and consequently $\alpha$ is not a square in $K$.

$\bullet$ If $n$ is even, as $q$ is odd ($\mathrm{char}(F) \ne 2$), $1 + q +\cdots+q^{n-1} \equiv 0 \pmod 2$, thus $\alpha^{(q^n-1)/2} = 1$, so $\alpha$ is a square in $K$.
\end{proof}

\paragraph{Ex. 7.8}

{\it  In a field with $2^n$ elements, what is the subgroup of squares.
}

\bigskip

Let $F$ a field with $q = 2^n$ elements. 

\bigskip

{\bf Proof 1}
\begin{proof}
$d = (q-1) \wedge 2 = (2^n - 1) \wedge 2 = 1$, thus each $\alpha \in F^*$ verifies $\alpha^{(q-1)/d} = \alpha^{q-1} = 1$. Theorem 7.1.2 show that $\alpha$ is a square in $F$, of exactly one root.
\end{proof}

{\bf Proof 2}
\begin{proof}
For all $x \in F$, $x = x^q = \left (x^{2^{n-1}}\right)^2$. So all elements in $F$ or $K$ are squares.
\end{proof}

\paragraph{Ex. 7.9}

{\it If $K \supset F$ are finite fields, $\vert F \vert = q, \alpha \in F, q\equiv 1 \pmod n$, and $x^n = \alpha$ is not solvable in $F$, show that $x^n = \alpha$ is not solvable in $K$ if $(n,[K:F]) = 1$.
}

\begin{proof}
Let $k = [K:F]$. From hypothesis, $k\wedge n = 1$, so there exist integers $u,v$ such that $uk+vn = 1$.

As $n \mid q-1$, $n \wedge (q-1) = n$, so the hypothesis "$x^n = \alpha$ is not solvable in $F$" implies that $\alpha^{(q-1)/n} \neq 1$ (Prop. 7.1.2).

Write $\omega =\alpha^{(q-1)/n}$, so $\omega \ne 1$ and $\omega^n = 1$.

As $n \mid q-1$, $n \mid q^k - 1$ and
$$\alpha^{(q^k-1)/n} = (\alpha^{(q-1)/n})^{1+q+q^2+\cdots+q^{k-1}} =  \omega^{1+q+q^2+\cdots+q^{k-1}}.$$
Moreover $1+q+\cdots+q^{k-1} \equiv k \pmod n$, and $\omega^n = 1$, so $\alpha^{(q^k-1)/n} = \omega^k$.

If $\omega^k = 1$, then $\omega = \omega^{uk+vn} = (\omega^k)^u (\omega^n)^v = 1$, which is in contradiction with $\omega = \alpha^{(q-1)/n} \ne 1$.

So $\alpha^{(q^k-1)/n} = \omega^k \ne 1$, and consequently the equation $x^n = \alpha$ has no solution in $K$.
\end{proof}

\paragraph{Ex. 7.10}

{\it If $K \supset F$ be finite fields and $[K:F] = 2$. For $\beta \in K$, show that $\beta^{1+q} \in F$ and moreover that every element in $F$ is of the form $\beta^{1+q}$ for some $\beta \in K$.
}

\begin{proof}
If $\beta = 0$, $\beta^{1+q} = 0 \in F$, and if $\beta \in K^*$, $\beta^{q^2-1} = 1$, so $\left(\beta ^{1+q}\right)^{q-1} = 1$, thus $ \beta^{1+q} \in F$ (Prop. 7.1.1, Corollary 1).

Let  $g$ a generator of $K^*$ : $K^* = \{1,g,g^2,\cdots,g^{q^2-2}\}$.

For every in integer $k \in \Z$, 
$$g^k  \in F^* \iff (g^k)^{q-1} = 1 \iff g^{k(q-1)}=1 \iff q^2-1 \mid k(q-1) \iff q+1 \mid k.$$
Thus $F^* = \{1,g^{q+1}, g^{2(q+1)},\cdots, g^{(q-2)(q+1)}\}$. I $\alpha \in F^*$, there exists $i, 0 \leq i \leq q-1$ such that $\alpha = g^{i(q+1)}$.
If we write $\beta = g^i$, then $\alpha = \beta^{1+q}$ (and for $\alpha = 0$, we take $\beta = 0$).

Conclusion : if $K$ is a quadratic extension of $F$ ($F,K$ finite fields), every element in $F$ is of the form $\beta^{1+q}$ for some $\beta \in K$.
\end{proof}

\paragraph{Ex. 7.11}

{\it With the situation being that of Exercise 10 suppose that $\alpha \in F$ has order $q-1$. Show that there is a $\beta \in K$ with order $q^2-1$ such that $\beta^{1+q} = \alpha$.
}

\bigskip

Write $|a|$ the order of an element $a$ in a group $G$. We recall the following lemma :

\bigskip

{\bf Lemma} If $|a| = d$, then for all $i \in \Z$, $| a^i | = \frac{d}{d\wedge i}$.
\begin{proof}
Indeed, for all $k \in \Z$,
$$(a^i)^k=e \iff a^{ik}=e \iff d \mid ik \iff \frac{d}{d \wedge i} \mid \frac{i}{d \wedge i}\,k \iff \frac{d}{d \wedge i} \mid k.$$
\end{proof}
\begin{proof}(Ex. 7.11)

Let $\alpha \in F^*$ with $|a| = q-1$, and $g$ a generator of $K^*$, so $|g| = q^2-1$. We know from exercise 7.10 that there exists an integer i such that $\alpha = g^{i(q+1)}$.

Let $h = g^{q+1}$. As $h^{q-1} = 1$, then $h \in F^*$, and since $|g| = q^2-1$, $|h| = q-1$, so $h$ is a generator of $F^*$.

Note that for all $s\in \Z$, $\alpha = g^{(i+s(q-1))(q+1)}$, since $g^{q^2-1} = 1$.

We will show that we can choose $s$ such that $j = i +s(q-1)$ is relatively prime with $q+1$. Then  $j$ is such that $\alpha = g^{j(q+1)} = h^j$.


$i$ is odd : if not $\alpha$ is an element of the subgroup of squares in $F^*$, so its order divides $(q-1)/2$, in contradiction with $|\alpha| = q-1$.

 $(q-1)\wedge (q+1) \mid 2$. Since $i-1$ is even, there exist integers $s,t$ verifying the B�zout's equation
$$i-1 = t(q+1)-s(q-1).$$
Then $j = i+s(q-1) = 1 + t(q+1)$ is relatively prime with $q+1$ : $j \wedge (q+1) = 1$.

Moreover, as $\alpha = h^j$, with $|\alpha| = |h| = q-1$, the lemme implies that
$$q-1 = |\alpha| = \frac{q-1}{(q-1)\wedge j},$$
so $(q-1) \wedge j = 1$.
As $(q+1) \wedge j = 1$ and $(q-1) \wedge j = 1$, then $(q^2-1) \wedge j = 1$.

Let $\beta = g^j$ : then $\alpha = \beta^{1+q}$, and using the lemma :
$$|\beta| = |g^j| = \frac{q^2-1}{(q^2-1)\wedge j} = q^2-1.$$
Conclusion :  there exists a $\beta \in K^*$ with order $q^2-1$ such that $\beta^{1+q} = \alpha$.
\end{proof}

\paragraph{Ex. 7.12}

{\it Use Proposition 7.2.1 to show that given a field $k$ and a polynomial $f(x) \in k[x]$ there is a field $K\supset k$ such that $[K:k]$ is finite and $f(x) = a(x-\alpha_1)(x-\alpha_2)\cdots(x-\alpha_n)$ in $K[x]$.
}

\begin{proof}
We show by induction on the degree $n$ of $f$ that for all polynomials $f \in k[x]$ with $\deg(f) = n \geq 1$, there exists a field extension $K$ such that $[K:k]$ is finite, and $f(x)$ splits in linear factors on $K$.

If $n = 1$, $f(x) = ax+b = a(x-\alpha_0)$, where $\alpha_0 = -b/a$ : $K = k$ is suitable.

Suppose that the property is true for all polynomials of degree less than $n$ on an arbitrary field $k$.

Let $f(x) \in k[x], \deg(f) = n$. From propoistion 7.2.1. applied to an irreducible factor of $f$, there exists a field $L, [L:K]<\infty$ and $\alpha \in L$ such that $f(\alpha_1) = 0$. Then $f(x) = (x-\alpha_1) g(x), g(x) \in L[x]$. 

Applying the induction hypothesis in the field $L$ on the polynomial $g \in L[x]$ with $\deg(g) = n-1$, we obtain a field $K, [K:L]<\infty$ such that $g(x) = a(x-\alpha_2)\cdots(x-\alpha_n)$ with $\alpha_i \in K$. So  $f(x) = a(x-\alpha_1)(x-\alpha_2)\cdots(x-\alpha_n)$ splits in linear factors in $K$. The induction is achieved.
\end{proof}

\paragraph{Ex. 7.13}

{\it Apply Exercise 7.12 to $k = \Z/p\Z$ and $f(x) = x^{p^n} -x$ to obtain another proof of Theorem 2.
}

\begin{proof}
Let $f(x) =  x^{p^n} -x$. We know from Ex. 7.12 that there exists a finite extension $K$ of $\F_p$ such that $f$ splits in linear factors on $K$ :
$$f(x) = \prod_{k=1}^{p^n} (x-\alpha_k), \qquad \alpha_1,\ldots, \alpha_{p^n} \in K.$$
The set $k =\{\alpha_1,\cdots,\alpha_{p_n}\} \subset K$ of the roots of $x^{p^n} -x$ is a subfield of $K$ : indeed, if $\alpha, \beta \in k$,
\begin{enumerate}
\item[(a)] $f(1) = 0$, so $1 \in k$
\item[(b)]$(\alpha- \beta)^{p^n} = \alpha^{p^n} - \beta^{p^n} = \alpha - \beta$, so $\alpha - \beta \in k$.
\item[(c)] $(\alpha\beta)^{p^n} = \alpha^{p^n} \beta^{p^n} = \alpha  \beta$, so $\alpha \beta \in k$.
\item[(d)] $(\alpha^{-1})^{p^n} = (\alpha^{p^n})^{-1} = \alpha^{-1}$, so $\alpha^{-1} \in k$ if $\alpha \ne 0$.
\end{enumerate}
As $f'(x) = -1$, $f(x) \wedge f'(x) = 1$, so $f$ has no multiple root, so the cardinality of $k$ is $p^n$.

Let $g(x) \in \F_p[x]$ a factor of $f(x)$, irreducible in $\F_p[x]$, with $d =\deg(g)$. As $g \mid f$, $g$ splits in linear factors in $k[x]$. Let $\alpha$ a root of $g(x)$ in $k$. As $g$ is irreducible on $\F_p$, 
$d = \deg(g) = [\F_p[\alpha] : \F_p]$. Moreover $ n = [k : \F_p] = [k : \F_p[\alpha]]\,[\F_p[\alpha] : \F_p]$, so $d \mid n$.

Reciprocally, suppose that $g$ is any irreducible polynomial in $\F_p[x]$, with $d = \deg(g) \mid n$. Then $K_0 = \F_p[x]/ \langle g\rangle$  contains a root $\alpha$ of $g$, and $[K_0:\F_p] = \deg(g) = d$, so $\alpha^{p^d} = \alpha$.

As $d\mid n$ , then $p^d-1 \mid p^n-1$ and $x^{p^d}-1 \mid x^{p^n}-1$ (Lemma 2,3 in section 1), so
$$x^{p^d} - x \mid x^{p^n}-x.$$
$f(\alpha) = \alpha^{p^n} - \alpha = 0$ and $g$ is the minimal polynomial of $\alpha$, so $g \mid f$.

Conclusion : 
$$ x^{p^n}-x = \prod_{d\mid n} F_d(x),$$
where $F_d(x)$ is the product of the monic irreducible polynomial of degree $d$.
\end{proof}

\paragraph{Ex. 7.14}

{\it  Let $F$ be a field with $q$ elements and $n$ a positive integer. Show that there exist irreducible polynomials in $F[x]$ of degree $n$.
}

\begin{proof}
Leq $F = \F_q$ a field with $q = p^m$ elements, and $n$ a positive integer.

From Theorem 2 Corollary 3, there exists an irreducible polynomial $f(x) \in \F_p[x]$ of degree $nm$. Let $g$ an irreducible factor of $f$ in $\mathbb{F}_q[x]$, and $\alpha$ a root of $g$ in an extension of $\F_q$.

\bigskip

We show that $\F_q \subset \F_p[\alpha]$.

$\F_q$ and $\F_p[\alpha]$ are two subfield of the same finite field $\F_q[\alpha]$. Moreover, $|\F_q| = p^m$, and $|\F_p[\alpha]| = p^{nm} $. As $m\mid n$, $\F_q \subset \F_p[\alpha]$ .

Indeed, for all  $\gamma \in \F_q[\alpha]$,$$ \gamma \in \F_q \Rightarrow  \gamma^{p^m} =  \gamma \Rightarrow  \gamma^{p^{mn}} =  \gamma \Rightarrow  \gamma \in \F_p[\alpha].$$
So $\F_q \subset \F_p[\alpha]$.

\bigskip

We show that $\F_q[\alpha] =  \F_p[\alpha]$.

As $\F_p \subset \F_q$, $\F_p[\alpha] \subset \F_q[\alpha]$.

Let $\beta \in \F_q[\alpha]$ : $\beta = \sum\limits_{i=1}^k a_i\alpha^i$, where $a_i \in \F[q] \subset \F_p[\alpha]$, so $a_i = p_i(\alpha), p_i \in \F_p[\alpha]$. Consequently
$$\beta = \sum\limits_{i=1}^k p_i(\alpha) \alpha^i \in \F_p[\alpha],$$
so $\F_q[\alpha] =  \F_p[\alpha]$.

$$nm = [\mathbb{F}_p[\alpha] : \mathbb{F}_p] = [\mathbb{F}_q[\alpha]:\mathbb{F}_p] = [\mathbb{F}_q[\alpha] : \mathbb{F}_q]\times [\mathbb{F}_q:\mathbb{F}_p] =  [\mathbb{F}_q[\alpha] : \mathbb{F}_q]\times m .$$
Thus  $[\mathbb{F}_q[\alpha] : \mathbb{F}_q] = n$, and $g$ is the minimal polynomial of $\alpha$ on $\F_q$, so $\deg(g) = n$.

Conclusion : if $F$ is a field with $q = p^m$ elements, there exist irreducible polynomials in $F[x]$ of degree $n$ for all positive integers $n$.
\end{proof}

\paragraph{Ex. 7.15}

{\it Let $x^n-1 \in F[x]$, where $F$ is a finite field with $q$ elements. Suppose that $(q,n) = 1$. Show that $x^n-1$ splits into linear factors in some extension field and that the least degree of such a field is the smallest integer $f$ such that $q^f \equiv 1 \pmod n$.

}

\begin{proof}
From exercise 7.12, we know that $x^n-1$ splits into linear factors in some extension field $K$, with $[K:F]<\infty$ :
$$u(x)= x^n-1 = (x-\zeta_0)(x-\zeta_1)\cdots(x-\zeta_{n-1}), \qquad \zeta_i \in K.$$
$u'(x)\wedge u(x) = nx^{n-1} \wedge (x^n-1) = 1$, since $x(nx^{n-1}) - n(x^n-1) = n$, and $n\neq 0$ in the field $F$, since we know from the hypothesis $q \wedge n=1$ that the characteristic $p$ doesn't divide $n$. So the $n$ roots of $x^n-1$ are distinct.

The set $G = \{x \in K \ \vert \ x^n=1\}$ is a subgroup of $K^*$, thus $G$ is cyclic of order $n$. Let $\zeta$ a generator of $G$. Then
$$x^n-1 = (x-1)(x-\zeta)(x-\zeta^2)\cdots(x-\zeta^{n-1}).$$

Let $p(x)$ the minimal polynomial of $\zeta$ on $F$, and $f$ the degree of $p$ :
$$f = \deg(p) = [F[\zeta] : F].$$
So $\mathrm{Card}\, F[\zeta] = q^f$, and since $\zeta \in F[\zeta]^*$, $\zeta^{q^f - 1} -1 = 0$.
As the order of $\zeta$ in the group $G$ is $n$, $n \mid q^f - 1$, namely $q^f \equiv 1 \pmod n$.

\bigskip

Let $k$ any positive integer such that $q^k \equiv 1 \pmod n$.

Then $n \mid q^k - 1$, so $\zeta^{q^k-1} - 1 = 0$, $\zeta^{q^k} - \zeta = 0$. Let $L$ an extension of $K$ such that $x^{q^k} - x$ splits in linear factors in $L$. As $\zeta^{q^k} - \zeta = 0$, $\zeta $ belongs to the subfield $M$ of $L$ with cardinality $q^k$, such that $[M:F]= k$. Thus $\F[\zeta] \subset M$, so $f = [F[\zeta] : F] \leq k = [M:F]$.

$f = [F[\zeta] : F]$ is the smallest $k \in \N^*$ such that $q^k\equiv 1 \pmod n$.

If $K$ is any extension of $F$ containing the roots of $x^n-1$, then $K \supset F[\zeta]$, where $\zeta$ is a primitive root of unity, so $[K : F] \geq [F[\zeta]:F] = f$.

Conclusion : the minimal degree of a extension $K \supset F$ containing the roots of $x^n - 1$, with $n\wedge q = 1$, is the smallest positive integer $f$ such that $q^f \equiv 1 \pmod n$, the order of $q$ modulo $n$.
\end{proof}

\paragraph{Ex. 7.16}

{\it Calculate the monic irreducible polynomials of degree 4 in $\Z/2\Z[x]$.

}

\begin{proof}
Write $F_d$ the product of irreducible monic polynomials in $\mathbb{F}_2[x]$.

Theorem 2 gives $$x^{16}-x=x^{2^4}-x = \prod\limits_{d\mid 4} F_d(x) = F_1(x) F_2(x) F_4(x)$$ 
and 

$$x^{4}-x=x^{2^2}-x = \prod\limits_{d\mid 2} F_d(x) = F_1(x) F_2(x)$$

so $F_4(x) = \frac{x^{16}-x}{x^4-x} = \frac{x^{15}-1}{x^3-1} = x^{12}+x^9+x^6+x^3+1$

$F_4(x) = (x^4+x^3+x^2+x+1)(x^4+x+1)(x^4+x^3+1)$

Among the 16 monic polynomials of degree 4 in $\mathbb{F}_2[x]$, 3 are irreducible :
\begin{align*}
P_1(x) &= x^4+x^3+x^2+x+1,\\
 P_2(x)&=x^4+x+1\\
  P_3(x)&=x^4+x^3+1
\end{align*}

With sage : 
\begin{verbatim}
sage: A = PolynomialRing(GF(2),'x')
sage: x = A.gen()
sage: f = (x^16-x)/(x^4-x)
sage: factor(f)
(x^4 + x + 1) * (x^4 + x^3 + 1) * (x^4 + x^3 + x^2 + x + 1)
\end{verbatim}
\end{proof}

\paragraph{Ex. 7.17}

{\it Let $q$ and $p$ be distinct odd primes. Show that the number of monic irreducibles of degree $q$ in $\Z/p\Z$ is $q^{-1}(p^q -p)$.
}

\begin{proof}
From Theorem 2 Corllary 2, we know that the number of irreducible polynomials on $\F_p$ of degree $q$ is given by
$$N_q = \frac{1}{q}\sum_{d\mid q} \mu\left(\frac{q}{d}\right) p^d.$$
As $q$ is prime, $d$ takes the values $1,q$, with $\mu(1) = 1, \mu(q) = -1$, so
$$N_q = \frac{p^q -p}{q}.$$
\end{proof}

\paragraph{Ex. 7.18}

{\it Let $p$ be a prime with $p\equiv 3 \pmod 4$. Show that the residue classes modulo $p$ in $\Z[i]$ form a field with $p^2$ elements.
}

\begin{proof}
If $p$ is a prime rational integer, with $p\equiv 3 \pmod 4$, then $p$ is a prime in $\Z[i]$.

Indeed, $p$ is irreducibel : if $p = uv,\ u,v \in \Z[i]$, where $u = c+di,v$ are not units, then $p^2 = N(u)N(v),\ N(u)>1,N(v)>1$, so $p = N(u) = u \overline{u} = c^2+d^2$.

 As $c^2\equiv 0,1 \pmod 4, d^2 \equiv 0,1 \pmod 4$, so $p \equiv1 \pmod 4$, which is in contradiction with the hypothesis.
 
 So $p$ is irreducible in $\Z[i]$, and since $\Z[i]$ is a principal ideal domain, $p$ is prime in $\Z[i]$, thus $\Z[i]/( p )$ is a field.
 
 Let $z = a+bi \in \Z[i]$. The Euclidean division of $a,b$ by $q$ gives 
 $$a = qp+r,\ 0\leq r < p, \qquad b = q'p+s, \ 0\leq s < p,$$
 so $$z \equiv r+is \pmod p,\ 0\leq r <p, 0 \leq s < p.$$
 Let's verify that these $p^2$ elements are in differnet classes of congruences modulo $p$.
 
 If $r+is \equiv r'+is' \pmod p$, then $(r-r')/p + i(s-s')/p \in \Z[i]$, so $r\equiv r',s\equiv s' \pmod p$.
 
 As $r,r',s,s'$ are between $0$ and $p-1$, $r=r', s= s'$.
 
 So the cardinality of the field $\Z[i]/( p )$ is $p^2$.
\end{proof}

\paragraph{Ex. 7.19}

{\it Let $F$ be a finite field with $q$ elements . If $f(x) \in F[x]$ has degree $t$, put $|f| = q^t$. Verify the formal identity $\sum_f |f|^{-s} = (1-q^{1-s})^{-1}$. The sum is over all monic polynomials.

}

\begin{proof}
Let $U$ the set of monic polynomials in $\mathbb{F}_q[x]$, and $U_t$ the set of monic polynomials of degree $t$, and $s\in \C$. Then $U = \coprod_{t \in \N} U_t$, so
\begin{align*}
\sum_{f \in U} \vert f \vert^{-s }&= \sum_{t=0}^\infty \sum_{f \in U_t} \vert f \vert^{-s}\\
&=\sum_{t=0}^\infty \frac{1}{q^{ts}} \sum_{f\in U_t} 1
\end{align*}
As $\sum_{f\in U_t} 1 = \mathrm{Card} \, (U_t) = q^t$, then, for $\mathrm{Re}(s) >1$

\begin{align*}
\sum_{f \in U} \vert f \vert^{-s } &= \sum_{t=0}^\infty \frac{1}{q^{t(s-1)}}\\
&=\frac{1}{1-\frac{1}{q^{s-1}}}\\
&= (1-q^{1-s})^{-1}
\end{align*}
As $\left | \frac{1}{q^{t(s-1)}} \right | = \frac{1}{q^{t(\mathrm{Re}(s)-1)}}$, the serie is absolutely convergent for $\mathrm{Re}(s)>1$. This justifies the grouping of terms in this sum.

\bigskip

Conclusion :  if $\mathrm{Re}(s)>1$, 
$$\sum_{f \in U} \vert f \vert^{-s } =(1-q^{1-s})^{-1},$$
where $U$ is the set of monic polynomials in  $\mathbb{F}_q[x]$.
\end{proof}

\paragraph{Ex. 7.20}

{\it With the notation of Exercise 19 let $d(f)$ be the number of monic divisors of $f$ and $\sigma(f) = \sum_{g\mid f} |g|$, where the sum is over the monic divisors of $f$. Verify the following identities :
\begin{enumerate}
\item[(a)] $\sum_f d(f) |f|^{-s} = (1 -q^{1-s})^{-2}$
\item[(b)] $\sum \sigma(f)|f|^{-s} = (1-q^{1-s})^{-1} (1-q^{2-s})^{-1}$
\end{enumerate}

}

\begin{proof}
(a) With the notation of 7.19, for $s \in \mathbb{C}, \mathrm{Re}(s) >1$, $ \sum_{f\in U} \vert f \vert^{-s}$ is absolutely convergent and

$$(1-q^{1-s})^{-1} = \sum_{f\in U} \vert f \vert^{-s}$$

Then
\begin{align*}
(1-q^{1-s})^{-2} &= \sum_{f\in U} \vert f \vert^{-s}\sum_{g\in U} \vert g\vert^{-s}\\
&=\sum_{(f,g)\in U^2} \vert fg\vert^{-s}\\
&= \sum_{h\in U} \sum_{g\in U,\, g\mid h} \vert h \vert^{-s},
\end{align*}
indeed, the application
\[\varphi :
 \left \{
\begin{array}{ccc}
U\times U & \to &   \{(h,g)\in U\times U, g\mid h\}  \\
  (f,g) &   \mapsto &(fg,g)    
\end{array}
\right.
\]
is a bijection.
 
So
 \begin{align*}
 (1-q^{1-s})^{-2} &=\sum_{h\in U}  \vert h \vert^{-s} \mathrm{Card}\{g \in U, g \mid h\}\\
 &= \sum_{h\in U} \vert h \vert^{-s} d(h)\\
 &= \sum_{f\in U} d(f) \vert f \vert^{-s}
 \end{align*}
 
 (b) Similarly,
 \begin{align*}
 (1-q^{1-s})^{-1}  (1-q^{2-s})^{-1} &= \sum_{f\in U} \vert f \vert^{-s} \sum_{g\in U} \vert g \vert^{-s+1}\\
 &=\sum_{(f,g)\in U^2} \vert g \vert\, \vert fg\vert^{-s}\\
 &= \sum_{h\in U} \sum_{g\in U,\, g\mid h} \vert g \vert \, \vert h \vert^{-s}\\
 &=\sum_{h\in U}\vert h \vert ^{-s}  \sum_{g\in U,\, g\mid h} \vert g \vert\\
 &=\sum_{h\in U}  \sigma(h) \vert h \vert ^{-s}\\
  &=\sum_{f\in U}  \sigma(f) \vert f \vert ^{-s}\\
 \end{align*}
\end{proof}

\paragraph{Ex. 7.21}

{\it Let $F$ be a field with $q=p^n$ elements. For $\alpha \in F$ set $f(x) = (x-\alpha)(x-\alpha^p)(x-\alpha^{p^2})\cdots(x-\alpha^{p^{n-1}})$. Show that $f(x) \in \Z/p\Z[x]$. In particular, $\alpha + \alpha^p+\cdots+\alpha^{p^{n-1}}$ and $\alpha\alpha^p\alpha^{p^2}\cdots\alpha^{p^{n-1}}$ are in $Z/p\Z$.

}

\begin{proof}
Let 
$F : 
\left\{
\begin{array}{ccc}
 \mathbb{F}_q &  \to  &  \mathbb{F}_q \\
 x & \mapsto     &  x^p    
\end{array}
\right. .
$


As the characteristic of $\F_q$ is $p$, $(x+y)^p=x^p+y^p$ et $(xy)^p = x^py^p$, and each homomorphism of field is injective, $F$ is a field automorphism (Frobenius automorphism).

For every automorphism $H$ in $\F_q$, and every polynomial $p(x) =\sum a_ix^i \in \F_q[x]$, write $(H.p)(x) = \sum_i H(a_i) x^i$. Then for all $(p,q) \in \F_q[x]^2$, $H.(pq) = (H.p)(H.q)$.

With this notation,
$$f(x) =(x-\alpha)(x-F\alpha)(x- F^2\alpha)\cdots (x-F^{n-1}\alpha),$$
$$(H.f)(x) =(x-F\alpha)(x-F^2\alpha)(x- F^3\alpha)\cdots (x-F^{n}\alpha).$$

Since $\alpha \in \mathbb{F}_{p^n}$, $F^n\alpha = \alpha^{p^n} = \alpha$ , thus
$$H.f = f.$$
In other words, if $f(x) =\sum_i a_i x^i$, then for all $i$, $H(a_i) = a_i$, so $a_i^p = a_i$, thus $a_i \in \F_p$, and $f \in \F_p[x]$.
In particular, the coefficients $a_{n-1}= \alpha + \alpha^p+\cdots+\alpha^{p^{n-1}}, a_0=\alpha\alpha^p\alpha^{p^2}\cdots\alpha^{p^{n-1}}$ are in $\F_p$.
\end{proof}

\paragraph{Ex. 7.22}

{\it (continuation) Set $\mathrm{tr}(\alpha) = \alpha+\alpha^p+\cdots+\alpha^{p^{n-1}}$. Prove that
\begin{enumerate}
\item[(a)] $\mathrm{tr}(\alpha) + \mathrm{tr}(\beta) = \mathrm{tr}(\alpha+\beta)$.
\item[(b)] $\mathrm{tr}(a\alpha) = a\, \mathrm{tr}(\alpha)$ for $a \in \Z/p\Z$.
\item[(c)] There is an $\alpha \in F$ such that $\mathrm{tr}(\alpha) \ne 0$.
\end{enumerate}
}

\begin{proof}
Let $F$ the Frobenius automorphism of $\F_q$ introduced in Ex.7.21.

(a),(b) : If $x,y \in \F_q$, and $a \in \F_p$, then $a^p = a$, so $F(x+y) =(x+y)^p = x^p+y^p = F(x)+F(y)$, and $F(ax) = a^p x^p = a x^p =a F(x)$, so $F$ is $\F_p$-linear, and also $tr = I + F + F^2+\cdots+F^{n-1}$. 

(c) The polynomial $p(x) = x+x^p+x^{p^2}+\cdots+x^{p^{n-1}}$ has degree $p^{n-1}$, so $p(x)$ has at most $p^{n-1}$ roots in $\F_q$, and $|\F_q| =p^n >deg(p) = p^{n-1}$. Therefore there exist in $\F_q$ some element $\alpha$ which is not a root of $p(x)$, and so  $\mathrm{tr}(\alpha) = p(\alpha) \ne 0$.
\end{proof}

\paragraph{Ex. 7.23}

{\it (continuation) For $\alpha \in F$ consider the polynomial $x^p-x-\alpha \in F[x]$. Show that this polynomial is either irreducible or the product of linear factors. Prove that the latter alternative holds iff $\mathrm{tr}(\alpha) = 0$.

}

\begin{proof}
Let $f(x) = x^p -x -\alpha \in F[x]$. There exists an extension $K \supset F$ with finite degree on $F$ which contains a root $\gamma$ of $f$.

As $\gamma^p - \gamma - \alpha = 0$, then for all $i\in \F_p$,
$$(\gamma + i)^p - (\gamma+i) - \alpha = (\gamma^p - \gamma - \alpha) + i^p - i = 0.$$
So $f$ has $n$ distinct roots in $K$ : $\gamma, \gamma+1,\ldots,\gamma+p-1$, and so
$$f(x) = (x-\gamma)(x-\gamma-1)\cdots(x-\gamma-(p-1)).$$
$F[\gamma] $ contains all roots of $f$.

$\bullet$ If $\gamma \in F$, $f(x)$ splits in linear factors in $F$. $f(x)$ is not irreducible, since $\deg(f) = p>1$.

$\bullet$ If $\gamma \not \in F$, we will show that $f$ is irreducible in $F[x]$.

If not, then $f(x) = g(x) h(x)$ is the product of two polynomials $g,h \in F[x]$ such that $1\leq \deg(g) \leq p-1$.

The unicity of the decomposition in irreducible factors in $F[\gamma][x]$ shows that
$$g(x) = \prod_{i\in A} (x-\gamma -i),$$
where $A$ is a subset of $\F_p$, with $A \ne \emptyset, A \ne \F_p$.
As $g(x) \in F[x]$, $\sum\limits_{i\in A} (\gamma+i) = k \gamma +l \in \F_p$, where $1 \leq k = |A| \leq p-1$ and $l = \sum\limits_{i\in A} i \in \F_p$.

So $k\gamma \in \F_p$. Since $\gamma \not \in \F_p$, $k$ is not invertible in $\F_p$, in contradiction with $1\leq k \leq p-1$. Consequently,  $f(x)$ is irreducible.

We conclude that $x^p-x-\alpha \in F[x]$ is irreducible iff $\gamma \not \in F$.

\bigskip

Let $F$ the Frobenius automorphism of $K$ (cf. Ex. 7.21).

$$\alpha = F(\gamma) - \gamma, F(\alpha) = F^2(\gamma) - F(\gamma), \ldots, F^{n-1}(\alpha) = F^n(\gamma) - F^{n-1}(\gamma).$$
The sum of these equalities gives
$$\mathrm{tr}(\alpha) = \alpha+F(\alpha)+\cdots+F^{n-1}(\alpha) = F^n(\gamma) - \gamma = \gamma^{p^n} - \gamma .$$
As the cardinality of $F$ is $q=p^n$,
$$\gamma \in F \iff \gamma^{p^n} - \gamma = 0 \iff \mathrm{tr}(\alpha) = 0.$$

Conclusion : $x^p - x - \alpha$ is irreducible iff $\mathrm{tr}(\alpha) \ne 0$. If $\mathrm{tr}(\alpha)= 0$, $x^p - x - \alpha$ splits in linear factors in $F[x]$.
\end{proof}

\paragraph{Ex. 7.24}

{\it Suppose that $f(x) \in \Z/p\Z[x]$ has the property that $f(x+y) = f(x) + f(y) \in \Z/p\Z[x,y]$. Show that $f(x)$ must be of the form $a_0x +a_1x^p+a_2x^{p^2}+\cdots+a_mx^{p^m}.$
}

\bigskip

{\bf Lemma} {\it If the prime number $p$ divides all binomial coefficients $\binom{n}{1}, \binom{n}{2},\ldots, \binom{n}{n-1}$, then $n$ is a power of $p$.
\begin{proof}
Let $ u(x) = (x+1)^n -x^n - 1 \in \F_p[x]$. Then $f(x) = \sum\limits_{k=1}^{n-1} \binom{n}{i} x^i = 0$.

Write $n = p^a q$, with $p\wedge q = 1$. With a reductio as absurdum, suppose that $q>1$. Then 
$$f(x)= 0 = (x+1)^{p^\alpha q} - x^{p^\alpha q} - 1=(x^{p^\alpha} + 1)^q -x^{p^\alpha q} - 1=\sum\limits_{k=1}^{q-1} \binom{q}{k} x^{kp^a}.$$
Consequently, the coefficient of $x^{p^a}$ is null, so $p \mid q$ : this is absurd. Therefore $q=1$ and $n = p^a$.
\end{proof}


\begin{proof}(Ex. 7.24)

Suppose that $f\in \F_p[x]$ verify in $\F_p[x,y]$ the equality $f(x+y) = f(x)+f(y)$.

Write $f(x) = \sum\limits_{k=1}^d c_i x^i$.
\begin{align*}
0 = f(x+y) - f(x) -f(y)&= \sum_{n=0}^d c_n[(x+y)^n-x ^n - y^n]\\
&=\sum_{n=0}^d \sum_{k=1}^{n-1} c_n\binom{n}{k} x^ky ^{n-k}\\
\end{align*}
So  for all $n$, for all $k$, $1\leq k \leq n-1$, $c_n \binom{n}{k} = 0$ in $\F_p$.

From the lemma, if $n$ is not a power of $p$, there exists a $k$, $1\leq k \leq n-1$ such that $\binom{n}{k} \not \equiv 0 \pmod p$, so $c_n = 0$.
If we write $a_k = c_{p^k}$, then $f(x)$ is of the form
$$f(x) = a_0 x+a_1x^p +a_2 x^{p^2}+\cdots+a_m x^{p^m}.$$
\end{proof}

{ \Large \bf Chapter 8} 

\paragraph{Ex. 8.1}

{\it Let $p$ be a prime and $d=(m,p-1)$. Prove that $N(x^m = a) =\sum \chi(a)$, the sum being over all $\chi$ such that $\chi^d = \varepsilon$.
}

\begin{proof}
Let $d = m\wedge (p-1)$ . we prove that $N(x^m =a) = N(x^d = a)$ for all $d \in \F_p$.

$\bullet$ If $a = 0$, $0$ is the only root of $x^m -a$ or $x^d -a$, so  $N(x^m = a) = N(x^d = a) = 1$.

$\bullet$ If $a \in \F_p^*$ and $x^n = a$ has a solution, then we know from the demonstration of Proposition 4.2.1 that $N(x^n-a) = d = N(x^d-a)$.

$\bullet$ If If $a \in \F_p^*$ and $x^n = a$ has no solution, then (Prop. 4.2.1) $a^{(p-1)/d} \ne 1$, so $x^d = a$ has no solution : $N(x^n-a) = 0 = N(x^d-a)$.

Using Prop. 8.1.5, as $d \mid n$, we obtain
$$N(x^n =a) = N(x^d = a) = \sum_{\chi^d = \varepsilon} \chi(a).$$
\end{proof}

\paragraph{Ex. 8.2, false sentence.}

{\it With the notation of Exercise 1 show that $N(x^m = a) = N(x^d = a)$ and conclude that if $d_i = (m_i,p-1)$, then $\sum_i a_ix^{m_i} = b$ and $\sum_i a_i x^{d_i} = b$ have the same number of solutions.
}

\bigskip

This result is false. I give a counterexample with $p=5$ :  $x + x^3 =0 \in \F_5[x]$ has 3 solutions $0,2,-2$.  As $3\wedge (p-1) = 3 \wedge 4 = 1$, the reduced equation is $x + x = 0$, which has an unique solution $0$. The true sentence is :

\bigskip\paragraph{Ex. 8.2}
{\it With the notation of Exercise 1 show that $N(x^m = a) = N(x^d = a)$ and conclude that if $d_i = (m_i,p-1)$, then $\sum_i a_ix_i^{m_i} = b$ and $\sum_i a_i x_i^{d_i} = b$ have the same number of solutions.
}

\begin{proof}
From Ex. 8.1, we know that
$$N(x^m=a) = \sum_{\chi^d=\varepsilon} \chi(a) = N(x^d=a).$$
Using this result, we obtain
\begin{align*}
N\left(\sum\limits_{i=1}^l a_i x_i^{m_i} = b\right) &= \sum\limits_{a_1u_1+\cdots+a_l u_l= b}\, \prod\limits_{i=1}^l N(x^{m_i} = u_i)\\
&= \sum\limits_{a_1u_1+\cdots+a_l u_l= b}\, \prod\limits_{i=1}^l N(x^{d_i} = u_i)\\
&=N\left(\sum\limits_{i=1}^l a_i x_i^{d_i} = b\right)
\end{align*}
\end{proof}

\paragraph{Ex. 8.3}

{\it Let $\chi$ be a non trivial multiplicative character of $\F_p$ and $\rho$ be the character of order 2. Show that $\sum_t\chi(1-t^2) = J(\chi,\rho)$.[Hint: Evaluate $J(\chi,\rho)$ using the relation $N(x^2 = a) = 1 + \rho(a)$.]

}

\begin{proof}
\begin{align*}
J(\chi,\rho) &= \sum\limits_{a+b=1} \chi(a)\rho(b)\\
&= \sum\limits_{a+b=1}\chi(a)(N(x^2=b) -1)\\
&=\sum\limits_{a+b=1}\chi(a)N(x^2=b) -\sum\limits_{a+b=1}\chi(a)\\
\end{align*}
As $\chi \neq \varepsilon$, 
$$\sum\limits_{a+b=1}\chi(a)= \sum\limits_{a\in\mathbb{F}_p}\chi(a)=0.$$
Let $C = \{x^2 \ \vert\  x \in \mathbb{F}^*\}$ the set of squares in $\F_p^*$ , $\overline{C}$ its complementary in  $\mathbb{F}_p^*$ : 
$$\mathbb{F}_p = \{0\} \cup C \cup \overline{C}.$$
Then 
\begin{align*}
J(\chi,\rho) &= \sum\limits_{a+b=1}\chi(a)N(x^2=b)\\
&= \sum\limits_{a+b=1, b=0}\chi(a)N(x^2=b)+\sum\limits_{a+b=1,b\in C}\chi(a)N(x^2=b)+\sum\limits_{a+b=1,b\in\overline{C}}\chi(a)N(x^2=b)\\
&=\chi(1)+ 2 \sum\limits_{b\in C} \chi(1-b)
 \end{align*}
 (because $N(x^2=b)=0$ if $x \in \overline{C}$, and $N(x^2=b)=2$ if $x\in C$).
 As each $b\in C$ has two roots, and as the set of roots of two distinct $b$ are disjointed,
 $$J(\chi,\rho) = \chi(1) + \sum\limits_{t \in \mathbb{F}_p^*} \chi(1-t^2) = \sum\limits_{t \in \mathbb{F}_p} \chi(1-t^2).$$
 
 Conclusion : if $\chi$ is a non trivial multiplicative character of $\F_p$ and $\rho$  the character of order 2,
 $$ J(\chi,\rho) = \sum_{t\in\F_p}\chi(1-t^2) .$$
\end{proof}

\paragraph{Ex. 8.4}

{\it Show, if $k \in \F_p, k\neq 0$, that $\sum_t \chi(t(k-t)) = \chi(k^2/2^2)J(\chi,\rho)$.

}

\begin{proof}
We know from Ex. 8.3 that  $J(\chi,\rho) = \sum_t \chi(1-t^2)$, so

\begin{align*}�
J(\chi,\rho) &= \sum_{t \in \F_p}\chi(1-t)\chi(1+t)\\
&=\sum_{u \in \F_p} \chi(u) \chi(2-u)\qquad(u = 1-t)\\
&=\chi(2^2) \sum_{u \in \F_p} \chi\left(\frac{u}{2}\right) \chi\left(1-\frac{u}{2}\right)\\
&= \chi(2^2) \sum_{v \in \F_p} \chi(v) \chi(1-v)\qquad(u = 2v)\\
&= \chi(2^2) \chi(k^{-2}) \sum_{w \in \F_p} \chi(kv) \chi(k-kv)\\
&=\chi(2^2/k^2) \sum_{t \in \F_p} \chi(t) \chi(k-t)\qquad(t=kv).
\end{align*}

Conclusion : if $k \in \mathbb{F}^*$, and $\chi$ is a non trivial character, $\rho$ the character of order 2,
$$ \sum_{t\in \F_p}\chi(t(k-t)) = \chi(k^2/2^2) J(\chi,\rho).$$
\end{proof}

\paragraph{Ex. 8.5}

{\it If $\chi^2 \ne \varepsilon$, show that $g(\chi)^2 = \chi(2)^{-2}J(\chi,\rho)g(\chi^2)$. [Hint: Write out $g(\chi)^2$ explicitly and use Exercise 4.]

}

\begin{proof}
Let  $\zeta = e^{2i\pi/p}$.
Using the result of Ex. 8.4, we obtain
\begin{align*}
g(\chi)^2 &= \left(\sum_t \chi(t) \zeta^t \right)\left(\sum_s \chi(s) \zeta^s\right)\\
&=\sum_{s,t} \chi(t)\chi(s) \zeta^{t+s}\\
&=\sum_k\left(\sum_{s+t=k} \chi(t) \chi(s)\right)\zeta^k\\
&=\sum_k\left( \sum_t \chi(t(k-t) \right) \zeta^k\\
&= \chi(-1)\sum_t \chi(t^2) + \sum_{k\neq0} \chi(k^2/2^2) J(\chi,\rho) \zeta^k\\
&= \chi(-1)\sum_t \chi^2(t) + \chi(2)^{-2}  J(\chi,\rho) \sum_{k\neq0} \chi^2(k)  \zeta^k\\
\end{align*}

If $\chi^2 \neq \varepsilon, \sum_t \chi^2(t)=0$, so

$$g(\chi)^2 = \chi(2)^{-2}  J(\chi,\rho) g(\chi^2).$$
\end{proof}

\paragraph{Ex. 8.6}

{\it (continuation) Show that $J(\chi,\chi) = \chi(2)^{-2} J(\chi,\rho)$.

}

\begin{proof}
As $\chi^2 \ne \rho$, Theorem 1 Chapter 8 gives $J(\chi,\chi) = g(\chi)^2/g(\chi^2)$, and Exercise 8.5 gives $g(\chi)^2/g(\chi^2) = \chi(2)^{-2} J(\chi,\rho)$, so
$$J(\chi,\chi) =  \chi(2)^{-2} J(\chi,\rho).$$
\end{proof}

\paragraph{Ex. 8.7}

{\it Suppose that $p\equiv 1 \pmod 4$ and that $\chi$ is a character of order 4. Then $\chi^2 = \rho$ and $J(\chi,\chi) = \chi(-1) J(\chi,\rho)$. [Hint: Evaluate $g(\chi)^4$ in two ways.]

}

\begin{proof}
As $\chi$ is a character of order 2, $\chi^2$ is a character of order, and $\rho$ (Legendre's character) is the unique character of order 2, so $\chi^4 = \rho$.

From Prop. 8.3.3 we have
$$g(\chi)^4 = \chi(-1) p J(\chi,\chi) J(\chi,\chi^2)  =  \chi(-1) p J(\chi,\chi) J(\chi,\rho).$$
Squaring the result of Ex. 8.5, we obtain
$$g(\chi)^4 = \chi(2)^{-4} J(\chi,\rho)^2 \left[g(\chi^2)\right]^2.$$
Moreover $\chi(2^4) = \chi^4(2) = \varepsilon(2) = 1$, and $g(\chi^2) = g(\rho) = g$, so $\left[g(\chi^2)\right]^2=g^2 = (-1)^{(p-1)/2} p = p $ (From Prop. 6.3.2 and $p\equiv 1 \pmod 4)$. 

Equating these two result, we obtain 
$$\chi(-1) p J(\chi,\chi) J(\chi,\rho)= J(\chi,\rho)^2 p.$$
As $g(\chi)^4 \ne 0$ since $|g(\chi)|^2 = p$, we have $J(\chi,\rho) \ne 0$, so
$$\chi(-1) J(\chi,\chi) = J(\chi,\rho).$$
$[\chi(-1)]^2 = \chi((-1)^2) = \chi(1) = 1$, so $\chi(-1) = \pm 1$, and $\chi(-1)^{-1} = \chi(-1)$, thus
$$ J(\chi,\chi) = \chi(-1) J(\chi,\rho).$$
\end{proof}

\paragraph{Ex. 8.8}

{\it Generalize Exercise 3 in the following way. Suppose that $p$ is a prime, $\sum_t \chi(1-t^m) = \sum_{\lambda} J(\chi,\lambda)$, where $\lambda$ varies over all characters such that $\lambda^m = \varepsilon$. Conclude that $\left | \sum_t \chi(1-t^m) \right | \leq (m-1)p^{1/2}$.

}

\begin{proof}
For all $y \in \F_p$, write  $A_y =\{x \in \mathbb{F}_p\ \vert \ x^m = y\}$. Then $\vert A_y\vert = N(x^m = y)$.

$\mathbb{F}_p = \coprod\limits_{y \in \mathbb{F}_p} A_y$ is the disjoint union of the $A_y$, so
$$\sum_{t\in \mathbb{F}_p} \chi(1-t^m) = \sum_{y\in \mathbb{F}_p} \sum\limits_{t \in A_y} \chi(1-t^m) = \sum_{y\in \mathbb{F}_p} \vert A_y \vert \chi(1-y) =\sum_{y\in \mathbb{F}_p} N(x^m = y) \chi(1-y).$$
Moreover, $N(x^m=y) = \sum\limits_{\lambda^m = \varepsilon} \lambda(y)$ (Prop. 8.1.5), so
\begin{align*}
\sum_{t\in \mathbb{F}_p} \chi(1-t^m) &= \sum_{y\in \mathbb{F}_p}  \sum_{\lambda^m = \varepsilon} \lambda(y)  \chi(1-y)\\
& =  \sum\limits_{\lambda^m = \varepsilon}  \sum\limits_{x+y=1} \chi(x) \lambda(y)\\
& = \sum\limits_{\lambda^m = \varepsilon} J(\chi,\lambda)
\end{align*}
Conclusion : $$\sum\limits_{t\in \mathbb{F}_p} \chi(1-t^m) = \sum\limits_{\lambda^m = \varepsilon} J(\chi,\lambda).$$

We know that there exist $m$ character whose order divides $m$. As $\chi \ne \varepsilon$, $J(\chi,\varepsilon)=0$, and  $\vert J(\chi,\lambda) \vert = \sqrt{p}$ for every $\lambda \ne \varepsilon$,
$$\left \vert \sum_{t\in \mathbb{F}_p} \chi(1-t^m)\right  \vert \leq \sum_{\lambda^m = \varepsilon, \lambda \neq \varepsilon} \vert J(\chi,\lambda) \vert = (m-1)\sqrt{p}.$$
\end{proof}


\end{document}