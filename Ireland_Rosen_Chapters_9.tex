%&LaTeX
\documentclass[11pt,a4paper]{article}
\usepackage[frenchb,english]{babel}
\usepackage[applemac]{inputenc}
\usepackage[OT1]{fontenc}
\usepackage[]{graphicx}
\usepackage{amsmath}
\usepackage{amsfonts}
\usepackage{amsthm}
\usepackage{amssymb}
\usepackage{yfonts}
%\input{8bitdefs}

% marges
\topmargin 10pt
\headsep 10pt
\headheight 10pt
\marginparwidth 30pt
\oddsidemargin 40pt
\evensidemargin 40pt
\footskip 30pt
\textheight 670pt
\textwidth 420pt

\def\imp{\Rightarrow}
\def\gcro{\mbox{[\hspace{-.15em}[}}% intervalles d'entiers 
\def\dcro{\mbox{]\hspace{-.15em}]}}

\newcommand{\D}{\mathrm{d}}
\newcommand{\Q}{\mathbb{Q}}
\newcommand{\Z}{\mathbb{Z}}
\newcommand{\N}{\mathbb{N}}
\newcommand{\R}{\mathbb{R}}
\newcommand{\C}{\mathbb{C}}
\newcommand{\F}{\mathbb{F}}
\newcommand{\re}{\,\mathrm{Re}\,}
\newcommand{\ord}{\mathrm{ord}}
\newcommand{\n}{\mathrm{N}}
\newcommand{\legendre}[2]{\genfrac{(}{)}{}{}{#1}{#2}}



\title{Solutions to Ireland, Rosen ``A Classical Introduction to Modern Number Theory''}
\author{Richard Ganaye}

\begin{document}

\maketitle


{ \Large \bf Chapter 9} 

\paragraph{Ex. 9.1}

{\it If $\alpha \in \Z[\omega]$, show that $\alpha$ is congruent to either $0,1$, or $-1$ modulo $1-\omega$.
}

\begin{proof}
Let $\lambda = 1 - \omega$, and $\alpha = a+b\omega \in D = \Z[\omega], a,b \in \Z$.

$\omega \equiv 1 \pmod \lambda$, so $\alpha \equiv a+b \pmod \lambda$, $\alpha \equiv c$ with $c = a+b \in \Z$.

$c\equiv 0,1,-1 \pmod 3$, and since $\lambda \mid 3$, $c \equiv 0,1,-1 \pmod \lambda$.

 Every $\alpha \in D$ is congruent to either $0,1$, or $-1$ modulo $\lambda = 1 -\omega$.

The classes of $0,1,-1$ in $D/\lambda D$ are distinct. Indeed, $1\not \equiv -1 \pmod \lambda$, if not $\lambda \mid 2$, so $2 = \lambda \lambda'$, $N(2) = N(\lambda) N(\lambda')$, thus $4 = 3 N(\lambda')$, so $3 \mid 4$, which is nonsense.

$\pm1 \equiv 0 \pmod \lambda$ implies $\lambda \mid 1$, so $\lambda$ would be a unit, in contradiction with $\lambda$ prime.

So there exist exactly three classes modulo $\lambda$ in $D$ : $ | D/\lambda D | = 3 = N(\lambda)$.

\end{proof}

\paragraph{Ex. 9.2}

{\it From now on we shall set $D = \Z[\omega]$ and $\lambda = 1 - \omega$. For $\mu$ in $D$ show that we can write $\mu = (-1)^a \omega^b \lambda^c\pi_1^{a_1}\pi_2^{a_2}\cdots\pi_t^{a_t}$, where $a,b,c$, and the $a_i$ are nonnegative integers and the $\pi_i$ are primary primes.
}


\begin{proof}

Let $S$ the set containing $\lambda = 1 - \omega$ and all primary primes. 

We show that
\begin{enumerate}
\item[(a)] every prime in $D$ is associate to a prime in $S$,
\item[(b)] no two primes in $S$ are associate.
\end{enumerate}

 Let $\pi$ be a prime in $D$. There are three cases.

  \begin{enumerate}
  \item[$\bullet$] If $\n(\pi) = 3$, then $\pi$ is associate to $\lambda \in S$, thus $\pi \in \{1-\omega, -1+\omega, -2 - \omega, 2 + \omega, 1 + 2\omega, -1-2\omega\}$, and no associate of $\lambda$ is primary.
  \item[$\bullet$] If $\n(\pi) = q^2$, where $q \equiv -1 \pmod 3, q>0,$ is a rational prime, then $\pi$ is associate to $q$ (Proposition 9.1.2), and $q$ is a primary prime.
  	The primes associate to $q$ are $q, -q, \omega q, -\omega q, -q - \omega q, q + \omega q$, so only $q$ is primary.
  \item[$\bullet$] If $\n(\pi) = p$, where $p \equiv 1 \pmod 3$, then the proposition 9.1.4. shows that among the associates to $\pi$ exactly one is primary.
  \end{enumerate}
   Moreover, the norm of two primes belonging to two different cases are distinct, so two such primes are not associate.

By Theorem 3, Chapter 1, as $D = \Z[\omega]$ is a principal ideal domain,  every $\mu \in D$ is of the form
$$\mu = u \prod_{\pi \in S} \pi^{e(\pi)},$$
where $u$ is a unit, so $u = (-1)^a\omega^b$. Thus
$$\mu = (-1)^a \omega^b \lambda^c\pi_1^{a_1}\pi_2^{a_2}\cdots\pi_t^{a_t},$$ where the $\pi$ are primary primes, and $a,b,c$ and the $a_i$ are nonnegative integers.
\end{proof}

\paragraph{Ex. 9.3}

{\it Let $\gamma$ a primary prime. To evaluate $\chi_\gamma(\mu)$ we see, by Exercise 2, that it is enough to evaluate $\chi_\gamma(-1), \chi_\gamma(\omega), \chi_\gamma(\lambda)$, and $\chi_\gamma(\pi)$, where $\pi$ is a primary prime. Since $-1 =(-1)^3$ we have $\chi_\gamma(-1) = 1$. We now consider $\chi_\gamma(\omega)$. Let $\gamma = a+b \omega$ and set $a = 3m -1$ and $b=3n$. Show that $\chi_\gamma(\omega) = \omega^{m+n}$.
}

\begin{proof}
Let $\gamma = a+b\omega=3m-1+3n\omega$. Then  $\chi_{\gamma}(\omega) = \omega^{\frac{N(\gamma)-1}{3}}$ (remark (b) of Theorem 1).
\begin{align*}
N(\gamma)-1 &= (3m-1)^2+(3n)^2-3n(3m-1)-1\\
&=9m^2 - 6m + 9n^2 -9nm + 3n\\
\frac{N(\gamma)-1}{3}&= 3m^2-2m+3n^2-3nm+n \equiv n+m \ [3]
\end{align*}
Thus, for $\gamma = a+b\omega=3m-1+3n\omega$,

$$\chi_{\gamma}(\omega) = \omega^{\frac{N(\gamma)-1}{3}} = \omega^{n+m}.$$
\end{proof}

\paragraph{Ex. 9.4}

{\it (continuation) Show that $\chi_\gamma(\omega) = 1,\omega$, or $\omega^2$ according to whether $\gamma$ is congruent to 8,2, or 5 modulo $3\lambda$. In particular, if $q$ is a rational prime, $q \equiv 2 \pmod 3$, then $\chi_q(\omega) = 1, \omega$, or $\omega^2$ according to whether $q \equiv 8,2$, or $5\pmod 9$. [Hint : $\gamma = a + b \omega = -1 + 3(m+n\omega)$, and so $\gamma \equiv -1 + 3(m+n) \pmod{3\lambda}$.]
}

\begin{proof}
$\lambda=1-\omega$, so $\omega\equiv1\ \pmod \lambda$. Thus
\begin{align*}
m+n\omega &\equiv m+n\ \pmod \lambda\\
3(m+n \omega) &\equiv 3 (m+n) \pmod{3\lambda}\\
\gamma = -1 + 3(m+n\omega) &\equiv -1+3(m+n) \pmod{3 \lambda}
\end{align*}
Moreover $9 = 3 \lambda \bar{\lambda} \equiv 0 \pmod{3 \lambda}$, thus $\gamma$ is congruent modulo $3\lambda$ to an integer between  $0$ and $8$ of the form $3k-1$ : $\gamma \equiv 8,2$ or $5 \pmod{3\lambda}$.

By Ex. 9.3, $\chi_\gamma(\omega) = 1 \iff m+n \equiv 0 \ [3]$, and $m+n  \equiv 0 \ [3]$ implies $m+n = 3k, k \in \Z$, so $\gamma  \equiv -1+ 9k \equiv -1 \equiv 8 \ [3\lambda]$.

Conversely, if $\gamma \equiv 8 \equiv -1 \ [3\lambda]$, then $3\lambda \mid 3(m+n)$, so $\lambda \mid m+n$, and $\n(\lambda) \mid\n(m+n)$, $3 \mid (m+n)^2$, thus $3 \mid m+n$,  $m+n \equiv 0 \ [3]$, and so $\chi_\gamma(\omega) = 1$. The two other cases are similar, so we obtain

\begin{align*}
\chi_\gamma(\omega) = 1 &\iff m+n \equiv 0\ [3] \iff \gamma \equiv 8 \ [3\lambda],\\
\chi_\gamma(\omega) = \omega &\iff m+n \equiv 1\ [3] \iff \gamma \equiv 2 \ [3\lambda],\\
\chi_\gamma(\omega) = \omega^2 &\iff m+n \equiv 2\ [3] \iff \gamma \equiv 5 \ [3\lambda].
\end{align*}

If $\gamma = q$ is a rational prime,
$q \equiv 8 \ [9]$ implies $q \equiv 8\ [3\lambda]$, since $3 \lambda \mid 9 = 3 \lambda \bar{\lambda}$, thus $\chi_q(\omega) = 1$.

Conversely, if $\chi_q(\omega) = 1$, then $q \equiv 8\ [3\lambda]$, $q-8 = \mu (3 \lambda), \mu \in D$, therefore 

$(q-8)^2 = N(\mu) 3^3, 3^3 \mid (q-8)^2$, thus $3^2 \mid q-8$ and so $q \equiv 8 \ [9]$. The two other cases are similar.

\begin{align*}
\chi_q(\omega) = 1 & \iff q \equiv 8 \ [9],\\
\chi_q(\omega) = \omega & \iff q \equiv 2 \ [9],\\
\chi_q(\omega) = \omega^2 &\iff q \equiv 5 \ [9].
\end{align*}
\end{proof}

\paragraph{Ex. 9.5}
{\it In the text we stated Eisenstein's result $\chi_\gamma(\lambda) = \omega^{2m}$. Show that $\chi_{\gamma}(3) = \omega^{2n}$.
}

\begin{proof} Here $\gamma = (3m-1) + 3n\omega$.

Note that $(1-\omega)^2 = -3 \omega$, thus $\chi_\gamma((1-\omega)^2) = \chi_\gamma(-1)\chi_\gamma(3)\chi_\gamma(\omega)$.

Using Eisenstein's result (see a proof in Ex.24-26),
$$\chi_\gamma((1-\omega)^2)= \chi_\gamma(\lambda^2) =\chi_\gamma(\lambda)^2 =\omega^{4m} = \omega^m.$$

As $-1 = (-1)^3, \chi_\gamma(-1) = 1$. Finally  $\chi_\gamma(\omega) = \omega^{m+n}$ by Exercise 9.3. Thus

$$\omega^m = \chi_\gamma(3) \omega^{m+n},\qquad  \chi_\gamma(3) = \omega^{-n} = \omega^{2n}.$$

In conclusion,
$$\chi_\gamma(3) = \omega^{2n}.$$
\end{proof}

\paragraph{Ex. 9.6}

{\it Prove that
\begin{enumerate}
\item[(a)] $\chi_\gamma(\lambda) = 1$ for $\gamma \equiv 8, 8 + 3\omega, 8 + 6 \omega \ [9]$.
\item[(b)] $\chi_\gamma(\lambda) = \omega$ for $\gamma \equiv 5, 5 + 3\omega, 5 + 6 \omega \ [9]$.
\item[(c)]$\chi_\gamma(\lambda) = \omega^2$ for $\gamma \equiv 2, 2 + 3\omega, 2 + 6 \omega \ [9]$.
\end{enumerate}
}

\begin{proof}
Here $\gamma = -1 + 3(m+n\omega)$ is a primary prime, and $\chi_\gamma(\lambda) = \omega^{2m}$.
\begin{align*}
\chi_\gamma(\lambda)=1 &\iff m\equiv0\ [3] \Rightarrow \gamma \equiv 8+3n\omega\ [9]\Rightarrow \gamma\equiv8,8+3\omega,8+6\omega\ [9]\\
\chi_\gamma(\lambda)=\omega&\iff m\equiv2\ [3] \Rightarrow \gamma \equiv5+3n\omega\ [9]\Rightarrow \gamma\equiv5,5+3\omega,5+6\omega\ [9]\\
\chi_\gamma(\lambda)=\omega^2& \iff m\equiv1\ [3] \Rightarrow \gamma \equiv2+3n\omega\ [9]\Rightarrow \gamma\equiv2,2+3\omega,2+6\omega\ [9]\\
\end{align*}

As $\chi_\gamma(\lambda) \in \{1,\omega,\omega^2\}$, these 9 cases are the only possibilities. Moreover these 9 cases are mutually exclusive, since $9$ doesn't divide  any difference. Thus the reciprocals are true.
\begin{align*}
\chi_\gamma(\lambda)=1 &\iff \gamma\equiv8,8+3\omega,8+6\omega\ [9]\\
\chi_\gamma(\lambda)=\omega&\iff \gamma\equiv5,5+3\omega,5+6\omega\ [9]\\
\chi_\gamma(\lambda)=\omega^2&\iff \gamma\equiv2,2+3\omega,2+6\omega\ [9]\\
\end{align*}
\end{proof}

\paragraph{Ex. 9.7}

{\it Find primary primes associate to $1-2\omega, -7-3\omega$, and $3-\omega$.
}

\begin{proof} 
: \\
\begin{enumerate}
\item[$\bullet$] $(1-2\omega)\omega = 2 + 3 \omega \equiv 2 \pmod 3$, so $2+3\omega$ is primary, and associate to $1-2\omega$.
$\n(2+3\omega) = 7$ and 7 is a rational prime, thus $2 + 3\omega$ is a primary prime.
\item[$\bullet$] $-7 - 3 \omega \equiv 2 \pmod 3$.

$\n(-7-3\omega) = 37$ and 37 is a rational prime, thus $-7 - 3\omega$ is a primary prime.
\item[$\bullet$] $(3-\omega)\omega^2 = -4 - 3\omega \equiv 2 \pmod 3$, so $-4-3\omega$ is primary, and  associate to $3-\omega$.

$\n(-4 - 3 \omega) = 13$ and 13 is a rational prime, thus $-4 - 3 \omega$ is a primary prime.
\end{enumerate}
\end{proof}

\paragraph{Ex. 9.8}

{\it Factor the following numbers into primes in $D$ : $7,21,45,22$, and $143$.
}

\begin{proof}
$7=\n(2+3\omega)$, thus $ 7 = (2+3\omega)(2+3\omega^2) = (2+3\omega)(-1-3\omega)$, where $2 + 3\omega$ and $-1-3\omega$ are primes in $D$, since their norm is a prime integer. Since these primes are primary, they are not associate.

$21 = 3\times 7 = -\omega^2 \lambda^2 (2+3\omega)(-1-3\omega)$  since $3=-\omega^2(1-\omega)^2$.

$45 = 3^2\times 5 = \omega \lambda^4 5$, where $5 \equiv 2 \pmod 3$ is a primary prime in $D$.

$22=2\times 11$ , where 2 and 11 are primes in $D$.

$143 = 11\times 13 = 11(-4-3\omega)(-4-3\omega^2) = 11  (-4-3\omega)(-1+3\omega)$.
\end{proof}

\paragraph{Ex. 9.9}
{\it Show that $\overline{\alpha} \ne 0$, the residue class of $\alpha$, is a cube in the field $D/\pi D$ iff $\alpha^{(N\pi -1)/3} \equiv 1 \pmod \pi$. Conclude that there are $(N\pi - 1)/3$ cubes in $(D/\pi D)^*$.
}

\bigskip
Solution 1:
\begin{proof}
Let $\pi$ be a prime in $D$, $N\pi \ne 3$, and $\alpha \in D,\pi \nmid \alpha$.

$\overline{\alpha}$ is a cube in $(D/\pi D)^*$

$\iff x^3 \equiv \alpha \pmod \pi$ has a solution in $D$

$\iff \chi_{\pi}(\alpha)= 1$ \qquad \qquad (by Prop. 9.3.3(a))

$\iff \alpha^{\frac{N\pi-1}{3}} \equiv 1 \pmod \pi$

$\iff \overline{\alpha}^{\frac{N\pi-1}{3}}  = \overline{1}$.

The cubes in $(D/\pi D)^*$ are then the roots of the polynomial $f(x) = x^{\frac{N\pi-1}{3}} - \overline{1}$ in $D/\pi D$. 

Let $q$ be the cardinal of the field $D/\pi D$. Since $q = |D/\pi D| = N\pi$, $\frac{N\pi - 1}{3} \mid q-1$, $f(x) \mid x^{q-1}-1 \mid x^q-x$. By Corollary 2 of Proposition 7.1.1, $f$ has $\deg(f) = \frac{N\pi-1}{3}$ roots.

Conclusion: there are exactly $\frac{N\pi-1}{3}$ cubes in $(D/\pi D)^*$.
\end{proof}

Solution 2:
\begin{proof}
Let $\varphi : (D/\pi D)^* \to (D/\pi D)^*$ be the group homomorphism defined by $\varphi(x) = x^3$.

Then $\mathrm{im}(\varphi)$ is the set of cubes in $(D/\pi D)^*$.

The equation $x^3 = \overline{1}$ has three distinct solutions $\overline{1}, \overline{\omega}, \overline{\omega}^2$  in $D/ \pi D$ if $N\pi \ne 3$ (see the demonstration of Proposition 9.3.1).

Thus $\ker(\varphi)= \{\overline{1}, \overline{\omega}, \overline{\omega}^2\}$ and $| \ker(\varphi)| = 3$. Therefore $| \mathrm{im} (\varphi) | = | (D/\pi D)^* | / |\ker(\varphi) | = (N\pi - 1)/3$. There exist exactly $\frac{N\pi-1}{3}$ cubes in $(D/\pi D)^*$.
\end{proof}

Note: if $N\pi = 3$, that is to say, if $\pi$ is associate to $1-\omega$, $D/\pi D = \{\overline{0},\overline{1}, \overline{2}\}$. As $\overline{1}^3 = \overline{1}, \overline{2}^3 = \overline{2}$, all the elements of $(D/\pi D)^*$ are cubes.

\paragraph{Ex. 9.10}

{\it What is the factorisation of $x^{24}-1$ in $D/5D$.
}

\begin{proof}
$\vert (D/5D)^*\vert = N(5) - 1 = 24$, thus $x^{24} - 1 = \prod\limits_{\alpha \in (D/ 5 D)^*} (x - \alpha)$.

(where the $\alpha \in (D/ 5 D)^*$ are of the form $\alpha = a + b\, [\omega],\  0 \leq a <5, 0 \leq b <5, (a,b) \ne (0,0))$.
\end{proof}

\paragraph{Ex. 9.11}

{\it How many cubes are there in $D/5D$ ?
}

\begin{proof}
By Exercise 9.9, there exist $(N(5) - 1)/3 = 8$ cubes in $(D/5D)^*$ (and $0 = 0^3$ is a cube).
\end{proof}

\paragraph{Ex. 9.12}

{\it Show that $\omega \lambda$ has order 8 in $D/5D$ and that $\omega^2 \lambda$ has order 24. [Hint : Show first that $(\omega \lambda)^2$ has order 4.]
}

\begin{proof} If $\alpha = (\omega \lambda)^2$, then
$$\alpha = (\omega \lambda)^2 = \omega^2 (1-\omega)^2 = \omega^2(1 + \omega^2-2 \omega) = -3\omega^3= -3.$$

So $\alpha^2 =9 \equiv -1\pmod 5, \alpha^4 \equiv 1 \pmod 5$ and $\alpha^2 \not \equiv 1 \pmod 5$, thus the class of $\alpha= (\omega \lambda)^2$ has order 4 in $(D/5D)^*$, and this implies that $\omega \lambda$ has order 8.

Let $\beta = \omega^2 \lambda$. $\vert (D/5 D)^* \vert = 24$, thus  $[\beta]^{24} = 1$ (where $[\beta]$ is the class of $\beta$ in $D/5D$.)

To verify that $[\beta]$ has order 24, it is sufficient to show that $[\beta]^{8}\neq 1,[\beta]^{12}\neq 1$.

$\beta^8 = \omega ^{16} \lambda^8 = \omega \lambda^8 = (\omega \lambda)^8 \omega^2 \equiv  \omega^2 \not \equiv 1 \pmod 5$.

$\beta^{12} = (\omega^2 \lambda)^{12} = \lambda^{12} = (\omega \lambda)^{12} \equiv  (\omega \lambda)^{4}  \equiv -1 \pmod 5$ (since $(\omega \lambda)$ has order 8 in $D/5D$).


Conclusion: $\omega \lambda$ has order 8, $\omega \lambda^2$ has order 24 in $(D/5D)^*$.
\end{proof}

\paragraph{Ex. 9.13}

{\it Show that $\pi$ is a cube in $D/5D$ iff $\pi \equiv 1,2,3,4,1+2\omega, 2 + 4\omega, 3 + \omega$, or $4+3\omega \pmod 5$.
}

\begin{proof}
Let $\pi \in D, [\pi] \neq 0$. Then $[\pi] $ is a cube in $D/5D$ iff $[\pi]^{(q^2-1)/3} = 1$, with $q = 5$, namely $[\pi]^8 = 1$ (Prop. 7.1.2, where $3 \mid q^2-1 = 24 = |(D/5D)^*| $).

By Exercise 9.12, the class of $\gamma = \omega \lambda$ has order 8, thus the 8 elements $[\gamma]^k, 0 \leq k \leq 7$ are distinct roots of the polynomial $x^8-1$, which has at most 8 roots. Therefore  the subgroup of cubes in $(D/5D)^*$ is
$$\{1 ,[\gamma], [\gamma]^2,\ldots, [\gamma]^7\}.$$
$\gamma=\omega(1-\omega) = \omega+1+\omega = 1 + 2\omega$, so
\begin{align*}
\gamma^0 &= 1\\
\gamma^1 & = 1 + 2\omega\\
\gamma^2 &\equiv -3 \equiv 2\ [5]\qquad (\mathrm{Ex}.\ 9.12)\\
\gamma^3 &= -3 - 6\omega \equiv 2 + 4 \omega \ [5]\\
\gamma^4&\equiv -1 \equiv 4\ [5]\\
\gamma^5 &\equiv -1 - 2 \omega \equiv 4 + 3\omega\ [5]\\
\gamma^6 &\equiv 3 \ [5]\\
\gamma^7 &\equiv 3+6\omega \equiv 3 + \omega \ [5]
\end{align*}
Conclusion: If $\pi \not \equiv 0 \pmod 5$, $\pi \equiv \alpha^3 \pmod 5,\alpha \in D$ iff 

$$\pi \equiv 1,2,3,4,1+2\omega,2+4\omega,3+\omega,4+3\omega\ [5].$$ 
\end{proof}

\paragraph{Ex. 9.14}

{\it For which primes $\pi \in D$ is $x^3 \equiv 5 \pmod \pi$ solvable ?
}

\begin{proof}
If $\pi$ is associate to 5, then $5^3 \equiv 0 \equiv 5 \pmod \pi$, so $x^3 \equiv 5 \pmod \pi$ is solvable.

If $\pi$ is a primary prime not associate to 5, the Law of Cubic Reciprocity gives
\begin{align*}
5\equiv x^3 \ [\pi], x \in D &\iff \chi_\pi(5)=1\\
&\iff \chi_5(\pi)=1\\
&\iff  \pi \ \mathrm{ is\ a\ cube\ in\  } D/5D\\
&\iff \pi \equiv1,2,3,4,1+2\omega,2+4\omega,3+\omega,4+3\omega \ [5]
\end{align*}
(see Ex. 9.13)

\bigskip

Conclusion: the equation $5\equiv x^3 \ [\pi], x \in D $ is solvable iff the primary prime associate to $\pi$ is congruent modulo 5 to $1,2,3,4,1+2\omega,2+4\omega,3+\omega,4+3\omega$ (or 0).

Examples: 

$\bullet$ $q=23$ is a primary prime congruent to 3 modulo 5, thus the equation $x^3 \equiv 5\pmod{23}$ has a solution $x  \in D$ ($x = 19)$.

$\bullet$ $-4 - 3 \omega$ is the primary prime associate to the prime $3- \omega$, and $-4 - 3 \omega \equiv 1 + 2 \omega \pmod 5$, thus the equation $x^3 \equiv 5 \pmod {3-\omega}$ has a solution $a + b \omega \in \Z[\omega]$.

Indeed, $7^3 \equiv 8^3 \equiv 11^3 \equiv 5 \pmod {13}$, and $3 - \omega \mid 13$, so $7^3 \equiv 8^3 \equiv 11^3 \equiv 5 \pmod{ 3 - \omega}$.
\end{proof}

\paragraph{Ex. 9.15}

{\it Suppose that $p \equiv 1 \pmod 3$ and that $p = \pi \overline{\pi}$, where $\pi$ is a primary prime in $D$. Show that $x^3 \equiv a \pmod p$ is solvable in $\Z$ iff $\chi_{\pi}(a) = 1$. We assume that $a \in \Z$.
}

\begin{proof}
Since $\pi \mid p$, if $ x^3 \equiv a \pmod p,\ x \in \Z$, then $ x^3 \equiv a \pmod \pi$, thus $\chi_{\pi}(a) = 1$.

Conversely, suppose that $\chi_{\pi}(a) = 1$. Then the equation $y^3 \equiv a \pmod \pi$ has a solution $y = u + v \omega, \ u,v \in \Z$. Moreover, the class of $y$ has a representative $x \in \Z$ modulo $\pi$ (see the proof of Proposition 9.2.1) :
$$y\equiv  x \pmod \pi, x \in \Z.$$

So $x^3 \equiv a \pmod \pi$ has a solution $x \in \Z$.

Thus $\pi \mid x^3 - a, N(\pi) = p \mid (x^3 -a)^2$, therefore $p \mid x^3 -a$ in $\Z$, and so $x^3 \equiv a \pmod p$.

Conclusion: if $p\equiv 1 \pmod 3$, $p = \pi \overline{\pi}$, where $\pi$ is a primary prime,  and $a \in \Z$,
$$\exists x \in \Z, \ x^3 \equiv a \pmod p \iff \chi_{\pi}(a) = 1.$$
In other words, $x^3 \equiv a \pmod \pi$ is solvable in $D$ iff it is solvable in $\Z$.
\end{proof}

\paragraph{Ex. 9.16}

{\it Is $x^3 \equiv 2 - 3\omega \pmod {11}$ solvable ? Since $D/11D$ has 121 elements this is hard to resolve by straightforward checking. Fill in the details of the following proof that it is not solvable. $\chi_{\pi}(2 - 3 \omega) = \chi_{2-3\omega}(11)$ and so we shall have a solution iff $x^3 \equiv 11 \pmod {2-3\omega}$ is solvable. This congruence is solvable iff $x^3 = 11\pmod 7$ is solvable in $\Z$. However, $x^3 \equiv a \pmod 7$ is solvable in $\Z$ iff $a \equiv 1$ or $6 \pmod 7$.
}

\bigskip
Warning: false sentence, since 
$$N(2 - 3 \omega) = (2-3\omega)(2 - 3 \omega^2) = 4 + 9 -6(\omega+\omega^2) = 4 + 9 + 6 = 19 \ (\mathrm{and}\ \mathrm{not}\ 7!).$$
\begin{proof}
Since $19$ is a rational prime, and since $\pi= 2 - 3\omega$ and $11$  are primary primes, by the Law of Cubic Reciprocity, and by Exercise 9.15 (with $p=19 \equiv 1 \pmod 3$),
\begin{align*}
\exists x \in D,\ 2-3\omega \equiv x^3\ [11]&\iff \chi_{11}(2-3\omega) = 1\\
&\iff \chi_{2-3\omega}(11)=1\\
&\iff \exists x \in D,\  x^3 \equiv 11\  [2 - 3 \omega]\\
&\iff \exists x \in \mathbb{Z},\  x^3 \equiv 11\  [19]
\end{align*}
Moreover, by Proposition 7.1.2 (with $p = 19$, $d = (p-1) \wedge 3 = 3, (p-1)/d = 6$),
$$\exists x \in \mathbb{Z},\  x^3 \equiv 11\  [19] \iff 11^6 \equiv 1 \pmod {19},$$
which is true : $11^6 = 121^3 = (19 \times 6 + 7)^3 \equiv 49 \times 7 \equiv 11 \times 7 \equiv 77 \equiv 1 \ [19]$.

Conclusion: there exists $x \in D$ such that $2-3\omega  \equiv x^3 \pmod {11}$.

With some computer code, we find a solution $x = 1 + 8\omega$ (and its associates $\omega^2 x =  7 - \omega, \omega x = -8- 7\omega \equiv 3 + 4 \omega \pmod{11} $) :
$$x^3 = (1+8\omega)^3 = 321 - 168 \omega \equiv 2 - 3 \omega \pmod {11}.$$
\end{proof}

Note: The sentence becomes true if we replace $2 - 3\omega$ by the primary prime $2+3\omega$. Since $N(2+3\omega) = 7$, with the same reasoning,
\begin{align*}
\exists x \in D,\ 2+3\omega \equiv x^3\ [11]
&\iff \chi_{2+3\omega}(11)=1\\
&\iff \exists x \in D,\  x^3 \equiv 11\  [2 + 3 \omega]\\
&\iff \exists x \in \mathbb{Z},\  x^3 \equiv 11 \equiv 4\  [7]\\
&\iff 4^2 \equiv 1 \pmod 7
\end{align*}
but $4^2 \equiv 2 \not \equiv 1 \pmod 7$, so the equation $x^3 \equiv 2 + 3\omega \pmod {11}$ is not solvable.

($x^3 \equiv a \pmod {11}$ is solvable in $\Z$ iff $a^\frac{7-1}{3} = a^2 \equiv 1 \pmod 7$ iff $a \equiv \pm 1 \pmod 7$.)



\paragraph{Ex. 9.17}

{\it An element $\gamma \in D$ is called primary if $\gamma \equiv2 \pmod 3$. If $\gamma$ and $\rho$ are primary, show that $-\gamma \rho$ is primary. If $\gamma$ is primary, show that $\gamma = \pm \gamma_1\gamma_2\ldots \gamma_t$, where the $\gamma_i$ are (not necessarily distinct) primary primes.
}

\begin{proof}
If $\gamma \equiv 2, \rho \equiv 2 \pmod 3$, then $-\gamma \rho \equiv -2 \times 2 \equiv 2 \pmod 3$, so $-\gamma \rho$ is primary.

\bigskip

By Ex. 9.2, $\gamma$ can be  written 
$$\gamma = (-1)^a \omega^b \lambda^c \pi_1^{a_1}\cdots \pi_t^{a_t},$$
where $\pi_i \equiv 2 \pmod 3, a \in \{0,1\}, b \in \{0,1,2\}$.

As $\pi_i \equiv -1 \pmod 3$, and $\gamma \equiv -1 \pmod 3$, we obtain $\omega^b \lambda^c \equiv \pm 1 \pmod 3$. We prove that $b = c = 0$.

Note that $\lambda^2 = (1-\omega)^2 = -3\omega \equiv 0 \pmod 3$. If $c \geq 2$, we would obtain $\gamma \equiv 0 \pmod 3$, in contradiction with the hypothesis, thus $c = 0$ or $c = 1$.

If $c=1$, $$\omega^b\lambda^c \in \{1-\omega, \omega(1-\omega),\omega^2(1-\omega)\} = \{1 - \omega, 1+2\omega, - 2 - \omega\}.$$ Since $1-\omega \not \equiv \pm 1, 1+2\omega \not \equiv \pm 1, - 2 - \omega \not \equiv \pm 1 \pmod 3$, this is impossible, so $c=0$.

Then $\omega^b \equiv \pm 1 \pmod 3$, where $\omega^b \in \{1,\omega, -1-\omega\}$. Since $\omega \not \equiv \pm1 \pmod 3$, and $-1 - \omega \not \equiv \pm 1 \pmod 3$, then $\omega^b = 1, 0 \leq b \leq 2$, thus $b=0$.

Finally, $\gamma = (-1)^a \pi_1^{a_1}\cdots \pi_t^{a_t}$.

Conclusion: Every primary $\gamma \in D$ is under the form
$$\gamma = \pm \gamma_1\gamma_2\cdots \gamma_t,$$
where the $\gamma_i$ are primary primes.

\end{proof}

\paragraph{Ex. 9.18}

{\it (continuation) If $\gamma = \pm \gamma_1 \gamma_2\cdots \gamma_t$ is a primary decomposition of the primary element $\gamma$, define $\chi_{\gamma}(\alpha) = \chi_{\gamma_1}(\alpha)\chi_{\gamma_2}(\alpha)\cdots\chi_{\gamma_t}(\alpha)$. Prove that $\chi_{\gamma}(\alpha) = \chi_{\gamma}(\beta)$ if $\alpha \equiv \beta \pmod \gamma$ and $\chi_{\gamma}(\alpha \beta) = \chi_\gamma(\alpha) \chi_\gamma(\beta)$. If $\rho$ is primary, show that $\chi_\rho(\alpha) \chi_\gamma(\alpha) = \chi_{-\rho \gamma}(\alpha)$.
}

\begin{proof}
If $\alpha \equiv \beta\ [\gamma]$, then $\alpha \equiv \beta\pmod {\gamma_i}, 1 \leq i \leq t$, so $\chi_{\gamma_i}(\alpha) = \chi_{\gamma_i}(\beta)$, thus $\chi_\gamma(\alpha) = \chi_\gamma(\beta)$.

By Proposition 9.3.3,
\begin{align*}
\chi_\gamma(\alpha \beta) &= \chi_{\gamma_1}(\alpha \beta)\chi_{\gamma_2}(\alpha \beta)\cdots\chi_{\gamma_t}(\alpha \beta)\\
&= \chi_{\gamma_1}(\alpha)\chi_{\gamma_2}(\alpha)\cdots\chi_{\gamma_t}(\alpha)\chi_{\gamma_1}(\beta)\chi_{\gamma_2}(\beta)\cdots\chi_{\gamma_t}(\beta)\\
&= \chi_\gamma(\alpha) \chi_{\gamma}(\beta)
\end{align*}
Finally, if $\rho = \pm \rho_1\rho_2\cdots\rho_l$ is primary, then $-\rho \gamma = \pm \rho_1\rho_2\cdots\rho_l\gamma_1\gamma_2\cdots\gamma_t$ is primary by Ex. 9.17, therefore

$$\chi_{-\rho \gamma}(\alpha) = (\chi_{\rho_1} \chi_{\rho_2}\cdots \chi_{\rho_l}\chi_{\gamma_1} \chi_{\gamma_2}\cdots \chi_{\gamma_t} )(\alpha)= \chi_\rho(\alpha)\chi_\gamma(\alpha).$$
\end{proof}

Note: The unit $-1$ is primary by definition, and $-1$ is the opposite of the empty product, so for all $\alpha$ in $D$, $\chi_{-1}(\alpha) = 1$ by definition. The result of the exercises remain true if we accept the unit $-1$ as a primary element.


\paragraph{Ex. 9.19}

{\it Suppose that $\gamma = A + B \omega$ is primary and that $A = 3M-1$ and $B = 3N$. Prove that $\chi_\gamma(\omega) = \omega^{M+N}$ and that $\chi_\gamma(\lambda) = \omega^{2M}$.
}

\begin{proof}
We verify first that if $\gamma = -\gamma_1 \gamma_2$, with
$$
\begin{array}{lll}
  \gamma = A+B\omega, & A=3M-1,  & B=3N,  \\
   \gamma_1 = A_1+B_1\omega,& A_1= 3M_1-1,  &  B_1=3N_1, \\
  \gamma_2 = A_2+B_2\omega,&  A_2= 3M_2-1, &  B_2=3N_2, 
\end{array}
$$
then $M \equiv M_1+M_2\pmod 3,N \equiv N_1+N_2 \pmod 3$.
$$-\gamma_1 \gamma_2 = -A_1A_2 + B_1 B_2  + (- A_1B_2-A_2B_1+B_1B_2) \omega=A+B\omega,$$
therefore
$$3M-1 = A =-A_1A_2 + B_1 B_2 \equiv 3(M_1+M_2) - 1\pmod 9,$$
thus $M\equiv M_1+M_2\pmod 3.$
$$3N = B = -A_1B_2-A_2B_1+B_1B_2 \equiv 3 (N_1+N_2)\pmod 9,$$ 
thus $N \equiv N_1+N_2 \pmod 3.$

By induction, if $\gamma = \pm \gamma_1 \gamma_2\cdots \gamma_t = (-1)^{t-1} \gamma_1 \gamma_2\cdots \gamma_t$, where $\gamma_i = A_i+B_i\omega,A_i= 3M_i-1,B_i=3N_i$, then
$$M \equiv M_1+\cdots+M_t\pmod 3, N \equiv N_1+\cdots + N_t \pmod 3.$$

By Exercise 9.3,
\begin{align*}
\chi_\gamma(\omega) &= \chi_{\gamma_1}(\omega)\cdots\chi_{\gamma_t}(\omega)\\
&=\omega^{M_1+N_1}\cdots \omega^{M_t+N_t}\\
&=\omega^{(M_1+\cdots+M_t)+(N_1+\cdots+N_t)}\\
&=\omega^{M+N},
\end{align*}

and by Eisenstein's result,
\begin{align*}
\chi_\gamma(\lambda) &= \chi_{\gamma_1}(\lambda)\cdots\chi_{\gamma_t}(\lambda)\\
&=\omega^{2M_1}\cdots \omega^{2M_t}\\
&=\omega^{2(M_1+\cdots+M_t)}\\
&=\omega^{2M}.
\end{align*}

Conclusion: if $\gamma = 3M-1+3N\omega$, then

$$\chi_\gamma(\omega) = \omega^{M+N}, \chi_\gamma(\lambda)=\omega^{2M}.$$
\end{proof}

\paragraph{Ex. 9.20}

{\it If $\gamma$ and $\rho$ are primary, show that $\chi_\gamma(\rho) = \chi_\rho(\gamma)$.
}

\bigskip


Important note: The following solution assumes that $\chi_{\pi_2}(\pi_1) = \chi_{\pi_1}(\pi_2)$ for any pair $\pi_1,\pi_2$ of primary primes. But Theorem 1 uses the hypothesis $N(\pi_1) \ne N(\pi_2)$ to prove Cubic Reciprocity.

We can complete the proof in the case where $N(\pi_1) = N(\pi_2)$. Since $\pi_1,\pi_2$ are primary primes, then $\pi_1 = \pi_2$ or $\pi_1 = \overline{\pi_2}$.

In the case $\pi_1 = \pi_2$, then $\chi_{\pi_2}(\pi_1) = 0 =  \chi_{\pi_1}(\pi_2)$.

To prove that $\legendre{\overline{\pi}}{\pi} = \legendre{\pi}{\overline{\pi}}$, we begin with a particular case of the proposition:

\medskip
{\bf Lemma.} {\it Let $n\in \Z$ be a primary element in $A$, and let $\pi$ be a primary prime such that $N(\pi) = p \equiv 1 \pmod 3$. Then
$$\legendre{n}{\pi}_3 = \legendre{\pi}{n}_3.$$
}



\begin{proof}
If $p \mid n$, then $\legendre{n}{\pi}_3 =0 =  \legendre{\pi}{n}_3.$ Now we assume that $p \wedge n = 1$.

The decomposition of $n$ is of the form
\begin{align*} n &= \pm p_1 \cdots p_s q_1 \cdots q_r\ (p_i \equiv 1 \ [3], q_j \equiv -1 \ [3])\\
&= \pm \pi_1\overline{\pi_1} \cdots \pi_s \overline{\pi}_s q_1 \cdots q_r,
\end{align*}
where $\pi_i, \overline{\pi_i} (1\leq i \leq s)$ and $q_j (1 \leq j \leq r)$ are primary prime.

Since $N(\pi_i) = p_i \ne p$ and $N(\pi) = p \ne N(q_j) =q_j^2$, Theorem 1 shows that
\begin{align*}
\legendre{n}{\pi}_3 &=\legendre{\pi_1}{\pi}_3\legendre{\overline{\pi_1}}{\pi}_3\cdots \legendre{\pi_s}{\pi}_3\legendre{\overline{\pi_s}}{\pi}_3 \legendre{q_1}{\pi}_3  \cdots \legendre{q_r}{\pi}_3 \\
&=\legendre{\pi}{\pi_1}_3\legendre{\pi}{\overline{\pi_1}}_3\cdots \legendre{\pi}{\pi_s}_3\legendre{\pi}{\overline{\pi_s}}_3 \legendre{\pi}{q_1}_3  \cdots \legendre{\pi}{q_r}_3 \\
&= \legendre{\pi}{n}_3.
\end{align*}
\end{proof}

We can now remove the useless hypothesis $N(\pi_1) \ne N(\pi_2)$ in Theorem 1.

\medskip

{\bf Proposition.}
Let $\pi_1,\pi_2$ be primary primes. Then
$$\legendre{\pi_2}{\pi_1}_3 = \legendre{\pi_1}{\pi_2}_3.$$


\begin{proof}
By theorem 1, it remains only the case where $N(\pi_1) = N(\pi_2)$.

If $\pi_1 = \pi_2$, then $\legendre{\pi_2}{\pi_1}_3 = \legendre{\pi_1}{\pi_2}_3 =0$.

If $\pi_1 \ne \pi_2$, since $\pi_1$ et $\pi_2$ are primary, then $\pi_1, \pi_2$ are primes such that $N(\pi_1) = N(\pi_2) = p \equiv 1 \pmod 3$, and $\pi_2 = \overline{\pi_1}$. Writing $\pi = \pi_1$, it is sufficient to prove
$$\legendre{\overline{\pi}}{\pi}_3 =  \legendre{\pi}{\overline{\pi}}_3.$$
We use the ``Evans' trick'' (see [Lemmermayer, Reciprocity Laws p. 215]). The element $n =-( \pi + \overline{\pi})$ is a  rationnal integer, which is primary. The Lemma gives then
\begin{align*}
\legendre{\overline{\pi}}{\pi}_3 &= \legendre{\pi + \overline{\pi}}{\pi}_3\\
&= \legendre{-\pi - \overline{\pi}}{\pi}_3\\
&= \legendre{\pi}{-\pi - \overline{\pi}}_3\\
&= \legendre{-\overline{\pi}}{-\pi - \overline{\pi}}_3\\
&=\legendre{\overline{\pi}}{-\pi - \overline{\pi}}_3\\
&=\legendre{-\pi  -\overline{\pi}}{\overline{\pi}}_3\\
&=\legendre{-\pi}{\overline{\pi}}_3\\
&=\legendre{\pi}{\overline{\pi}}_3
\end{align*}
\end{proof}

We can now give the solution of Exercise 9.20.
\begin{proof}


$\rho, \gamma$ are written
\begin{align*}
\rho&=\pm\rho_1\rho_2\cdots\rho_l,\\
\gamma &= \pm \gamma_1 \gamma_2\cdots \gamma_m, 
\end{align*}
 where $\rho_i,\gamma_i$ are primary primes.
By the law of Cubic Reciprocity, we obtain
\begin{align*}
\chi_\gamma(\rho) &= \prod_{j=1}^m \chi_{\gamma_j}(\rho)\\
& = \prod_{j=1}^m\prod_{i=1}^l \chi_{\gamma_j}(\rho_{i})\\
& = \prod_{i=1}^l\prod_{j=1}^m \chi_{\gamma_j}(\rho_{i})\\
& = \prod_{i=1}^l\prod_{j=1}^m \chi_{\rho_i}(\gamma_j)\\
&=\prod_{i=1}^l \chi_{\rho_i}(\gamma)\\
&=\chi_{\rho}( \gamma).
\end{align*}

(if $\gamma = -1$, or $\rho = -1$, some products are empty, but the result remains true: $\chi_{-1}(\rho) = 1 = \chi_{\rho}(-1)$.)



\end{proof}




\paragraph{Ex. 9.21}
{\it If $\gamma$ is primary, show that there are infinitely many primary primes $\pi$ such that $x^3 \equiv \gamma \pmod \pi$ is not solvable. Show also that there are infinitely many primary primes $\pi$ such that $x^3 \equiv \omega \pmod \pi$ is not solvable and the same for $x^3 \equiv \lambda \pmod \pi$. (Hint: Imitate the proof of Theorem 3 of Chapter 5.)
}

\begin{proof}
\begin{enumerate}
\item[a)] As some primary elements of $D$ may be cubes, by example $53 + 36 \omega = (-1 + 3 \omega)^3$, we must of course suppose that $\gamma$ is not the cube of some element of $D$ (in the contrary case $x^3 \equiv \gamma \pmod \pi $ is solvable for all prime $\pi$).

Note first that for all primes $\pi$ in $D$, there exists $\sigma \in D$ such that $\chi_\pi(\sigma) = \omega$. Indeed, there exist $(N\pi - 1)/3$ cubes in $(D/\pi D)^*$, which has $N\pi - 1$ elements, so there exists an element $\overline{\tau} \in (D/\pi D)^*$ which is not a cube, therefore there exists $\tau \in D$ such that $\chi_\pi(\tau) \neq 1$. If $\chi_\pi(\tau) = \omega$, we put $\sigma = \tau$ and if $\chi_\pi(\tau) = \omega^2$, we put $\sigma = \tau^2$. In the two cases, $\chi_\pi(\sigma) = \omega$.

\bigskip

Let $\gamma \in D$, where $\gamma$ is primary. Then $\gamma = \pm \gamma_1^{n_1}\gamma_2^{n_2}\cdots\gamma_p^{n_p}$, where the $\gamma_i$ are distinct primary primes. 
Write $n_i = 3q_i + r_i, \ r_i \in \{0,1,2\}$. Then grouping in $\gamma'$ the $\gamma^{r_i}$ such that $r_i \ne 0$, we can write $\gamma = \delta^3 \gamma', \gamma' = \gamma_1^{r_1} \gamma_2^{r_2}\cdots \gamma_l^{r_l}, r_i \in \{1,2\}, \delta = \pm \gamma_1^{q_1}\cdots\gamma_p^{q_p} \in D$ ($-1$ is a cube). Since by hypothesis $\gamma$ is not a cube, $l\geq 1$. Moreover the equation $x^3 \equiv \gamma \pmod \pi$ is solvable iff $x^3 \equiv \gamma' \pmod \pi$ is solvable. We may then suppose that $$\gamma = \gamma_1^{r_1} \gamma_2^{r_2}\cdots \gamma_l^{r_l}, 1\leq r_i \leq 2,$$ without cubic factors.

Note that the $\gamma_i$ are not associate to $\lambda = 1-\omega$ (see Ex. 9.17).

Let $A = \{\lambda_1,\lambda_2,\ldots, \lambda_k\}$  a set (possibly empty) of distinct primary primes $\lambda_i$ (therefore  they are not associate), and not associate neither to $\gamma_i, 1 \leq i \leq l$, nor to $\lambda = 1 - \omega$.

We will show that we can find a primary prime $\lambda_{k+1}$ distinct of the $\lambda_i$ with the same properties and such that the equation $x^3 \equiv \lambda \pmod {\lambda_{k+1}}$ is not solvable. This will prove the existence of infinitely many primes $\pi$ such that the equation $x^3 \equiv \lambda \pmod \pi$ is not solvable.

Using the initial note, let $\sigma \in D$ such that $\chi_{\gamma_l}(\sigma) = \omega$. As $D$ is a principal ideal domain, the Chinese Remainder Theorem is valid. Since $3 = \lambda \overline{\lambda} = -\omega^2 \lambda^2$ is relatively prime to $\gamma_i, \lambda_i$, there exists $\beta \in D$ such that
\begin{align*}
\beta &\equiv 2 \ [3],\\
\beta &\equiv 1 \ [\lambda_i] \hspace{1cm} (1\leq i\leq k),\\
\beta &\equiv 1 \ [\gamma_i] \hspace{1cm} (1 \leq i \leq l-1),\\
\beta &\equiv  \sigma \ [\gamma_l].\\
\end{align*}
The first equation show that $\beta$ is primary, so $\beta= (-1)^{m-1} \beta_1\ldots \beta_m$, where the $\beta_i$ are primary primes.

By Exercise 9.20,
$$\chi_\beta(\gamma) = \chi_\beta(\gamma_1)^{r_1}\cdots\chi_\beta(\gamma_l)^{r_l} = \chi_{\gamma_1}(\beta)^{r_1}\cdots\chi_{\gamma_l}(\beta)^{r_l}.$$
As $\chi_{\gamma_i}(1) = 1 \ (1\leq i \leq l-1)$, and $\chi_{\gamma_l}(\beta) = \chi_{\gamma_l}(\sigma) =\omega$, we obtain $\chi_\beta(\gamma) = \omega^{r_l} \neq 1$, since $r_l =1$ or $r_l = 2$.

By Exercise 9.18, $\chi_\rho(\alpha) \chi_\gamma(\alpha) = \chi_{-\rho \gamma}(\alpha)$, with primary $\rho, \gamma$, so by induction, as $\beta= (-1)^{m-1} \beta_1\cdots\beta_m$,
$$\chi_\beta(\gamma) = \chi_{\beta_1}(\gamma)\cdots\chi_{\beta_m}(\gamma) \neq 1.$$
Thus there exists a subscript $j$ such that $\chi_{\beta_j}(\gamma) \neq 1$.

We can then take $\lambda_{k+1} = \beta_j$. Indeed, since $\beta \equiv 1 \ [\lambda_i]$ and $\beta \not \equiv 0\ [\gamma_i]$, $\beta_j$ is distinct of the $\lambda_i$ and $\gamma_i$, and $\beta_j$ is not associate to $\lambda$ since $\beta \equiv 2 \pmod 3$.

As $\chi_{\lambda_{k+1}}(\gamma) \neq 1$, the equation $x^3 \equiv \gamma \ [\lambda _{k+1}]$ is not solvable, so $\lambda_{k+1}$ is convenient. 

Conclusion : if $\gamma \in D$ is primary and is not a cube in $D$, there exist infinitely many primes $\pi \in D$ such that the equation $x^3 \equiv \gamma \ [\pi]$ is not solvable.
\item[b)] We show that $x^3 \equiv \omega \ [\pi]$ has no solution for infinitely many primes $\pi$.

To initialize the induction, we display such a prime $\pi$, namely $\pi = 2+ 3 \omega$. Indeed, $N(\pi) = 4 + 9 - 6 = 7$, 7 is a rational prime, so $\pi$ is a primary prime in $D$, of the form $\pi = 3m-1 + 3n\omega$, with $n = m =1$, so $\chi_\pi(\omega) = \omega^{m+n} = \omega^2 \neq 1$ : the equation $x^3 \equiv \omega \ [\pi]$ is not solvable. Moreover $\pi$ is not associate to $\lambda = 1 - \omega$.

Suppose now the existence of a set $A = \{\lambda_1,\lambda_2,\ldots,\lambda_l\}, l \geq 1$, of distinct primary primes $\lambda_i$, not associate to $\lambda$ and such the equation $x^3 \equiv \omega\ [\lambda_i]$ is not solvable for each $i, \ 1\leq i \leq l$. We will show that we can add a prime $\lambda_{l+1}$  to the set $A$ with the same properties.


Let $$\beta = 3(-1)^{l-1} \lambda_1\cdots\lambda_l - 1.$$

$(-1)^{l-1} \lambda_1\cdots\lambda_l$ is primary, so $(-1)^{l-1} \lambda_1\cdots\lambda_l = 3m-1 + 3n\omega,\ m,n \in \Z$.

$\beta = 3(3m-1 + 3n\omega) - 1 = 3(3m-1) - 1 + 9n\omega = 3M-1 + 3 N \omega$, where $M = 3m-1, N = 3n$. By Exercise 9.19,
$$\chi_\beta(\omega) = \omega^{M+N} = \omega^{3m-1+3n} = \omega^2 \neq 1.$$
As $\beta = \pm \beta_1\cdots\beta_m$, where the $\beta_i$ are primary primes, $\chi_\beta(\omega) = \chi_{\beta_1}(\omega) \cdots \chi_{\beta_m}(\omega) \neq 1$, so there exists a subscript $i$ such that $\chi_{\beta_i}(\omega) \ne 1$. 

Since $\beta = 3(-1)^{l-1} \lambda_1\cdots\lambda_l - 1$, $\beta_i$ is associate neither to $\lambda_i$ nor to $\lambda$. Moreover $\chi_{\beta_i}(\omega)\ne 1$, thus the equation $x^3 \equiv \omega\ [\beta_i]$ is not solvable: $\lambda_{l+1} = \beta_i$ is convenient.

Conclusion: the equation $x^3 \equiv \omega \ [\pi]$ is not solvable for infinitely many primes $\pi$.

\item[c)] We show that $x^3 \equiv \lambda \ [\pi]$ has no solution for infinitely many primes $\pi$.

To initialize the induction, we display such a prime $\pi$, namely $\pi = -4 + 3 \omega$. Indeed, $N(\pi) = 16 + 9 + 12 = 37$, 37 is a rational prime, so $\pi$ is a primary prime in $D$, of the form $\pi = 3m-1 + 3n\omega$, with $m=-1,n=1$, so $\chi_\pi(\lambda) = \omega^{2m} = \omega \neq 1$ : the equation $x^3 \equiv \lambda \ [\pi]$ is not solvable.

Suppose now the existence of a set $A = \{\lambda_1,\lambda_2,\ldots,\lambda_l\}, l \geq 1$, of distinct primary primes $\lambda_i$, not associate to $\lambda$ and such the equation $x^3 \equiv \lambda\ [\lambda_i]$ is not solvable. We will show that we can add a prime $\lambda_{l+1}$  to the set $A$ with the same properties.

Let $$\beta = 3(-1)^{l-1} \lambda_1\cdots\lambda_l - 1.$$
$(-1)^{l-1} \lambda_1\cdots\lambda_l$ is primary, so $(-1)^{l-1} \lambda_1\cdots\lambda_l = 3m-1 + 3n\omega,\ m,n \in \Z$.

$\beta = 3(3m-1 + 3n\omega) - 1 = 3(3m-1) - 1 + 9n\omega = 3M-1 + 3 N \omega$, where $M = 3m-1, N = 3n$. By Exercise 9.19,
$$\chi_\beta(\lambda) = \omega^{2M} = \omega^{2(3m-1)} = \omega \neq 1.$$
As $\beta = \pm \beta_1\cdots\beta_m$, where the $\beta_i$ are primary primes, $\chi_\beta(\omega) = \chi_{\beta_1}(\omega) \cdots \chi_{\beta_m}(\omega) \neq 1$, so there exists a subscript $i$ such that $\chi_{\beta_i}(\lambda) \ne 1$. 

Since $\beta = 3(-1)^{l-1} \lambda_1\cdots\lambda_l - 1$, $\beta_i$ is associate neither to $\lambda_i$ nor to $\lambda$. Moreover $\chi_{\beta_i}(\lambda)\ne 1$, thus the equation $x^3 \equiv \lambda\ [\beta_i]$ is not solvable : $\lambda_{l+1} = \beta_i$ is convenient.

Conclusion : the equation $x^3 \equiv \lambda \ [\pi]$ is not solvable for infinitely many primes $\pi$.

\end{enumerate}
\end{proof}

\paragraph{Ex. 9.22}

{\it (continuation) Show in general that if $\gamma \in D$ and $x^3 \equiv \gamma \pmod \pi$ is solvable for all but finitely many primary primes $\pi$, then $\gamma$ is a cube in $D$.
}

\begin{proof}
Let $\gamma \in D$ and suppose that $\gamma$ is not a cube in $D$. We will show that the equation $x^3 \equiv \gamma \ [\pi]$ is not solvable for infinitely primes $\pi \in D$.

By Exercise 9.2, we can write
$$\gamma = (-1)^u\omega^v \lambda^w \gamma_1^{n_1}\cdots\gamma_p^{n_p},$$
where the $\gamma_i$ are distinct primary primes, not associate to $\lambda$. Let $v = 3q + b, w = 3q' +c, n_i = 3q_i +r_i$, with the remainders $b,c,r_i$ in $\{0,1,2\}$. Grouping the factors with null remainders, we obtain $\gamma = \delta^3 \gamma', \gamma' = \omega^b \lambda^c\gamma_1^{r_1}\cdots \gamma_l^{r_l}$, with  $b,c,r_i$ in $\{1,2\}, \delta \in D, l \geq 0$ ($-1$ is a cube).

Moreover the equation $x^3 \equiv \gamma \ [\pi]$ is solvable iff the equation $x^3 \equiv \gamma'\ [\pi]$ is solvable. So we may suppose that 
$$\gamma =  \omega^b \lambda^c\gamma_1^{r_1}\cdots \gamma_l^{r_l},\qquad b\in \{1,2\}, c\in \{1,2\}, r_i \in \{1,2\},$$
without cubic factors.

\begin{enumerate}
\item[$\bullet$] Case 1 : $l\geq 1$.

Let $A = \{\lambda_1,\ldots,\lambda_k\}$ a possibly empty set of distinct primary primes $\lambda_i$, distinct of the $\gamma_i$, not associate to $\lambda$,  and such that the equation $x^3 \equiv \gamma \ [\lambda_i]$ is not solvable. We will show that we can add a prime $\lambda_{k+1}$ with the same properties.

Suppose that $l\geq 1$. We have proved in Ex. 9.21 that there exists $\sigma \in D$ such that $\chi_{\gamma_l}(\sigma) = \omega$. Since $9,\lambda_i, \gamma_j$ are relatively prime, there exists $\beta \in D$ such that
\begin{align*}
\beta &\equiv -1\ [9]\\
\beta & \equiv 1 \ [\lambda_i], 1 \leq i \leq k\\
\beta & \equiv 1 \ [\gamma_i], 1 \leq i \leq l-1\\
\beta & \equiv \sigma\  [\gamma_l]
\end{align*}
$\beta \equiv -1 \ [9]$, thus $\beta \equiv -1 \ [3]$ : $\beta$ is primary, of the form $\beta = 3M-1+3N\omega$.

$\beta = 3M-1+3N\omega \equiv -1 \ [9]$, so $3M+3N\omega \equiv 0 \ [9]$, $M+N\omega \equiv 0 \ [3]$, thus $3\mid M,3\mid N$.

By Exercise 9.18,
\begin{align*}
\chi_\beta(\omega) &= \omega^{M+N} = 1\\
\chi_\beta(\lambda) &= \omega^{2M} = 1
\end{align*}
As $\beta$ and $\gamma_i$ are primary, $\chi_\beta(\gamma_i) = \chi_{\gamma_i}(\beta) = \chi_{\gamma_i}(1) = 1\ (1\leq i \leq l-1)$. 

$\chi_\beta(\gamma) =\chi_\beta(\omega)^b \chi_\beta(\lambda)^c\chi_\beta(\gamma_1)^{r_1}\cdots\chi_\beta(\gamma_l)^{r_l} = \chi_{\beta}(\gamma_l)^{r_l} = \chi_{\gamma_l}(\beta)^{r_l} = \chi_{\gamma_l}(\sigma)^{r_l} = \omega^{r_l}  \neq 1$, since $r_l \in \{1,2\}$. 

$\beta = \pm \beta_1\cdots \beta_m$, with $\beta_i$ primary primes, therefore
$$\chi_\beta(\gamma) = (\chi_{\beta_1}\cdots \chi_{\beta_m})(\gamma) \neq 1. $$ Thus there exists a subscript $i$ such that $\chi_{\beta_i}(\gamma) \ne 1$, so $x^3 \equiv \gamma\ [\beta_i]$ is not solvable. Moreover $\beta \equiv 1 \ [\gamma_i]$, so $\beta_i$ is not associate to any $\gamma_j$. Similarly, $\beta_i$ is not associate to any $\gamma_j$, and $\beta \equiv -1\ [9]$, therefore $\beta_i$ is not associate to $\lambda$. So $\lambda_{k+1} = \beta_i$ is convenient.

There exist infinitely many $\pi$ such that $x^3 \equiv \gamma \ [\pi]$ is not solvable.


\item[$\bullet$] Case 2 : $l = 0$, so $\gamma = \omega^b\lambda^c,\ 1\leq b \leq 2, 1 \leq c \leq 2$.

$\pi_0 = 2 - 3 \omega$ is a primary prime ($N(\pi_0) = 19$).

Let $A = \{\lambda_1,\ldots,\lambda_k\}$ a possibly empty set of distinct primary primes $\lambda_i \neq \pi_0$ such that the equation $x^3 \equiv \gamma \ [\lambda_i]$ is not solvable. We will show that we can add a prime $\lambda_{k+1}$ with the same properties.

Let $\beta = 9 (-1)^{k-1} \lambda_1\cdots \lambda_k + 2-3\omega$.

$\beta\equiv 2 \ [3]$ : $\beta$ is primary.

Moreover $(-1)^{k-1} \lambda_1\cdots \lambda_k $ is primary, so 

$$(-1)^{k-1} \lambda_1\cdots \lambda_k  = 3m-1+3n\omega, m \in \mathbb{Z},n\in \mathbb{Z}.$$
Then
\begin{align*}
\beta&=9(3m-1+3n\omega)+2-3\omega\\
&=27m-7+(27n-3)\omega\\
&=3(9m-2)-1+3(9n-1)\omega\\
&=3M-1+3N\omega,
\end{align*}
where $M=9m-2,N=9n-1$.
Therefore
\begin{align*}
\chi_\beta(\omega) &= \omega^{M+N} = \omega^{9m-2+9n-1}=1\\
\chi_\beta(\lambda) &=\omega^{2M} = \omega^{2(9m-2)}=\omega^2\neq 1
\end{align*}

$\beta = \pm\beta_1\cdots\beta_m$, where the $\beta_i$ are primary primes.

$\chi_\beta(\gamma) = \chi_\beta(\omega)^b \chi_\beta(\lambda)^c = \omega^{2c} \neq 1$ since $c = 1$ or $c = 2$.
$$\chi_\beta(\gamma) = (\chi_{\beta_1}\cdots \chi_{\beta_m})(\gamma) \neq 1. $$ Thus there exists a subscript $i$ such that $\chi_{\beta_i}(\gamma) \ne 1$, so $x^3 \equiv \gamma\ [\beta_i]$ is not solvable. 

As $\beta_i \mid \beta = 9 (-1)^{k-1} \lambda_1\cdots \lambda_k + 2-3\omega$, if $\beta_i = \lambda_j$ for some subscript $j$, $\lambda_j \mid \pi_0 = 2 - 3\omega$, so $\lambda_j = \pi_0$, which is a contradiction, thus $\beta_i \not \in A$. Similarly, if  $\beta_i = \pi_0 = 2 - 3\omega$, then $\pi_0 \mid 9 \lambda_1\cdots\lambda_k$, and $\pi_0$ is relatively prime to $\lambda$, so $\pi_0 = \lambda_j$ for some subscript $j$ : this is a contradiction, thus $\beta_i \ne \pi_0$. $\lambda_{k+1} = \beta_i$ is convenient.

So there exist infinitely many $\pi$ such that $x^3 \equiv \gamma \ [\pi]$ is not solvable.

$\bullet$ Conclusion :

if $\gamma$ is not a cube in $D$, there exist infinitely many primes $\pi$ such that $x^3 \equiv \gamma \ [\pi]$ is not sovable.

By contraposition, if the equation $x^3 \equiv \gamma \ [\pi]$ is solvable for every prime $\pi$, at the exception perhaps of the primes in a finite set, then $\gamma$ is a cube in $D$.

\end{enumerate}
\end{proof}

\paragraph{Ex. 9.23}

{\it Suppose that $p\equiv 1 \pmod 3$. Use Exercise 5 to show that $x^3 \equiv 3 \pmod p$ is solvable in $\Z$ iff $p$ is of the form $4p = C^2 + 243 B^2$.
}

\begin{proof}
Let $p$ be a rational prime, $p \equiv 1 \pmod 3$, then $p = \pi \overline{\pi}$, where $\pi \in D$ is a primary prime : $\pi = a + b \omega = 3m-1 + 3 n\omega$.
\begin{enumerate}
\item[$\bullet$] Suppose that there exists $x \in \Z$ such that $x^3 \equiv 3 \pmod p$. Then $x^3 \equiv 3 \pmod \pi$, so $\chi_\pi(3) = 1$. By Exercise 9.5, $\omega^{2n} = \chi_\pi(3) = 1$, thus $3 \mid n$, therefore $9 \mid b = 3n$, namely $b = 9B, B \in \Z$.

$p = N\pi = a^2+b^2 -ab, 4p = (2a-b)^2+3b^2 = C^2 + 243 B^2$, where $C = 2a-b,B=b/9$.
So there exists $C,B\in \Z$ such that $4p = C^2 + 243B^2$.
\item[$\bullet$] Conversely, suppose that there exist $C,B \in \Z$ such that $4p = C^2 + 243 B^2$.

As $4p = (2a-b)^2 + 3 b^2 = C^2 + 3(9B)^2$, from the unicity proved in Exercise 8.13, we obtain $b = \pm9B$, so $9 \mid b=3n, 3 \mid n$, and $\chi_\pi(3) = \omega^{2n} = 1$.

Thus there exists $x \in D$ such that $x^3 \equiv 3 \pmod \pi$. As $p \equiv 1\pmod 3$, $D/\pi D = \{\overline{0},\ldots,\overline{p-1}\}$, so there exists $h \in \Z$ such that $x \equiv h \pmod \pi$, and $h^3 \equiv 3 \pmod \pi$.

Therefore $p = N\pi \mid N(h^3 - 3)$, namely $p \mid (h^3-3)^2$, where $p$ is a rational prime, thus $p \mid h^3 -3$ : there exists $x \in \Z$ such that $x^3 \equiv 3 \pmod p$.

Moreover $4p = C^2 + 243 B^2$ implies $p \equiv 1 \pmod 3$.

$$(p\equiv 1 \ [3]\ \mathrm{and}\ \exists x \in \mathbb{Z},\ x^3 \equiv 3 \ [p] )\iff \exists C \in \mathbb{Z}, \exists B \in \mathbb{Z},\ 4p = C^2+243 B^2.$$
\end{enumerate}
\end{proof}

\paragraph{Ex. 9.24}

{\it Let $\pi = a +b \omega$ be a complex primary element of $D = \Z[\omega]$. Put $a = 3m-1,b=3n,p = N(\pi)$.
\begin{enumerate}
\item[(a)] $(p-1)/3 \equiv -2m + n \pmod 3$.
\item[(b)] $(a^2-1)/3 \equiv m \pmod 3$.
\item[(c)] $\chi_\pi(a) = \omega^m$.
\item[(d)] $\chi_\pi(a+b) = \omega^{2n} \chi_\pi(1-\omega)$.
\end{enumerate}
}

\bigskip

{\bf Lemma.} {\it Let $a \in \Z$, $a \equiv -1 \pmod 3$, and $b\in \Z$ such that $a \wedge b = 1$. Then $\chi_a(b)= 1$.}

\bigskip

\begin{proof}(of Lemma.)

If $q$ is a rational prime, $q \equiv 2 \pmod 3$, and $q \wedge b = 1$, then $\chi_q(b) = 1$ (Prop. 9.3.4, Corollary).

If $p$ is a rational prime, $p \equiv 1 \pmod 3$ and $p \wedge b = 1$, then $p = \pi \overline{\pi}$, with $\pi$ primary prime in $D$ (and also $\overline{\pi}$), and by definition of $\chi_p$, $\chi_p(b) = \chi_\pi(b) \chi_{\overline{\pi}}(b)$.

As $\chi_{\overline{\pi}}(b) = \chi_{\overline{\pi}}(\overline{b}) = \overline{\chi_\pi(b)}$ (Prop. 9.3.4(b)), then $\chi_p(b)  = \chi_\pi(b) \chi_{\overline{\pi}}(b) = \chi_\pi(b)\overline{\chi_\pi(b)}= 1$.

$a$ has a decomposition in prime  factors of the form :
$$a = \pm q_1q_2\cdots q_k p_1p_2\cdots p_l = \pm q_1q_2\cdots q_k \pi_1 \overline{\pi_1} \pi_2 \overline{\pi_2}\cdots \pi_l \overline{\pi_l},$$
where $q_i \equiv -1 , p_j \equiv 1 \pmod 3$, and  the $\pi_k$ are primary primes (since all these elements are primary, the symbol $\pm$ is $(-1)^{k-1}$).
Thus, by definition of $\chi_a$,
$$\chi_a(b) = \chi_{q_1}(b)\cdots \chi_{q_k}(b) \chi_{\pi_1}(b)\chi_{\overline{\pi_1}}(b)\cdots \chi_{\pi_l}(b)\chi_{\overline{\pi_l}}(b) = 1.$$
The result remains true if $a = -1$ : then, by definition, $\chi_a(b) = 1$.

\end{proof}

\bigskip

\begin{proof}(of Ex 9.24.)
By hypothesis, $\pi$ is a primary element, so $\pi = 3m-1 + 3n \pi,\ m,n\in \Z$.
We don't suppose in this proof that $\pi$ is a prime element, so $p = N(\pi)$ is not necessarily prime. 
 \begin{enumerate}
\item[(a)]$p-1 = (3m-1)^2+(3n)^2-3n(3m-1) - 1 \equiv -6m + 3n \pmod 9$, thus $$\frac{p-1}{3} \equiv -2m + n \pmod 3.$$
\item[(b)]$a^2 - 1 = (3m-1)^2-1 \equiv -6m \pmod 9$, thus
$$\frac{a^2-1}{3} \equiv m \pmod 3.$$
\item[(c)]  
 As $\pi,a$ are primary, by Exercise 9.20, $\chi_\pi(a) = \chi_a(\pi)$.

Since $\pi \equiv b\omega \pmod a$, $\chi_a(\pi) = \chi_a(b) \chi_a(\omega)$.

By Exercise 9.18,  as $a = 3m-1$, $\chi_a(\omega) = \omega^{M+N}$, where $M = m, N = 0$, so $$\chi_a(\omega) = \omega^m.$$

Here $a$ is relatively prime to $b$ in $\Z$ : if a rational prime $r$ divides $a,b$, then $r \mid \pi$ in $D$, thus $r \mid \overline{\pi}$, so $r^2 \mid \pi \overline{\pi} = p$ in $D$, thus $r^2 \mid p$ in $\Z$, which is absurd. The Lemma gives then $\chi_a(b) = 1$.

 We conclude that $\chi_a(b) = 1,\chi_a(\omega) = \omega^m$, so $\chi_\pi(a) = \chi_a(\pi) = \chi_a(b) \chi_a(\omega) = \omega^m$.
 $$\chi_\pi(a) = \omega^m.$$
 
 \item[(d)] 
 $$a+b = [(a+b) \omega] \omega^{-1},$$
and 
$$(a+b) \omega = (a+b\omega) +a\omega -a \equiv a(\omega-1)\pmod \pi,$$ thus
$$a+b \equiv a(1-\omega) \omega^{-1} \ [\pi],$$
$$\chi_\pi (a+b) = \chi_\pi(1-\omega) \chi_\pi(a) \chi_\pi(\omega)^{-1},$$
$\chi_\pi(a) = \omega^m$ by (c), and $\chi_\pi(\omega) = \omega^{m+n}$(as in Ex. 9.3), thus

$$\chi_\pi(a+b) = \omega^{2n} \chi_\pi(1-\omega).$$

\end{enumerate}
\end{proof}

\paragraph{Ex. 9.25}

{\it Show that $\chi_{a+b}(\pi)$ may be computed as follows.
\begin{enumerate}
\item[(a)] $\chi_{a+b}(\pi) = \chi_{a+b}(1-\omega)$.
\item[(b)] $\chi_{a+b}(\pi) = \omega^{2(m+n)}$.
\end{enumerate}
}

\begin{proof}
\begin{enumerate}
\item[(a)] $\pi = a + b \omega$ and $a \equiv -b \pmod {a+b}$, thus $\pi \equiv -b(1-\omega) \pmod {a+b}$. So
$$\chi_{a+b}(\pi) = \chi_{a+b}(b) \chi_{a+b}(1-\omega).$$
Since $a \wedge b = 1$, $(a+b) \wedge b = 1$ : as in Ex. 9.24, $\chi_{a+b}(b) = 1$. So
$$\chi_{a+b}(\pi) =  \chi_{a+b}(1-\omega).$$
\item[(b)] Since the character $\chi_{a+b}$ has order 3,
\begin{align*}
\chi_{a+b}(1-\omega) &= (\chi_{a+b}((1-\omega)^2))^2\\
&=(\chi_{a+b}(-3\omega))^2\\
&=[\chi_{a+b}(3) \chi_{a+b}(\omega)]^2
\end{align*}

$\chi_{a+b}(3) = 1$ because $(a+b) \wedge 3 = (3(m+n)-1) \wedge 3 = 1$.

$\chi_{a+b}(\omega) = \omega^{m+n}$ (Ex. 9.19).

Conclusion : $$\chi_{a+b}(1-\omega) = \omega^{2(m+n)}.$$ 
\end{enumerate}
\end{proof}

\paragraph{Ex. 9.26}

{\it Combine the previous two exercises to conclude that $\chi_\pi(1-\omega) = \omega^{2m}$.
}

\begin{proof}
Since $\pi$ and $a+b$ are primary elements of $D$, by Exercise 9.20,
$$\chi_\pi(a+b) = \chi_{a+b}(\pi).$$
By Exercises 9.24 and 9.25,
\begin{align*}
\chi_\pi(a+b) &= \omega^{2n} \chi_\pi(1-\omega)\\
\chi_{a+b}(\pi) &= \omega^{2(m+n)}
\end{align*}
Thus $\omega^{2n} \chi_\pi(1-\omega) = \omega^{2(m+n)}$.
Consequently
$$\chi_\pi(1-\omega) = \omega^{2m}.$$
\end{proof}

\paragraph{Ex. 9.27}

{\it Let $\pi = a+bi$ be a primary irreducible in $\Z[i], b\ne 0$. Show
\begin{enumerate}
\item[(a)] $a \equiv (-1)^{(p-1)/4} \pmod 4,\ p = N(\pi)$.
\item[(b)] $b \equiv (-1)^{(p-1)/4} - 1 \pmod 4$.
\end{enumerate}
(Wrong sentence for (b) in the edition 1990.)
}

\begin{proof}
Let $\pi=a+bi$ be a primary prime in $\mathbb{Z}[i]$, $b\neq 0$, such that $p = N(\pi)$. Then
$$p = \pi \bar{\pi}=a^2+b^2\equiv 1 \ [4].$$
By Lemma 6, Section 7, $a$ is odd, $b$ even, and 
 $$(a\equiv 1\ [4], b\equiv 0 \ [4])\ \mathrm{or}\  (a\equiv 3\  [4], b\equiv 2 \ [4]) .$$
\begin{enumerate}
\item[(a)]
	\begin{enumerate}
	\item[$\bullet$] Case 1: $a \equiv 1 \ [4], b \equiv 0 [4]$. Then
	$a=4A+1,b=4B,\ A,B \in \Z$, so  $(a^2+b^2-1)/4= 4A^2+4B^2+2A$ is even : 

	$(-1)^{(p-1)/4} = (-1)^{(a^2+b^2-1)/4} = 1$, and $a\equiv 1 [4]$, thus $a \equiv (-1)^{(p-1)/4} \ [4]$.
	\item[$\bullet$] Case 2: $ a\equiv 3 \ [4], b \equiv 2 \pmod 4$.
	
	$a=4A+3,b=4B+2, a^2+b^2-1 = 16 A^2+24A+9+16B^2+16B+4-1 \equiv 4 \ [8]$, so $(a^2+b^2-1)/4 \equiv 1 \ [2], (-1)^{(p-1)/4} = (-1)^{(a^2+b^2-1)/4} = -1$, and $a \equiv -1\ [4]$, thus  $a \equiv (-1)^{(p-1)/4} \ [4]$.
	\end{enumerate}
	In both cases, 
	$$ a \equiv (-1)^{(p-1)/4} \ [4].$$
\item[(b)] In every case, $b \equiv a-1 \ [4]$, thus
$$ b \equiv (-1)^{(p-1)/4}-1 \ [4].$$

\end{enumerate}
In other words, for all primary primes $\pi = a + bi$ such that $N(\pi) = p$,
\begin{align*}
p \equiv 1\ [8] &\iff \pi  \equiv 1 \ [4],\\
p \equiv 5\ [8] &\iff \pi  \equiv 3 + 2i \ [4].
\end{align*}
\end{proof}

\paragraph{Ex. 9.28}

{\it The notation being as in Exercise 27 show $\chi_\pi(\overline{\pi}) = \chi_\pi(2) \chi_\pi(a)$.
}

\begin{proof}
$\pi = a+bi,\overline{\pi} = a-bi = 2a-\pi \equiv 2a \ [\pi]$, thus, by Proposition 9.8.3 (e) :
 $$\chi_\pi(\overline{\pi}) = \chi_\pi(2a) = \chi_\pi(2) \chi_\pi(a).$$
\end{proof}

\paragraph{Ex. 9.29} 

{\it By Exercise 9.27, $a(-1)^{(p-1)/4}$ is primary. Use biquadratic reciprocity to show $\chi_\pi(a(-1)^{(p-1)/4}) = (-1)^{(a^2-1)/8}$.
}

\begin{proof}
$a \equiv (-1)^{(p-1)/4}\ [4]$ (Ex. 9.27(a)), $a  (-1)^{(p-1)/4} \equiv 1\  [4]$, thus $a  (-1)^{(p-1)/4}$ is primary (if $a \neq \pm 1$).

If $a = \pm 1$ is an unit, $a  (-1)^{(p-1)/4} = 1$ and $\chi_\pi(a(-1)^{(p-1)/4}) = 1 = (-1)^{(a^2-1)/8}$, so we can suppose that $a$ is not an unit.

As $a  (-1)^{(p-1)/4} \equiv 1 \pmod 4$, the Law of Biquadratic Reciprocity (Prop. 9.9.8) gives
\begin{align*}
\chi_\pi(a(-1)^{(p-1)/4}) &= \chi_{a(-1)^{(p-1)/4}}(\pi) \\
&= \chi_a(\pi) \qquad (\mathrm{Prop. 9.8.3(f)})\\
&=\chi_a(a+bi)\\
&=\chi_a(bi)\\
&=\chi_a(b) \chi_a(i).
\end{align*}

As $a \wedge b=1$ (since $p = a^2+b^2$), $\chi_a(b) = 1$ (Prop. 9.8.5, with $a\neq 1$), so
$$\chi_\pi(a(-1)^{(p-1)/4}) =  \chi_a(i).$$
 
 If $a\equiv 1 \ [4]$, Proposition 9.8.6 gives $\chi_a(i) = (-1)^{(a-1)/4}$. Write $a = 4A + 1,\ A\in \Z$. Then 
 $$(-1)^{(a^2-1)/8} = (-1)^{2A^2+A} = (-1)^A = (-1)^{(a-1)/4} = \chi_a(i).$$
 
 If $a \equiv -1 \ [4]$, then $\chi_a(i) = \chi_{-a}(i) = (-1)^{(-a-1)/4}$ by the same proposition. Write $a = 4A-1,\ A \in \Z$. Then
 $$(-1)^{(a^2-1)/8} = (-1)^{2A^2 - A} =(-1)^{-A} = (-1)^{(-a-1)/4} = \chi_a(i).$$
 So, for each odd $a$, $a \ne \pm 1$, $$\chi_a(i) = i^{(a^2-1)/8}.$$
 Conclusion : if $\pi = a + bi$ is a primary irreducible such that $N(\pi) = p$, then
  $$\chi_\pi(a (-1)^{(p-1)/4}) =  (-1)^{(a^2-1)/8}.$$
\end{proof}

\paragraph{Ex. 9.30}

{\it Use the preceding two exercises to show $\chi_\pi(\overline{\pi}) = \chi_\pi(2) (-1)^{(a^2-1)/8}$.
}

\begin{proof}
By Exercises 9.28, 9.29, and $\chi_\pi(-1) =(-1)^{(a-1)/2}$ (Prop. 9.8.3(d)),
\begin{align*}
\chi_{\pi}(\overline{\pi})&=  \chi_\pi(2) \chi_\pi(a)\\
&=\chi_\pi(2) \chi_\pi(a(-1)^{(p-1)/4})(\chi_\pi(-1))^{(p-1)/4}\\
&=\chi_\pi(2) (-1)^{(a^2-1)/8} ((-1)^{(a-1)/2})^{(p-1)/4}\\
&=\chi_\pi (-2)(-1)^{(a^2-1)/8} ((-1)^{(a-1)/2})^{(p+3)/4}\\
&=\chi_\pi(-2) (-1)^{(a^2-1)/8}  (-1)^{((a-1)/2)\,((p+3)/4)}.
\end{align*}

If $a \equiv 1 \pmod 4$, then $(-1)^{(a-1)/2}=1$.

If $a \equiv 3 \pmod 4$, then $b \equiv 2 \ [4]$ : $$a=4A+3,b=4B+2,p+3 = a^2+b^2+3 = (4A+3)^2+(4B+2)^2+3 \equiv 0 \ [8],$$ so $(p+3)/4 \equiv 0 \ [2].$

In both cases  $  (-1)^{((a-1)/2)\,((p+3)/4)}=1$, and so

$$\chi_{\pi}(\overline{\pi})=\chi_\pi(-2) (-1)^{(a^2-1)/8}.  $$
\end{proof}

\paragraph{Ex. 9.31}

{\it Let $p$ be prime, $p\equiv 1 \pmod 4$. Show that $p = a^2+b^2$ where $a$ and $b$ are uniquely determined by the conditions $a \equiv 1 \pmod 4, b \equiv -((p-1)/2)! a \pmod p$.
}

\begin{proof}
Recall the following lemma : 

{\bf Lemma :} 

{\it Let $p$ be  prime, $p \equiv 1 \ [4]$, then  $\left [ \left ( \frac{p-1}{2} \right)!\right ]^2 \equiv -1 \ [p]$.}

\medskip
 
By Wilson's theorem (Prop. 4.1.1, Corollary), $(p-1)! \equiv -1 \ [p]$.

\begin{align*}
-1 \equiv (p-1)! &= 1.2. \cdots .\left(\frac{p-1}{2}\right)\left(\frac{p+1}{2}\right)\cdots(p-2)(p-1)\\
&\equiv 1.2.\cdots\frac{p-1}{2}\left[-\left(\frac{p-1}{2}\right)\right]\cdots(-2)(-1) \\
&\equiv (-1)^{(p-1)/2} \left [ \left ( \frac{p-1}{2} \right)!\right ]^2 \\
&\equiv \left [ \left ( \frac{p-1}{2} \right)!\right ]^2 \ [p],\\
\end{align*}
since $p\equiv 1 \ [4]$.
\begin{enumerate}
\item[$\bullet$] We show that there exists a pair $a,b \in \Z$ which verifies the sentence.

By lemma 5 section 7, as $p \equiv 1 \ [4]$, there exists an irreducible $\pi$ such that $N(\pi) = p$, and we can choose $\pi$ such that $\pi = A + Bi$ is primary (lemma 7 section 7), so $A$ is odd.

If $A\equiv 1 \pmod 4$, we take $a=A$, and if $A \equiv 3 \pmod 4$, we take $a = -A$ : then $a \equiv 1 \pmod 4$.

Let $u = \left ( \frac{p-1}{2} \right)!$. Then $0 \equiv p = A^2 + B^2 \pmod p$,  $B^2 \equiv -A^2 \equiv (uA)^2 \ \pmod p$.

$p \mid (B-uA)(B+uA)$, thus $B \equiv \pm uA\pmod p$.

Since $a = \pm A$, $B \equiv \pm ua \pmod p$.

If $B \equiv -ua \pmod p$, we take $b = B$, if not $b = -B$. 

Then $a,b$ are such that $p=a^2+b^2, a\equiv 1 \ [4], b \equiv -((p-1)/2)!\, a\ [p]$.

\item[$\bullet$] Unicity of the pair $(a,b)$ such that
$$p=a^2+b^2, a\equiv 1 \ [4], b \equiv -((p-1)/2)!\, a\ [p].$$

Suppose that $c,d$ are such that $p = c^2+d^2,c\equiv 1 \ [4], d \equiv -((p-1)/2)! c \ [p]$.

Let $\pi = a + ib, \lambda = c+id$. As $p = N\pi = N \lambda$ is a rational prime, $\pi$ and $\lambda$ are primes in $D$, and $p = \pi \overline{\pi} = \lambda \overline{\lambda}$, thus $\lambda$ is associate to $\pi$ or $\overline{\pi}$. :
$$\lambda \in \{\pi, - \pi, i \pi, -i\pi, \overline{\pi},-\overline{\pi},i\overline{\pi},-i\overline{\pi}\}.$$
As $a,c$ are odd, and $b,d$ even, it remains only the possibilities $\lambda =\pm \pi, \lambda = \pm \overline{\pi}$, thus $c = \pm a$. Moreover $a \equiv c \equiv 1 \ [4]$, thus $a=c$, and $d \equiv -((p-1)/2)!c \equiv -((p-1)/2)!a \equiv b \ [p]$.

$p=a^2+b^2 = a^2 + d^2$, so $d = \pm b$, and $d \equiv b \ [p]$.

If $d = -b$, then $p\mid 2b$, thus $p \mid b$, and also $p \mid a$, so $p^2 \mid a^2+b^2 = p$: this is impossible. So $a=c,b=d$. Unicity is proved.

Conclusion : if $p\equiv 1 \ [4]$, there exists an unique pair $a,b$ such that
$$p=a^2+b^2, a \equiv 1 \pmod 4, b \equiv -((p-1)/2)! a \pmod p.$$
\end{enumerate}
\end{proof}

\paragraph{Ex. 9.32}

{\it Let $p$ be a prime, $p\equiv 1 \pmod 4$ and write $p = \pi \overline{\pi}, \pi \in \Z[i]$. Show $\chi_p(1+i) = i^{(p-1)/4}$.
}

\begin{proof}


\begin{align*}
\chi_p(1+i) &= \chi_\pi(1+i) \chi_{\bar{\pi}}(1+i) \\
&= \chi_\pi(1+i) \overline{\chi_\pi(1-i)} \qquad ( \mathrm{Prop.}\  \mathrm{9.8.3(c)})\\
& =\frac{\chi_\pi(1+i)}{\chi_\pi(1-i) }= \chi_\pi(i) \qquad (\mathrm{since}\  (1-i)i = 1+i)\\
&=i^{\frac{p-1}{4}}.
\end{align*}
The last equality is a consequence of the definition of $\chi_\pi$ : $\chi_\pi(i) \equiv i^{\frac{p-1}{4}} \pmod \pi$, and the classes of $1,i,i^2,i^3$ modulo $\pi$ are distinct.
\end{proof}

\paragraph{Ex. 9.33}

{\it  Let $q$ be a positive prime, $q\equiv 3 \pmod 4$. Show $\chi_q(1+i) = i^{(q+1)/4}$. [Hint : $(1+i)^{q-1} \equiv -i \pmod q$.]
}

\bigskip

The sentence is false and must be replaced by
$$ \chi_q(1+i) = (-i)^{(q+1)/4 } = i^{-(q+1)/4} .$$
We verify this on the example $q=11$ :
\begin{align*}
\chi_{q}(1+i) &\equiv (1+i)^{(q^2-1)/4}\\
& \equiv (1+i)^{30}\\
&\equiv -2^{15} i \equiv -32 i \equiv i \pmod {11},
\end{align*}
so $\chi_{11}(1+i) = i$, and $i^{(-q-1)/4} = i^{-3} = i$ (but $i^{(q+1)/4} = -i$).

\begin{proof}

Write $q = 4k + 3, k \in \N$.

As $(1+i)^2 = 2i$, $(1+i)^{q-1} =  (2i)^{(q-1)/2}$.

$2^{(q-1)/2} \equiv \legendre{2}{q} \ [q]$ and $ \legendre{2}{q} = (-1)^{(q^2-1)/8} = (-1)^{2k^2+3k+1} = (-1)^{k+1}$

$i^{(q-1)/2} = i^{2k+1} = (-1)^k i$. 

So
$$ (1+i)^{q-1} \equiv -i\ [q].$$
$N(q) = q^2$, so $\chi_q(1+i) \equiv (1+i)^{(q^2-1)/4} = [(1+i)^{q-1}]^{(q+1)/4}\equiv (-i)^{(q+1)/4} \ [q]$ :

$$\chi_q(1+i) =(-i)^{(q+1)/4} = i^{(-q-1)/4}.$$
\end{proof}

\paragraph{Ex. 9.34}

{\it Let $\pi = a +bi$ be a primary irreducible, $(a,b) = 1$. Show
\begin{enumerate}
\item[(a)] if $\pi \equiv 1 \pmod 4$, then $\chi_\pi(a) = i^{(a-1)/2}$.
\item[(b)] if $\pi \equiv 3 + 2i\pmod 4$, then $\chi_\pi(a) = -i^{(-a-1)/2}$.
\end{enumerate}
}

\begin{proof}
Let $\pi = a+bi$ be a primary irreducible, with $a\wedge b=1$, so $b \neq 0$. We can apply the result of Exercise 9.29:
$$\chi_\pi(a(-1)^{(p-1)/4}) = (-1)^{(a^2-1)/8}.$$
\begin{enumerate}
\item[(a)] Suppose that  $\pi \equiv 1 \ [4]$. 

Then $a\equiv 1\ [4], b\equiv 0 \ [4], a = 4A+1,b=4B,\ A,B \in \mathbb{Z}$.

As $\chi_\pi(-1) = (-1)^{(a-1)/2}$, 
$$\chi_\pi(a) = (-1)^{\frac{a-1}{2} \frac{p-1}{4}}(-1)^{\frac{a^2-1}{8}},$$
where 
$$ p = N \pi = a^2+b^2, (-1)^{(p-1)/4} = (-1)^{\frac{a^2-1}{4}+\frac{b^2}{4}} = (-1)^{4A^2+2A+4B^2}=1,$$
thus $(-1)^{\frac{a-1}{2} \frac{p-1}{4}}=1$.
$$\chi_\pi(a) = (-1)^{(a^2-1)/8} = (-1)^{2A^2+A} = (-1)^A = (-1)^{(a-1)/4} = i^{(a-1)/2}.$$
Conclusion: if $\pi \equiv 1 \ [4]$, $\chi_\pi(a) =  i^{(a-1)/2}$.
\item[(b)] Suppose that $\pi \equiv 3 + 2i\ [4]$.

Then $a\equiv 3 \ [4], b \equiv 2\ [4],a = 4A+3,b=4B+2,\ A,B\in\mathbb{Z}$. As in (a),
$$\chi_\pi(a) = (-1)^{\frac{a-1}{2} \frac{p-1}{4}}(-1)^{\frac{a^2-1}{8}},$$
where $a^2+b^2-1 = 16 A^2+24A+16B^2+16B+12 \equiv 4 \ [8]$, so $\frac{a^2+b^2-1}{4} \equiv 1 \ [2]$, thus $(-1)^{(p-1)/4} = (-1)^{(a^2+b^2-1)/4} = -1$.

$$ (-1)^{\frac{a-1}{2} \frac{p-1}{4}} = (-1)^{\frac{a-1}{2}} = (-1)^{2A+1} = -1,$$

$$\frac{a^2-1}{8} = 2A^2+3A+1, (-1)^{(a^2-1)/8} = (-1)^{3A+1} = (-1)^{A+1}=(-1)^{(a+1)/4},$$

$$\chi_\pi(a)= -(-1)^{(a+1)/4} = -i^{(a+1)/2}.$$

Moreover $$\frac{a+1}{2} \equiv \frac{-a-1}{2} \ [4] \iff a+1 \equiv -a-1\ [8] \iff 2a\equiv -2 \ [8] \iff a\equiv 3 \ [4],$$ thus $i^{(a+1)/2} = i ^{(-a-1)/2}$.

Conclusion : if $\pi \equiv 3+2i \ [4]$, $\chi_\pi(a) = - i^{(-a-1)/2}$.

\end{enumerate}
\end{proof}

\paragraph{Ex. 9.35}

{\it If $\pi = a+bi$ is as in Exercise 9.34 show $\chi_\pi(a)\chi_\pi(1+i) = i ^{(3(a+b-1))/4}$. [Hint: $a(1+i) = a+b +i(a+bi)$. Generalize Exercises 32 and 33 to any integer $\equiv 1 \pmod 4$ and use Proposition 9.9.8. Note $a+b \equiv 1 \pmod 4$.]
}

\begin{proof}
We give a generalization of Exercises 9.32 and 9.33 : if $n\equiv1\ [4], n \neq 1$, then $\chi_n(1+i) = i^{(n-1)/4}$.

By Exercises 9.32 and 9.33, we know that if  $p\equiv 1\ [4]$ is a rational prime, then 
$$\chi_p(1+i) = i^{(p-1)/4},$$
and if $q \equiv 3 \ [4]$, in other words $-q \equiv 1 \ [4]$, where $q$ is a rational prime, then
$$\chi_{-q}(1+i) =  \chi_{q}(1+i) = i^{(-q-1)/4}.$$
Let  $n \in \mathbb{Z}, n\equiv 1 \ [4],n\neq 1$.

If $n>0$, $n = q_1q_2\cdots q_kp_1p_2\cdots p_l$, where $q_i \equiv -1 \ [4], p_i \equiv 1 \ [4]$, thus $k$ is even.

If $n<0$, $n = -q_1q_2\cdots q_kp_1p_2\cdots p_l$, with $k$ odd.
In both cases, 
$$n = (-q_1)(-q_2)\cdots(-q_k)p_1p_2\cdots p_l,$$
so we can write
$$n = s_1s_2\cdots s_N, \qquad \mathrm{where}\ s_i = -q_i, 1\leq i \leq k, s_i = p_{i-k}, \ k+1 \leq i \leq k+l = N,$$
where $s_i\equiv 1 \ [4], \ 1 \leq i \leq N$.
\begin{align*}
\chi_n(1+i)&=  \chi_{-q_1}(1+i)\cdots\chi_{-q_k}(1+i)\chi_{p_1}(1+i)\cdots\chi_{p_l}(1+i)\\
&=i^{(-q_1-1)/4}\cdots i^{(-q_k-1)/4}  i^{(p_1-1)/4} \cdots i^{(p_l-1)/4} \\
&=i^{(s_1-1)/4}\cdots  i^{(s_N-1)/4} \\
&=i^{\sum_{i=1}^N \frac{s_i-1}{4}}\\
&=i^{(n-1)/4},
\end{align*}
the last equality resulting of Exercise 9.44.

Conclusion : if $n \in \mathbb{Z}, n\equiv1\ [4], n \neq 1$, then $\chi_n(1+i)=i^{(n-1)/4}$.

\bigskip

Let $\pi = a+bi, a\wedge b=1$ a primary irreducible. As $a(1+i) = a+b + i(a+bi)$, $a(1+i) \equiv a+b \ [\pi]$, so
$$\chi_{\pi}(a) \chi_{\pi}(1+i) = \chi_{\pi}(a+b).$$
As $\pi =a +bi$ is primary, $a+b \equiv 1 \ [4]$.

If $a+b = 1$, then $\chi_{\pi}(a) \chi_{\pi}(1+i) = \chi_{\pi}(a+b)=1 = i^{3(a+b-1)/4}$. If not, the Law of Biquadratic Reciprocity (Proposition 9.9.8) gives
$$\chi_\pi(a+b) = \chi_{a+b}(\pi).$$
Now $b \equiv -a\pmod{a+b}$, so $ a + bi \equiv a(1-i) \equiv -i a (1+i) \pmod{a+b}$. Therefore
$$\chi_{a+b}(\pi) = \chi_{a+b}(-1) \chi_{a+b}(a) \chi_{a+b}(i) \chi_{a+b}(1+i).$$
Since $n\equiv 1 \ [4]$, $\chi_n(i) = (-1)^{(n-1)/4}$ (Prop.9.8.6), thus $$\chi_n(-1) = \chi_n(i^2) = (-1)^{\frac{n-1}{2}}=1.$$

Consequently, since $a+b \equiv 1 \ [4]$, $\chi_{a+b}(-1) = 1$ .

As $a\wedge b = 1, (a+b) \wedge a = 1$, thus $\chi_{a+b}(a) = 1$ (Prop 9.8.5).

$a+b \equiv 1\ [4]$, thus $\chi_{a+b}(i) = (-1)^{(a+b-1)/4}$ (Prop. 9.8.6).

From the first part of this proof, $\chi_{a+b}(1+i) = i^{(a+b-1)/4}$, so
\begin{align*}
 \chi_{a+b}(\pi) &= \chi_{a+b}(-1) \chi_{a+b}(a) \chi_{a+b}(i) \chi_{a+b}(1+i)\\
 &=(-1)^{(a+b-1)/4} i ^{(a+b-1)/4}\\
 &=i^{(a+b-1)/2} i ^{(a+b-1)/4}\\
 &=i^{3(a+b-1)/4}
 \end{align*}
 
 Conclusion : if $\pi = a+b i$ is a primary irreducible, such that $a\wedge b=1$, then $$\chi_{\pi}(a) \chi_{\pi}(1+i) = i^{3(a+b-1)/4}$$
\end{proof} 

\paragraph{Ex. 9.36}

{\it Remove the restriction $(a,b)=1$ in Exercise 9.34.
}

\begin{proof}
Suppose that $q = a\wedge b >1$. Then $a = qa',b = qb', \ a',b' \in \Z$, so $\pi = q(a'+ib')$.

As $\pi$ is irreducible, and as $q$ is not an unit, $u = a'+b'i$ is an unit, and so $\pi = uq$ is associate to $q$ : the rational integer $q$ is then a prime in $D$, so a rational prime $q \equiv 3 \pmod 4$.

If $u = \pm i$, then $\pi = \pm q = a + bi$ is such that $b$ is odd, in contradiction with $\pi$ primary. Thus $u = \pm 1$, and $\pi = \varepsilon q, \varepsilon = \pm 1$. As $\pi$ is primary, $\varepsilon = -1$, so $\pi = -q$.

Then $\chi_\pi(a) = \chi_{-q}(-q) = 0$, the result of Ex. 34 is false if $b=0$.

Conclusion : if $\pi = a+bi$ is a primary irreducible, and $b\neq 0$, then 
\begin{enumerate}
\item[(a)] if $\pi \equiv 1 \ [4], \chi_\pi(a) = i^{(a-1)/2}$,
\item[(b)] if $\pi \equiv 3+2i\ [4]$, $\chi_\pi(a) = -i^{(-a-1)/2}$.
\end{enumerate}
\end{proof}

\paragraph{Ex. 9.37}

{\it Combine Exercises 32, 33, 34, and 35 to show $\chi_\pi(1+i) = i^{(a-b-b^2-1)/4}$. Show that this result implies Exercise 26 of Chapter 5 ``the biquadratic character of 2").
}

\bigskip

{\bf Lemma.} If $\pi = a +bi$ is a primary prime, then $$\chi_\pi(i) = i^\frac{-a+1}{2}.$$

\begin{proof}{(of Lemma.)}
Let $\pi = a +bi$ a primary prime in $\Z[i]$.
\begin{enumerate}
\item[$\bullet$] If $\pi = -q$, where $q \equiv 3 \pmod 4, q>0$ is a rational prime, then $a= -q, b=0$. By definition of the quartic character,
$$\chi_q(i) = i^\frac{N(q) - 1}{4} = i^\frac{q^2-1}{4}.$$

Write $-q = a = 4k+1,\ k \in \Z$. Then
\begin{align*}
\frac{q^2-1}{4} &= 4k^2 + 2k\\
&\equiv 2k  =\frac{a-1}{2}\pmod 4.
\end{align*}
Therefore $$\chi_{-q}(i)  = \chi_{q}(i) = i^{\frac{q^2-1}{4}} = i^\frac{a-1}{2} = \left(\frac{1}{i}\right)^\frac{-a+1}{2}= (-i)^\frac{-a+1}{2} =  (-1)^\frac{-a+1}{2} i ^\frac{-a+1}{2} = i ^\frac{-a+1}{2},$$
since $(-1)^\frac{-a+1}{2} = (-1)^{-2k} = 1$.

Suppose now that $N(\pi) = p$, where $p\equiv 1 \pmod 4$ is a rational prime. Then 
$$\chi_\pi(i) = i^\frac{N(\pi)-1}{4} = i^\frac{p-1}{4}.$$
 Since $\pi = a + bi$ is primary, there are two cases.
\item[$\bullet$] If $a\equiv 1 \pmod 4, b \equiv 0 \pmod 4$, then $a = 4A + 1, b = 4B,\ A,B \in \Z$.
\begin{align*}
\frac{p-1}{4} &= \frac{a^2+b^2-1}{4}\\
&=\frac{16 A^2 +8A + 16 B^2}{4}\\
&=4A^2 +2A + 4B^2\\
&\equiv 2A  = \frac{a-1}{2}
\end{align*}
Therefore $$\chi_{\pi}(i) = i^{\frac{p-1}{4}} = i^{\frac{a-1}{2}} = \left(\frac{1}{i}\right)^\frac{-a+1}{2}= (-i)^\frac{-a+1}{2} =  (-1)^\frac{-a+1}{2} i ^\frac{-a+1}{2} = i ^\frac{-a+1}{2},$$
since $(-1)^\frac{-a+1}{2} = (-1)^{-2A} = 1$.

\item[$\bullet$] If $a\equiv 3\pmod 4, b \equiv 2 \pmod 4$, then $a = 4A -1, B = 4B+2,\ A,B\in \Z$.
\begin{align*}
\frac{p-1}{4} &= \frac{a^2+b^2-1}{4}\\
&=\frac{16 A^2 - 8A + 16 B^2+ 16B +4}{4}\\
&= 4A^2 - 2A + 4B^2 4B+1\\
&\equiv -2A + 1  = \frac{-a+1}{2} \pmod 4
\end{align*}
Therefore $\chi_\pi(i) = (-1)^\frac{-a+1}{4}$.

The equality $\chi_\pi(i) = (-1)^\frac{-a+1}{4}$ is verified for all primary primes $\pi$.
\end{enumerate}

\end{proof}

\begin{proof}{(of Ex.9.37)}
Let $\pi = a + ib$ be a primary irreducible in $\Z[i]$.
\begin{enumerate}
\item[$\bullet$] If $b = 0$, then $\pi = a \in \Z$.
As $\pi$ is primary, $\pi = -q, q\equiv 3 \pmod 4$, where $q$ is a rational prime, so $a=-q, b = 0$.
By Ex. 9.32 (or its generalization 9.35),
 $$\chi_\pi(1+i) = \chi_{-q}(1+i) = i^{(-q-1)/4} = i^{(a-b-b^2-1)/4}.$$
 \item[$\bullet$] If $b \ne  0$, then $a\wedge b = 1$ (see Ex. 9.36), and by Ex. 9.35, 
 $$\chi_\pi(a) \chi_\pi(1+i) = i^{3(a+b-1)/4}.$$
 	\begin{enumerate}
	\item[$\bullet$]  If $\pi \equiv 1\ [4]$, $a \equiv 1\ [4], b \equiv 0 \ [4]$ : $a=4A+1,b=4B,\ A,B\in \mathbb{Z}$.
	
By Ex. 9.34(a),
$$\chi_\pi(a) = i^{(a-1)/2}, \chi_\pi(a)^{-1} = (-i)^{(a-1)/2} = i^{(a-1)/2}.$$
\begin{align*}
\chi_{\pi}(1+i) &= i^{3 \frac{a+b-1}{4} - 2 \frac{a-1}{4}}\\
&=i^{\frac{a+3b-1}{4}}\\
&=i^{\frac{a-b-b^2-1}{4}},
\end{align*}
since $\left( \frac{a+3b-1}{4} \right)- \left( \frac{a-b-b^2-1}{4}\right) = b + \frac{b^2}{4} = 4B+4B^2 \equiv 0 \ [4]$.

	\item[$\bullet$] If $\pi \equiv 3+2i\  [4]$, $a \equiv 3[4], b \equiv 2 \ [4]$ : $a=4A-1,b=4B+2,\ A,B\in\mathbb{Z}$.
	
By Ex. 9.34(b),

$$\chi_\pi(a) =- i^{(-a-1)/2}, \chi_\pi(a)^{-1} = -i^{(a+1)/2} = i^{(a-3)/2},$$
so
$$\chi_\pi(1+i) = i^{(3a+3b-3+2a-6)/4} = i^{(5a+3b-9)/4}.$$

Now $\frac{1}{4}[(a-b-b^2-1) - (5a+3b-9)] = \frac{1}{4}(-4a-4b-b^2+8) = -a-b+2 -\frac{b^2}{4} = -4A+1-4B-2+2 -(2B+1)^2 \equiv 0 \ [4]$,

thus $\chi_\pi(1+i) = i^{(a-b-b^2-1)/4}$.



	\end{enumerate}
	Conclusion : if $\pi =a+ib$ is primary irreducible, then

$$\chi_\pi(1+i) = i^{(a-b-b^2-1)/4}$$

\end{enumerate}

\bigskip

Second part : the biquadratic character of 2 (see Ex. 5.25 to 5.28).

Let $p\equiv 1 \ [4]$. Then $p=N(\pi)$, where $\pi = a+bi$ is a primary prime.

We show first that $\chi_\pi(2)=i^\frac{ab}{2}$.

Since $2 = i^3(1+i)^2$, the first part of the exercise, and the Lemma, give
\begin{align*}
\chi_\pi(2) &= \chi_\pi(i)^3 \chi_\pi(1+i)^2\\
&=i^\frac{3(-a+1)}{2} i ^\frac{a-b-b^2-1}{2}\\
&=i^{1-a -(b+1)\frac{b}{2}}
\end{align*}
Since $\pi$ is primary, $a\equiv b+1 \equiv -b+1\pmod 4$, therefore
\begin{align*}
1-a -(b+1)\frac{b}{2} &\equiv - b -(b+1)\frac{b}{2}\\
&\equiv \frac{b}{2}(-b-3)\\
&\equiv \frac{b}{2}(-b+1)\\
&\equiv \frac{ab}{2} \qquad \pmod 4,
\end{align*}
so $\chi_\pi(2) = i^\frac{ab}{2} $.
 
   \bigskip
   
 Now we show that $p$ is of the form $p=A^2+64B^2$ if and only if $p\equiv 1 \pmod 4$ and if $x^4 \equiv 2$ has a solution $x \in \Z$.
 
 If $p=A^2+64B^2 = A^2 + (8B)^2$, then the prime number $p$ is a sum of two squares, and $p\ne 2$, therefore $p\equiv 1 \pmod 4$. Since $p=A^2 + 64B^2$, $A$ is odd. Put $b = 8B$, and $a = A$ if $A\equiv 1 \pmod 4$, $a = -A$ if $A\equiv -1 \pmod 4$. Then $\pi =a +bi$ is such that $N(\pi) = a^2 +b^2 = p$, and $a\equiv 1, b\equiv 0 \pmod 4$, therefore $\pi$ is a primary prime. Then
 $$\chi_\pi(2) = i^\frac{ab}{2} = i^{4aB} = 1.$$
  Therefore there exists $\alpha \in D$ such that $2\equiv \alpha^4 \pmod \pi$. As $D/\pi D$ is the set of classes of $0,1,\cdots,p-1$, there exists $x\in \Z$ such that $x\equiv \alpha \pmod \pi$, so $2 \equiv x^4 \pmod \pi$.
 
 Then $p = N(\pi) \mid N(x^4-2) = (x^4-2)^2$, thus $p \mid x^4 -2$, in other words $2\equiv x^4 \pmod p$.
   
Conversely, suppose that $p\equiv 1 \pmod 4$ and that $2$ is a biquadratic residue modulo $p$. As $p\equiv 1 \pmod 4$, $p = \pi \overline{\pi}$, where $\pi =a + bi$ is a primary prime. Since $2\equiv x^4 \pmod p$ for some $x \in \Z$, then $2 \equiv x^4 \pmod \pi$, so $\chi_\pi(2)=1$. Moreover
$$1 = \chi_\pi(2) = i^\frac{ab}{2}.$$
Since $a$ is odd, $8 \mid b$, therefore $p=A^2 + 64B^2$, where $A=a, B =b/8$.
  
 Conclusion :
 $$\exists (A,B) \in \Z^2,\ p = A^2+64B^2 \iff (p\equiv 1 \ [4] \ \mathrm{and}\ \exists x \in \Z, \ x^4 \equiv 2 \ [p]).$$
\end{proof}

\paragraph{Ex. 9.38}

{\it Prove part (d) of Proposition 9.8.3.
}

\bigskip

{\bf Proposition 9.8.3(d)} {\it 
If $\pi$ is a primary irreducible then $\chi_{\pi}(-1)=(-1)^{(a-1)/2}$, where $\pi = a+bi$.}



\begin{proof}
Let $\pi = a + bi$ a primary irreducible. Then $a$ is odd, and $b$ is even, and $N(\pi) = a^2+b^2$. 
Then 
$$\chi_\pi(-1) = (-1)^{\frac{N(\pi)-1}{4}} = (-1)^{\frac{a^2-1}{4}+ \frac{b^2}{4}} = [(-1)^{\frac{a+1}{2}}]^{\frac{a-1}{2}} (-1)^{\frac{b^2}{4}}.$$
By Lemma 6, section 7, $a\equiv 1\ [4], b\equiv 0\ [4]$, or $a \equiv 3 [4], b\equiv 2 [4]$.
\begin{enumerate}
	\item[$\bullet$] If $a\equiv 1\ [4], b\equiv 0\ [4]$, then $(-1)^{\frac{a+1}{2}} = -1, (-1)^{\frac{b^2}{4}}=+1$ , so
	$$\chi_\pi(-1) =(-1)^{\frac{a-1}{2}}.$$
	\item[$\bullet$] If $a \equiv 3 \ [4], b\equiv 2 \ [4]$, then $(-1)^{\frac{a+1}{2}} = 1, (-1)^{\frac{b^2}{4}}=-1$, so
	$$\chi_\pi(-1) =-1 = (-1)^{\frac{a-1}{2}}.$$

\end{enumerate}

Conclusion: if $\pi$ is a primary irreducible in $\Z[i]$, then
$$\chi_{\pi}(-1)=(-1)^{(a-1)/2}.$$
\end{proof} 

\paragraph{Ex. 9.39}

{\it Let $p\equiv 1 \pmod 6$ and write $4p = A^2 + 27 B^2, \ A \equiv 1 \pmod 3$. Put $m=(p-1)/6$. Show $\binom{3m}{m} \equiv -1 \pmod p \iff 2 \mid B$.
}

\begin{proof}
Let $p$ be a rational prime, $p\equiv 1 \pmod 6$. As $p\equiv 1 \pmod 3$, we know from Theorem 2, Chapter 8, that there are integers $A$ and $B$ such that $4p = A^2 + 27B^2, A\equiv 1 \pmod 3$, and that $A$ is uniquely determined by these conditions.

Then $A,B$ have same parities. If we take $a =\frac{A+3B}{2}, b = 3B$, then $A = 2a -b, B = \frac{b}{3}$, and $4p = (2a-b)^2 + 3 b^2$, so $p = a^2 -ab + b^2$. If $\pi = a + b \omega$, then $N(\pi)= p$. Since $A = 2a-b \equiv 1 \ [3]$, and $b = 3B \equiv 0 \ [3]$, then $a \equiv -1 \ [3]$, so $\pi$ is a primary prime.

\bigskip

$\bullet$ Suppose that $2\mid B$.
Since $p =a^2 -ab +b^2$ is odd, and $b = 3B$,
$$2 \mid B \iff 2 \mid b \iff (b \equiv 0 \  [2], a\equiv 1\ [2]) \iff \pi \equiv 1 \ [2].$$
By Proposition 9.6.1, 
 $$\pi \equiv 1 \ [2] \iff x^3-2 \text{ is solvable in } D \iff  \chi_\pi(2) = 1.$$
 Therefore
 $$2 \mid B \iff \chi_\pi(2) = 1.$$
 Here $\chi_\pi$ is of order 3, so $\chi_\pi^2 \ne \varepsilon$. By Exercise 8.6,
$$J(\chi_\pi,\chi_\pi) = \chi_\pi(2)^{-2} J(\chi_\pi,\rho),$$
where $\rho $ is the Legendre's character.

In this case, $2 \mid B$, $\chi_\pi(2) = 1$, so $J(\chi_\pi,\chi_\pi) =  J(\chi_\pi,\rho)$, and by Lemma 1 section 4, where $p\equiv 1 \ [3]$ and $p = N(\pi)$, $$\pi = a + b \omega = J(\chi_\pi,\chi_\pi) = J(\chi_\pi,\rho).$$

By Exercise 8.15, $$N(y^2 = x^3+1) = p+A,$$ and the Exercise 8.27(b) gives
$$N(y^2 = x^3 + 1) = N(y^2 +x^3 = 1) = p+ 2 \re J(\chi_\pi,\rho).$$
thus
$$A = 2 \re J(\chi_\pi,\rho) = 2 \re \pi = 2a -b.$$
Moreover, since $J(\chi_\pi,\rho) = \pi = a + b \omega$, by Exercise 8.27(c),
$$2a-b \equiv -\binom{(p-1)/2}{(p-1)/3}.$$
Therefore
$$-A \equiv \binom{(p-1)/2}{(p-1)/3} =\binom{(p-1)/2}{(p-1)/2 - (p-1)/6}  =  \binom{(p-1)/2}{(p-1)/6}=\binom{3m}{m} \pmod p,$$
where $m=(p-1)/6.$
Since $A\equiv 1 \pmod 3$,
$$\binom{3m}{m} \equiv -1 \pmod p.$$

\bigskip

$\bullet$ Conversely, suppose that $\binom{3m}{m} \equiv -1 \pmod p$. Then $A = 2a -b \equiv  -\binom{3m}{m} \pmod p$.
Write $J(\chi_\pi,\rho) = c + d\omega$. By Exercise 8.27(c), $2c - d \equiv -\binom{3m}{m} \pmod p$. thus
$$2a -b \equiv 2c - d \pmod p.$$

Since $|J(\chi_\pi,\rho)| = \sqrt{p}$, 
$$4p = (2a-b)^2 + 3b^2 = (2c-d)^2 + 3d^2,$$
thus $d \equiv \pm b \pmod p$.

By Exercise 8.6,
$$\pi = J(\chi_\pi,\chi_\pi) = \chi_\pi(2)^{-2} J (\chi_\pi,\rho),$$
Here $\chi_\pi$ is of order 3, therefore $\chi_\pi(2)^{-2} = \chi_\pi(2) \in \{1,\omega,\omega^2\}$, so
$$\pi = J(\chi_\pi,\chi_\pi) = \chi_\pi(2) J (\chi_\pi,\rho).$$

If $\chi_\pi(2) = \omega$, then $a+b\omega = \omega(c+d\omega) = -d + \omega(c-d)$. Then $a = -d \equiv \pm b \pmod p$. As $a \equiv -b\omega \pmod \pi$, we would have $-b\omega \equiv \pm b \pmod \pi$. Here $\pi \nmid b$,  otherwise $p = N(\pi) \mid N(b) = b^2$, so $p \mid b$, and $p =a^2 -ab +b^2$, so $p \mid a$, and $p^2 \mid p$, which is a nonsense. Therefore $\pi \mid \omega \pm 1$, where $\pi$ is a primary prime: it's impossible. Indeed $\omega + 1$ is a unit and $\omega - 1$ is prime, so $\pi \mid \omega - 1 = - \lambda$ implies that $\pi$ and $\lambda$ are associate, in contradiction with $N(\pi) = p \ne 3 = N(\lambda)$.

If $\chi_\pi(2) = \omega^2$, then $a+b\omega = \omega^2(c+d\omega) = (d-c) - \omega c$, so $a = d-c, b = -c$. 

Reasoning modulo $\overline{\pi} = a + b \omega^2 = (a-b) + b \omega$, where $\overline{\pi} \mid \pi \overline{\pi} =p$, we obtain
$$d = a-b \equiv -b \omega \pmod{\overline{\pi}},$$
where $d \equiv \pm b \pmod{\overline{\pi}}$, so $-b\omega \equiv \pm b\pmod{\overline{\pi}}$. Since $N(\overline{\pi}) = p$, we obtain the same contradiction as above.


So $\chi_\pi(2) = 1$, and the previously proved equivalence $2\mid B \iff \chi_\pi(2) = 1$ show that $2 \mid B$.

Conclusion: $$\binom{(p-1)/2}{(p-1)/6} \equiv -1 \pmod p \iff 2 \mid B.$$
\end{proof}

\paragraph{Ex. 9.40}
{\it Let $p\equiv 1 \pmod 6$, and put $p = \pi \overline{\pi}$ where $\pi$ is primary. Write $\pi = a + b\omega$ and show
\begin{enumerate}
\item[(a)] If $\chi_\pi(2) = \omega$ then $2b-a \equiv -\binom{3m}{m} \pmod p$.
\item[(b)] If $\chi_\pi(2) = \omega^2$ then $a+b \equiv \binom{3m}{m} \pmod p$.
\item[(c)] If $\chi_\pi(2) = \omega$ put $A=2a-b,B = b/3$. Show $(A-9B)/2 \equiv \binom{3m}{m} \pmod p$.
\item[(d)] If $\chi_\pi(2) = \omega^2$ put $2a-b = A$ and $B = -b/3$. Show $(A-9B)/2 \equiv \binom{3m}{m} \pmod p$.
\item[(e)] Show that the ``normalization" of $B$ in (c) and (d) is equivalent to $A \equiv B \pmod 4$.
[Recall $\chi_\pi(2) \equiv \pi \pmod 2$ by cubic reciprocity.]
\end{enumerate}
}

\begin{proof}
Here $p = 6m + 1, m \in \Z$, and $p = \pi \overline{\pi}$, where $\pi = a + b\omega$ is a primary prime.

We have proved in Exercise 39 that
\begin{align}
\pi = J(\chi_\pi,\chi_\pi) = \chi_\pi(2) J(\chi_\pi,\rho).
\end{align}
Write $J(\chi_\pi,\rho) = c + d \omega$. The Exercise 8.27(c) shows that
\begin{align}
2c -d \equiv -\binom{3m}{m} \pmod p.
\end{align}

\begin{enumerate}
\item[(a)] If $\chi_\pi(2) = \omega$, then (1) gives
\begin{align*}
a + b \omega &= \omega (c+ d\omega) =-d+ \omega(c-d),
\end{align*}
so $a = -d, b = c-d$, therefore the equality (2) gives
$$2b-a = 2(c-d)+ d = 2c -d \equiv -\binom{3m}{m} \pmod p.$$


\item[(b)] If $\chi_\pi(2) = \omega^2$, then
$$a + b \omega = \omega^2 (c+ d\omega) = d-c -c\omega,$$
so $a = d-c, b = -c$, and
$$a + b =d-2c\equiv \binom{3m}{m} \pmod p.$$

\item[(c)] Suppose that $\chi_\pi(2) = \omega$, and put $A = 2a-b, B = b/3$, so $$4p = A^2 + 27 B^2,\quad A\equiv 1 \ [3],$$
which shows that $A,B$ have same parities. Then, by part (a),
\begin{align*}
\frac{A-9B}{2} &= \frac{2a-b - 3b}{2}\\
&=a-2b\\
&\equiv \binom{3m}{m} \pmod p
\end{align*}

\item[(d)] Suppose that $\chi_\pi(2) = \omega^2$, and put $A = 2a-b, B = -b/3$, so we have again
 $$4p = A^2 + 27 B^2,\quad A\equiv 1 \ [3].$$
In this case, by part (b)
\begin{align*}
 \frac{A-9B}{2} &= \frac{2a-b +3b}{2}\\
 &=a+b\\
&\equiv \binom{3m}{m} \pmod p
 \end{align*}
 
 \item[(e)] The conditions $4p = A^2 + 27 B^2,\ A\equiv 1 \ [3],$ determine $A,B$, except the sign of $B$. So $4p = A^2 + 27 B^2 = (2a-b)^2 + 3b^2$, implies $A = 2a-b$ and $B = \pm \frac{b}{3}$.
 
 By Exercise 39, since $A,B$ have same parity, the condition $A,B$ odd is equivalent to $\chi_\pi(2) \in \{\omega, \omega^2\}$. We choose this sign of $B$ so that
 $$ \frac{A-9B}{2} \equiv \binom{3m}{m} \pmod p.$$
 
 By parts (d) and (e), where $A,B$ are odd, this choice is given by $B = b/3$ if $\chi_\pi(2) = \omega$, and $B = -b/3$ if $\chi_\pi(2) = \omega^2$. We show that these conditions are equivalent to $A\equiv B \pmod 4$.
 
 $\bullet$ If $\chi_\pi(2) = \omega$, then $A = 2a-b, B = b/3$.
 
 By cubic reciprocity, $\chi_\pi(2) \equiv \pi \pmod 2$ (see section 6). Here $\chi_\pi(2) = \omega$, so $\omega \equiv a + b \omega \pmod 2$, therefore $a \equiv 0 \pmod 2, b\equiv 1 \pmod 2$, 
$$A = 2a - b \equiv -b \equiv \frac{b}{3}  = B \pmod 4,$$

so $A \equiv B \pmod 4$.


 
 $\bullet$ If $\chi_\pi(2) = \omega^2$, then $A = 2a-b, B = -b/3$. In this case,
 $$\omega^2 = - 1 - \omega \equiv a + b \omega \pmod 2,$$
 therefore $a \equiv 1 \equiv b \pmod 2$, and
 $$A  = 2a - b \equiv 2 - b \equiv b \equiv -\frac{b}{3} = B \pmod 4.$$
 In both cases, the choice of the sign of $B$ implies that $A \equiv B \pmod 4$.
 
 \bigskip
 
Conversely, suppose that $A \equiv B \pmod 4$.  Write $B = \varepsilon \frac{b}{3}$, where $\varepsilon = \pm 1$. Then $A \equiv B \pmod 4$ gives
$$2a-b \equiv \varepsilon \frac{b}{3} \equiv -\varepsilon b \pmod 4,$$
thus $a \equiv \frac{1-\varepsilon}{2} b \pmod 2$. Then
\begin{align*}
\chi_\pi(2) &\equiv \pi = a + b \omega\\
&\equiv b\left( \frac{1-\varepsilon}{2} + \omega\right) \pmod 2
\end{align*}

 If $\chi_\pi(2) = \omega$, since $b = 3B$ is odd, $ \frac{1-\varepsilon}{2}  \equiv 0 \pmod 2$, therefore $\varepsilon = 1$, and $B = \frac{b}{3}$.
 
 If $\chi_\pi(2) = \omega^2 = - 1 - \omega$, $ \frac{1-\varepsilon}{2}  \equiv 1 \pmod 2$, therefore $\varepsilon = -1$, and $B = -\frac{b}{3}$.
 
 The normalisation given in parts (c) and (d) for the choice of the sign of $B$ is equivalent to $A \equiv B \pmod 4$ (where $A,B$ are odd).
\end{enumerate}

\end{proof}

\paragraph{Ex. 9.41}
{\it Let $p\equiv 1 \pmod 6, 4p = A^2 + 27 B^2, A \equiv 1 \pmod 3$, $A$ and $B$ odd. Put $\pi = a+b\omega, 2a-b =A, b = 3B$. Let $\chi_\pi$ be the cubic residue character.
\begin{enumerate}
\item[(a)] If $\chi_\pi(2) = \omega$ show $N(x^3+2y^3 = 1) = p+1 + 2b-a \equiv 0 \pmod 2$.
\item[(b)] If $\chi_\pi(2) = \omega^2$ show $N(x^3 + 2y^3 = 1) = p+1 -a-b \equiv 0 \pmod 2$.
\item[(c)] Show that if $A \equiv B \pmod 4$, then assuming $\chi_\pi(2) \ne 1$, one has $\chi_\pi(2) = \omega$.
\item[(d)] If $\chi_\pi(2) \ne 1, A \equiv B \pmod 4$ then $$2^{(p-1)/3} \equiv (-A-3B)/6B \equiv (A+9B)/(A-9B) \pmod \pi.$$

(This generalization of Euler's criterion is due to E.Lehmer [174]. See also K.Williams [243].)
\end{enumerate}
}

\begin{proof}
With the help of Theorem 1, Chapter 8, we obtain, writing $\chi_\pi(2) = \omega^k$,
\begin{align*}
N(x^3 + 2 y^3 = 1)&= \sum_{a+2b = 1} N(x^3 =a) N(y^3 = b)\\
&=\sum_{a+2b = 1} \left(\sum_{i=0}^2 \chi_\pi^i(a)\right)\left( \sum_{j=0}^2 \chi_\pi^j(b)\right)\\
&=\sum_{i=0}^2\sum_{j=0}^2 \sum_{a+2b=1} \chi_\pi^i(a)\chi_\pi^j(b)\\
&=\sum_{i=0}^2\sum_{j=0}^2  \sum_{a+b' = 1} \chi_\pi^i(a) \chi_\pi^j(2^{-1} b')\\
&=\sum_{i=0}^2\sum_{j=0}^2 \chi_\pi(2)^{-j} J(\chi_\pi^i, \chi_\pi^j)\\
&=\sum_{i=0}^2\sum_{j=0}^2 \omega^{-kj} J(\chi_\pi^i, \chi_\pi^j)\\
&= p + \omega^{-k} J(\chi_\pi^2,\chi_\pi) + \omega^{-2k} J(\chi_\pi,\chi_\pi^2)\\
&\phantom{= p .} + \omega^{-k} J(\chi_\pi, \chi_\pi) + \omega^{-2k} J(\chi_\pi^2, \chi_\pi^2)\\
&=p - \omega^{-k} \chi_\pi(-1) - \omega^{-2k} \chi_\pi(-1)^2 + 2 \re(\omega^{-k} J(\chi_\pi,\chi_\pi)\\
&= p - \omega^{-k} -  \omega^{-2k} +  2 \re(\omega^{-k} J(\chi_\pi,\chi_\pi)).
\end{align*}

\begin{enumerate}

\item[(a)] If $\chi_\pi(2) = \omega$, then $k=1$. Using $\chi_\pi^2 = \chi_\pi^{-1} =\overline{\chi_\pi}$, we obtain
\begin{align*}
N(x^3 + 2 y^3 = 1)&= p+1+ 2 \re(\omega^{2} J(\chi_\pi,\chi_\pi))\\
&=p+1 + 2 \re(\omega^2 \pi),
\end{align*}
since $J(\chi_\pi,\chi_\pi) = \pi$ (Lemma 1, section 4).
\begin{align*}
\omega^2\pi &= \omega^2(a + b \omega) = b - a -\omega a,\\
2 \re(\omega^2\pi) &= (b-a -\omega a) + (b - a - \omega^2 a) = 2b-2a + a = 2b-a,
\end{align*}
therefore
$$N(x^3 + 2 y^3 = 1) = p+1 + 2b -a\qquad (\text{if }\chi_\pi(2) = \omega).$$
Since in then case $\chi_\pi(2) = \omega$, then $a \equiv 0 \pmod 2$ (see Ex. 39, part (e)), so $p + 1 + 2b -a \equiv 0 \pmod 2$.

(b) If $\chi_\pi(2) = \omega^2 = \omega^{-1}$, then $k=-1$, and
\begin{align*}
N(x^3 + 2 y^3 = 1)&= p+1+ 2 \re(\omega \pi),\\
\end{align*}
with 
\begin{align*}
\omega \pi &= \omega (a + b \omega) = -b + (a-b)\omega,\\
2 \re(\omega \pi) &= (-b + (a-b) \omega) +( -b + (a-b)\omega^2) = -2b - (a-b) = -a -b,
\end{align*}
therefore
$$N(x^3 + 2y^3 = 1) = p+1 -a-b \qquad (\text {if }\chi_\pi(2) = \omega^2).$$


\item[(c)] Suppose that $A\equiv B \pmod 4$, and $\chi_\pi(2)\ne 1$. By hypothesis, $b = 3B$, and this implies by Exercise 40 (e) that $\chi_\pi(2) = \omega$ (if not, $\chi_\pi(2) = \omega^2$, and $A\equiv B \pmod 4$ gives $B = -b/3$).

\item[(d)] Suppose that $\chi_\pi(2) \ne 1, A\equiv B \pmod 4$. By part (c), $\chi_\pi(2) = \omega$.

Since $2a-b = A, B = b/3$, then $a = \frac{A+3B}{2}, b = 3B$.

Starting from $a + b \omega \equiv 0 \pmod \pi$, we obtain
$$ 3B \omega \equiv -\frac{A+3B}{2} \pmod \pi.$$
Since $pa = a^2 - ab + b^2$, $a$ is relatively prime with $p$, therefore $\pi \wedge b = 1$, so $\pi \wedge B = 1$, and $\pi \wedge 6 = 1$, since $p \equiv 1 \pmod 6$, thus
$$\chi_\pi(2)  = \omega \equiv \frac{-A-3B}{6B} \pmod \pi,$$
where we must read in this fraction the product of $A + 3B$ by the inverse modulo $p$ of $6B$.
By definition, using $N(\pi) = p$, 
$$\chi_\pi(2) \equiv 2^\frac{p-1}{3} \pmod \pi,$$
so
$$2^{\frac{p-1}{3}} \equiv \frac{-A-3B}{6B}  \pmod \pi.$$

Moreover, since  $4p = A^2 + 27 B^2$, $A^2 + 27 B^2 \equiv 0 \pmod p$, therefore
$$ 6B(A+9B) + (A+3B)(A-9B) \equiv 0 \pmod p.$$
If $p \mid A-9B$, since $p \nmid 6B$, this equality implies that $p\mid A+9B$, therefore $p \mid (A-9B) + (A+9B) = 2A$, which is false. Therefore $A-9B \not \equiv 0 \pmod p$, and
$$2^{\frac{p-1}{3}} \equiv \frac{-A-3B}{6B}  \equiv \frac{A+9B}{A-9B} \pmod \pi.$$
\end{enumerate}

\end{proof}

Note : By a usual argument, if $h \in \Z$, $2^{\frac{p-1}{3}} \equiv h \pmod \pi \iff 2^{\frac{p-1}{3}} \equiv h \pmod p$. Note that the hypothesis $\chi_\pi(2) \ne 1$ means that $2$ is not a cubic residue modulo $p$, which is equivalent to $A,B$ odd by Exercise 39. We can conclude

\bigskip

{\it
Suppose that $p \equiv 1 \pmod 6$, and let $(A,B)$ be the unique solution of $4p = A^2 + 27 B^2$ such that $A \equiv 1 \pmod 3$, and $B\equiv A \pmod 4$ if $B$ odd, and $B>0$ otherwise.

If $B$ is even, then $2$ is a cubic residue modulo $p$, and $2^\frac{p-1}{3} = 1$.

If $B$ is odd, then $2$ is not a cubic residue modulo $p$, and $B$ satisfies $B \equiv A \pmod 4$. 

Writing $a = \frac{A+3B}{2}, b = 3B$, and $\pi = a + b \omega$, then  $\chi_\pi(2) = \omega$, and
$$2^{\frac{p-1}{3}}  \equiv \frac{A+9B}{A-9B} \pmod p.$$
}

\bigskip

The three roots of $x^3 - 1$ in $\F_p$ are $1, \frac{A+9B}{A-9B}, \frac{A-9B}{A+9B}$. Here $2$ is not a cubic residue modulo $p$, and $2^{\frac{p-1}{3}}$ is also a cubic root of unity modulo $p$, so $2^{\frac{p-1}{3}} \equiv  \frac{A\pm 9B}{A\mp9B} \pmod p$. The proposition  explicits the choice of the sign of $B$ which gives $2^{\frac{p-1}{3}}  \equiv \frac{A+9B}{A-9B} \pmod p$.

\bigskip

Numerical example : Let $p$ be the prime $967$. If we decompose $p$ on the form $p = \pi \overline{\pi}$, we obtain $\pi =  a + b \omega = -34 -27 \omega$. To obtain these result without tries, I find $k = 682$ such that $p \mid k^2 + 3$ with the Tonelli-Shanks algorithm, and I compute $\gcd(p, k+1 + 2\omega) = a + b \omega$, where $a + b \omega$ is primary, with a small Python program using the class of elements in $\Z[\omega]$ and the Euclid algorithm in $\Z[\omega]$. This gives the decompositions 
$$967 = p = a^2 -ab + b^2 = 34^2 - 34\times 27 + 27^2,$$
 and 
 $$3868 = 4p = (2a -b)^2 + 3b^2 = A^2 + 27 B^2 = 41^2 + 27 \times 9^2,$$
 where $A \equiv 1 \pmod 3$, and I choose the sign of $B$ such that $B \equiv A \pmod 4$. We obtain $A = -41, B = -9$, and $a,b$ must verify $A = 2a-b, B = b/3$.
 
  Then $\chi_\pi(2) = \omega$, where $\pi = -41 - 9 \omega$. In $\F_{967}$, the cubic roots of unity modulo $p$ are $1, 142, 824$: $142^3 \equiv 824^3 \equiv 1 \pmod {967}$. 
  
  Here $(A + 9B) (A - 9B)^{-1} = 142$, and we verify with a fast exponentiation that $2^\frac{p-1}{3} = 2^{322} \equiv 142 \pmod {967}.$


I give here an extract of a table obtained with this program, which for each $p$ gives $A,B$ such that $4p = A^2 + 27 B^2, A \equiv 1 \pmod 3$ and such that $B \equiv A \pmod 4$ if $A,B$ odd, and $\pi =a +b \omega$ satisfies $\chi_\pi(2) = \omega$ (or $\chi_\pi(2) = 1$ if $A,B$ even, which corresponds to the case $a\equiv 1, b \equiv 0 \pmod 2$).
$$
\begin{array}{l|l|l|l|l|l|l|l|l}
p    &  A   & B  & \pi = a + b \omega   & a \text{\%} 2 & b \% 2 & 2^\frac{p-1}{3}\ \% \ p & \frac{A-9B}{A+9B} & \chi_\pi(2)\\
\hline
787 & 31 & -9 & 2  -27\omega & 0 & 1 & 379 & 379 &\omega \\
811 & -56 & -2 & -31  -6 \omega & 1 & 0 & 1 & 130 & 1 \\
823 & -5 & 11 & 14 + 33\omega & 0 & 1 & 648 & 648 & \omega \\
829 & 7 & 11 & 20 + 33\omega & 0 & 1 & 125 & 125 & \omega  \\
853 & -35 & 9 & -4 + 27\omega & 0 & 1 & 632 & 632 & \omega \\
859 & 13 & -11 & -10  -33\omega & 0 & 1 & 260 & 260 & \omega \\
877 & -59 & 1 & -28 + 3 \omega& 0 & 1 & 594 & 594 & \omega  \\
883 & -47 & -7 & -34  -21\omega & 0 & 1 & 545 & 545 & \omega \\
907 & 19 & 11 & 26 + 33\omega & 0 & 1 & 384 & 384 & \omega  \\
919 & 52 & -6 & 17  -18\omega & 1 & 0 & 1 & 52 & 1 \\
937 & 61 & 1 & 32 + 3\omega & 0 & 1 & 614 & 614 & \omega \\
967 & -41 & -9 & -34  -27\omega & 0 & 1 & 142 & 142 & \omega  \\
991 & 61 & -3 & 26  -9\omega & 0 & 1 & 113 & 113 & \omega  \\
997 & 10 & -12 & -13  -36\omega & 1 & 0 & 1 & 692 & 1 \\
\end{array}
$$
As a verification I compute $\chi_\pi(2)$ with a fast exponentiation in $\Z[\omega]$ : $\chi_\pi(2) = 2^\frac{p-1}{3} \pmod \pi$.

We obtain the primary prime $\mu$ such that $N(\mu) = p, \chi_\mu(2) = \omega^2$ by taking the conjugate of $\pi$. For instance, with $p = 787$,  $\pi =2 - 27\omega$ satisfies $\chi_\pi(2) = \omega$, therefore $\chi_{\overline{\pi}}(2) = \chi_{29 + 27\omega} = \omega^2$.


The lines where $\chi_\pi(2) = 1$, corresponding to the case where $A,B$ are even (or equivalently $a$ odd, $b$ even), give the decomposition ${p = x^2 + 27 y^2}$, ${(x = A/2,y = B/2)}$. For instance $997 = 5^2 + 27\times 6^2$. If $p$ is prime,
$$\exists x\in \Z, \exists y \in \Z,\  p = x^2 + 27y^2 \iff p\equiv 1 \pmod 3 \text{ and } \exists a \in \Z,\ 2 \equiv a^3 \pmod p.$$
\paragraph{Ex. 9.42}

{\it The notation being as in Section 12 show that the minimal polynomial of $g(\chi_\pi)$ is $x^3 - 3px-Ap$.
}

Note : we must read ``the minimal polynomial of $G = g(\chi_\pi) + \overline{g(\chi_\pi)}$ is $x^3 - 3px-Ap$''.

\begin{proof}
Write $f(x) = \sum_{i=0}^3 a_i x^i = x^3 - 3px - Ap$.

Then $a_3 = 1$, $p \mid a_0 = Ap, p \mid a_1 = -3p, p\mid a_2 = 0$.

Moreover, since $4p = A^2 + 27 B^2$, $p\nmid A$, therefore $p^2 \nmid a_0$.

The Eisenstein's Irreducibility Criterion (Ex. 6.23) shows that $f(x)$ is irreducible over $\Q$. By section 12, $G$ is a root of $f$, so $f$ is the minimal polynomial of $G$.
\end{proof}


\paragraph{Ex. 9.43}

{\it Find the local maxima and minima of $x^3 - 3px -Ap$ and show that each of the intervals $(-2\sqrt{p}, - \sqrt{p}), (-\sqrt{p},\sqrt{p}), (\sqrt{p},2 \sqrt{p})$ contains exactly one of the values $2 \mathrm{Re}(\omega^k g(\chi_\pi)),\ k=0,1,2$.
}

\begin{proof}
Write $\chi = \chi_\pi$, and for $k \in \{0,1,2\}$, 
$$G_k = 2 \mathrm{Re}(\omega^k g(\chi)) = \omega^k g(\chi) + \overline{\omega}^k \overline{g(\chi)},$$
so $G = G_0$.
As in section 12, since $g(\chi)^3 = p \pi$, and $|g(\chi)|^2 = p$,
\begin{align*}
G_k^3 &= g(\chi)^3 + \overline{g(\chi)}^3 + 3 \omega^{2k} g(\chi)^2 \overline{g(\chi)} + 3 \omega^k g(\chi) \overline{\omega}^{2k} \overline{g(\chi)}^2\\
&= p \pi + p \overline{\pi} + 3 g(\chi) \overline{g(\chi)} (\omega^k g(\chi) + \overline{\omega}^k \overline{g(\chi)}\\
&= 3 p G_k + p(2a -b)\\
&= 3p G_k + pA
\end{align*}
So $G_0, G_1,G_2$ are the three roots of $f(x) = x^3 - 3p x - Ap$.

$f'(x) = 3(x^2-p)<0$ iff $-\sqrt{p} < x < \sqrt{p}$. $f$ is decreasing on $[-\sqrt{p},\sqrt{p}]$, and increasing on $]-\infty, -\sqrt{p}[$, and on $[\sqrt{p}, + \infty[$.

Since $4 p = A^2 + 27 B^2$, $|A| < 2 \sqrt{p}$, therefore
\begin{align*}
f(\sqrt{p}) &= p \sqrt{p} - 3 p \sqrt{p} - Ap\\
&=-p(2\sqrt{p} + A)<0,
\end{align*}
and
\begin{align*}
f(-\sqrt{p}) &= - p \sqrt{p} + 3 p \sqrt{p} - Ap\\
&= p(2\sqrt{p} - A)>0.
\end{align*}
\end{proof}

Since $\lim\limits_{x \to - \infty} f(x)= -\infty$ and $\lim\limits_{x \to  +\infty} f(x)= +\infty$, the intermediate value theorem shows that $f$ has a unique root in each of the intervals $]-\infty, -\sqrt{p}[,]-\sqrt{p},\sqrt{p}[, [\sqrt{p}, + \infty[$.

Moreover
\begin{align*}
f(2\sqrt{p}) &= 8p\sqrt{p} - 6p\sqrt{p} - Ap = p(2\sqrt{p} - A) >0,\\
f(-2\sqrt{p}) &= -8p\sqrt{p} + 6 p \sqrt{p} - Ap = p(-2\sqrt{p} - A) <0,
\end{align*}
therefore $f$ has a unique root in each of the intervals  $]-2 \sqrt{p}, -\sqrt{p}[,]-\sqrt{p},\sqrt{p}[, [\sqrt{p}, 2 \sqrt{p}[$.








\paragraph{Ex. 9.44}

{\it Let $n \in \Z$, $n = s_1\cdots s_t, n\equiv 1 \pmod 4, i = 1,\ldots,t$. Show $(n-1)/4\equiv \sum_{i=1}^t (s_i-1)/4 \pmod 4$.
}

\begin{proof}
If  $n = s t , s \equiv1,t\equiv1\ [4]$, then $s = 4k+1, t = 4 l +1, k,t \in \mathbb{Z}$, so
$$n = (4k+1)(4l+1) = 16 kl + 4k+4l+1, \frac{n-1}{4} = 4 kl + k + l \equiv k+l = \frac{s-1}{4}+\frac{l-1}{4} \ [4].$$
Reasoning by induction on $t$, suppose that  every product of $t$ factors $ n = s_1s_2\cdots s_t$, where $s_i\equiv 1 \ [4]$ verifies 
$$\frac{n-1}{4} \equiv \sum\limits_{i=1}^t \frac{s_i-1}{4} [4].$$
If $n' = s_1 s_2 \cdots s_t s_{t+1} = n s_{t+1}, s_i \equiv 1[4]$, then $n\equiv1,s_{t+1} \equiv 1 \ [4]$, so
$$\frac{n'-1}{4} \equiv \frac{n-1}{4} + \frac{s_{t+1}-1}{4} \equiv \sum\limits_{i=1}^t \frac{s_i-1}{4} + \frac{s_{t+1}-1}{4} \equiv \sum\limits_{i=1}^{t+1} \frac{s_i-1}{4} \ [4].$$
Conclusion :  if  $n = s_1 s_2 \cdots s_t, s_i \equiv 1 [4]$, then $\frac{n-1}{4} \equiv \sum\limits_{i=1}^t \frac{s_i-1}{4} [4]$.
\end{proof}

\paragraph{Ex. 9.45}

{\it Let $\pi = a + bi \in \Z[i]$ and $q\equiv 3 \ [4]$ a rational prime. Show $\pi^q \equiv \overline{\pi} \ [q]$.
}

\begin{proof}
Let $\pi=a+bi \in \mathbb{Z}[i]$, and $q\equiv 3 \ [4]$ a rational prime.

As $\binom{q}{k} \equiv 0 \pmod q$ for $1\leq k \leq q-1$, the Fermat's Little Theorem gives
\begin{align*}
\pi^q &=(a+bi)^q\\
&\equiv a^q+b^qi^q\ [q]\\
&\equiv a + b i^3 \ [q]\\
&= a-bi \\
&= \overline{\pi}
\end{align*}

Conclusion : $\pi^q \equiv \bar{\pi}\ [q]$ ($\pi \in \mathbb{Z}[i]$, and $q\equiv 3 \ [4]$)
\end{proof}



\end{document}
