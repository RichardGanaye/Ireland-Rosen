%&LaTeX
\documentclass[11pt,a4paper]{article}
\usepackage[frenchb,english]{babel}
\usepackage[applemac]{inputenc}
\usepackage[OT1]{fontenc}
\usepackage[]{graphicx}
\usepackage{amsmath}
\usepackage{amsfonts}
\usepackage{amsthm}
\usepackage{amssymb}
\usepackage{yfonts}
%\input{8bitdefs}

% marges
\topmargin 10pt
\headsep 10pt
\headheight 10pt
\marginparwidth 30pt
\oddsidemargin 40pt
\evensidemargin 40pt
\footskip 30pt
\textheight 670pt
\textwidth 420pt

\def\imp{\Rightarrow}
\def\gcro{\mbox{[\hspace{-.15em}[}}% intervalles d'entiers 
\def\dcro{\mbox{]\hspace{-.15em}]}}

\newcommand{\D}{\mathrm{d}}
\newcommand{\Q}{\mathbb{Q}}
\newcommand{\Z}{\mathbb{Z}}
\newcommand{\N}{\mathbb{N}}
\newcommand{\R}{\mathbb{R}}
\newcommand{\C}{\mathbb{C}}
\newcommand{\F}{\mathbb{F}}
\newcommand{\re}{\,\mathrm{Re}\,}
\newcommand{\ord}{\mathrm{ord}}
\newcommand{\legendre}[2]{\genfrac{(}{)}{}{}{#1}{#2}}



\title{Solutions to Ireland, Rosen ``A Classical Introduction to Modern Number Theory''}
\author{Richard Ganaye}

\begin{document}

\maketitle


{ \Large \bf Chapter 9} 

\paragraph{Ex. 9.1}

{\it If $\alpha \in \Z[\omega]$, show that $\alpha$ is congruent to either $0,1$, or $-1$ modulo $1-\omega$.
}

\begin{proof}
Let $\lambda = 1 - \omega$, and $z = a+b\omega \in D = \Z[\omega], a,b \in \Z$.

$\omega \equiv 1 \pmod \lambda$, so $z \equiv a+b \pmod \lambda$, with $c = a+b \in \Z$.

$c\equiv 0,1,-1 \pmod 3$, and since $\lambda \mid 3$, $\lambda \equiv 0,1,-1 \pmod \lambda$.

Conclusion : every $z \in D$ is congruent to either $0,1$, or $-1$ modulo $\lambda = 1 -\omega$.

Note :  $1\not \equiv -1 \pmod \lambda$, if not $\lambda \mid 2$, so $2 = \lambda \lambda'$, $N(2) = N(\lambda) N(\lambda')$, thus $4 = 3 N(\lambda')$, so $3 \mid 4$ : this is absurd.

$\pm1 \equiv 0 \pmod \lambda$ implies $\lambda \mid 1$, so $\lambda$ would be an unit, in contradiction with $\lambda$ prime.

So there exist exactly three classes modulo $\lambda$ in $D$ : $ | D/\lambda D | = 3 = N(\lambda)$.

\end{proof}

\paragraph{Ex. 9.2}

{\it From now on we shall set $D = \Z[\omega]$ and $\lambda = 1 - \omega$. For $\mu$ in $D$ show that we can write $\mu = (-1)^a \omega^b \lambda^c\pi_1^{a_1}\pi_2^{a_2}\cdots\pi_t^{a_t}$, where $a,b,c$, and the $a_i$ are nonnegative integers and the $\pi_i$ are primary primes.
}

\begin{proof}
Let $S$ the set containing $\lambda = 1 - \omega$ and all primary primes. By Proposition 9.3.5,
\begin{enumerate}
\item[(a)] Every prime in $D$ is associate to a prime in $S$.
\item[(b)] No two primes in $S$ are associate.
\end{enumerate}
By Theorem 3, Chapter 1, as $D = \Z[\omega]$ is a principal ideal domain,  every $\mu \in D$ is of the form
$$\mu = u \prod_{\lambda \in S} \lambda^{e(\lambda)},$$
where $u$ is a unit, so $u = (-1)^a\omega^b$. Thus
$$\mu = (-1)^a \omega^b \lambda^c\pi_1^{a_1}\pi_2^{a_2}\cdots\pi_t^{a_t},$$ where the $\pi$ are primary primes, and $a,b,c$ and the $a_i$ are nonnegative integers.
\end{proof}

\paragraph{Ex. 9.3}

{\it Let $\gamma$ a primary prime. To evaluate $\chi_\gamma(\mu)$ we see, by Exercise 2, that it is enough to evaluate $\chi_\gamma(-1), \chi_\gamma(\omega), \chi_\gamma(\lambda)$, and $\chi_\gamma(\pi)$, where $\pi$ is a primary prime. Since $-1 =(-1)^3$ we have $chi_\gamma(-1) = 1$. We now consider $\chi_\gamma(\omega)$. Let $\gamma = a+b \omega$ and set $a = 3m -1$ and $b=3n$. Show that $\chi_\gamma(\omega) = \omega^{m+n}$.
}

\begin{proof}
Let $\gamma = a+b\omega=3m-1+3n\omega$. Then  $\chi_{\gamma}(\omega) = \omega^{\frac{N(\gamma)-1}{3}}$.
\begin{align*}
N(\gamma)-1 &= (3m-1)^2+(3n)^2-3n(3m-1)-1\\
&=9m^2 - 6m + 9n^2 -9nm + 3n\\
\frac{N(\gamma)-1}{3}&= 3m^2-2m+3n^2-3nm+n \equiv n+m \ [3]
\end{align*}
Thus, for $\gamma = a+b\omega=3m-1+3n\omega$,

$$\chi_{\gamma}(\omega) = \omega^{\frac{N(\gamma)-1}{3}} = \omega^{n+m}$$
\end{proof}

\paragraph{Ex. 9.4}

{\it (continuation) Show that $\chi_\gamma(\omega) = 1,\omega$, or $\omega^2$ according to whether $\gamma$ is congruent to 8,2, or 5 modulo $3\lambda$. In particular, if $q$ is a rational prime, $q \equiv 2 \pmod 3$, then $\chi_q(\omega) = 1, \omega$, or $\omega^2$ according to whether $q \equiv 8,2$, or $5,\pmod 9$. [Hint : $\gamma = a + b \omega = -1 + 3(m+n\omega)$, and so $\gamma \equiv -1 + 3(m+n) \pmod{3\lambda}$.]
}

\begin{proof}
$\lambda=1-\omega$, so $\omega\equiv1\ \pmod \lambda$. Thus
\begin{align*}
m+n\omega &\equiv m+n\ \pmod \lambda\\
3(m+n \omega) &\equiv 3 (m+n) \pmod{3\lambda}\\
\gamma = -1 + 3(m+n\omega) &\equiv -1+3(m+n) \pmod{3 \lambda}
\end{align*}
Moreover $9 = 3 \lambda \bar{\lambda} \equiv 0 \pmod{3 \lambda}$, thus $\gamma$ is congruent modulo $3\lambda$ to an integer between  $0$ and $8$ of the form $3k-1$ : $\gamma \equiv 8,2$ or $5 \pmod{3\lambda}$.

By Ex. 9.3, $\chi_\gamma(\omega) = 1 \iff m+n \equiv 0 \ [3]$, and $m+n  \equiv 0 \ [3]$ implies $m+n = 3k, k \in \Z$, so $\gamma  \equiv -1+ 9k \equiv -1 \equiv 8 \ [3\lambda]$.

Reciprocally, if $\gamma \equiv 8 \equiv -1 \ [3\lambda]$, then $3\lambda \mid 3(m+n)$, so $\lambda \mid m+n$, and $N(\lambda) \mid N(m+n)$, $3 \mid (m+n)^2$, thus $3 \mid m+n$,  $m+n \equiv 0 \ [3]$, and so $\chi_\gamma(\omega) = 1$. As the two other cases are similar, we obtain

\begin{align*}
\chi_\gamma(\omega) = 1 &\iff m+n \equiv 0\ [3] \iff \gamma \equiv 8 \ [3\lambda]\\
\chi_\gamma(\omega) = \omega &\iff m+n \equiv 1\ [3] \iff \gamma \equiv 2 \ [3\lambda]\\
\chi_\gamma(\omega) = \omega^2 &\iff m+n \equiv 2\ [3] \iff \gamma \equiv 5 \ [3\lambda]
\end{align*}

If $\gamma = q$ is a rational prime,
$q \equiv 8 \ [9]$ implies $q \equiv 8\ [3\lambda]$, since $3 \lambda \mid 9 = 3 \lambda \bar{\lambda}$, thus $\chi_q(\omega) = 1$.

Reciprocally, if $\chi_q(\omega) = 1$, then $q \equiv 8\ [3\lambda]$, $q-8 = \mu (3 \lambda), \mu \in D$, therefore 

$(q-8)^2 = N(\mu) 3^3, 3^3 \mid (q-8)^2$, thus $3^2 \mid q-8$ and so $q \equiv 8 \ [9]$. The two other cases are similar.

\begin{align*}
\chi_q(\omega) = 1 & \iff q \equiv 8 \ [9]\\
\chi_q(\omega) = \omega & \iff q \equiv 2 \ [9]\\
\chi_q(\omega) = \omega^2 &\iff q \equiv 5 \ [9]
\end{align*}
\end{proof}

\paragraph{Ex. 9.5}
{\it In the text we stated Eisenstein's result $\chi_\gamma(\lambda) = \omega^{2m}$. Show that $\chi_{\gamma}(3) = \omega^{2n}$.
}

\begin{proof}
$(1-\omega)^2 = -3 \omega$, thus $\chi_\gamma((1-\omega)^2) = \chi_\gamma(-1)\chi_\gamma(3)\chi_\gamma(\omega)$.

$\chi_\gamma((1-\omega)^2)= \chi_\gamma(\lambda^2) = \omega^{4m} = \omega^m$

As $-1 = (-1)^3, \chi_\gamma(-1) = 1$. Finally  $\chi_\gamma(\omega) = \omega^{m+n}$ by Exercise 9.3. Thus

$$\omega^m = \chi_\gamma(3) \omega^{m+n},\qquad  \chi_\gamma(3) = \omega^{-n} = \omega^{2n}.$$

Conclusion : 
$$\chi_\gamma(3) = \omega^{2n}$$
\end{proof}

\paragraph{Ex. 9.6}

{\it Prove that
\begin{enumerate}
\item[(a)] $\chi_\gamma(\lambda) = 1$ for $\gamma \equiv 8, 8 + 3\omega, 8 + 6 \omega \ [9]$.
\item[(b)] $\chi_\gamma(\lambda) = \omega$ for $\gamma \equiv 5, 5 + 3\omega, 5 + 6 \omega \ [9]$.
\item[(c)]$\chi_\gamma(\lambda) = \omega^2$ for $\gamma \equiv 2, 2 + 3\omega, 2 + 6 \omega \ [9]$.
\end{enumerate}
}

\begin{proof}
$\gamma = -1 + 3(m+n\omega)$, et $\chi_\gamma(\lambda) = \omega^{2m}$.
\begin{align*}
\chi_\gamma(\lambda)=1 &\iff m\equiv0\ [3] \Rightarrow \gamma \equiv 8+3n\omega\ [9]\Rightarrow \gamma\equiv8,8+3\omega,8+6\omega\ [9]\\
\chi_\gamma(\lambda)=\omega&\iff m\equiv2\ [3] \Rightarrow \gamma \equiv5+3n\omega\ [9]\Rightarrow \gamma\equiv5,5+3\omega,5+6\omega\ [9]\\
\chi_\gamma(\lambda)=\omega^2& \iff m\equiv1\ [3] \Rightarrow \gamma \equiv2+3n\omega\ [9]\Rightarrow \gamma\equiv2,2+3\omega,2+6\omega\ [9]\\
\end{align*}

As $\chi_\gamma(\lambda) \in \{1,\omega,\omega^2\}$, these 9 cases are the only possibilities. Moreover these 9 cases are mutually exclusive, since $9$ doesn't divide  any difference. Thus the reciprocals are true.
\begin{align*}
\chi_\gamma(\lambda)=1 &\iff \gamma\equiv8,8+3\omega,8+6\omega\ [9]\\
\chi_\gamma(\lambda)=\omega&\iff \gamma\equiv5,5+3\omega,5+6\omega\ [9]\\
\chi_\gamma(\lambda)=\omega^2&\iff \gamma\equiv2,2+3\omega,2+6\omega\ [9]\\
\end{align*}
\end{proof}

\paragraph{Ex. 9.7}

{\it Find primary primes associate to $1-2\omega, -7-3\omega$, and $3-\omega$.
}

\begin{proof} 
: \\
\begin{enumerate}
\item[$\bullet$] $(1-2\omega)\omega = 2 + 3 \omega \equiv 2 \pmod 3$, so $2+3\omega$ is primary, and associate to $1-2\omega$.
$N(2+3\omega) = 7$ and 7 is a rational prime, thus $2 + 3\omega$ is a primary prime.
\item[$\bullet$] $-7 - 3 \omega \equiv 2 \pmod 3$.

$N(-7-3\omega) = 37$ and 37 is a rational prime, thus $-7 - 3\omega$ is a primary prime.
\item[$\bullet$] $(3-\omega)\omega^2 = -4 - 3\omega \equiv 2 \pmod 3$, so $-4-3\omega$ is primary, and  associate to $3-\omega$.

$N(-4 - 3 \omega) = 13$ and 13 is a rational prime, thus $-4 - 3 \omega$ is a primary prime.
\end{enumerate}
\end{proof}

\paragraph{Ex. 9.8}

{\it Factor the following numbers into primes in $D$ : $7,21,45,22$, and $143$.
}

\begin{proof}
$7=N(2+3\omega)$, thus $ 7 = (2+3\omega)(2+3\omega^2) = (2+3\omega)(-1-3\omega)$.

$21 = 3\times 7 = -\omega^2 \lambda^2 (2+3\omega)(-1-3\omega)$  since $3=-\omega^2(1-\omega)^2$.

$45 = 3^2\times 5 = \omega \lambda^4 5$

$22=2\times 11$ (2 and 11 are primes in $D$)

$143 = 11\times 13 = 11(-4-3\omega)(-4-3\omega^2) = 11  (-4-3\omega)(-1+3\omega)$
\end{proof}

\paragraph{Ex. 9.9}
{\it Show that $\overline{\alpha} \ne 0$, the residue class of $\alpha$, is a cube in the field $D/\pi D$ iff $\alpha^{(N\pi -1)/3} \equiv 1 \pmod \pi$. Conclude that there are $(N\pi - 1)/3$ cubes in $(D/\pi D)^*$.
}

\bigskip
Solution 1 : 
\begin{proof}
Let $\pi$ a prime in $D$, $N\pi \ne 3$, and $\alpha \in D,\pi \nmid \alpha$.

$\overline{\alpha}$ is a cube in $(D/\pi D)^*$

$\iff x^3 \equiv \alpha \pmod \pi$ has a solution

$\iff \chi_{\pi}(\alpha)= 1$ (by Prop. 9.3.3(a))

$\iff \alpha^{\frac{N\pi-1}{3}} \equiv 1 \pmod \pi$

$\iff \overline{\alpha}^{\frac{N\pi-1}{3}}  = \overline{1}$.

The cubes in $(D/\pi D)^*$ are then the roots of the polynomial $f(x) = x^{\frac{N\pi-1}{3}} - \overline{1}$ in $D/\pi D$. 

As $d = |D/\pi D| = N\pi$, $(N\pi - 1)/3 \mid q-1$, $f(x) \mid x^{q-1}-1 \mid x^q-x$. By Corollary 2 of Proposition 8.1.1, $f$ has $\deg(f) = \frac{N\pi-1}{3}$ roots.

Conclusion  : there exist exactly $\frac{N\pi-1}{3}$ cubes in $(D/\pi D)^*$.
\end{proof}

Solution 2 :
\begin{proof}
Let $\varphi : (D/\pi D)^* \to (D/\pi D)^*$ the group homomorphism defined by $\varphi(x) = x^3$.

Then $\mathrm{im}(\varphi)$ is the set of cubes in $(D/\pi D)^*$.

The equation $x^3 = \overline{1}$ has three distinct solutions $\overline{1}, \overline{\omega}, \overline{\omega}^2$  in $D/ \pi D$ if $N\pi \ne 3$ (see the demonstration of Proposition 9.3.1).

So $\ker(\varphi)= \{\overline{1}, \overline{\omega}, \overline{\omega}^2\}$ and $| \ker(\varphi)| = 3$. Thus $| \mathrm{im} \varphi | = | (D/\pi D)^* \ / |\ker(\varphi) | = (N\pi - 1)/3$. There exist exactly $\frac{N\pi-1}{3}$ cubes in $(D/\pi D)^*$.
\end{proof}

Note : if $N\pi = 3$, that is to say if $\pi$ is associate to $1-\omega$, $D/\pi D = \{\overline{0},\overline{1}, \overline{2}\}$. As $\overline{1}^3 = \overline{1}, \overline{2}^3 = \overline{2}$, all the elements of $(D/\pi D)^*$ are cubes.

\paragraph{Ex. 9.10}

{\it What is the factorisation of $x^{24}-1$ in $D/5D$.
}

\begin{proof}
$\vert (D/5D)^*\vert = N(5) - 1 = 24$, thus $x^{24} - 1 = \prod\limits_{\alpha \in (D/ 5 D)^*} (x - \alpha)$.

( $\alpha = a + b\, \overline{\omega},\  0 \leq a <5, 0 \leq b <5$).
\end{proof}

\paragraph{Ex. 9.11}

{\it How many cubes are there in $D/5D$ ?
}

\begin{proof}
By Exercise 9.9, there exist $(N(5) - 1)/3 = 8$ cubes in $D/5D$.
\end{proof}

\paragraph{Ex. 9.12}

{\it Show that $\omega \lambda$ has order 8 in $D/5D$ and that $\omega^2 \lambda$ has order 24. [Hint : Show first that $(\omega \lambda)^2$ has order 4.]
}

\begin{proof}
$\alpha = (\omega \lambda)^2 = \omega^2 (1-\omega)^2 = \omega^2(1 + \omega^2-2 \omega) = \-3\omega^3= -3$.

$\alpha^2 =9 \equiv -1\pmod 5, \alpha^4 \equiv 1 \pmod 5$, thus $\alpha= (\omega \lambda)^2$ is of order 4 in $D/5D$, and $\omega \lambda$ of order 8.

Let $\beta = \omega^2 \lambda$. $\vert (D/5 D)^* \vert = 24$, thus  $\overline{\beta}^{24} = 1$.

To verify that $\overline{\beta}$ has order 24, it is sufficient to verify $\overline{\beta}^{8}\neq 1,\overline{\beta}^{12}\neq 1$ :

$\beta^8 = \omega ^{16} \lambda^8 = \omega \lambda^8 = (\omega \lambda)^8 \omega^2 \equiv  \omega^2 \not \equiv 1 \pmod 5$.

$\beta^{12} = (\omega^2 \lambda)^{12} = \lambda^{12} = (\omega \lambda)^{12} \equiv  (\omega \lambda)^{4}  \equiv -1 \pmod 5$ (since $(\omega \lambda)$ has order 8 in $D/5D$).


Conclusion : $\omega \lambda$ has order 8, $\omega \lambda^2$ has order 24.
\end{proof}

\paragraph{Ex. 9.13}

{\it Show that $\pi$ is a cube in $D/5D$ iff $\pi \equiv 1,2,3,4,1+2\omega, 2 + 4\omega, 3 + \omega$, or $4+3\omega \pmod 5$.
}

\begin{proof}
Let $\pi \in D, \overline{\pi} \neq 0$. Then $\overline{\pi} $ is a cube in $D/5D$ iff $\overline{\pi}^{(q^2-1)/3} = 1$, with $q = 5$, namely $\overline{\pi}^8 = 1$ (Prop. 7.1.2, where $3 \mid q^2-1 = 24 = |(D/5D)^*| $).

By Exercise 9.12, the class of $\gamma = \omega \lambda$ has order 8, thus the 8 elements $\overline{\gamma}^k, 0 \leq k \leq 7$ are distinct roots of the polynomial $x^8-1$, which has at most 8 roots. Therefore  the subgroup of cubes in $(D/5D)^*$ is
$$\{1 ,\overline{\gamma}, \overline{\gamma}^2,\ldots, \overline{\gamma}^7\}.$$
$\gamma=\omega(1-\omega) = \omega+1+\omega = 1 + 2\omega$, so
\begin{align*}
\gamma^0 &= 1\\
\gamma^1 & = 1 + 2\omega\\
\gamma^2 &\equiv -3 \equiv 2\ [5]\qquad (\mathrm{Ex}.\ 9.12)\\
\gamma^3 &= -3 - 6\omega \equiv 2 + 4 \omega \ [5]\\
\gamma^4&\equiv -1 \equiv 4\ [5]\\
\gamma^5 &\equiv -1 - 2 \omega \equiv 4 + 3\omega\ [5]\\
\gamma^6 &\equiv 3 \ [5]\\
\gamma^7 &\equiv 3+6\omega \equiv 3 + \omega \ [5]
\end{align*}
Conclusion : If $\pi \not \equiv 0 \pmod 5$, $\pi \equiv \alpha^3 \pmod 5,\alpha \in D$ iff 

$$\pi \equiv 1,2,3,4,1+2\omega,2+4\omega,3+\omega,4+3\omega\ [5].$$ 
\end{proof}

\paragraph{Ex. 9.14}

{\it For which primes $\pi \in D$ is $x^3 \equiv 5 \pmod \pi$ solvable ?
}

\begin{proof}
If $\pi$ is a primary prime, and not an associate of 5, the Law of Cubic Reciprocity gives
\begin{align*}
5\equiv x^3 \ [\pi], x \in D &\iff \chi_\pi(5)=1\\
&\iff \chi_5(\pi)=1\\
&\iff  \pi \ \mathrm{ is\ a\ cube\ in\  } D/5D\\
&\iff \pi \equiv1,2,3,4,1+\omega,2+4\omega,3+\omega,4+3\omega \ [5]
\end{align*}
(see Ex. 9.13)

\bigskip

Conclusion : the equation $5\equiv x^3 \ [\pi], x \in D $ is solvable iff the primary prime associate to $\pi$ is congruent modulo 5 to $1,2,3,4,1+2\omega,2+4\omega,3+\omega,4+3\omega$.

Examples : 

$\bullet$ $q=23$ is a primary prime congruent to 3 modulo 5, thus the equation $x^3 \equiv 5\pmod{23}$ has a solution $x  \in D$ ($x = 19)$.

$\bullet$ $-4 - 3 \omega$ is the primary prime associate to the prime $3- \omega$, and $-4 - 3 \omega \equiv 1 + 2 \omega \pmod 5$, thus the equation $x^3 \equiv 5 \pmod {3-\omega}$ has a solution $a + b \omega \in \Z[\omega]$.

Indeed , $7^3 \equiv 5^3 \equiv 11^3 \equiv 5 \pmod {13}$, and $3 - \omega \mid 13$, so $7^3 \equiv 5^3 \equiv 11^3 \equiv 5 \pmod{ 3 - \omega}$.
\end{proof}

\paragraph{Ex. 9.15}

{\it Suppose that $p \equiv 1 \pmod 3$ and that $p = \pi \overline{\pi}$, where $\pi$ is a primary prime in $D$. Show that $x^3 \equiv a \pmod p$ is solvable in $\Z$ iff $\chi_{\pi}(a) = 1$ . We assume that $a \in \Z$.
}

\begin{proof}
As $\pi \mid p$, if $ a \equiv x^3 \pmod p, x \in \Z$, then $ a \equiv x^3 \pmod \pi$, thus $\chi_{\pi}(a) = 1$.

Reciprocally, suppose that $\chi_{\pi}(a) = 1$. Then the equation $a \equiv y^3 \pmod \pi$ has a solution $y = u + v \omega, \ u,v \in \Z$. Moreover, $\overline{y}$ has a representant $x \in \Z$ modulo $\pi$ :
$$y\equiv  x \pmod \pi, x \in \Z.$$

So $a \equiv x^3$ has a solution $x \in \Z$.

Thus $\pi \mid a-x^3, N(\pi) = p \mid (a-x^3)^2$, therefore $p \mid a-x^3$ and so $a \equiv x^3 \pmod p$.

Conclusion ; if $p\equiv 1 \pmod 3$ , $p = \pi \overline{\pi}$, where $\pi$ is is a primary prime  and $a \in \Z$,
$$\exists x \in \Z, \ a \equiv x^3 \pmod p \iff \chi_{\pi}(a) = 1.$$
In other words, $x^3 \equiv a \pmod \pi$ is solvable in $D$ iff it is solvable in $\Z$.
\end{proof}

\paragraph{Ex. 9.16}

{\it Is $x^3 \equiv 2 - 3\omega \pmod {11}$ solvable ? Since $D/11D$ has 121 elements this is hard to resolve by straightforward checking. Fill in the details of the following proof that it is not solvable. $\chi_{\pi}(2 - 3 \omega) = \chi_{2-3\omega}(11)$ and so we shall have a solution iff $x^3 \equiv 11 \pmod {2-3\omega}$ is solvable. This congruence is solvable iff $x^3 = 11\pmod 7$ is solvable in $\Z$. However, $x^3 \equiv a \pmod 7$ is solvable in $\Z$ iff $a \equiv 1$ or $6 \pmod 7$.
}

\bigskip
Warning : false sentence, since 
$$N(2 - 3 \omega) = (2-3\omega)(2 - 3 \omega^2) = 4 + 9 -6(\omega+\omega^2) = 4 + 9 + 6 = 1\ (\mathrm{and}\ \mathrm{not}\ 7!).$$
\begin{proof}
As $19$ is a rational prime, and $\pi= 2 - 3\omega$ and $11$  are primary primes, by Exercise 9.15,
\begin{align*}
\exists x \in D,\ 2-3\omega \equiv x^3\ [11]&\iff \chi_{11}(2-3\omega) = 1\\
&\iff \chi_{2-3\omega}(11)=1\\
&\iff \exists x \in \mathbb{Z},\  x^3 \equiv 11\  [19]
\end{align*}
Moreover 
$$\exists x \in \mathbb{Z},\  x^3 \equiv 11\  [19] \iff 11^6 \equiv 1 \pmod {19},$$
which is true : $11^6 = 121^3 = (19 \times 6 + 7)^3 \equiv 49 \times 7 \equiv 11 \times 7 \equiv 77 \equiv 1 \ [19]$.

Conclusion : there exists $x \in D$ such that $2-3\omega  \equiv x^3 \pmod {11}$.

We a little programming, we find a solution $x = 1 + 8\omega$ (and its associates $\omega^2 x =  7 - \omega, \omega x = -8- 7\omega \equiv 3 + 4 \omega \pmod{11} $) :
$$x^3 = (1+8\omega)^3 = 321 - 168 \omega \equiv 2 - 3 \omega \pmod {11}.$$
\end{proof}


\paragraph{Ex. 9.17}

{\it An element $\gamma \in D$ is called primary if $\gamma \equiv2 \pmod 3$. If $\gamma$ and $\rho$ are primary, show that $-\gamma \rho$ is primary. If $\gamma$ is primary, show that $\gamma = \pm \gamma_1\gamma_2\ldots \gamma_t$, where the $\gamma_i$ are (not necessarily distinct) primary primes.
}

\begin{proof}
If $\gamma \equiv 2, \rho \equiv 2 \pmod 3$, then $-\gamma \rho \equiv -2 \times 2 \equiv 2 \pmod 3$, so $-\gamma \rho$ is primary.

\bigskip

By Ex. 9.2, $\gamma$ can be  written 
$$\gamma = (-1)^a \omega^b \lambda^c \pi_1^{a_1}\cdots \pi_t^{a_t},$$
where $\pi_i \equiv 2 \pmod 3, a \in \{0,1\}, b \in \{0,1,2\}$.

As $\pi_i \equiv -1 \pmod 3$, and $\gamma \equiv -1 \pmod 3$, we obtain $\omega^b \lambda^c \equiv \pm 1 \pmod 3$. We prove that $b = c = 0$.

$\lambda^2 = (1-\omega)^2 = -3\omega \equiv 0 \pmod 3$. If $c \geq 2$, we woulfd obtain $\gamma \equiv 0 \pmod 3$, in contradiction with the hypothesis, thus $c = 0$ or $c = 1$.

If $c=1$, $\omega^b\lambda^c \in \{1-\omega, \omega(1-\omega) = 1+2\omega, \omega^2(1-\omega) = - 2 - \omega$. Since $1-\omega \not \equiv \pm 1, 1+2\omega \not \equiv \pm 1, - 2 - \omega \not \equiv \pm 1 \pmod 3$, this is impossible, so $c=0$.
$\omega^b \in \{1,\omega, -1-\omega$. Since $\omega \not \equiv \pm1 \pmod 3$, and $-1 - \omega \not \equiv \pm 1 \pmod 3$, then $\omega^b = 1, 0 \leq b \leq 2$, thus $b=0$.

Finally, $\gamma = (-1)^a \pi_1^{a_1}\cdots \pi_t^{a_t}$.

Conclusion : every primary $\gamma \in D$ is under the form
$$\gamma = \pm \gamma_1\gamma_2\cdots \gamma_t,$$
where the $\gamma_i$ are primary primes.
\end{proof}

\paragraph{Ex. 9.18}

{\it (continuation) If $\gamma = \pm \gamma_1 \gamma_2\cdots \gamma_t$ is a primary decomposition of the primary element $\gamma$, define $\chi_{\gamma}(\alpha) = \chi_{\gamma_1}(\alpha)\chi_{\gamma_2}(\alpha)\cdots\chi_{\gamma_t}(\alpha)$. Prove that $\chi_{\gamma}(\alpha) = \chi_{\gamma}(\beta)$ if $\alpha \equiv \beta \pmod \gamma$ and $\chi_{\gamma}(\alpha \beta) = \chi_\gamma(\alpha) \chi_\gamma(\beta)$. If $\rho$ is primary, show that $\chi_\rho(\alpha) \chi_\gamma(\alpha) = \chi_{-\rho \gamma}(\alpha)$.
}

\begin{proof}
If $\alpha \equiv \beta\ [\gamma]$, then $\alpha \equiv \beta\pmod {\gamma_i}, 1 \leq i \leq t$, so $\chi_{\gamma_i}(\alpha) = \chi_{\gamma_i}(\beta)$, thus $\chi_\gamma(\alpha) = \chi_\gamma(\beta)$.

By Proposition 9.3.3,
\begin{align*}
\chi_\gamma(\alpha \beta) &= \chi_{\gamma_1}(\alpha \beta)\chi_{\gamma_2}(\alpha \beta)\cdots\chi_{\gamma_t}(\alpha \beta)\\
&= \chi_{\gamma_1}(\alpha)\chi_{\gamma_2}(\alpha)\cdots\chi_{\gamma_t}(\alpha)\chi_{\gamma_1}(\beta)\chi_{\gamma_2}(\beta)\cdots\chi_{\gamma_t}(\beta)\\
&= \chi_\gamma(\alpha) \chi_{\gamma}(\beta)
\end{align*}
Finally if $\rho = \pm \rho_1\rho_2\cdots\rho_l$ is primary, then $-\rho \gamma = \pm \rho_1\rho_2\cdots\rho_l\gamma_1\gamma_2\cdots\gamma_t$ is primary by Ex. 9.17, therefore

$$\chi_{-\rho \gamma}(\alpha) = (\chi_{\rho_1} \chi_{\rho_2}\cdots \chi_{\rho_l}\chi_{\gamma_1} \chi_{\gamma_2}\cdots \chi_{\gamma_t} )(\alpha)= \chi_\rho(\alpha)\chi_\gamma(\alpha).$$
\end{proof}

\paragraph{Ex. 9.19}

{\it Suppose that $\gamma = A + B \omega$ is primary and that $A = 3M-1$ and $B = 3N$. Prove that $\chi_\gamma(\omega) = \omega^{M+N}$ and that $\chi_\gamma(\lambda) = \omega^{2M}$.
}

\begin{proof}
We verify first that if $\gamma = -\gamma_1 \gamma_2$, with
$$
\begin{array}{lll}
  \gamma = A+B\omega, & A=3M-1,  & B=3N,  \\
   \gamma_1 = A_1+B_1\omega,& A_1= 3M_1-1,  &  B_1=3N_1, \\
  \gamma_2 = A_2+B_2\omega,&  A_2= 3M_2-1, &  B_2=3N_2, 
\end{array}
$$
then $M \equiv M_1+M_2\pmod 3,N \equiv N_1+N_2 \pmod 3$.
$$-\gamma_1 \gamma_2 = -A_1A_2 + B_1 B_2  + (- A_1B_2-A_2B_1+B_1B_2) \omega=A+B\omega,$$
therefore
$$3M-1 = A =-A_1A_2 + B_1 B_2 \equiv 3(M_1+M_2) - 1\pmod 9,$$
thus $M\equiv M_1+M_2\pmod 3.$
$$3N = B = -A_1B_2-A_2B_1+B_1B_2 \equiv 3 (N_1+N_2)\pmod 9,$$ 
thus $N \equiv N_1+N_2 \pmod 3.$

By induction, if $\gamma = \pm \gamma_1 \gamma_2\cdots \gamma_t = (-1)^{t-1} \gamma_1 \gamma_2\cdots \gamma_t$, where $\gamma_i = A_i+B_i\omega,A_i= 3M_i-1,B_i=3N_i$, then
$$M \equiv M_1+\cdots+M_t\pmod 3, N \equiv N_1+\cdots + N_t \pmod 3.$$

By Exercise 9.3,
\begin{align*}
\chi_\gamma(\omega) &= \chi_{\gamma_1}(\omega)\cdots\chi_{\gamma_t}(\omega)\\
&=\omega^{M_1+N_1}\cdots \omega^{M_t+N_t}\\
&=\omega^{(M_1+\cdots+M_t)+(N_1+\cdots+N_t)}\\
&=\omega^{M+N}
\end{align*}

and by Eisenstein's result,
\begin{align*}
\chi_\gamma(\lambda) &= \chi_{\gamma_1}(\lambda)\cdots\chi_{\gamma_t}(\lambda)\\
&=\omega^{2M_1}\cdots \omega^{2M_t}\\
&=\omega^{2(M_1+\cdots+M_t)}\\
&=\omega^{2M}
\end{align*}

Conclusion : if $\gamma = 3M-1+3N\omega$, then

$$\chi_\gamma(\omega) = \omega^{M+N}, \chi_\gamma(\lambda)=\omega^{2M}.$$
\end{proof}

\paragraph{Ex. 9.20}

{\it If $\gamma$ and $\rho$ are primary, show that $\chi_\gamma(\rho) = \chi_\rho(\gamma)$.
}

\begin{proof}

\end{proof}
$\rho, \gamma$ are written
\begin{align*}
\rho&=\pm\rho_1\rho_2\cdots\rho_l,\\
\gamma &= \pm \gamma_1 \gamma_2\cdots \gamma_m, 
\end{align*}
 where $\rho_i,\gamma_i$ are primary primes.
By the law of Cubic Reciprocity, we obtain
\begin{align*}
\chi_\gamma(\rho) &= \prod_{j=1}^m \chi_{\gamma_j}(\rho)\\
& = \prod_{j=1}^m\prod_{i=1}^l \chi_{\gamma_j}(\rho_{i})\\
& = \prod_{i=1}^l\prod_{j=1}^m \chi_{\gamma_j}(\rho_{i})\\
& = \prod_{i=1}^l\prod_{j=1}^m \chi_{\rho_i}(\gamma_j)\\
&=\prod_{i=1}^l \chi_{\rho_i}(\gamma)\\
&=\chi_{\rho}( \gamma)
\end{align*}

\paragraph{Ex. 9.21}

{\it If $\gamma$ is primary, show that there are infinitely many primary primes $\pi$ such that $x^3 \equiv \gamma \pmod \pi$ is not solvable. Show also that there are infinitely many primary primes $\pi$ such that $x^3 \equiv \omega \pmod \pi$ is not solvable and the same for $x^3 \equiv \lambda \pmod \pi$. (Hint: Imitate the proof of Theorem 3 of Chapter 5.)
}

\begin{proof}
\begin{enumerate}
\item[a)] As some primary elements of $D$ may be cubes, by example $53 + 36 \omega = (-1 + 3 \omega)^3$, we must of course suppose that $\gamma$ is not the cube of some element of $D$ (in the contrary case $x^3 \equiv \gamma \pmod \pi $ is solvable for all prime $\pi$).

Note first that for all prime $\pi$ in $D$, there exists $\sigma \in D$ such that $\chi_\pi(\sigma) = \omega$. Indeed, there exist $(N\pi - 1)/3$ cubes in $(D/\pi D)^*$, which has $N\pi - 1$ elements, so there exists an element $\overline{\tau} \in (D/\pi D)^*$ which is not a cube, therefore there exists $\tau \in D$ such that $\chi_\pi(\tau) \neq 1$. If $\chi_\pi(\tau) = \omega$, we put $\sigma = \tau$ and if $\chi_\pi(\tau) = \omega^2$, we put $\sigma = \tau^2$. In the two cases, $\chi_\pi(\sigma) = \omega$.

\bigskip

Let $\gamma \in D$, where $\gamma$ is primary. Then $\gamma = \pm \gamma_2^{n_1}\gamma_1^{n_2}\cdots\gamma_p^{n_p}$, where the $\gamma_i$ are distinct primary primes. 
Write $n_i = 3q_i + r_i, \ r_i \in \{0,1,2\}$. Then grouping in $\gamma'$ the $r_i \ne 0$, we can write $\gamma = \delta^3 \gamma', \gamma' = \gamma_1^{r_1} \gamma_2^{r_2}\cdots \gamma_l^{r_l}, r_i \in \{1,2\}, \delta \in D$ ($-1$ is a cube). Since by hypothesis $\gamma$ is not a cube, $l\geq 1$. Moreover the equation $x^3 \equiv \gamma \pmod \pi$ is solvable iff $x^3 \equiv \gamma' \pmod \pi$ is solvable. We may then suppose $$\gamma = \gamma_1^{r_1} \gamma_2^{r_2}\cdots \gamma_l^{r_l}, 1\leq r_i \leq 2,$$ without cubic factors.

Note that the $\gamma_i$ are not associate to $\lambda = 1-\omega$ (see Ex. 9.17).

Let $A = \{\lambda_1,\lambda_2,\ldots, \lambda_k\}$  a set (possibly empty) of distinct primary primes $\lambda_i$ (therefore  they are not associate), and not associate neither to $\gamma_i, 1 \leq i \leq l$, nor $\lambda = 1 - \omega$.

We will show that we can find a primary prime $\lambda_{k+1}$ distinct of the $\lambda_i$ with the same properties and such that the equation $x^3 \equiv \lambda \pmod {\lambda_{k+1}}$ is not solvable. This proves the existence of infinitely many primes $\pi$ such that the equation $x^3 \equiv \lambda \pmod \pi$ is not solvable.

With the initial note, let $\sigma \in D$ such that $\chi_{\gamma_l}(\sigma) = \omega$. As $D$ is a principal ideal domain, the Chinese Remainder Theorem is valid. Since $3 = \lambda \overline{\lambda}$ is relatively prime to $\gamma_i, \lambda_i$, there exists $\beta \in D$ such that
\begin{align*}
\beta &\equiv 2 \ [3]\\
\beta &\equiv 1 \ [\lambda_i] \hspace{1cm} (1\leq i\leq k)\\
\beta &\equiv 1 \ [\gamma_i] \hspace{1cm} (1 \leq i \leq l-1)\\
\beta &\equiv  \sigma \ [\gamma_l]\\
\end{align*}
The first equation show that $\beta$ is primary, so $\beta= (-1)^{m-1} \beta_1\ldots \beta_m$, where the $\beta_i$ are primary primes.

By Exercise 9.20,
$$\chi_\beta(\gamma) = \chi_\beta(\gamma_1)^{r_1}\cdots\chi_\beta(\gamma_l)^{r_l} = \chi_{\gamma_1}(\beta)^{r_1}\cdots\chi_{\gamma_l}(\beta)^{r_l}.$$
As $\chi_\beta(\gamma) = \chi_{\gamma_i}(1) = 1 \ (1\leq i \leq l-1)$, and $\chi_{\gamma_l}(\beta) = \chi_{\gamma_l}(\sigma) =\omega$, we obtain $\chi_\beta(\gamma) = \omega^{r_l} \neq 1$, since $r_l =$ or $r_l = 2$.

By Exercise 9.18, $\chi_\rho(\alpha) \chi_\gamma(\alpha) = \chi_{-\rho \gamma}(\alpha)$, with primary $\rho, \gamma$, so by induction, as $\beta= (-1)^{m-1} \beta_1\cdots\beta_m$,
$$\chi_\beta(\gamma) = \chi_{\beta_1}(\gamma)\cdots\chi_{\beta_m}(\gamma) \neq 1.$$
Thus there exists a subscript $j$ such that $\chi_{\beta_j}(\gamma) \neq 1$.

We can then take $\lambda_{k+1} = \beta_j$. Indeed, as $\beta \equiv 1 \ [\lambda_i]$ and $\beta \not \equiv 0\ [\gamma_i]$, $\beta_j$ is distinct of the $\lambda_i$ and $\gamma_i$, and $\beta_j$ is not associate to $\lambda$ since $\beta \equiv 2 \pmod 3$.

As $\chi_{\lambda_{k+1}}(\gamma) \neq 1$, the equation $x^3 \equiv \gamma \ [\lambda _{k+1}]$ is not solvable, so $\lambda_{k+1}$ is convenient. 

Conclusion : if $\gamma \in D$ is primary and is not a cube in $D$, there exist infinitely many primes $\pi \in D$ such that the equation $x^3 \equiv \lambda \ [\pi]$ is not solvable.
\item[b)] We show that $x^3 \equiv \omega \ [\pi]$ has no solution for infinitely many primes $\pi$.

To begin the induction, we display such a prime $\pi$, namely $\pi = 2+ 3 \omega$. Indeed, $N(\pi) = 4 + 9 - 6 = 7$, 7 is a rational prime, so $\pi$ is a primary prime in $D$, of the form $\pi = 3m-1 + 3n\omega$, with $n = m =1$, so $\chi_\pi(\omega) = \omega^{m+n} = \omega^2 \neq 1$ : the equation $x^3 \equiv \omega \ [\pi]$ is not solvable. Moreover $\pi$ is not associate to $\lambda = 1 - \omega$.

Suppose now the existence of a set $A = \{\lambda_1,\lambda_2,\ldots,\lambda_l\}, l \geq 1$, of distinct primary primes $\lambda_i$, not associate to $\lambda$ and such the equation $x^3 \equiv \omega\ [\lambda_i]$ is not solvable. We will show that we can add a prime $\lambda_{l+1}$  to the set $A$ with the same properties.


Let $$\beta = 3(-1)^{l-1} \lambda_1\cdots\lambda_l - 1.$$

$(-1)^{l-1} \lambda_1\cdots\lambda_l$ is primary, so $(-1)^{l-1} \lambda_1\cdots\lambda_l = 3m-1 + 3n\omega,\ m,n \in \Z$.

$\beta = 3(3m-1 + 3n\omega) - 1 = 3(3m-1) - 1 + 9n\omega = 3M-1 + 3 N \omega$, where $M = 3m-1, N = 3n$. By Exercise 9.19,
$$\chi_\beta(\omega) = \omega^{M+N} = \omega^{3m-1+3n} = \omega^2 \neq 1.$$
As $\beta = \pm \beta_1\cdots\beta_m$, where the $\beta_i$ are primary primes, $\chi_\beta(\omega) = \chi_{\beta_1}(\omega) \cdots \chi_{\beta_m}(\omega) \neq 1$, so there exists a subscript $i$ such that $\chi_{\beta_i}(\omega) \ne 1$. 

Since $\beta = 3(-1)^{l-1} \lambda_1\cdots\lambda_l - 1$, $\beta_i$ is associate neither to $\lambda_i$ nor to $\lambda$. Moreover $\chi_{\beta_i}(\omega)\ne 1$, thus the equation $x^3 \equiv \omega\ [\beta_i]$ is not solvable : $\lambda_{l+1} = \beta_i$ is convenient.

Conclusion : the equation $x^3 \equiv \omega \ [\pi]$ is not solvable for infinitely many primes $\pi$.

\item[c)] We show that $x^3 \equiv \lambda \ [\pi]$ has no solution for infinitely many primes $\pi$.

To begin the induction, we display such a prime $\pi$, namely $\pi = -4 + 3 \omega$. Indeed, $N(\pi) = 16 + 9 + 12 = 37$, 37 is a rational prime, so $\pi$ is a primary prime in $D$, of the form $\pi = 3m-1 + 3n\omega$, with $m=-1,n=1$, so $\chi_\pi(\lambda) = \omega^{2m} = \omega \neq 1$ : the equation $x^3 \equiv \lambda \ [\pi]$ is not solvable.

Suppose now the existence of a set $A = \{\lambda_1,\lambda_2,\ldots,\lambda_l\}, l \geq 1$, of distinct primary primes $\lambda_i$, not associate to $\lambda$ and such the equation $x^3 \equiv \lambda\ [\lambda_i]$ is not solvable. We will show that we can add a prime $\lambda_{l+1}$  to the set $A$ with the same properties.

Let $$\beta = 3(-1)^{l-1} \lambda_1\cdots\lambda_l - 1.$$
$(-1)^{l-1} \lambda_1\cdots\lambda_l$ is primary, so $(-1)^{l-1} \lambda_1\cdots\lambda_l = 3m-1 + 3n\omega,\ m,n \in \Z$.

$\beta = 3(3m-1 + 3n\omega) - 1 = 3(3m-1) - 1 + 9n\omega = 3M-1 + 3 N \omega$, where $M = 3m-1, N = 3n$. By Exercise 9.19,
$$\chi_\beta(\lambda) = \omega^{2M} = \omega^{2(3m-1)} = \omega \neq 1.$$
As $\beta = \pm \beta_1\cdots\beta_m$, where the $\beta_i$ are primary primes, $\chi_\beta(\omega) = \chi_{\beta_1}(\omega) \cdots \chi_{\beta_m}(\omega) \neq 1$, so there exists a subscript $i$ such that $\chi_{\beta_i}(\lambda) \ne 1$. 

Since $\beta = 3(-1)^{l-1} \lambda_1\cdots\lambda_l - 1$, $\beta_i$ is associate neither to $\lambda_i$ nor to $\lambda$. Moreover $\chi_{\beta_i}(\lambda)\ne 1$, thus the equation $x^3 \equiv \lambda\ [\beta_i]$ is not solvable : $\lambda_{l+1} = \beta_i$ is convenient.

Conclusion : the equation $x^3 \equiv \lambda \ [\pi]$ is not solvable for infinitely many primes $\pi$.

\end{enumerate}
\end{proof}

\paragraph{Ex. 9.22}

{\it (continuation) Show in general that if $\gamma \in D$ and $x^3 \equiv \gamma \pmod \pi$ is solvable for all but finitely finitely many primary primes $\pi$, then $\gamma$ is a cube in $D$.
}

\begin{proof}
Let $\gamma \in D$ and suppose that $\gamma$ is not a cube in $D$. We will show that the equation $x^3 \equiv \gamma \ [\pi]$ is not solvable for infinitely primes $\pi \in D$.

By Exercise 9.2, we can write
$$\gamma = (-1)^u\omega^v \lambda^w \gamma_1^{n_1}\cdots\gamma_p^{n_p},$$
where the $\gamma_i$ are distinct primary primes. Let $v = 3q + b, w = 3q' +c, n_i = 3q_i +r_i$, with the remainders $b,c,r_i$ in $\{0,1,2\}$. Grouping the factors with null remainders, we obtain $\gamma = \delta^3 \gamma', \gamma' = \omega^b \lambda^c\gamma_1^{r_1}\cdots \gamma_l^{r_l}$, with  $b,c,r_i$ in $\{1,2\}, \delta \in D, l \geq 0$ ($-1$ is a cube).

Moreover the equation $x^3 \equiv \gamma \ [\pi]$ is solvable iff the equation $x^3 \equiv \gamma'\ [\pi]$ is solvable. So we may suppose that 
$$\gamma =  \omega^b \lambda^c\gamma_1^{r_1}\cdots \gamma_l^{r_l},\qquad b\in \{1,2\}, c\in \{1,2\}, r_i \in \{1,2\},$$
without cubic factors.

\begin{enumerate}
\item[$\bullet$] Case 1 : $l\geq 1$.

Let $A = \{\lambda_1,\ldots,\lambda_k\}$ a possibly empty set of distinct primary primes $\lambda_i$, distinct of the $\gamma_i$ and such that the equation $x^3 \equiv \gamma \ [\lambda_i]$ is not solvable. We will show that we can add a prime $\lambda_{k+1}$ with the same properties.

Suppose that $l\geq 1$. We have proved in Ex. 9.21 that there exists $\sigma \in D$ such that $\chi_{\gamma_l}(\sigma) = \omega$. Let $\beta \in D$ such that
\begin{align*}
\beta &\equiv -1\ [9]\\
\beta & \equiv 1 \ [\lambda_i], 1 \leq i \leq k\\
\beta & \equiv 1 \ [\gamma_i], 1 \leq i \leq l-1\\
\beta & \equiv \sigma\  [\gamma_l]
\end{align*}
$\beta \equiv -1 \ [9]$, thus $\beta \equiv -1 \ [3]$ : $\beta$ is primary, of the form $\beta = 3M-1+3N\omega$.

$\beta = 3M-1+3N\omega \equiv -1 \ [9]$, so $3M+3N\omega \equiv 0 \ [9]$, $M+N\omega \equiv 0 \ [3]$, thus $3\mid M,3\mid N$.

By Exercise 9.18,
\begin{align*}
\chi_\beta(\omega) &= \omega^{M+N} = 1\\
\chi_\beta(\lambda) &= \omega^{2M} = 1
\end{align*}
As $\beta$ and $\gamma_i$ are primary, $\chi_\beta(\gamma_i) = \chi_{\gamma_i}(\beta) = \chi_{\gamma_i}(1) = 1\ (1\leq i \leq l-1)$. 

$\chi_\beta(\gamma) =\chi_\beta(\omega)^b \chi_\beta(\lambda)^c\chi_\beta(\gamma_1)^{r_1}\cdots\chi_\beta(\gamma_l)^{r_l} = \chi_{\beta}(\gamma_l)^{r_l} = \chi_{\gamma_l}(\beta)^{r_l} = \chi_{\gamma_l}(\sigma)^{r_l} = \omega^{r_l}  \neq 1$, since $r_l \in \{1,2\}$. 

$\beta = \pm \beta_1\cdots \beta_m$, with $\beta_i$ primary primes, therefore
$$\chi_\beta(\gamma) = (\chi_{\beta_1}\cdots \chi_{\beta_m})(\gamma) \neq 1. $$ Thus there exists a subscript $i$ such that $\chi_{\beta_i}(\gamma) \ne 1$, so $x^3 \equiv \gamma\ [\beta_i]$ is not solvable. Moreover $\beta \equiv 1 \ [\gamma_i]$, so $\beta_i$ is not associate to any $\gamma_j$. Similarly, $\beta_i$ is not associate to any $\gamma_j$. $\lambda_{k+1} = \beta_i$ is convenient.

So there exist infinitely many $\pi$ such that $x^3 \equiv \gamma \ [\pi]$ is not solvable.


\item[$\bullet$] Case 2 : $l = 0$, so $\gamma = \omega^b\lambda^c,\ 1\leq b \leq 2, 1 \leq c \leq 2$.

$\pi_0 = 2 - 3 \omega$ is a primary prime ($N(\pi_0) = 19$).

Let $A = \{\lambda_1,\ldots,\lambda_k\}$ a possibly empty set of distinct primary primes $\lambda_i \neq \pi_0$ such that the equation $x^3 \equiv \gamma \ [\lambda_i]$ is not solvable. We will show that we can add a prime $\lambda_{k+1}$ with the same properties.

Let $\beta = 9 (-1)^{k-1} \lambda_1\cdots \lambda_k + 2-3\omega$.

$\beta\equiv 2 \ [3]$ : $\beta$ is primary.

Moreover $(-1)^{k-1} \lambda_1\cdots \lambda_k $ is primary, of the form 

$$(-1)^{k-1} \lambda_1\cdots \lambda_k  = 3m-1+3n\omega, m \in \mathbb{Z},n\in \mathbb{Z}.$$

\begin{align*}
\beta&=9(3m-1+3n\omega)+2-3\omega\\
&=27m-7+(27n-3)\omega\\
&=3(9m-2)-1+3(9n-1)\omega\\
&=3M-1+3N\omega
\end{align*}

where $M=9m-2,N=9n-1$

\begin{align*}
\chi_\beta(\omega) &= \omega^{M+N} = \omega^{9m-2+9n-1}=1\\
\chi_\beta(\lambda) &=\omega^{2M} = \omega^{2(9m-2)}=\omega^2\neq 1
\end{align*}

$\beta = \pm\beta_1\cdots\beta_m$, where the $\beta_i$ are primary primes.

$\chi_\beta(\gamma) = \chi_\beta(\omega)^b \chi_\beta(\lambda)^c = \omega^{2c} \neq 1$ since $c = 1$ or $c = 2$.
$$\chi_\beta(\gamma) = (\chi_{\beta_1}\cdots \chi_{\beta_m})(\gamma) \neq 1. $$ Thus there exists a subscript $i$ such that $\chi_{\beta_i}(\gamma) \ne 1$, so $x^3 \equiv \gamma\ [\beta_i]$ is not solvable. 

As $\beta_i \mid \beta = 9 (-1)^{k-1} \lambda_1\cdots \lambda_k + 2-3\omega$, if $\beta_i = \lambda_j$ for some subscript $j$, $\lambda_j \mid \pi_0 = 2 - 3\omega$, so $\lambda_j = \pi_0$, which is a contradiction, thus $\beta_i \not \in A$. Similarly, if  $\beta_i = \pi_0 = 2 - 3\omega$, then $\pi_0 \mid 9 \lambda_1\cdots\lambda_k$, and $\pi_0$ is relatively prime to $\lambda$, so $\pi_0 = \lambda_j$ for some subscript $j$ : this is a contradiction, thus $\beta_i \ne \pi_0$. $\lambda_{k+1} = \beta_i$ is convenient.

So there exist infinitely many $\pi$ such that $x^3 \equiv \gamma \ [\pi]$ is not solvable.

$\bullet$ Conclusion :

if $\gamma$ is not a cube in $D$, there exist infinitely many primes $\pi$ such that $x^3 \equiv \gamma \ [\pi]$ is not sovable.

By contraposition, if the equation $x^3 \equiv \gamma \ [\pi]$ is solvable for every prime $\pi$, at the exception perhaps of the primes in a finite set, then $\gamma$ is a cube in $D$.

\end{enumerate}
\end{proof}

\paragraph{Ex. 9.23}

{\it Suppose that $p\equiv 1 \pmod 3$. Use Exercise 5 to show that $x^3 \equiv 3 \pmod p$ is solvable in $\Z$ iff $p$ is of the form $4p = C^2 + 243 B^2$.
}

\begin{proof}
Let $p$ a rational prime, $p \equiv 1 \pmod 3$, then $p = \pi \overline{\pi}$, where $\pi \in D$ is a primary prime : $\pi = a + b \omega = 3m-1 + 3 \omega$.
\begin{enumerate}
\item[$\bullet$] Suppose that there exists $x \in \Z$ such that $x^3 \equiv 3 \pmod p$. Then $x^3 \equiv 3 \pmod \pi$, so $\chi_\pi(3) = 1$. By Exercise 9.5, $\omega^{2n} = \chi_\pi(3) = 1$, thus $3 \mid n$, therefore $9 \mid b = 3n$, namely $b = 9B, B \in \Z$.

$p = N\pi = a^2+b^2 -ab, 4p = (2a-b)^2+3b^3 = C^2 + 243 B^2$, where $C = 2a-b,B=b/9$.
So there exists $C,B\in \Z$ such that $4p = C^2 + 243B^2$.
\item[$\bullet$] Reciprocally, suppose that there exist $C,B \in \Z$ such that $4p = C^2 + 243 B^2$.

As $4p = (2a-b)^2 + 3 b^2 = C^2 + 3(9B)^2$, from the unicity proved in Exercise 8.13, we obtain $b = \pm9B$, so $9 \mid b=3n, 3 \mid n$, and $\chi_\pi(3) = \omega^{2n} = 1$.

Thus there exists $x \in D$ such that $x^3 \equiv 3 \pmod \pi$. As $p \equiv 1\pmod 3$, $D/\pi D = \{\overline{0},\ldots,\overline{p-1}\}$, so there exists $h \in \Z$ such that $x \equiv h \pmod \pi$, and $h^3 \equiv 3 \pmod \pi$.

Therefore $p = N\pi \mid N(h^3 - 3)$, namely $p \mid (h^3-3)^2$, where $p$ is a rational prime, thus $p \mid h^3 -3$ : there exists $x \in \Z$ such that $x^3 \equiv 3 \pmod p$.

Moreover $4p = C^2 + 243 B^2$ implies $p \equiv 1 \pmod 3$.

$$(p\equiv 1 \ [3]\ \mathrm{and}\ \exists x \in \mathbb{Z}, x^3 \equiv 3 \ [p] )\iff \exists C \in \mathbb{Z}, \exists B \in \mathbb{Z}, 4p = C^2+243 B^2.$$
\end{enumerate}
\end{proof}

\paragraph{Ex. 9.24}

{\it Let $\pi = a +b \omega$ be a complex primary element of $D = \Z[\omega]$. Put $a = 3m-1,b=3n,p = N(\pi)$.
\begin{enumerate}
\item[(a)] $(p-1)/3 \equiv -2m + n \pmod 3$.
\item[(b)] $(a^2-1)/3 \equiv m \pmod 3$.
\item[(c)] $\chi_\pi(a) = \omega^m$.
\item[(d)] $\chi_\pi(a+b) = \omega^{2n} \chi_\pi(1-\omega)$.
\end{enumerate}
}

\begin{proof}
As $N\pi = p$ is a rational prime, $\pi$ is a primary prime.
\begin{enumerate}
\item[(a)]$p-1 = (3m-1)^2+(3n)^2-3n(3m-1) - 1 \equiv -6m + 3n \pmod 9$, thus $$\frac{p-1}{3} \equiv -2m + n \pmod 3.$$
\item[(b)]$a^2 - 1 = (3m-1)^2-1 \equiv -6m \pmod 9$, thus
$$\frac{a^2-1}{3} \equiv m \pmod 3.$$
\item[(c)]  As $\pi,a$ are primary, by Exercise 9.20, $\chi_\pi(a) = \chi_a(\pi)$.

Since $\pi \equiv b\omega \pmod a$, $\chi_a(\pi) = \chi_a(b) \chi_a(\omega)$.

By Exercise 9.3,  as $a = 3m-1$, $\chi_a(\omega) = \omega^{M+N}$, where $M = m, N = 0$, so $$\chi_a(\omega) = \omega^m.$$

If $q$ is a rational prime, $q \equiv 2 \pmod 3$, and $q \wedge b = 1$, then $\chi_q(b) = 1$ (Prop. 9.3.4, Corollary).

If $p$ is a rational prime, $p \equiv 1 \pmod 3$ and $p \wedge b = 1$, then $p = \pi \overline{\pi}$, with $\pi$ primary prime in $D$ (and also $\overline{\pi}$), and by definition of $\chi_p$, $\chi_p(b) = \chi_\pi(b) \chi_{\overline{\pi}}(b)$.

As $\chi_{\overline{\pi}}(b) = \chi_{\overline{\pi}}(\overline{b}) = \overline{\chi_\pi(b)}$ (Prop. 9.3.4(b)), so $\chi_p(b) = 1$.
$a$ has a decomposition in prime  factors of the form :
$$a = \pm q_1q_2\cdots q_k p_1p_2\cdots p_l = \pm q_1q_2\cdots q_k \pi_1 \overline{\pi_1} \pi_2 \overline{\pi_2}\cdots \pi_l \overline{\pi_l},$$
where $q_i \equiv -1 , p_j \equiv 1 \pmod 3$, and  the $\pi_k$ are primary primes (since all these elements are primary, the symbol $\pm$ is $(-1)^{k-1}$).
Thus, by Ex. 9.21,
$$\chi_a(b) = \chi_{q_1}(b)\cdots \chi_{q_k(b)} \chi_{\pi_1}(b)\chi_{\overline{\pi_1}}(b)\cdots \chi_{\pi_l}(b)\chi_{\overline{\pi_l}}(b) = 1.$$
($a$ is relatively prime to $b$ in $\Z$ : if a rational prime $r$ divides $a,b$, then $r \mid \pi$ in $D$, thus $r \mid \overline{\pi}$, so $r^2 \mid \pi \overline{\pi} = p$ in $D$, thus $r^2 \mid p$ in $\Z$, which implies $r = p$.
 But then $p \mid \pi, N(p) \mid N(\pi), p^2 \mid p$ : this is absurd.
 As $a$ is relatively prime to $b$ in $\Z$, $ua + vb = 1,\ u,v \in \Z$, so $a,b$ are relatively prime in $D$, each prime factor $q_i,\pi_i,\overline{\pi_i}$ of $b$ is relatively prime to $a$.)
 
 We conclude that $\chi_a(b) = 1,\chi_a(\omega) = \omega^m$, so $\chi_\pi(a) = \chi_a(\pi) = \chi_a(b) \chi_a(\omega) = \omega^m$.
 $$\chi_\pi(a) = \omega^m.$$
 \item[(d)] 
 $$a+b = [(a+b) \omega] \omega^{-1},$$
and 
$$(a+b) \omega = (a+b\omega) +a\omega -a \equiv a(\omega-1)\pmod \pi,$$ thus
$$a+b \equiv -a(1-\omega) \omega^{-1} \ [\pi],$$
$$\chi_\pi (a+b) = \chi_\pi(1-\omega) \chi_\pi(a) \chi_\pi(\omega)^{-1},$$
$\chi_\pi(a) = \omega^m$ by (c), and $\chi_\pi(\omega) = \omega^{m+n}$(Ex. 9.3), thus

$$\chi_\pi(a+b) = \omega^{2n} \chi_\pi(1-\omega).$$
\end{enumerate}
\end{proof}

\paragraph{Ex. 9.25}

{\it Show that $\chi_{a+b}(\pi)$ may be computed as follows.
\begin{enumerate}
\item[(a)] $\chi_{a+b}(\pi) = \chi_{a+b}(1-\omega)$.
\item[(b)] $\chi_{a+b}(\pi) = \omega^{2(m+n)}$.
\end{enumerate}
}

\begin{proof}
\begin{enumerate}
\item[(a)] $\pi = a + b \omega$ and $a \equiv -b \pmod {a+b}$, thus $\pi \equiv -b(1-\omega) \pmod {a+b}$. So
$$\chi_{a+b}(\pi) = \chi_{a+b}(b) \chi_{a+b}(1-\omega).$$
As $a \wedge b = 1$, $(a+b) \wedge b = 1$ : as in Ex. 9.24, $\chi_{a+b}(b) = 1$. So
$$\chi_{a+b}(\pi) =  \chi_{a+b}(1-\omega).$$
\item[(b)] Since $\chi_{a+b}$ is a character of order 3,
\begin{align*}
\chi_{a+b}(1-\omega) &= (\chi_{a+b}((1-\omega)^2))^2\\
&=(\chi_{a+b}(-3\omega))^2\\
&=[\chi_{a+b}(3) \chi_{a+b}(\omega)]^2
\end{align*}

$\chi_{a+b}(3) = 1$ car $(a+b) \wedge 3 = (3(m+n)-1) \wedge 3 = 1$.

$\chi_{a+b}(\omega) = \omega^{m+n}$ (Ex. 9.19).

Conclusion : $$\chi_{a+b}(1-\omega) = \omega^{2(m+n)}.$$
\end{enumerate}
\end{proof}

\paragraph{Ex. 9.26}

{\it Combine the previous two exercises to conclude that $\chi_\pi(1-\omega) = \omega^{2m}$.
}

\begin{proof}
$\pi$ and $a+b$ are primary elements of $D$, so
$$\chi_\pi(a+b) = \chi_{a+b}(\pi).$$
By Exercises 9.24 and 9.24,
\begin{align*}
\chi_\pi(a+b) &= \omega^{2n} \chi_\pi(1-\omega)\\
\chi_{a+b}(\pi) &= \omega^{2(m+n)}
\end{align*}
Thus $\omega^{2n} \chi_\pi(1-\omega) = \omega^{2(m+n)}$.

In conclusion,
$$\chi_\pi(1-\omega) = \omega^{2m}.$$
\end{proof}

\paragraph{Ex. 9.27}

{\it Let $\pi = a+bi$ be a primary irreducible in $\Z[i], b\ne 0$. Show
\begin{enumerate}
\item[(a)] $a \equiv (-1)^{(p-1)/4} \pmod 4, p = N(\pi)$.
\item[(b)] $b \equiv (-1)^{(p-1)/4} - 1 \pmod 4$.
\end{enumerate}
(Wrong sentence for (b) in an older edition.)
}

\begin{proof}
Let $\pi=a+bi$ a primary prime in $\mathbb{Z}[i]$, $b\neq 0$.
$$p = \pi \bar{\pi}=a^2+b^2\equiv 1 \ [4].$$
By Lemma 6 Section 7, $a$ is odd, $b$ even, and 
 $$(a\equiv 1\ [4], b\equiv 0 \ [4])\ \mathrm{or}\  (a\equiv 3\  [4], b\equiv 2 \ [4]) .$$
\begin{enumerate}
\item[(a)]
	\begin{enumerate}
	\item[$\bullet$] Case 1 : $a \equiv 1 \ [4], b \equiv 0 [4]$.
	$a=4A+1,b=4B$, so  $(a^2+b^2-1)/4= 4A^2+4B^2+2A$ is even : 

	$(-1)^{(p-1)/4} = (-1)^{(a^2+b^2-1)/4} = 1$, and $a\equiv 1 [4]$, thus $a \equiv (-1)^{(p-1)/4} \ [4]$.
	\item[$\bullet$] Case 2 : $ a\equiv 3 \ [4], b \equiv 2 \pmod 4$.
	
	$a=4A+3,b=4B+2, a^2+b^2-1 = 16 A^2+24A+9+16B^2+16B+4-1 \equiv 4 \ [8]$, so $(a^2+b^2-1)/4 \equiv 1 \ [2], (-1)^{(p-1)/4} = (-1)^{(a^2+b^2-1)/4} = -1$, and $a \equiv -1\ [4]$, thus  $a \equiv (-1)^{(p-1)/4} \ [4]$.
	\end{enumerate}
	In both cases, 
	$$ a \equiv (-1)^{(p-1)/4} \ [4].$$
\item[(b)] In every case, $b \equiv a-1 \ [4]$, thus
$$ b \equiv (-1)^{(p-1)/4}-1 \ [4].$$

\end{enumerate}
\end{proof}

\paragraph{Ex. 9.28}

{\it The notation being as in Exercise 27 show $\chi_\pi(\overline{\pi}) = \chi_\pi(2) \chi_\pi(a)$.
}

\begin{proof}
$\pi = a+bi,\overline{\pi} = a-bi = 2a-\pi \equiv 2a \ [\pi]$, thus, by Proposition 9.8.3 (e) :
 $$\chi_\pi(\overline{\pi}) = \chi_\pi(2a) = \chi_\pi(2) \chi_\pi(a).$$
\end{proof}

\paragraph{Ex. 9.29} 

{\it By Exercise 9.27, $a(-1)^{(p-1)/4}$ is primary. Use biquadratic reciprocity to show $\chi_\pi(a(-1)^{(p-1)/4}) = (-1)^{(a^2-1)/8}$.
}

\begin{proof}
$a \equiv (-1)^{(p-1)/4}\ [4]$ (Ex. 9.27(a)), $a  (-1)^{(p-1)/4} \equiv 1\  [4]$, thus $a  (-1)^{(p-1)/4}$ is primary (if $a \neq \pm 1$).

If $a = \pm 1$ is an unit, $a  (-1)^{(p-1)/4} = 1$ and $\chi_\pi(a(-1)^{(p-1)/4}) = 1 = (-1)^{(a^2-1)/8}$, so we can suppose that $a$ is not an unit.

As $a  (-1)^{(p-1)/4} \equiv 1 \pmod 4$, the Law of Biquadratic Reciprocity (Prop. 9.9.8) gives
\begin{align*}
\chi_\pi(a(-1)^{(p-1)/4}) &= \chi_{a(-1)^{(p-1)/4}}(\pi) \\
&= \chi_a(\pi) \qquad (\mathrm{Prop. 9.8.3(f)})\\
&=\chi_a(a+bi)\\
&=\chi_a(bi)\\
&=\chi_a(b) \chi_a(i)
\end{align*}

As $a \wedge b=1$ (since $p = a^2+b^2$), $\chi_a(b) = 1$ (Prop. 9.8.5, with $a\neq 1$), so
$$\chi_\pi(a(-1)^{(p-1)/4}) =  \chi_a(i).$$
 $a$ is not an unit, and $2 \nmid a$, $\chi_a(i) \equiv  i^{(N(a) -1)/4} \pmod a$, thus $\chi_a(i) =  i^{(N(a) -1)/4}$.

As $a$ is odd, $(a^2-1)/4$ is even, so
\begin{align*}
 \chi_\pi(a(-1)^{(p-1)/4}) &= \chi_a(i)\\
 &= i^{(N(a) -1)/4}\\
 &= i^{(a^2-1)/4}\\
 &=(-1)^{(a^2-1)/8}
\end{align*}
 Conclusion : 
  $$\chi_\pi(a (-1)^{(p-1)/4}) =  (-1)^{(a^2-1)/8}$$
\end{proof}

\paragraph{Ex. 9.30}

{\it Use the preceding two exercises to show $\chi_\pi(\overline{\pi}) = \chi_\pi(2) (-1)^{(a^2-1)/8}$.
}

\begin{proof}
By Exercises 9.28, 9.29, and $\chi_\pi(-1) =(-1)^{(a-1)/2}$ (Prop. 9.8.3(d)),
\begin{align*}
\chi_{\pi}(\overline{\pi})&=  \chi_\pi(2) \chi_\pi(a)\\
&=\chi_\pi(2) \chi_\pi(a(-1)^{(p-1)/4})(\chi_\pi(-1))^{(p-1)/4}\\
&=\chi_\pi(2) (-1)^{(a^2-1)/8} ((-1)^{(a-1)/2})^{(p-1)/4}\\
&=\chi_\pi (-2)(-1)^{(a^2-1)/8} ((-1)^{(a-1)/2})^{(p+3)/4}\\
&=\chi_\pi(-2) (-1)^{(a^2-1)/8}  (-1)^{((a-1)/2)\,((p+3)/4)}
\end{align*}

If $a \equiv 1 \pmod 4$, then $(-1)^{(a-1)/2}=1$.

If $a \equiv 3 \pmod 4$, then $b \equiv 2 \ [4]$ : $$a=4A+3,b=4B+2,p+3 = a^2+b^2+3 = (4A+3)^2+(4B+2)^2+3 \equiv 0 \ [8],$$ so $(p+3)/4 \equiv 0 \ [2].$

In both cases  $  (-1)^{((a-1)/2)\,((p+3)/4)}=1$, and so

$$\chi_{\pi}(\overline{\pi})=\chi_\pi(-2) (-1)^{(a^2-1)/8}.  $$
\end{proof}

\paragraph{Ex. 9.31}

{\it Let $p$ be prime, $p\equiv 1 \pmod 4$. Show that $p = a^2+b^2$ where $a$ and $b$ are uniquely determined by the conditions $a \equiv 1 \pmod 4, b \equiv -((p-1)/2)! a \pmod p$.
}

\begin{proof}
Recall the following lemma : 

{\bf Lemma :} 

lemme : Let $p$ be prime, $p \equiv 1 \ [4]$, then  $\left [ \left ( \frac{p-1}{2} \right)!\right ]^2 \equiv -1 \ [p]$.

By Wilson's theorem (Prop. 4.1.1, Corollary), $(p-1)! \equiv -1 \ [p]$.

\begin{align*}
-1 \equiv (p-1)! &= 1.2. \cdots .(\frac{p-1}{2})(\frac{p+1}{2})\cdots(p-2)(p-1)\\
&\equiv 1.2.\cdots\frac{p-1}{2}[-(\frac{p-1}{2})]\cdots(-2)(-1) \\
&\equiv (-1)^{(p-1)/2} \left [ \left ( \frac{p-1}{2} \right)!\right ]^2 \\
&\equiv \left [ \left ( \frac{p-1}{2} \right)!\right ]^2 \ [p]\\
\end{align*}
since $p\equiv 1 \ [4]$.
\begin{enumerate}
\item[$\bullet$] We show that there exists a pair $a,b \in \Z$ which verifies the sentence.

By lemma 5 section 7, as $p \equiv 1 \ [4]$, there exists an irreducible $\pi$ such that $N(\pi) = p$, and we can choose $\pi$ such that $\pi = A + Bi$ is primary (lemma 7 section 7), so $A$ is odd.

If $A\equiv 1 \pmod 4$, we take $a=A$, and if $A \equiv 3 \pmod 4$, we take $a = -A$ : then $a \equiv 1 \pmod 4$.

Let $u = \left ( \frac{p-1}{2} \right)!$. Then $0 \equiv p \equiv A^2 + B^2 \pmod p$,  $B^2 \equiv -A^2 \equiv (uA)^2 \ \pmod p$.

$p \mid (B-uA)(B+uA)$, thus $B \equiv \pm uA\pmod p$.

If $B \equiv -ua \pmod p$, we take $b = B$, if not $b = -B$.

$a,b$ are such that $p=a^2+b^2, a\equiv 1 \ [4], b \equiv -((p-1)/2)!\, a\ [p]$.

\item[$\bullet$] Unicity of the pair $(a,b)$ such that
$$p=a^2+b^2, a\equiv 1 \ [4], b \equiv -((p-1)/2)!\, a\ [p].$$

Suppose that $c,d$ are such that $p = c^2+d^2,c\equiv 1 \ [4], d \equiv -((p-1)/2)! c \ [p]$.

Let $\pi = a + ib, \lambda = c+id$. As $p = N\pi = N \lambda$ is a rational prime, $\pi$ and $\lambda$ are primes in $D$, and $p = \pi \overline{\pi} = \lambda \overline{\lambda}$, thus $\lambda$ is associate to $\pi$ or $\overline{\pi}$. :
$$\lambda \in \{\pi, - \pi, i \pi, -i\pi, \overline{\pi},-\overline{\pi},i\overline{\pi},-i\overline{\pi}\}.$$
As $a,c$ are odd, and $b,d$ even, it remains only the possibilities $\lambda =\pm \pi, \lambda = \pm \overline{\pi}$, thus $c = \pm a$. Moreover $a \equiv c \equiv 1 \ [4]$, thus $a=c$, and $d \equiv -((p-1)/2)!c \equiv -((p-1)/2)!a \equiv b \ [p]$.

$p=a^2+b^2 = a^2 + d^2$, so $d = \pm b$, and $d \equiv b \ [p]$.

If $d = -b$, then $p\mid 2b$, thus $p \mid b$, and also $p \mid a$, so $p^2 \mid p$: this is impossible. So $a=b,c=d$. Unicty is proved.

Conclusion : if $p\equiv 1 \ [4]$, there exists an unique pair $a,b$ such that
$$p=a^2+b^2, a \equiv 1 \pmod 4, b \equiv -((p-1)/2)! a \pmod p.$$
\end{enumerate}
\end{proof}

\paragraph{Ex. 9.32}

{\it Let $p$ be a prime, $p\equiv 1 \pmod 4$ and write $p = \pi \overline{\pi}, \pi \in \Z[i]$. Show $\chi_p(1+i) = i^{(p-1)/4}$.
}

\begin{proof}

\end{proof}
\begin{align*}
\chi_p(1+i) &= \chi_\pi(1+i) \chi_{\bar{\pi}}(1+i) \\
&= \chi_\pi(1+i) \overline{\chi_\pi(1-i)} \qquad ( \mathrm{Prop.}\  \mathrm{9.8.3(c)})\\
& =\frac{\chi_\pi(1+i)}{\chi_\pi(1-i) }= \chi_\pi(i) \qquad (\mathrm{since}\  (1-i)i = 1+i)\\
&=i^{\frac{p-1}{4}}
\end{align*}

\paragraph{Ex. 9.33}

{\it  Let $q$ be a positive prime, $q\equiv 3 \pmod 4$. Show $\chi_q(1+i) = i^{(q+1)/4}$. [Hint : $(1+i)^{q-1} \equiv -i \pmod q$.]
}

\bigskip

The sentence is false and must be replaced by
$$ \chi_q(1+i) = (-i)^{(q+1)/4 } = i^{-(q+1)/4} .$$
Verify this on the example $q=11$ :
\begin{align*}
\chi_{q}(1+i) &\equiv (1+i)^{(q^2-1)/4} \pmod q\\
& \equiv (1+i)^{30} \pmod {11}\\
&\equiv -2^{15} i \equiv -32 i \equiv i \pmod {11}
\end{align*}
so $\chi_{11}(1+i) = i$, and $i^{(-q-1)/4} = i^{-3} = i$ ( but $i^{(q+1)/4} = -i$).

\begin{proof}

Write $q = 4k + 3, k \in \N$.

As $(1+i)^2 = 2i$, $(1+i)^{q-1} =  (2i)^{(q-1)/2}$.

$2^{(q-1)/2} \equiv \legendre{2}{q} \ [q]$ et $ \legendre{2}{q} = (-1)^{(q^2-1)/8} = (-1)^{2k^2+3k+1} = (-1)^{k+1}$

$i^{(q-1)/2} = i^{2k+1} = (-1)^k i$. 

So
$$ (1+i)^{q-1} \equiv -i\ [q].$$
$N(q) = q^2$, so $\chi_q(1+i) \equiv (1+i)^{(q^2-1)/4} = [(1+i)^{q-1}]^{(q+1)/4}\equiv (-i)^{(q+1)/4} \ [q]$ :

$$\chi_q(1+i) =(-i)^{(q+1)/4} = i^{(-q-1)/4}.$$
\end{proof}

\paragraph{Ex. 9.34}

{\it Let $\pi = a +bi$ be a primary irreducible, $(a,b) = 1$. Show
\begin{enumerate}
\item[(a)] if $\pi \equiv 1 \pmod 4$, then $\chi_\pi(a) = i^{(a-1)/2}$.
\item[(b)] if $\pi \equiv 3 + 2i\pmod 4$, then $\chi_\pi(a) = -i^{(-a-1)/2}$.
\end{enumerate}
}

\begin{proof}
Let $\pi = a+bi$ be a primary irreducible, with $a\wedge b=1$, so $b \neq 0$ : we can apply the result of Exercise 9.29 :
$$\chi_\pi(a(-1)^{(p-1)/4}) = (-1)^{(a^2-1)/8}.$$
\begin{enumerate}
\item[(a)] Suppose that  $\pi \equiv 1 \ [4]$. 

Then $a\equiv 1\ [4], b\equiv 0 \ [4], a = 4A+1,b=4B,\ A,B \in \mathbb{Z}$.

As $\chi_\pi(-1) = (-1)^{(a-1)/2}$, 
$$\chi_\pi(a) = (-1)^{\frac{a-1}{2} \frac{p-1}{4}}(-1)^{\frac{a^2-1}{8}},$$
where 
$$ p = N \pi = a^2+b^2, (-1)^{(p-1)/4} = (-1)^{\frac{a^2-1}{4}+\frac{b^2}{4}} = (-1)^{4A^2+2A+4B^2}=1,$$
thus $(-1)^{\frac{a-1}{2} \frac{p-1}{4}}=1$.
$$\chi_\pi(a) = (-1)^{(a^2-1)/8} = (-1)^{2A^2+A} = (-1)^A = (-1)^{(a-1)/4} = i^{(a-1)/2}.$$
Conclusion : if $\pi \equiv 1 \ [4]$, $\chi_\pi(a) =  i^{(a-1)/2}$.
\item[(b)] Suppose that $\pi \equiv 3 + 2i\ [4]$.

Then $a\equiv 3 \ [4], b \equiv 2\ [4],a = 4A+3,b=4B+2,\ A,B\in\mathbb{Z}$. As in (a),
$$\chi_\pi(a) = (-1)^{\frac{a-1}{2} \frac{p-1}{4}}(-1)^{\frac{a^2-1}{8}},$$
where $a^2+b^2-1 = 16 A^2+24A+16B^2+16B+12 \equiv 4 \ [8]$, so $\frac{a^2+b^2-1}{4} \equiv 1 \ [2]$, thus $(-1)^{(p-1)/4} = (-1)^{(a^2+b^2-1)/4} = -1$.

$$ (-1)^{\frac{a-1}{2} \frac{p-1}{4}} = (-1)^{\frac{a-1}{2}} = (-1)^{2A+1} = -1,$$

$$\frac{a^2-1}{8} = 2A^2+3A+1, (-1)^{(a^2-1)/8} = (-1)^{3A+1} = (-1)^{A+1}=(-1)^{(a+1)/4},$$

$$\chi_\pi(a)= -(-1)^{(a+1)/4} = -i^{(a+1)/2}.$$

Moreover $$\frac{a+1}{2} \equiv \frac{-a-1}{2} \ [4] \iff a+1 \equiv -a-1\ [8] \iff 2a\equiv -2 \ [8] \iff a\equiv 3 \ [4],$$ thus $i^{(a+1)/2} = i ^{(-a-1)/2}$

Conclusion : if $\pi \equiv 3+2i \ [4]$, $\chi_\pi(a) = - i^{(-a-1)/2}$.

\end{enumerate}
\end{proof}

\paragraph{Ex. 9.35}

{\it If $\pi = a+bi$ is as in Exercise 9.34 show $\chi_\pi(a)\chi_\pi(1+i) = i ^{(3(a+b-1))/4}$. [Hint: $a(1+i) = a+b +i(a+bi)$. Generalize Exercises 32 and 33 to any integer $\equiv 1 \pmod 4$ and use Proposition 9.9.8. Note $a+b \equiv 1 \pmod 4$.]
}

\begin{proof}
We give a generalization of Exercises 9.32 and 9.34 : if $n\equiv1\ [4], n \neq 1$, then $\chi_n(1+i) = i^{(n-1)/4}$.

By Exercises 9.33 and 9.34, we know that if  $p\equiv 1\ [4]$ is a rational prime, then 
$$\chi_p(1+i) = i^{(p-1)/4},$$
and if $q \equiv 3 \ [4]$, in other words $-q \equiv 1 \ [4]$, where $q$ is a rational prime, then
$$\chi_{-q}(1+i) =  \chi_{q}(1+i) = i^{(-q-1)/4}.$$
Let  $n \in \mathbb{Z}, n\equiv 1 \ [4],n\neq 1$.

If $n>0$, $n = q_1q_2\cdots q_kp_1p_2\cdots p_l$, where $q_i \equiv -1 \ [4], o_i \equiv 1 \ [4]$, thus $k$ is odd.

If $n<0$, $n = -q_1q_2\cdots q_kp_1p_2\cdots p_l$, with $k$ odd.
In both cases, 
$$n = (-q_1)(-q_2)\cdots(-q_k)p_1p_2\cdots p_l,$$
so of the form
$$n = s_1s_2\cdots s_N, \qquad \mathrm{where}\ s_i = -q_i, 1\leq i \leq k, s_i = p_{i-k}, \ k+1 \leq i \leq k+l = N,$$
so $s_i\equiv 1 \ [4], \ 1 \leq i \leq N$.
\begin{align*}
\chi_n(1+i)&=  \chi_{-q_1}(1+i)\cdots\chi_{-q_k}(1+i)\chi_{p_1}(1+i)\cdots\chi_{p_l}(1+i)\\
&=i^{(-q_1-1)/4}\cdots i^{(-q_k-1)/4}  i^{(p_1-1)/4} \cdots i^{(p_l-1)/4} \\
&=  \chi_{-q_1}(1+i)\cdots\chi_{-q_k}(1+i)\chi_{p_1}(1+i)\cdots\chi_{p_l}(1+i)\\
&=i^{(s_1-1)/4}\cdots i^{(s_k-1)/4}  i^{(s_{k+1}-1)/4} \cdots i^{(s_N-1)/4} \\
&=i^{\sum_{i=1}^N \frac{s_i-1}{4}}\\
&=i^{(n-1)/4},
\end{align*}
the last equality resulting of Exercise 9.44.

Conclusion : if $n \in \mathbb{Z}, n\equiv1\ [4], n \neq 1$, then $\chi_n(1+i)=i^{(n-1)/4}$.

\bigskip

Let $\pi = a+bi, a \wedge b = 1$ a primary irreducible. As $a(1+i) = a+b + i(a+bi)$, $a(1+i) \equiv a+b \ [\pi]$, so
$$\chi_{\pi}(a) \chi_{\pi}(1+i) = \chi_{\pi}(a+b).$$
As $\pi =a +bi$ is primary, $a+b \equiv 1 \ [4]$.

If $a+b = 1$, then $\chi_{\pi}(a) \chi_{\pi}(1+i) = \chi_{\pi}(a+b)=1 = i^{3(a+b-1)/4}$. If not, the Law of Biquadratic Reciprocity (Proposition 9.9.8) gives
$$\chi_\pi(a+b) = \chi_{a+b}(\pi).$$
Now $b \equiv -a\pmod{a+b}$, so $ a + bi \equiv a(1-i) \equiv -i a (1+i) \pmod{a+b}$. Therefore
$$\chi_{a+b}(\pi) = \chi_{a+b}(-1) \chi_{a+b}(a) \chi_{a+b}(i) \chi_{a+b}(1+i).$$
Since $n\equiv 1 \ [4]$, $\chi_n(i) = (-1)^{(n-1)/4}$ (Prop.9.8.6), thus $$\chi_n(-1) = \chi_n(i^2) = (-1)^{\frac{n-1}{2}}=1.$$

Consequently, since $a+b \equiv 1 \ [4]$, $\chi_{a+b}(-1) = 1$ .

As $a\wedge b = 1, (a+b) \wedge a = 1$, thus $\chi_{a+b}(a) = 1$ (Prop 9.8.5).

$a+b \equiv 1\ [4]$, thus $\chi_{a+b}(i) = (-1)^{(a+b-1)/4}$ (Prop. 9.8.6).

From the first part of this proof, $\chi_{a+b}(1+i) = i^{(a+b-1)/4}$, so
\begin{align*}
 \chi_{a+b}(\pi) &= \chi_{a+b}(-1) \chi_{a+b}(a) \chi_{a+b}(i) \chi_{a+b}(1+i)\\
 &=(-1)^{(a+b-1)/4} i ^{(a+b-1)/4}\\
 &=i^{(a+b-1)/2} i ^{(a+b-1)/4}\\
 &=i^{3(a+b-1)/4}
 \end{align*}
 
 Conclusion : if $\pi = a+b i, a \wedge b = 1$ is a primary irreducible, then $$\chi_{\pi}(a) \chi_{\pi}(1+i) = i^{3(a+b-1)/4}$$
\end{proof} 

\paragraph{Ex. 9.36}

{\it Remove the restriction $(a,b)=1$ in Exercise 9.34.
}

\begin{proof}
Suppose that $q = a\wedge b >1$. Then $a = qa',b = qb', \ a',b' \in \Z$, so $\pi = q(a'+ib')$.

As $\pi$ is irreducible, and as $q$ is not an unit, $u = a'+b'i$ is an unit, and so $\pi = uq$ is associate to $q$ : the rational integer $q$ is then a prime in $D$, so a rational prime $q \equiv 3 \pmod 4$.

If $u = \pm i$, then $\pi = \pm q = a + bi$ is such that $b$ is odd, in contradiction with $\pi$ primary. Thus $u = \pm 1$, and $\pi = \varepsilon q, \varepsilon = \pm 1$. As $\pi$ is primary, $\varepsilon = -1$, so $\pi = -q$.

Then $\chi_\pi(a) = \chi_{-q}(-q) = 0$, the result of Ex. 34 is false if $b=0$.

Conclusion : if $\pi = a+bi$ is a primary irreducible, and $b\neq 0$, then 
\begin{enumerate}
\item[(a)] if $\pi \equiv 1 \ [4], \chi_\pi(a) = i^{(a-1)/2}$,
\item[(b)] if $\pi \equiv 3+2i\ [4]$, $\chi_\pi(a) = -i^{(-a-1)/2}$.
\end{enumerate}
\end{proof}

\paragraph{Ex. 9.37}

{\it Combine Exercises 32, 33, 34, and 35 to show $\chi_\pi(1+i) = i^{(a-b-b^2-1)/4}$. Show that this result implies Exercise 26 of Chapter 5 ("the biquadratic character of 2").
}

\begin{proof}
Let $\pi = a + ib$ be a primary irreducible in $\Z[i]$.
\begin{enumerate}
\item[$\bullet$] If $b = 0$, then $\pi = a \in \Z$.
As $\pi$ is primary, $\pi = -q, q\equiv 3 \pmod 4$, where $q$ is a rational prime, so $a=-q, b = 0$.
By Ex. 9.32 (or its generalization 9.35),
 $$\chi_\pi(1+i) = \chi_{-q}(1+i) = i^{(-q-1)/4} = i^{(a-b-b^2-1)/4}.$$
 \item[$\bullet$] If $b \ne  0$, by Ex. 9.35, 
 $$\chi_\pi(a) \chi_\pi(1+i) = i^{3(a+b-1)/4}.$$
 	\begin{enumerate}
	\item[$\bullet$]  If $\pi \equiv 1\ [4]$, $a \equiv 1\ [4], b \equiv 0 \ [4]$ : $a=4A+1,b=4B,\ A,B\in \mathbb{Z}$.
	
By Ex. 9.34(a),
$$\chi_\pi(a) = i^{(a-1)/2}, \chi_\pi(a)^{-1} = i^{(-a+1)/2}.$$
\begin{align*}
\chi_{\pi}(1+i) &= i^{3 \frac{a+b-1}{4} - 2 \frac{a-1}{4}}\\
&=i^{\frac{a+3b-1}{4}}\\
&=i^{\frac{a-b-b^2-1}{4}}
\end{align*}
since $\left( \frac{a-3b-1}{4} \right)- \left( \frac{a-b-b^2-1}{4}\right) = b + \frac{b^2}{4} = 4B+4B^2 \equiv 0 \ [4]$.

	\item[$\bullet$] If $\pi \equiv 3+2i\  [4]$, $a \equiv 3[4], b \equiv 2 \ [4]$ : $a=4A-1,b=4B+2,\ A,B\in\mathbb{Z}$.
	
By Ex. 9.34(b),

$$\chi_\pi(a) =- i^{(-a-1)/2}, \chi_\pi(a)^{-1} = -i^{(a+1)/2} = i^{(a-3)/2},$$
so
$$\chi_\pi(1+i) = i^{(3a+3b-3+2a-6)/4} = i^{(5a+3b-9)/4}.$$

Now $\frac{1}{4}[(a-b-b^2-1) - (5a+3b-9)] = \frac{1}{4}(-4a-4b-b^2+8) = -a-b+2 -\frac{b^2}{4} = -4A+1-4B-2+2 -(2B+1)^2 \equiv 0 \ [4]$,

thus $\chi_\pi(1+i) = i^{(a-b-b^2-1)/4}$.



	\end{enumerate}
	Conclusion : if $\pi =a+ib$ is primary irreducible, then

$$\chi_\pi(1+i) = i^{(a-b-b^2-1)/4}$$

\end{enumerate}

\bigskip

Second part : the biquadratic character of 2 (see Ex. 5.25 to 5.28).

Let $p\equiv 1 \ [4]$. Then $p=N(\pi)$, where $\pi = a+bi$ is a primary irreducible.

We show first $\chi_\pi(2)=1 \iff 8 \mid b$.

 $2 = -i(1+i)^2$, so $$\chi_\pi(2) = \chi_\pi(-1) \chi_\pi(i) \chi_\pi(1+i)^2.$$
 
 By Proposition 9.8.5(d) (see Exercise 9.38), and Exercise 9.35,
 \begin{align*}
&\chi_\pi(-1) = (-1)^{(a-1)/2},\\
&\chi_\pi(i) = i^{(p-1)/4} = i^{(a^2-1)/4 + b^2/4}\\
&\chi_\pi(1+i)^2 = (-1)^{(a-b-b^2-1)/4}.
\end{align*}
So
$$\chi_\pi(2) = (-1)^{(a-1)/2}(-1) ^{(a-b-b^2-1)/4}i^{(a^2-1)/4 + b^2/4}.$$

$\bullet$ If $8 \mid b$, then $b\equiv 0 \pmod 8$, and since $\pi$ is primary, $a \equiv 1 \pmod 4$, so $a = 4A+1, \ A\in \Z$. Therefore
$$\chi_\pi(2) = (-1)^{2A} (-1)^A \, i^{4A^2+2A} = (-1)^A (-1)^A = 1.$$

$\bullet$ Reciprocally, if $\chi_\pi(2)=1$, the exponent of $i$ is even, thus $2 \mid(p-1)/4$, so $8 \mid p-1$.

As $\pi$ is primary, $a$ is odd and $b$ even : $a = 2a'+1,b=2b',\ a',b'\in \Z$, and
$$8 \mid p-1 = 4a'^2+4a'+4b'^2 = 8 \frac{a'(a'+1)}{2} + 4 b',$$
thus $b'$ is even, $b\equiv 0 \pmod 4$, and as $\pi$ is primary, $a \equiv 1 \pmod 4$ : we can write
$$a=4A+1,b=4B, \qquad A,B\in \Z.$$
Therefore 
 $$(-1)^{(a-1)/2}=1,$$
 and
  $$\frac{a-b-b^2-1}{4} = \frac{4A+1-4B-16B^2-1}{4} = A-B-4B^2 \equiv A-B\ [2],$$
  thus
    $$(-1) ^{(a-b-b^2-1)/4} = (-1)^{A-B},$$
     $$i^{(a^2-1)/4 + b^2/4} = i^{4A^2+2A+4B^2}= (-1)^A,$$
   so $1 = \chi_\pi(2) = (-1)^B$, $B$ is even, so $8 \mid b$.
  
  \bigskip
  
 We have proved
 $$\chi_\pi(2)=1 \iff 8 \mid b.$$
 
 If there exists $x \in \Z$ such that $2\equiv x^4 \pmod p$, then $2 \equiv x^4 \pmod \pi$, thus $\chi_\pi(2)=1$, and $8 \mid b$ :
 $$p = A^2+64B^2,\qquad \mathrm{where}\ A=a,B =b/8.$$
 
 Reciprocally, if $p=A^2+64B^2$, then the rational prime $p>2$ is the sum of two squares, so $p\equiv 1 \pmod 4$, and $A$ is odd.
 
 As $p = N(\pi) = a^2 +b^2 = A^2 + 64 B^2$ (with $a,A$ odd numbers), the unicity of the decomposition in sum of two squares (Ex. 8.12) gives $b^2 = 64 B^2$, so $8 \mid b$, thus $\chi_\pi(2) = 1$.
 
 Therefore there exists $\alpha \in D$ such that $2\equiv \alpha^4 \pmod \pi$. As $D/\pi D$ is the set of classes of $0,1,\cdots,p-1$, there exists $x\in \Z$ such that $x\equiv \alpha \pmod \pi$, so $2 \equiv x^4 \pmod \pi$.
 
 Then $p = N(\pi) \mid N(x^4-2) = (x^4-2)^2$, thus $p \mid x^2$, in other words $2\equiv x^4 \pmod p$.
 
 Conclusion :
 $$\exists (A,B) \in \Z^2,\ p = A^2+64B^2 \iff (p\equiv 1 \ [4] \ \mathrm{and}\ \exists x \in \Z, \ x^4 \equiv 2 \ [p]).$$
\end{proof}

\paragraph{Ex. 9.38}

{\it Prove part (d) of Proposition 9.8.3.
}

\bigskip

{\bf Proposition 9.8.3(d)} {\it 
If $\pi$ is a primary irreducible then $\chi_{\pi}(-1)=(-1)^{(a-1)/2}$, where $\pi = a+bi,b\neq 0$.}



\begin{proof}
Let $\pi = a + bi$ a primary irreducible.
\begin{enumerate}
\item[case 1.] $N(\pi) = p = a^2+b^2$ ($a$ odd, $b$ even) is a rational prime, $p \equiv 1 \pmod 4$.

Then 
$$\chi_\pi(-1) = (-1)^{\frac{p-1}{4}} = (-1)^{\frac{a^2-1}{4}+ \frac{b^2}{4}} = [(-1)^{\frac{a+1}{2}}]^{\frac{a-1}{2}} (-1)^{\frac{b^2}{4}}.$$
By Lemma 6, section 7, $a\equiv 1\ [4], b\equiv 0\ [4]$, or $a \equiv 3 [4], b\equiv 2 [4]$.
	\begin{enumerate}
	\item[$\bullet$] If $a\equiv 1\ [4], b\equiv 0\ [4]$, then $(-1)^{\frac{a+1}{2}} = -1, (-1)^{\frac{b^2}{4}}=+1$ , so
	$$\chi_\pi(-1) =(-1)^{\frac{a-1}{2}}.$$
	\item[$\bullet$] If $a \equiv 3 \ [4], b\equiv 2 \ [4]$, then $(-1)^{\frac{a+1}{2}} = 1, (-1)^{\frac{b^2}{4}}=-1$, so
	$$\chi_\pi(-1) =-1 = (-1)^{\frac{a-1}{2}}.$$

	\end{enumerate}
\item[case 2.] $N(\pi) = q^2, q \equiv -1 \pmod 4$.

As $\pi$ is primary, $\pi = -q$, so $a = -q \equiv 1 \pmod 4, b=0$.
$$\chi_\pi(-1)= (-1)^{\frac{q^2-1}{4}} = [(-1)^{q-1}]^{\frac{q+1}{4}}\equiv 1 \equiv (-1)^{\frac{a-1}{2}}[4].$$
\end{enumerate}
Conclusion : if $\pi$ is a primary irreducible in $\Z[i]$, then
$$\chi_{\pi}(-1)=(-1)^{(a-1)/2}.$$
\end{proof} 

\paragraph{Ex. 9.39}

{\it Let $p\equiv 1 \pmod 6$ and write $4p = A^2 + 27 B^2, \ A \equiv 1 \pmod 3$. Put $m=(p-1)/6$. Show $\binom{3m}{m} \equiv -1 \pmod p \iff 2 \mid B$.
}

\begin{proof}
Let $p$ a rational prime, $p\equiv 1 \pmod 6$. As $p\equiv 1 \pmod 3$, $p = N(\pi)$, where $\pi = a + b\omega$ is a primary prime. $p = N(\pi) =a^2 -ab+b^2, 4p = (2a-b)^2 + 3b^2$. As $\pi$ is primary, $a\equiv 2 \pmod 3, b \equiv 0 \pmod 3$, so $4p = A^2 + 27 B^2$, with $A = 2a-b \equiv 1 \pmod 3, b = B/3$.

\bigskip

Suppose that $2\mid B$.
Since $\pi$ is primary,
$$2 \mid B \iff 2 \mid b \iff (b \equiv 0 \  [2], a\equiv 1\ [2]).$$
By Proposition 9.6.1,  $2 \mid B$ iff $\pi \equiv 1 \ [2]$, iff $x^3-2$ is solvable in $D$, iff $\chi_\pi(2) = 1$.

By Exercise 8.6,
$$J(\chi_\pi,\chi_\pi) = \chi_\pi(2)^{-2} J(\chi_\pi,\rho),$$
where $\rho $ is the Legendre's character.

Here $\chi_\pi((2) = 1$, so $J(\chi_\pi,\chi_\pi) =  J(\chi_\pi,\rho)$, and by Lemma 1 section 4, $$\pi = a + b \omega = J(\chi_\pi,\chi_\pi) = J(\chi_\pi,\rho).$$

By Exercise 8.15, $$N(y^2 = x^3+1) = p+A,$$ and the Exercise 8.27 gives
$$N(y^2 = x^3 + 1) = N(y^2 +x^3 = 1) = p+ 2 \re J(\chi_m,\rho),$$
and also
$$-A \equiv \binom{(p-1)/2}{(p-1)/3} =\binom{(p-1)/2}{(p-1)/2 - (p-1)/3}  =  \binom{(p-1)/2}{(p-1)/6}=\binom{3m}{m} \pmod p, m=(p-1)/6.$$
Therefore
$$\binom{3m}{m} \equiv -1 \pmod p.$$

\bigskip

Reciprocally, suppose that $\binom{3m}{m} \equiv -1 \pmod p$. Then $A = 2a -b \equiv  -\binom{3m}{m} \pmod p$.
Write $J(\chi_\pi,\rho) = c + d\omega$. By Exercise 8.27(c), $2c - d \equiv -\binom{3m}{m} \pmod p$. thus
$$2a -b \equiv 2c - d \pmod p.$$

Since $|J(\chi_\pi,\rho)| = \sqrt{p}$, 
$$4p = (2a-b)^2 + 3b^2 = (2c-d)^2 + 3d^2,$$
thus $d \equiv \pm b \pmod p$.

By Exercise 8.6,
$$\pi = J(\chi_\pi,\chi_\pi) = \chi_\pi(2)^{-2} J (\chi_\pi,\rho),$$
where $\chi_\pi(2)^{-2} = \chi_\pi(2) \in \{1,\omega,\omega^2\}$.

If $\chi_\pi(2) = \omega$, then $a+b\omega = \omega(c+d\omega) = -d + \omega(c-d)$. Then $a = -d \equiv \pm b \pmod p$. As $a \equiv -b\omega \pmod \pi$, we would have $-b\omega \equiv \pm b \pmod \pi$. As $\pi \nmid b$, $\pi \mid \omega \pm 1$, with $\pi$ primary : it's impossible ($\omega + 1$ is a unit and $\omega - 1$ is prime).

If $\chi_\pi(2) = \omega^2$, then $a+b\omega = \omega^2(c+d\omega), a + b\omega^2 = \omega(c+d\omega^2)$ : same contradiction.

So $\chi_\pi(2) = 1$, and the previously proved equivalence $2\mid B \iff \chi_\pi(2) = 1$ show that $2 \mid B$.

Conclusion : $$\binom{(p-1)/2}{(p-1)/6} \equiv -1 \pmod p \iff 2 \mid B.$$
\end{proof}

\paragraph{Ex. 9.44}

{\it Let $n \in \Z$, $n = s_1\cdots s_t, n\equiv 1 \pmod 4, i = 1,\ldots,t$. Show $(n-1)/4\equiv \sum_{i=1}^t (s_i-1)/4 \pmod 4$.
}

\begin{proof}
If  $n = s t , s \equiv1,t\equiv1\ [4]$, then $s = 4k+1, t = 4 l +1, k,t \in \mathbb{Z}$, so
$$n = (4k+1)(4l+1) = 16 kl + 4k+4l+1, \frac{n-1}{4} = 4 kl + k + l \equiv k+l = \frac{s-1}{4}+\frac{l-1}{4} \ [4].$$
Reasoning by induction on $t$, suppose that  every product of $t$ factors $ n = s_1s_2\cdots s_t$, where $s_i\equiv 1 \ [4]$ verifies 
$$\frac{n-1}{4} \equiv \sum\limits_{i=1}^t \frac{s_i-1}{4} [4].$$
If $n' = s_1 s_2 \cdots s_t s_{t+1} = n s_{t+1}, s_i \equiv 1[4]$, then $n\equiv1,s_{t+1} \equiv 1 \ [4]$, so
$$\frac{n'-1}{4} \equiv \frac{n-1}{4} + \frac{s_{t+1}-1}{4} \equiv \sum\limits_{i=1}^t \frac{s_i-1}{4} + \frac{s_{t+1}-1}{4} \equiv \sum\limits_{i=1}^{t+1} \frac{s_i-1}{4} \ [4].$$
Conclusion :  if  $n = s_1 s_2 \cdots s_t, s_i \equiv 1 [4]$, alors $\frac{n-1}{4} \equiv \sum\limits_{i=1}^t \frac{s_i-1}{4} [4]$.
\end{proof}

\paragraph{Ex. 9.45}

{\it Let $\pi = a + bi \in \Z[i]$ and $q\equiv 3 \ [4]$ a rational prime. Show $\pi^q \equiv \overline{\pi} \ [4]$.
}

\begin{proof}
Let $\pi=a+bi \in \mathbb{Z}[i]$, and $q\equiv 3 \ [4]$ a rational prime.

As $\binom{q}{k} \equiv 0 \pmod q$ for $1\leq k \leq q-1$,
\begin{align*}
\pi^q &=(a+bi)^q\\
&\equiv a^q+b^qi^q\ [q]\\
&\equiv a + b i^3 \ [q]\\
&= a-bi \\
&= \overline{\pi}
\end{align*}

Conclusion : $\pi^q \equiv \bar{\pi}\ [q]$ ($\pi \in \mathbb{Z}[i]$, and $q\equiv 3 \ [4]$)
\end{proof}


\end{document}