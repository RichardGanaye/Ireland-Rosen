%&LaTeX
\documentclass[11pt,a4paper]{article}
\usepackage[frenchb,english]{babel}
\usepackage[applemac]{inputenc}
\usepackage[OT1]{fontenc}
\usepackage[]{graphicx}
\usepackage{amsmath}
\usepackage{amsfonts}
\usepackage{amsthm}
\usepackage{amssymb}
\usepackage{yfonts}
%\input{8bitdefs}

% marges
\topmargin 10pt
\headsep 10pt
\headheight 10pt
\marginparwidth 30pt
\oddsidemargin 40pt
\evensidemargin 40pt
\footskip 30pt
\textheight 670pt
\textwidth 420pt

\def\imp{\Rightarrow}
\def\gcro{\mbox{[\hspace{-.15em}[}}% intervalles d'entiers 
\def\dcro{\mbox{]\hspace{-.15em}]}}

\newcommand{\D}{\mathrm{d}}
\newcommand{\Q}{\mathbb{Q}}
\newcommand{\Z}{\mathbb{Z}}
\newcommand{\N}{\mathbb{N}}
\newcommand{\R}{\mathbb{R}}
\newcommand{\C}{\mathbb{C}}
\newcommand{\F}{\mathbb{F}}
\newcommand{\ord}{\mathrm{ord}}
\newcommand{\legendre}[2]{\genfrac{(}{)}{}{}{#1}{#2}}



\title{Solutions to Ireland, Rosen ``A Classical Introduction to Modern Number Theory''}
\author{Richard Ganaye}
\begin{document}

\maketitle

{\large \bf Chapter 4}

\paragraph{Ex. 4.1}

{\it Show that $2$ is a primitive root modulo $29$.
}

\begin{proof}
Let $p=29$ : $p-1 = 2^2\times 7$.

$2^4 = 16 \not \equiv1 [29]$

$2^{14} =  4 ^7 =4 \times 16^3 = 64\times 256 \equiv 6 \times (-34)=-204\equiv 86=3\times 29 -1 \equiv -1[29]$

$2^{28} \equiv 1[29]$ and $2^d \not \equiv 1$ if $d \mid 28, d < 28$, hence 2 is a primitive element modulo 29.
\end{proof}

\paragraph{Ex. 4.2}

{\it Compute all primitive roots for $p = 11, 13, 17$, and $19$.
}

\begin{proof}
$\bullet$ $p=11$. Then $p-1 = 10 = 2\times 5$.

$2^2 = 4 \not \equiv 1 \pmod {11}$, and $2^5 = 32 \equiv -1 \not \equiv 1 \pmod{11}$, so $2$ is a primitive element modulo $11$.

The other primitive elements modulo $11$ are congruent to the powers $2^i, i\wedge 10 = 1, 1\leq i <10$, namely $2,2^3,2^7,2^9$.

$2^7 \equiv 7 \pmod {11},2^9 \equiv 6 \pmod {11}$, so

$ \{\overline{2}, \overline{8}, \overline{7}, \overline{6}\}$ is the set of the generators of $U(\Z/11\Z)$.

Similarly :

$\bullet$ $p=13$ : $\{2,6,11,7\}$ is the set of the generators of $U(\Z/13\Z)$.

$\bullet$ $p=17$ : $\{3, 10, 5, 11, 14, 7, 12, 6\}$ is the set of the generators of $U(\Z/17\Z)$.

$\bullet$ $p=19$ : $\{2, 13, 14, 15, 3, 10\}$ is the set of the generators of $U(\Z/19\Z)$.

I obtain these results with the direct orders in S.A.G.E. :
\begin{verbatim}
p = 19; Fp = GF(p); a = Fp.multiplicative_generator()
print([a^k for k in range(1,p) if gcd(k,p-1) == 1])
\end{verbatim}
\end{proof}

\paragraph{Ex. 4.3}

{\it Suppose that $a$ is a primitive root modulo $p^n$, $p$ an odd prime. Show that $a$ is a primitive root modulo $p$.
}

\begin{proof}
Suppose that $a$ is a primitive root modulo $p^n$ : then $\overline{a}$ is a generator of $U(\Z/p^n\Z)$.

If $a$ was not a primitive root modulo $p$, $\overline{a}$ is not a generator of $U(\Z/p\Z)$, so there exists $b \in \Z, b \wedge p = 1$ such that $a^k \not \equiv b \pmod p$ for all $k\in \Z$. A fortiori $a^k \not \equiv b \pmod {p^n}$, and $b \wedge p^n = 1$, so $\overline{b} \in U(\Z/p^n\Z)$ and $\overline{b} \not \in \langle \overline{a} \rangle$ in $U(\Z/p^n\Z)$, in contradiction with the hypothesis. So $a$ is a primitive root modulo $p$.

(the reasoning on the orders of $a$, modulo $p$ and modulo $p^n$, is possible, but not so easy.)
\end{proof}

\paragraph{Ex. 4.4}

{\it 
Consider a prime $p$ of the form $4t + 1$. Show that $a$ is a primitive root modulo $p$ iff $-a$ is a primitive root modulo $p$.
}

\begin{proof} 
Solution 1.

As. $p-1$ is even, $(-a)^{p-1} = a^{p-1} \equiv 1 \pmod p$.

If $(-a)^n \equiv 1 \pmod p$, with $n \in \N$, then $a^n \equiv (-1)^n \pmod p$. 

If $n$ is odd, then $a^n \equiv -1 , a^{2n} \equiv 1 \pmod p$. As $a$ is a primitive root modulo $p$, $p-1 \mid 2n$, $2t \mid n$, so $n$ is even : this is a contradiction.

Consequently, $n$ is even, and $a^n \equiv 1 \pmod p$, so $p-1 \mid n$, so the least $n\in \N^*$ such that $a^n \equiv 1 \pmod p$ is $p-1$ : the order of $a$ modulo $p$ is $p-1$, $a$ is a primitive root modulo $p$.

Reciprocally, if $-a$ is a primitive root modulo $p$, we apply the previous  result at $-a$ to to obtain that $-(-a) = a$ is a primitive root.
\bigskip

Solution 2.

Let $p-1 = 2^{a_0}p_1^{a_1}\cdots p_k^{a_k}$ the decomposition of $p-1$ in prime factors. 

 As $p_i$ is odd for $i=1,2,\cdots k$, $(p-1)/p_i$ is even, and $a$ is primitive, so 
 \begin{align*}
& (-a)^{(p-1)/p_i} = a^{(p-1)/p_i} \not \equiv 1 \pmod p,\\
&(-a)^{(p-1)/2}  = (-a)^{2k} = a^{2k} = a^{(p-1)/2} \not \equiv 1 \pmod p.
 \end{align*}
 So the order of $a$ is $p-1$ modulo $p$ (see Ex. 4.8) : $a$ is a primitive element modulo $p$.
\end{proof}

\paragraph{Ex. 4.5}

{\it Consider a prime $p$ of the form $4t + 3$. Show that $a$ is a primitive root modulo $p$ iff $-a$ has order $(p - 1)/2$.
}

\begin{proof}
Let $a$ a primitive root modulo $p$. 

As $a^{p-1} \equiv 1 \pmod p$, $p \mid (a^{(p-1)/2} - 1)(a^{(p-1)/2} + 1)$, so $p \mid a^{(p-1)/2} - 1$ or $p \mid a^{(p-1)/2} + 1$. As $a$ is a primitive root modulo $p$, $a^{(p-1)/2} \not \equiv 1 \pmod p$, so 
$$a^{(p-1)/2} \equiv -1 \pmod p.$$

Hence $(-a)^{(p-1)/2} = (-1)^{2t+1} a^{(p-1)/2}  \equiv (-1)\times (-1) =1 \pmod p$.

Suppose that  $(-a)^n \equiv 1 \pmod p$, with $n \in \N$.


Then $a^{2n} = (-a)^{2n} \equiv 1 \pmod p$, so $p-1 \mid 2n, \frac{p-1}{2} \mid n$.

So $-a$ has order $(p-1)/2$ modulo $p$.

\bigskip

Reciprocally, suppose that $-a$ has order $(p-1)/2 = 2t+1$ modulo $p$. Let $2,p_1,\ldots p_k$ the prime factors of $p-1$, where $p_i$ are odd.

$a^{(p-1)/2} = a^{2t+1} = -(-a)^{2t+1} = -(-a)^{(p-1)/2} \equiv - 1$, so $a^{(p-1)/2} \not \equiv 1 \pmod 2$.

As $p-1$ is even, $(p-1)/p_i$ is even, so

$a^{(p-1)/p_i} = (-a)^{(p-1)/p_i} \not \equiv 1 \pmod p$ (since $-a$ has order $p-1)$.

So the order of $a$ is $p-1$ (see Ex. 4.8) : $a$ is a primitive root modulo $p$.
\end{proof}


\paragraph{Ex. 4.6}

{\it If $p = 2^{2^n} + 1$ is a Fermat prime, show that $3$ is a primitive root modulo $p$.
}

\begin{proof}

Solution 1 (with quadratic reciprocity).

Write $p = 2^k + 1$, with $k = 2^n$.

We suppose that $n>0$, so $k\geq 2, p \geq 5$. As $p$ is prime, $3^{p-1} \equiv 1 \pmod p$. 

In other words, $3^{2^k} \equiv 1 \pmod p$ : the order of $3$ is a divisor of $2^k$, a power of $2$.

$3$ has order $2^k$ modulo $p$ iff $3^{2^{k-1}} \not \equiv 1 \pmod p$. As $\left (3^{2^{k-1}} \right)^2 \equiv 1 \pmod p$, where $p$ is prime, this is equivalent to $3^{2^{k-1}}  \equiv -1 \pmod p$, which remains to prove.

$3^{2^{k-1}} = 3^{(p-1)/2} \equiv \legendre{3}{p} \pmod p$.

As the result is true for $p=5$, we can suppose $n\geq 2$.
From the law of quadratic reciprocity :
$$\legendre{3}{p} \legendre{p}{3} = (-1)^{(p-1)/2} = (-1)^{2^{k-1}} = 1.$$
So $\legendre{3}{p} = \legendre{p}{3}$
 
\begin{align*}
p = 2^{2^n}+1 &\equiv (-1)^{2^n} + 1 \pmod 3\\
&\equiv 2 \equiv -1 \pmod 3,
\end{align*}
so $\legendre{3}{p} = \legendre {p}{3} = -1$, that is to say
$$3^{2^{k-1}}  \equiv -1 \pmod p.$$
The order of $3$ modulo $p = 2^{2^n} + 1$ is $p-1 = 2^{2^n}$ : $3$ is a primitive root modulo $p$.

(On the other hand, if $3$ is of order $p-1$ modulo $p$, then $p$ is prime, so
$$ F_n = 2^{2^n} + 1 \ \mathrm{is}\ \mathrm{prime}\ \iff 3^{(F_n-1)/2} = 3^{2^{2^n - 1}} \equiv -1 \pmod {F_n}.)$$

\bigskip

Solution 2 (without quadratic reciprocity, with the hint of  chapter 4).

As above, if if we suppose that $3$ is not a primitive root modulo $p$, then $3^{2^{n-1}}  \equiv 1 \pmod p$, so  $n\geq 2$, and $(-3)^{(p-1)/2} = 3^{2^{n-1}}  \equiv 1 \pmod p$, so $-3$ is a square modulo $p$ : there exists $a \in \Z$ such that $-3 \equiv a^2 \pmod p$. 

As $2\wedge p = 1$, there exists $u \in \Z $ such that $2u \equiv -1+a \pmod p$ ($\overline{u}$ is similar to $\omega = \frac{-1 + i \sqrt{3}}{2} \in \C$). Then
\begin{align*}
8u^3 &\equiv (-1+a)^3\\
&\equiv -1+3a - 3a^2 +a^3\\
&\equiv -1 +3a + 9 -3a\\
&\equiv 8 \pmod p
\end{align*}
As $p\wedge 2 = p \wedge 8 = 1, u^3 \equiv 1 \pmod p$. Moreover, if $u \equiv 1 \pmod 3$, then $a \equiv 3 \pmod p$, $-3 \equiv 9 \pmod p, p\mid 12$, so $p = 2$ or $p=3$, in contradiction with $p\geq 5$. So the order of $u$ modulo $p$ is $3$ : $(\Z/p\Z)^*$ contains an element $\overline{u}$ of order $3$. So $3 \mid p-1$, $p\equiv 1 \pmod 3$, but $p \equiv (-1)^{2^n} + 1 \equiv 2 \equiv -1 \pmod 3$ : this is a contradiction, so $3$ is a primitive root modulo $p = 2^{2^n} +1$.
\end{proof}

\paragraph{Ex. 4.7}

{\it Suppose that $p$ is a prime of the form $8t + 3$ and that $q = (p -1)/2$ is also a prime.  Show that $2$ is a primitive root modulo $p$.
}

\begin{proof}
The first examples of such couples $(q,p)$ are $(5,11), (29,59) ,(41,83) ,(53,107) ,(89,179)$.

$p = 2q+1 = 8t+3$ and $p,q$ are prime numbers.

From Fermat's little theorem, $2^{p-1} \equiv 1 \pmod p$, so $2^{2q} \equiv 1 \pmod p$.

The order of $2$ modulo $p$ divides $2q$ : to prove that the order of $2$ is $2q = p-1$, it is suffisant to prove 
$$2^2 \not \equiv 1 \pmod p, \quad 2^q \not \equiv 1 \pmod p.$$

If $2^2 \equiv 1 \pmod p$, then $p \mid 3$, $p=3$ and $q = 1$ : $q$ is not a prime, so $2^2 \not \equiv 1 \pmod p$.

If $2^q  = 2^{(p-1)/2} \equiv 1 \pmod p$, then $2$ is a square modulo $p$ (prop. 4.2.1) : there exists $a \in \Z$ such that $2\equiv a^2 \pmod p$.


From the complementary case of law of quadratic reciprocity (see next chapter, prop. 5.1.3),  $2$ is a square modulo $p$ iff 
$$1 = \legendre{2}{p} = (-1)^{(p^2-1)/8}.$$
Yet  $p\equiv 3 \pmod 8$, so $p^2 \equiv 1 \pmod {16}$, $\legendre{2}{p} = (-1)^{(p^2-1)/8} = -1$, so $2$ is not a square modulo $p$. This is a contradiction, so $2^q\not \equiv 1 \pmod p$ : $2$ is a primitive root modulo $p$.
\end{proof}

\paragraph{Ex. 4.8}

{\it  Let $p$ be an odd prime. Show that $a$ is a primitive root modulo $p$ iff $a^{(p-1)/q} \not \equiv 1 \pmod p$ for all prime divisors $q$ of $p - 1$.
}

\begin{proof}
$\bullet$ If $a$ is a primitive root, then $a^k \not \equiv 1$ for all $k, 1\leq k < p-1$, so $a^{(p-1)/q} \not \equiv 1 \pmod p$ for all prime divisors $q$ of $p - 1$.

$\bullet$ In the other direction, suppose $a^{(p-1)/q} \not \equiv 1 \pmod p$ for all prime divisors $q$ of $p - 1$.

Let $\delta$ the order of $a$, and $p-1 = q_1^{a_1}q_2^{a_2}\cdots q_k^{a_k}$ the decomposition of $p-1$ in prime factors. As $\delta \mid p-1, \delta = q_1^{b_1}p_2^{b_2}\cdots q_k^{b_k}$, with $b_i \leq a_i, i=1,2,\ldots,k$. If $b_i < a_i$ for some index $i$, then $\delta \mid (p-1)/q_i$, so $a^{(p-1)/q_i} \equiv 1 \pmod p$, which is in contradiction with the hypothesis. Thus $b_i = a_i$ for all $i$, and $\delta = q-1$ : $a$ is a primitive root modulo $p$.
\end{proof}

\paragraph{Ex. 4.9}

{\it Show that the product of all the primitive roots modulo $p$ is congruent to $(-1)^{\phi(p-1)}$ modulo $p$.
}

\begin{proof} Here we suppose $p$ prime, $p>2$. Let $g$ a primitive root modulo $p$. $U(\Z/p\Z)$ is cyclic, generated by $\overline{g}$:
$$U(\Z/p\Z) = \{\overline{1},\overline{g}, \overline{g}^2, \ldots,\overline{g}^{p-2}\},\qquad \overline{g}^{p-1} = \overline{1}.$$
$\overline{g}^k$ is a primitive element iff $k \wedge (p-1) = 1$, so the product of primitive elements in $U(\Z/p\Z)$ is
$$\overline{P} = \prod_{\overset{k\wedge (p-1) = 1}{ 1\leq k < p-1}} \overline{g}^k.$$
so $\overline{P} = \overline{g}^S$, where $S = \sum\limits_{\overset{k\wedge (p-1) = 1}{ 1\leq k < p-1}}  k$.

From Ex. 2.22, we know that for $n\geq 2$,
$$\sum_{\overset{k\wedge n = 1}{ 1\leq k < n}}  k = \frac{1}{2} n \phi(n).$$
So $S = \sum\limits_{\overset{k\wedge (p-1) = 1}{ 1\leq k < p-1}}  k = \frac{1}{2} (p-1) \phi(p-1)$.

As $p>2$, $p-1$ is even. $(\overline{g}^{(p-1)/2})^2 = \overline{g}^{p-1} = \overline{1}$, and $\overline{g}^{(p-1)/2} \ne \overline{1}$. As $\Z/p\Z$ is a field, $\overline{g}^{(p-1)/2} = -\overline{1}$.

Thus $\overline{P} = (-\overline{1})^{\phi(p-1)}$ : so the product $P$ of all the primitive roots modulo $p$ is such that
$$P \equiv  (-1)^{\phi(p-1)} \pmod p.$$
\end{proof}

\paragraph{Ex. 4.10}

{\it Show that the sum of all the primitive roots modulo $p$ is congruent to $\mu(p-1)$ modulo $p$.
}


\begin{proof}
Notation : $\F_p=\Z/p\Z$ is the field with $p$ elements, $\vert x \vert$ the multiplicative order of an element $x \in \F_p^*$, $\N^* = \{1,2,3,\ldots\}$.

Let
$$
\psi : 
\left\{
\begin{array}{ccc}
\N^*  & \to   &  \F_p \\
 n  &\mapsto   &  \psi(n) = \sum\limits_{d \in \F_p^*,\vert d \vert =n} d   
\end{array}
\right.
$$
$\psi(n)$ is the sum of the elements  with order $n$ in $\F_p^*$. So $\psi(n) = 0$ if $n \nmid p-1$, and $S = \psi(p-1)$ is the sought sum of all the primitive roots modulo $p$.

We compute for all $n \in \N^*$
$$f(n) = \sum\limits_{d \mid n} \psi(d).$$

$f(n)$ is the sum of elements whose order divides $n$, in other worlds the sum of the roots of $x^n - 1$. This sum is, up to the sign, the coefficient of $x^{n-1}$, so is null, except in the case $n=1$, where the sum of the unique root $1$ of $x-1$ is $1$. So
$$f(1) = 1, \qquad \forall n > 1,f(n) = 0,$$
($f  = \chi_{\{1\}}$ is the characteristic function of $\{1\}$).

From the M�bius inversion formula, for all $n \in \N^*, \psi(n) = \sum_{d\mid m} \mu\left (\frac{n}{d}\right) f(d)$, so
$$\psi(p-1) =  \sum_{d\mid p-1} \mu\left (\frac{p-1}{d}\right) f(d) = \mu(p-1).$$

Conclusion : $$S = \sum\limits_{d \in \F_p^*,\vert d \vert =p-1} d = \mu(p-1) :$$ 
the sum of all the primitive roots modulo $p$ is congruent to $\mu(p-1)$ modulo $p$.
\end{proof}

\paragraph{Ex. 4.11}

{\it Prove that $1^k + 2^k + . . . + (p-1)^k \equiv 0 \pmod p$ if $p-1 \nmid k$, and $ -1 \pmod p$ if $p-1 \mid k$.
}

\begin{proof}
Let $S_k = 1^k+2^k+\cdots+(p-1)^k$.

Let $g$ a primitive root modulo $p$ : $\overline{g}$ a generator of $\F_p^*$.

As $(\overline{1},\overline{g},\overline{g}^{2}, \ldots, \overline{g}^{p-2}) $ is a permutation of $ (\overline{1},\overline{2}, \ldots,\overline{p-1})$,
\begin{align*}
\overline{S_k} &= \overline{1}^k + \overline{2}^k+\cdots+ \overline{p-1}^k\\
&= \sum_{i=0}^{p-2} \overline{g}^{ki} =
\left\{
\begin{array}{ccc}
\overline{ p-1} = -\overline{1} &  \mathrm{if} &  p-1 \mid k  \\
 \frac{ \overline{g}^{(p-1)k} -1}{ \overline{g}^k -1} = \overline{0}&  \mathrm{if}  &   p-1 \nmid k
\end{array}
\right.
\end{align*}
since $p-1 \mid k \iff \overline{g}^k = \overline{1}$.

Conclusion :
\begin{align*}
1^k+2^k+\cdots+(p-1)^k&\equiv 0 \pmod p\ \mathrm{if} \ p-1 \nmid k\\
1^k+2^k+\cdots+(p-1)^k&\equiv -1 \pmod p\ \mathrm{if} \ p-1 \mid k\\
\end{align*}
\end{proof}

\paragraph{Ex. 4.12}

{\it Use the existence of a primitive root to give another proof of Wilson's theorem$(p - 1)! \equiv -1  \pmod p$.
}

\begin{proof} As the result is trivial if $p=2$, we suppose that $p$ is an odd prime. 

Let $g$ a primitive root modulo $p$ : $\overline{g}$ a generator of $\F_p^*$.

As $(\overline{g}^{(p-1)/2}) ^2 = \overline{g}^{p-1} = \overline{1}$, and $\overline{g}^{(p-1)/2} \neq 1$ in the field $\F_p^*$,  then $\overline{g}^{(p-1)/2} = -1$, and $(\overline{1},\overline{g},\overline{g}^{2}, \ldots, \overline{g}^{p-2}) $ is a permutation of $ (\overline{1},\overline{2}, \ldots,\overline{p-1})$, so
\begin{align*}
\overline{(p-1)!} &= \prod_{k=0}^{p-2} \overline{g}^k\\
&= \overline{g}^{\sum_{k=0}^{p-2} k}\\
&=\overline{g}^{(p-2)(p-1)/2}\\
&=\left(\overline{g}^{(p-1)/2}\right)^{p-2}\\
&=(-\overline{1})^{p-2}\\
&= - 1.
\end{align*}
Hence $(p-1)! \equiv -1\pmod p$ for each prime $p$.
\end{proof}

\paragraph{Ex. 4.13}

{\it Let $G$ be a finite cyclic group and $g \in G$ a generator. Show that all the other generators are of the form $g^k$, where $(k, n) = 1$, $n$ being the order of $G$.
}

\begin{proof}
Suppose $G = \langle g \rangle$, with $\mathrm{Card}\, G = n$, so the order of $g$ is $n$.

Let $x$ another generator of $G$, then $x= g^k$, and $g = x^l, \ k,l \in \Z$, so $g = g^{kl}, g^{kl-1} = e : n \mid kl-1$, then $kl - 1 = qn, q\in \Z$, so $n\wedge k = 1$.

Reciprocally, if $u\wedge k = 1$, there exist $u,v \in \Z$ such that $un+vk=1$, so $g = g^{un+vk} = (g^n)^u(g^k)v = x^v \in \langle x \rangle$, so $G \subset \langle x \rangle$, $G = \langle x \rangle$ : $x$ is a generator of $G$.

Conclusion : if $g$ is a generator of $G$, all the other generators are the elements $g^k$, where $k \wedge n  = 1$, $n = \vert G \vert$.
\end{proof}

\paragraph{Ex. 4.14}

{\it Let $A$ be a finite abelian group and $a, b \in A$ elements of order $m$ and $n$, respectively.  If $(m, n) = 1$, prove that $ab$ has order $mn$.
}

\begin{proof}
Suppose $\vert a \vert = m, \vert b \vert = n, m\wedge n = 1$.

$\bullet$ If $(ab)^k = e$, then $a^k= b^{-k}$, so $a^{kn} = b^{-kn} = (b^n)^{-k} = e$, so $m\mid kn$, with $m\wedge n = 1$, so $m \wedge k$.

Similarly, $ b^{km} =a^{-km} = (a^m)^{-k} = e$, so $n \mid km, n \wedge m = 1$ : $n \mid k$. 

As $n \mid k, m \mid k, n \wedge m = 1, nm \mid k$.

$\bullet$ Reciprocally, if $nm \mid k, nm = qnm, q\in \Z$, so $(ab)^k = a^k b^k = (a^m)^{qn} (b^n)^{qm} = e$.
$$\forall k \in \Z,\ (ab)^k = e \iff nm \mid k.$$
So $\vert ab \vert = nm$.
\end{proof}

\paragraph{Ex. 4.15}

{\it Let $K$ be a field and $G \subset K^*$ a finite subgroup of the multiplicative group of $K$.  Extend the arguments used in the proof of Theorem 4.1 to show that $G$ is cyclic.
}

\bigskip

 {\bf Solution 1.}

\begin{proof}
Let $n = \vert G \vert$. From Lagrange's theorem, $a^n = 1$ for all $a \in G$, so the polynomial $x^n-1\in K[x] $ has exactly $n$  roots in $G$, and so
$$\forall x \in K, x \in G \iff x^n=1.$$

If $d \mid n$, the polynomial $x^d-1 \in K[x]$ has exactly $d$ roots in $K$ otherwise $x^n-1 = (x^d-1) g(x), g(x) \in K[x]$, and $\deg(g) = n-d$ has at most $n-d$ roots, so $x^n - 1$ would have less than $n$ roots in $K$. As $x_0^d = 1 \Rightarrow x_0^n = 1$, all these roots are in $G$ : $x^d - 1$ has $d$ roots in $G$.

Let $\psi(d)$ the number of elements in $G$ of order $d$ ( $\psi(d) = 0$ if $d \nmid n$). Then $\sum_{c\mid d} \psi(c) = d$. Applying the M�bius inversion theorem, $\psi(d) = \sum_{c\mid d} \mu(c) d/c=\Phi(d)$ (Prop. 2.2.5), in particular, $\psi(n) = \phi(n)>1$ if $n>2$. Since a group of order 2 is cyclic, we have shown in all cases the existence of an element of order $n$ in $G$, so $G$ is cyclic.

(variation : $\psi(d) = 0$ if there exists no element of order $d$, and $\psi(d) = \phi(d)$ otherwise : see Ex.4.13. So $\psi(d) \leq \phi(d)$ for all $d\mid n$. As $\sum_{d\mid n} \psi(d) = \sum_{d\mid n} \phi(d) = n$, $\psi(d) = \phi(d)$ for all $d\mid n$. So there exists in $G$ an element of order $n$, and $G$ is cyclic.)
\end{proof}

\bigskip

{\bf Solution 2.}
\begin{proof}
Let $n = \vert G \vert = p_1^{a_1}\cdots p_k^{a_k}$. From Lagrange's theorem, $y^n = 1$ for all $y \in G$.

$p(x) = x^{n/p_1} - 1 \in K[x]$ has at most $n/p_1 < n$ roots in $K^*$, a fortiori in $G$, so there exists $a \in G$ such that $a^{n/p_1} \neq 1$.

Let $c_1 = a^{n/p_1^{a_1}} = a^{p_2^{a_2}\cdots p_k^{a_k}}$. Then $c_1^{p_1^{a_1}} = 1$ and $c_1^{p_1^{a_1-1}} = a^{n/p_1} \neq 1$, so $\vert c_1 \vert = p_1^{a_1} $.

Similarly, there exist $c_2,\ldots,c_k$ with respective orders $\vert c_i \vert = p_i^{a_i}$.

From exercise 4.14, we obtain by induction that $c =c_1\cdots c_k$ has order $ p_1^{a_1}\cdots p_k^{a_k} = n$, so $G$ is cyclic.
\end{proof}

\paragraph{Ex. 4.16}

{\it Calculate the solutions to $x^3 \equiv 1 \pmod {19}$ and $x^4 \equiv 1\pmod {17}$.
}

\begin{proof}
Here we note $a$ the class of $a$ in $\Z/p\Z$.

Let $x \in \F_{19}$.
$x^3 - 1 = 0 \iff x-1 = 0$ or $x^2+x+1=0$.
\begin{align*}
x^2+x+1 = 0 &\iff (x + 10) - 99 = 0\\
&\iff (x+10)^2 - 4 = 0\\
&\iff (x+8)(x+12) = 0
\end{align*}
So, for all $x\in \Z$,
$$x^3 \equiv 1 \pmod {19} \iff x \equiv 1,7,11 \pmod {19}.$$

Let $x \in \F_{17}$.
\begin{align*}
x^4 = 1 &\iff x^2 = 1 \ \mathrm{or}\ x^2 = -1 = 4^2\\
&\iff x = \pm 1 \ \mathrm{or}\  x =\pm 4
\end{align*}
So, for all $x\in \Z$,
$$x^4 \equiv 1 \pmod {17} \iff x\equiv -1,1,-4,4 \pmod {17}.$$

Alternatively, we can take primitives roots modulo 19 and 17.

$2$ is a primitive root modulo $19$, Let $x = 2^k \in \F_{19}$.
\begin{align*}
x^3 = 1 &\iff 2^{3k} = 1\\
&\iff 18 \mid 3k\\
&\iff 6 \mid k\\
&\iff x = 1, 2^6 = 7, 2^{12} = 11
\end{align*}
$3$ is a primitive root modulo 17. Let $x = 3^k \in \F_{17}$.
\begin{align*}
x^4 = 1 &\iff 3^{4k} = 1\\
&\iff 16 \mid 4 k\\
&\iff 4 \mid k\\
&\iff x = 1, 3^4 = -4, 3^8 = -1, 3^{12} = 4
\end{align*}
\end{proof}

\paragraph{Ex. 4.17}

{\it 
Use the fact that $2$ is a primitive root modulo $29$ to find the seven solutions to $x^7 \equiv 1 \pmod {29}$.
}

\begin{proof}
Let $x \in \Z$, then $x \equiv 2^k \pmod{29} , k \in \N$.
\begin{align*}
x^7 \equiv 1 \pmod{29} &\iff 2^{7k} \equiv 1 \pmod {29}\\
&\iff 28 \mid 7k\\
&\iff 4 \mid k
\end{align*}
So the group cyclic  $S$ of the roots of $x^7 - 1$ in $\F_{29}$ are
$$S = \{1,2^4,2^8,2^{12},2^{16},2^{20},2^{24}\},$$
$$S = \{1,16,24,7,25,23,20\}.$$
\end{proof}

\paragraph{Ex. 4.18}

{\it Solve the congruence $1 + x + \cdots + x^6 \equiv 0 \pmod {29}$.
}

\begin{proof} As $(1 + x + \cdots + x^6)(1-x) = 1-x^7$,
$$1 + x + \cdots + x^6 \equiv 0 \pmod {29} \iff
\left\{
\begin{array}{ccc}
  x^7&  \equiv & 1 \pmod{29}  \\
  x &  \not \equiv &1 \pmod {29}      
\end{array}
\right.
$$
From Ex. 4.17, the solutions are congruent to $2^4,2^8,2^{12},2^{16},2^{20},2^{24}$ modulo $29$.
\end{proof}

\paragraph{Ex. 4.19}

{\it  Determine the numbers $a$ such that $x^3 \equiv a \pmod {p}$ is solvable for $p = 7, 11, 13$.
}

\begin{proof}
\begin{enumerate}
\item[(a)] If $p=7$, then $3 \mid p-1, d = 3\wedge (p-1) = 3$. From Prop. 4.2.1,
$$\exists x \in \Z, \ a \equiv x^3 \pmod 7 \iff a \equiv 0 \pmod 7 \ \mathrm{or}\ a^{(p-1)/3} = a^2 \equiv 1 \pmod  7.$$
So the numbers $a$ such that $x^3 \equiv a \pmod {7}$ is solvable are congruent at $0,1,-1$ modulo $7$.
\item[(b)]  If $p = 11$, then $d = 3 \wedge (p-1) = 1$. With the same proposition, 
$$\exists x \in \Z, \ a \equiv x^3 \pmod {11} \iff a \equiv 0 \pmod {11} \ \mathrm{or}\ a^{p-1} = a^6 \equiv 1 \pmod  {11}.$$
So all integers $a$ are cube modulo $11$, in only one way.

For an alternative proof, the application 
$$f : 
\left\{
\begin{array}{ccc}
  \F_{11}^*& \to   &\F_{11}^*  \\
  x & \mapsto   &   x^3
\end{array}
\right.
$$
$f$ is a bijection. Indeed,

$\bullet$ $f$ is a group homomorphism, 

$\bullet$ $x^3 = 1 \Rightarrow (x^3)^7 = 1 \Rightarrow x=1$ so $\ker(f) = \{1\}$, 

$\bullet$ $f:\F_{11}^*\to \F_{11}^*$ is injective and $\F_{11}^*$ is finite, so $f$ is bijective.

In $\F_{11}$, $0=0^3,1=1^3,2=7^3,3=9^3,4=5^3,5=3^3,6=8^3,7=6^3,8=2^3,9=4^3,10 = 10^3$.
\item[(c)] If $p=13$, then $3 \mid p-1, 3 \wedge (p-1) = 3$, so
\begin{align*}
\exists x \in \Z, \ a \equiv x^3 \pmod {13} & \iff a \equiv 0 \pmod {13} \ \mathrm{or}\ a^{(p-1)/3} = a^4 \equiv 1 \pmod  {13}\\
&\iff a \equiv 0,1,-1,5,-5 \pmod {13}
\end{align*}
($5 \equiv 8^3 \pmod {13}$.)
\end{enumerate}
\end{proof}

\paragraph{Ex. 4.20}

{\it Let $p$ be a prime, and $d$ a divisor of $p - 1$. Show that $d$th powers form a subgroup of $U(\Z/p\Z)$ of order $(p-1)/d$. Calculate this subgroup for $p = 11, d = 5$, for $p = 17, d = 4$, and for $p = 19, d = 6$.
}

\begin{proof}
Here $p$ is a prime number, and $d \mid p-1$. Let
$$f : 
\left\{
\begin{array}{ccc}
  \F_p^*&  \to  &\F_p^*   \\
  x&  \to  &   x^d
\end{array}
\right.
$$
Then $f$ is a group homomorphism, and $\mathrm{im}(f)$ is the set of $d$th powers, and consequently is a subgroup of $U(\F_p) = \F_p^*$.
$\ker(f)$ is the group of the roots of $x^d-1$. As $d \mid p-1$, the polynomial $x^d-1$ has exactly $d$ roots (Prop. 4.1.2), so $\vert \ker(f) \vert = d$. 

As $\mathrm{im}(f) \simeq \F_p^* /\ker(f)$,
$$\vert \mathrm{im}(f) \vert = \vert \F_p^*  \vert / \vert \ker(f) \vert = (p-1)/d.$$
So there exist exactly $(p-1)/d$ $d$th powers in $(\Z/p\Z)^*$.

\bigskip

From Prop. 4.2.1, as $d \mid p-1, d \wedge p-1$, for all $x \in \F_p^*$, 

$$x \in \mathrm{im}(f) \iff x^{(p-1)/d} = 1.$$

So the group of $d$th powers is the group of the roots of $x^{(p-1)/d} - 1$.

$\bullet$ If $p=11,d=5$, $\mathrm{im}(f) = \{1,-1\}$.

$\bullet$ If $p = 17, d= 4$, $x \in \mathrm{im}(f) \iff x^4 = 1$ : $\mathrm{im}(f) = \{1,-1,4,-4\}$.

$\bullet$ If $p=19,d=6$, $x \in \mathrm{im}(f) \iff x^3 = 1$ : $\mathrm{im}(f) = \{1,7,7^2 = 11\}$, 

where $7 \equiv 2^6 \pmod {19}$.
\end{proof}

\paragraph{Ex. 4.21}

{\it If $g$ is a primitive root modulo $p$, and $d|p-1$, show that $g^{(p-1)/d}$ has order $d$. Show also that $a$ is a $d$th power iff $a \equiv g^{kd} \pmod p$ for some $k$. Do Exercises 16-20 making use of those observations.
}

\begin{proof}
Let $x = \overline{g}^{(p-1)/d} \in \F_p^*$, where $g$ is a primitive root modulo $p$. For all $k \in \Z$, 
\begin{align*}
x^k = 1 &\iff g^{k \frac{p-1}{d}} = 1\\
&\iff p-1 \mid k \frac{p-1}{d}\\
&\iff d \mid k
\end{align*}
So the ordre of $\overline{g}^{(p-1)/d}$ is $d$.

$\bullet$ If $\overline{a} = \overline{g}^{kd}$, then $\overline{a} = x^d$, where $x =\overline{g}^k$, so $\overline{a}$ is a $d$th power.

$\bullet$ If $\overline{a}\neq \overline{0}$ is a $d$th power, $\overline{a} = x^d, x \in \F_p^*$. As $x \in \langle \overline{g} \rangle$, $x = \overline{g}^k$, so $\overline{a} = \overline{g}^{kd}$.

So, if $a\not \equiv 0 \pmod p$, $a$ is a $d$th power iff $a \equiv g^{kd} \pmod p$ for some $k$.

By example (Ex. 4.20), $2$ is a primitive root modulo $19$, so the 6th powers modulo 19 are $2^0 = 1,2^6 = 7,2^{12} = 11$.
\end{proof}

\paragraph{Ex. 4.22}

{\it If $a$ has order $3$ modulo $p$, show that $1+a$ has order $6$.
}
\begin{proof}
If $a$ has order $3$ modulo $p$, then $0 \equiv a^3-1 = (a-1)(a^2+a+1) \pmod p$, with $a\not \equiv 1 \pmod p$, so $a^2+a+1 \equiv 0 \pmod p$. Thus
\begin{align*}
(1+a)^3 &\equiv 1 + 3a +3a^2+a^3\\
&\equiv 1 + 3a +3(-1-a)+1\\
&\equiv -1 \pmod p
\end{align*}
So $(1+a)^6 \equiv 1 \pmod p$.

$(1+a)^2 \equiv 1+2a+a^2 =1+2a+(-1-a) \equiv a \not \equiv 1 \pmod p$. 

So $(1+a)^6 \equiv 1,  (1+a)^2 \not \equiv 1, (1+a)^3 \not \equiv 1 \pmod p$, so the order of $1+a$ divides $6$, but doesn't divides $2$ or $3$, so $1+a$ has order $6$ modulo $p$.
\end{proof}

\paragraph{Ex. 4.23}

{\it Show that $x^2 \equiv -1 \pmod p$ has a solution iff $p \equiv 1 \pmod 4$, and that $x^4 \equiv -1 \pmod p$ has a solution iff $p \equiv 1 \pmod 8$.
}

\begin{proof}
If $x^2\equiv -1 \pmod p$, then  $\overline{x}$ has order 4 in  $\F_p^*$, hence from Lagrange's theorem, $4 \mid p-1$.

Reciprocally, suppose $4 \mid p-1$, so $p = 4k+1, k \in \N^*$.  From proposition 4.2.1, as $2 \mid p-1$, $-1$ is a square modulo $p$ iff $(-1)^{(p-1)/2} \equiv 1 \pmod p$, which is true because $(-1) ^{(p-1)/2} =(-1)^{2k} = 1$.

If $x^4 \equiv -1 \pmod p$, then $\overline{x}^8 = 1 \in \F_p^*$, and $\overline{x}^4 \ne 1$, so $x$ has order $8$ in $\F_p^*$, so $8 \mid p-1$.

Reciprocally, if $p\equiv 1 \pmod 8$, $p = 8K+1, K\in \N^*$. From Prop.4.2.1, as $4 \mid p-1$, there exists $x \in \Z$ such that $-1 = x^4$ iff $(-1)^{(p-1)/4} \equiv 1 \pmod 8$, which is true because $(-1)^{(p-1)/4} = (-1)^{2K} = 1$.

Conclusion : 
$$\exists x \in \Z, \ x^4 \equiv -1 \pmod p \iff p \equiv 1 \pmod 8.$$
\end{proof}

\paragraph{Ex. 4.24}

{\it Show that $a x^m + b y^n \equiv c \pmod p$ has the same number of solutions as $a x^{m'} + b y^{n'} \equiv c \pmod p$, where $m' = (m,p-1)$ and $n' = (n, p-1)$.
}

\begin{proof}
If $a\wedge b \nmid c$, the two equations have no solution. So we can suppose $a\wedge b \mid c$, and after division by $\delta = a\wedge b$, we obtain an equation $a'x^m+b'y^n = c'$, $a' = a/\delta,b' = b\delta,c'=c\delta$, and $a' \wedge b' = 1$. So it remains to prove that $a x^m + b y^n \equiv c \pmod p$ has the same number of solutions as $a x^{m'} + b y^{n'} \equiv c \pmod p$ when $a\wedge b = 1$.

In this case the equation $au + bv=c$ has solutions. Let $N$ the number of solutions $(\overline{x},\overline{y})$  of the equation $\overline{a}\, \overline{x}^m + \overline{b}\,  \overline{y}^n = \overline{c}$,$N'$ the number of solutions $(\overline{x},\overline{y})$  of the equation $\overline{a}\, \overline{x}^{m'} + \overline{b}\,  \overline{y}^{n' }= \overline{c}$. Then 
\begin{align*}
N &= \mathrm{Card} \{(\overline{x},\overline{y}) \in \F_p\times \F_p\ \vert\  \overline{a}\, \overline{x}^m + \overline{b}\,  \overline{y}^n = \overline{c}\}\\
&=\sum_{\overline{a}\overline{u}+\overline{b} \overline{v}=\overline{c}} \mathrm{Card} \{(\overline{x},\overline{y}) \in \F_p\times \F_p\ \vert\  \overline{x}^m = \overline{u} ,\overline{y}^n = \overline{v}\}\\
&= \sum_{\overline{a}\overline{u}+\overline{b} \overline{v}=\overline{c}} \mathrm{Card}\{\overline{x} \in \F_p\ \vert \overline{x}^m = \overline{u}\} \times \mathrm{Card}\{\overline{y} \in \F_p\ \vert \overline{y}^n = \overline{v}\}.
\end{align*}
The same is true for $N'$, so it is suffisant to prove that
$$\mathrm{Card}\{\overline{x} \in \F_p\ \vert\  \overline{x}^m  = \overline{u}\} = \mathrm{Card}\{\overline{x} \in \F_p\ \vert \ \overline{x}^{m'} = \overline{u}\},$$ 
where $m' = m \wedge (p-1)$, and a similar equality for the equation $\overline{y}^n = \overline{v}$.

Let $\overline{g}$ a generator of $\F_p^*$. Write $\overline{u} = \overline{g}^r , r \in \N$.
\begin{align*}
\exists \overline{x} \in \F_p,\ \overline{x}^m = \overline{u} &\iff \exists k \in \Z, \ \overline{g}^{mk} = \overline{g}^r\\
&\iff \exists k \in \Z,\ p-1 \mid mk -r\\
&\iff \exists k \in \Z, \exists l \in \Z,\ r = mk + l (p-1)\\
&\iff m \wedge (p-1) \mid r
\end{align*}
So $$\{\overline{x} \in \F_p\ \vert\  \overline{x}^m  = \overline{u}\} \ne \emptyset \iff m\wedge (p-1) \mid r,$$ and similarly $$\{\overline{x} \in \F_p\ \vert\  \overline{x}^{m'}  = \overline{u}\} \ne \emptyset \iff m' \wedge (p-1) \mid r.$$ 
Since $m' \wedge (p-1) = (m\wedge (p-1)) \wedge (p-1) = m \wedge (p-1)$, these two conditions are equivalent, so these two sets are empty for the same values of $\overline{u}$.

Let $\overline{u}$ is such that $\{\overline{x} \in \F_p\ \vert\  \overline{x}^m  = \overline{u}\} \ne \emptyset$, and $x_0$ a fixed solution of $\overline{x}^m  = \overline{u}$.

Write $\overline{x} = \overline{g}^k, \overline{x_0} = g^{k_0}$. Let $d = m\wedge (p-1) (=m')$.
\begin{align*}
\overline{x}^m = u &\iff \overline{x}^m = \overline{x_0}^m\\
&\iff \overline{g}^{mk} = \overline{g}^{mk_0}\\
&\iff  p-1 \mid m(k -k_0)\\
&\iff \frac{p-1}{d} \mid \frac{m}{d} (k-k_0)\\
&\iff \frac{p-1}{d} \mid k-k_0\\
&\iff \exists j \in \Z, k = k_0 + j \frac{p-1}{d} 
\end{align*}
As $g$ is a primitive root modulo $p$, the distinct solutions are $x_0, x_0 g^{\frac{p-1}{d} },\ldots, x_0g^{k\frac{p-1}{d}}, \ldots x_0g^{ (d-1)\frac {p-1}{d}}$, so in this case
$$\mathrm{Card}\{\overline{x} \in \F_p\ \vert\  \overline{x}^m  = \overline{u}\} = d = m\wedge(p-1).$$
As $m'\wedge (p-1) = m\wedge (p-1)$,
$$\mathrm{Card}\{\overline{x} \in \F_p\ \vert\  \overline{x}^m  = \overline{u}\} = \mathrm{Card}\{\overline{x} \in \F_p\ \vert \ \overline{x}^{m'} = \overline{u}\}.$$ 
So $N=N'$ : $a x^m + b y^n \equiv c \pmod p$ has the same number of solutions as $a x^{m'} + b y^{n'} \equiv c \pmod p$, where $m' = (m,p-1)$ and $n' = (n, p-1)$.
\end{proof}

\paragraph{Ex. 4.25}

{\it Prove Propositions 4.2.2 and 4.2.4.
}

\medskip 

{\bf Proposition 4.2.2.} {\it Suppose that $a$ is odd, $e\geq 3$, and consider the congruence $x^n \equiv a \pmod {2^e}$. If $n$ is odd, a solution always exists and it is unique.

If $n$ is even, a solution exists iff $a \equiv 1 \pmod 4, a^{2^{e-2}/d} \equiv 1 \pmod{2^e}$, where $d = (n,2^{e-2})$. When a solution exists there are exactly $2d$ solutions.}
\begin{proof}
We suppose that $a$ is odd and $e \geq 3$.

 From Theorem 2', we know that $\{(-1)^a5^b\ \vert \ 0\leq a \leq 1, 0 \leq b \leq 2^{e-2}\}$ constitutes a reduced residue system modulo $2^e$, so we can write 
\begin{align*}
a &\equiv (-1)^s 5^t \pmod {2^e}, 0 \leq s \leq 1, 0 \leq t \leq 2^{e-2},\\
x &\equiv (-1)^y 5^z \pmod {2^e},0 \leq y \leq 1, 0 \leq z \leq 2^{e-2}.
\end{align*}
For all $x \in \Z$,
\begin{align*}
x^n \equiv a\pmod {2^e} &\iff (-1)^{ny} 5^{nz} \equiv (-1)^s 5^t \pmod {2^e}\\
\end{align*}
Then  $(-1)^{ny} \equiv (-1)^s \pmod4, ny\equiv s \pmod 2, (-1)^{ny} = (-1)^s$, so $5^{nz} \equiv 5^t \pmod{2^e}$.

Reciprocally, if $ny \equiv s \pmod {2}$ and $5^{nz} \equiv 5^t \pmod{2^e}$, then $x^n\equiv a \pmod{2^e}$, so
$$
x^n\equiv a \pmod{2^e} \iff 
\left\{
\begin{array}{ccl}
  ny& \equiv   &s \pmod{2}   \\
  5^{nz}& \equiv  &    5^t \pmod {2^{e}}
\end{array}
\right.
\iff 
\left\{
\begin{array}{ccl}
  ny& \equiv   &s \pmod{2}   \\
  nz& \equiv  &    t\pmod {2^{e-2}}
\end{array}
\right.
$$
since the order of 5 modulo $2^e$ is $2^{e-2}$.

$\bullet$ Suppose that $n$ is an odd integer. Then
$$
\left\{
\begin{array}{ccl}
  ny& \equiv   &s \pmod{2}   \\
  nz& \equiv  &    t\pmod {2^{e-2}}
\end{array}
\right.
\iff
\left\{
\begin{array}{ccl}
  y& \equiv   &s \pmod{2}   \\
  z& \equiv  &   n' t\pmod {2^{e-2}}
\end{array}
\right.
$$
where $n'$ is an inverse of $n$ modulo $2^{e-2}$ : $nn'\equiv 1 \pmod {2^{e-2}}$.

So $x^n\equiv a \pmod{2^e}$ has an unique solution modulo $2^e$.

$\bullet$ Suppose that $n$ is an even integer. 

Then $\left\{
\begin{array}{ccl}
  ny& \equiv   &s \pmod{2}   \\
  nz& \equiv  &    t\pmod {2^{e-2}}
\end{array}
\right.
$ implies $s \equiv 0 \pmod 2$ and $d = n\wedge2^{e-2} \mid t$.

Then $a \equiv (-1)^s 5^t \equiv 5^t  \pmod {2^{e}}$, so $a\equiv 1 \pmod 4$.

 Hence $a^{\frac{2^{e-2}}{d}} \equiv \left(5^{2^{e-2}}\right)^{\frac{t}{d}}\equiv 1 \pmod {2^e}$, since $5$ has order $2^{e-2}$, and $d\mid t$. 
 
 So, if $n$ is even, and $d =  n\wedge2^{e-2}$, 
 $$\exists x \in \Z,\ x^n\equiv a \pmod{2^e} \Rightarrow 
 \left\{
 \begin{array}{ccl}
  a& \equiv   &1 \pmod{4}   \\
  a^{\frac{2^{e-2}}{d}} & \equiv  &    1 \pmod {2^{e}}
\end{array}
\right.
$$
Reciprocally, suppose that $\left\{
 \begin{array}{ccl}
  a& \equiv   &1 \pmod{4}   \\
  a^{\frac{2^{e-2}}{d}} & \equiv  &    1 \pmod {2^{e}}
\end{array}
\right.$.
Then $a\equiv(-1)^s 5^t \pmod {2^e} $ implies $ a \equiv (-1)^s \pmod 4$, so $s$ is even, and $a \equiv 5^t \pmod{2^e}$.

Therefore $5^{t \frac{2^{e-2}}{d}} \equiv 1 \pmod {2^e}$, which  implies $2^{e-2} \mid t \frac{2^{e-2}}{d}$, so $d \mid t$.
\begin{align*}
\exists x \in \Z,\ x^n\equiv a \pmod{2^e} &\iff \exists y \in \Z,\ \exists z \in \Z,\ \left\{
\begin{array}{ccl}
  ny& \equiv   &s \pmod{2}   \\
  nz& \equiv  &    t\pmod {2^{e-2}}
\end{array}
\right.\\
&\iff \exists z \in \Z,\ nz \equiv t \pmod{2^{e-2}}\qquad (\mathrm{since}\ n,s\ \mathrm{even})\\
&\iff \exists z \in \Z,\ 2^{e-2} \mid nz - t\\
&\iff \exists z \in \Z,\ \frac{2^{e-2}}{d} \mid \frac{n}{d} z - \frac{t}{d}\\
&\iff \exists z \in \Z,\ \exists q \in \Z, \ q \frac{2^{e-2}}{d} + z \frac{n}{d}   = \frac{t}{d}
\end{align*}
As $\frac{2^{e-2}}{d} \wedge \frac{n}{d} = 1$, there exists a solution $(q,z_0)$ of this last equation, where $0\leq z_0 < \frac{2^{e-2}}{d}$, and so $x_0 = 5^{z_0}$ is a particular solution of $x^n\equiv a \pmod{2^e}$, therefore
 $$\exists x \in \Z,\ x^n\equiv a \pmod{2^e} \iff
 \left\{
 \begin{array}{ccl}
  a& \equiv   &1 \pmod{4}   \\
  a^{\frac{2^{e-2}}{d}} & \equiv  &    1 \pmod {2^{e}}
\end{array}
\right.
$$
If there exists a particular solution $x_0 \equiv (-1)^{y_0}5^{z_0}$, then
\begin{align*}
x^n \equiv a \pmod{2^e} &\iff x^n\equiv x_0^n \pmod {2^e}\\
&\iff
\left\{
\begin{array}{ccl}
  ny& \equiv   &ny_0 \pmod{2}   \\
  nz& \equiv  & nz_0   \pmod {2^{e-2}}
\end{array}
\right.\\
&\iff n(z -z_0)\equiv   0   \pmod {2^{e-2}} \qquad(\mathrm{since}\ n\  \mathrm{even})\\
&\iff \frac{2^{e-2}}{d} \mid \frac{n}{d} (z-z_0)\\
&\iff \frac{2^{e-2}}{d} \mid z-z_0,\qquad (\mathrm{since} \ \frac{2^{e-2}}{d} \wedge \frac{n}{d} = 1)\\
&\iff \exists k \in \Z, \ z = z_0+ k \frac{2^{e-2}}{d}
\end{align*}
As the order of $5$ modulo $2^e$ is $2^{e-2}$, the solutions of $x^n \equiv a \pmod{2^e}$ are $$x_k = (-1)^y 5^{z_0+ k \frac{2^{e-2}}{d}},\  0\leq y < 2,\  0\leq k < d,$$ so there are exactly $2d$ solutions modulo $2^e$.
\end{proof}

\bigskip
{\bf Proposition 4.2.4.} 
{\it Let $2^l$ be the highest power of $2$ dividing $n$. Suppose that $a$ is odd and that $x^n \equiv a \pmod {2^{2l+1}}$ is solvable. Then $x^n \equiv a \pmod{2^e}$ is solvable for all $e\geq 2l+1$, and consequently for all $e\geq 1$). Moreover, all these congruences have the same number of solutions.
}
\begin{proof}
We suppose that $a$ is odd, and that $x^n\equiv a \pmod {2^{2l+1}}$ is solvable. $l$ is such that $n = 2^l n'$, where $n'$ is an odd integer.

Let the induction hypothesis  be, for a fixed integer $m\geq 2l+1$,
$$\exists x_0 \in \Z,\  x_0^n\equiv a \pmod{2^m}.$$
Let $x_1 = x_0 + b 2^{m-l}$ : we show that for an appropriate choice of $b\in \{0,1\}$,  $x_1^n \equiv a \pmod {2^{m+1}}$.

$x_1^n = x_0^n + nb2^{m-l}x_0^{n-1} + 2^{2m-2l} A,\  A \in \Z$.

Since $m\geq 2l+1, 2m-2l \geq m+1$, so
$$x_1^n \equiv x_0^n + nb2^{m-l}x_0^{n-1} \pmod {2^{m+1}}.$$
\begin{align*}
x_1^n \equiv a \pmod {2^{m+1}} &\iff (x_0^n -a) + n' b x_0^{n-1} 2^m \equiv 0 \pmod {2^{n+1}}\\
&\iff\frac{x_0^n - a}{2^m} + n' b x_0^{n-1} \equiv 0 \pmod 2
\end{align*}
As $a$ is odd, and  $x_0^n \equiv a \pmod {2^m}, m\geq 1$, $x_0$ is odd, and $n'$ is odd, so there exists an unique $b \in \{0,1\}$ such that $\frac{x_0^n - a}{2^m} + n' b x_0^{n-1} \equiv 0 \pmod 2$. So there exists $x_1 \in \Z$ such that $x_1^b \equiv a \pmod {2^{m+1}}$, and the induction is completed. Therefore, $x^n \equiv a \pmod{2^e}$ is solvable for all $e\geq 2l+1$, and consequently for all $e\geq 1$). 

\bigskip

From the Proposition 4.2.2., with the hypothesis $e\geq 3$,  we know that the number of solutions of the solvable equation $x^n \equiv a \pmod {2^e}, e\geq 2l+1$,  is $1$ if $n$ is odd, $2(n\wedge 2^{e-2})$ if $n$ is even.

If $n$ is even, $l\geq 1$, $e \geq 2l+1 \geq 3$.
Since $e\geq 2l+1$, and $n = 2^l n'$ for an odd $n'$, $l\leq \frac{e-1}{2} \leq e-2$, so $n \wedge 2^{e-2} = n'2^{l} \wedge 2^{e-2}= 2^l$, and the number of solutions is $2^{l+1}$, independent of $e\geq 2l+1$.

Conclusion : under the hypothesis $x^n \equiv a \pmod {2^{2l+1}}$, where $l = \mathrm{ord}_2(n)$, then $x^n \equiv a \pmod {2^e}$ is solvable for all $e\geq 1$, and all these congruences have the same number of solutions for $e\geq 2l+1, e\geq 3$.
\end{proof}

\bigskip

{ \Large \bf Chapter 5}

\paragraph{Ex. 5.1}

{\it  Use Gauss' lemma to determine $\legendre{5}{7}, \legendre{3}{11}, \legendre{6}{13},\legendre{-1}{p}$.
}

\begin{proof}
$\bullet$ $a = 5, p=7$.

The array of values of the least residues modulo $p=7$, for $1\leq k \leq (p-1)/2$.
$$
\begin{array}{rcccc}
  k \mod 7& \vline & 1 & 2 & 3\\
 \hline
    5k \mod 7& \vline & -2  & 3 & 1
\end{array}
$$
So  the number of negative least residues is $\mu = 1$, and $\legendre{5}{7} =(-1)^{\mu} =  -1$.

$\bullet$ $a=3,p=11$.
$$
\begin{array}{rcccccc}
  k \mod 11& \vline & 1 & 2 & 3 &4 &5 \\
 \hline
    3k \mod 11& \vline & 3  & -5 & -2 & 1&  4
\end{array}
$$
So $\mu = 2$, $\legendre{3}{11} = (-1)^\mu = 1$.

$\bullet$ $a=6,p=13$.
$$
\begin{array}{rccccccc}
  k \mod 13& \vline & 1 & 2 & 3 &4 &5 &6\\
 \hline
    6k \mod 13& \vline & 6  & -1& 5 & -2 &  4& -3
\end{array}
$$
So $\mu = 3$, $\legendre{6}{13} = (-1)^\mu = -1$.

$\bullet$ If $a=-1$, and $p$ an odd prime, the values of the least residues of $-k$ modulo $p$ for $k=1,2,\ldots,(p-1)/2$ are $-k$, all negative. So  the number of negative least residues is $\mu = (p-1)/2$, and 
$\legendre{-1}{p} = (-1)^{(p-1)/2}$.
\end{proof}

\paragraph{Ex. 5.2}

{\it  Show that the number of solutions to $x^2 \equiv a \pmod p$ is equal to $1 + (a/p)$.
}

\begin{proof} Let $N$ the number of solutions of $x^2\equiv a \pmod p$, where $p$ is a prime number.

$\bullet$ If $\legendre{a}{p} = 0$, then $p\mid a$, $a\equiv 0 \pmod p$, so the unique solution of $x^2 \equiv a = 0$ is $x\equiv 0 \pmod p$, so $N = 1 = 1 + \legendre{a}{p}$.

$\bullet$ If $\legendre{a}{p} = -1$, then $N = 0 = 1 + \legendre{a}{p}$.

$\bullet$ If $\legendre{a}{p} = 1$, then $x^2\equiv a \pmod p$ has a solution $x_0$, and $x^2 \equiv a \pmod p \iff x^2 \equiv x_0^2 \pmod p\iff p \mid (x-x_0)(x+x_0) \iff x\equiv \pm x_0 \pmod p$, so $N = 2 = 1 + \legendre{a}{p}$.
\end{proof}

\end{document}