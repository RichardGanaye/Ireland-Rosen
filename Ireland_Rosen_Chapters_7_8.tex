%&LaTeX
\documentclass[11pt,a4paper]{article}
\usepackage[frenchb,english]{babel}
\usepackage[applemac]{inputenc}
\usepackage[OT1]{fontenc}
\usepackage[]{graphicx}
\usepackage{amsmath}
\usepackage{amsfonts}
\usepackage{amsthm}
\usepackage{amssymb}
\usepackage{yfonts}
%\input{8bitdefs}

% marges
\topmargin 10pt
\headsep 10pt
\headheight 10pt
\marginparwidth 30pt
\oddsidemargin 40pt
\evensidemargin 40pt
\footskip 30pt
\textheight 670pt
\textwidth 420pt

\def\imp{\Rightarrow}
\def\gcro{\mbox{[\hspace{-.15em}[}}% intervalles d'entiers 
\def\dcro{\mbox{]\hspace{-.15em}]}}

\newcommand{\D}{\mathrm{d}}
\newcommand{\Q}{\mathbb{Q}}
\newcommand{\Z}{\mathbb{Z}}
\newcommand{\N}{\mathbb{N}}
\newcommand{\R}{\mathbb{R}}
\newcommand{\C}{\mathbb{C}}
\newcommand{\F}{\mathbb{F}}
\newcommand{\re}{\,\mathrm{Re}\,}
\newcommand{\ord}{\mathrm{ord}}
\newcommand{\legendre}[2]{\genfrac{(}{)}{}{}{#1}{#2}}



\title{Solutions to Ireland, Rosen ``A Classical Introduction to Modern Number Theory''}
\author{Richard Ganaye}

\begin{document}

\maketitle

{\large \bf Chapter 7}

\paragraph{Ex. 7.1}

{\it Use the method of Theorem 1 to show that a finite subgroup of the
multiplicative group of a field is cyclic.
}

\bigskip

A solution is already given in Ex. 4.15

\paragraph{Ex. 7.2}

{\it Find the finite subgroups of $\R^*$ and $\C^*$ and show directly that they are cyclic.
}

\begin{proof}
If $G$ is a finite subgroup of $\R$ or $\C$, and $n = \vert G \vert$, then from Lagrange's Theorem, $x^n = 1$ for all $x \in G$.

$\bullet$ If  $G$ is a finite subgroup of $\R^*$, then the solutions of $x^n=1$ are in $\{-1,1\}$, so $\{1\} \subset G\subset \{-1,1\}$ : 
$G = \{1\}$ or $G = \{-1,1\}$, both cyclic.

$\bullet$ If  $G$ is a finite subgroup of $\C^*$, then $G \subset \mathbb{U}_n = \{e^{2ik\pi/n}\ \vert \ 0 \leq k \leq n-1\}$. As $\vert G \vert = \vert \mathbb{U}_n \vert = n$, then  $G = \mathbb{U}_n \simeq \Z/n\Z$ is cyclic.
\end{proof}

\paragraph{Ex. 7.3}

{\it  Let $F$ a field with $q$ elements and suppose that $q\equiv 1 \pmod n$. Show that for $\alpha \in \F^*$, the equation $x^n = \alpha$ has either no solutions or $n$ solutions.
}

\begin{proof} This is a particular case of Prop. 7.1.2., where $d = n \wedge (q-1) = n$ : the equation $x^n =\alpha$ has solutions iff $\alpha^{(q-1)/n} = 1$. In this case, there are exactly $d = n$ solutions.

We give  here a direct proof.

Let $g$ be a generator of $F^*$.  Write $x = g^y, \alpha = g^a$. Then
$$x^n = \alpha \iff g^{ny} = g^a\iff q-1 \mid ny -a.$$

Suppose that there exists $x\in F$ such that $x^n = \alpha$. Then there exists $y\in \Z$ such that $q-1 \mid ny -a$. Since $n \mid q-1$, then  $n \mid a$.
$$q-1 \mid ny -a \iff \frac{q-1}{n} \mid y - \frac{a}{n}\iff y= \frac{a}{n} + k \frac{q-1}{n}, k\in \mathbb{Z}.$$

As  $\frac{a}{n} + (k+n) \frac{q-1}{n} = \frac{a}{n} + k \frac{q-1}{n}, k\in \mathbb{Z}$, the values $k=0,1\cdots,n-1 $ are sufficient :
$$x^n=\alpha \iff y = \frac{a}{n} + k \frac{q-1}{n}, k\in \{0,1,\cdots,n-1\}.$$

Moreover, these solutions are all distinct : if $k,l\in  \{0,1,\cdots,n-1\}$,
\begin{align*}
g^{\frac{a}{n} + k \frac{q-1}{n}} = g^{\frac{a}{n} + l \frac{q-1}{n}} & \Rightarrow g^{(k-l) \frac{q-1}{n}} = 1\\
&\Rightarrow q-1 \mid (k-l) \frac{q-1}{n}\\
& \Rightarrow n \mid k-l \\
&\Rightarrow k\equiv l \ [n] \Rightarrow k=l.
\end{align*}

Conclusion: if  $F$ is a field with $q$ elements and $n \mid q-1$, the equation $x^n = \alpha$ has either no solutions or $n$ solutions in $F$.

Note: $$\exists x \in F^*, x^n = \alpha \iff n \mid a \iff \alpha^{(q-1)/n} = 1.$$

Indeed, if $x^n = \alpha$ has a solution, we have proved that $n\mid a$, thus $\alpha^{(q-1)/n} = (g^{a/n})^{q-1} = 1$.

Conversely, if $\alpha^{(q-1)/n} = 1$, $g^{a(q-1)/ n }=1$, thus $q-1 \mid a(q-1)/n$, so $ n \mid a$ :  $\alpha = x^n$, with $x = g^{a/n}$.
\end{proof}

\paragraph{Ex. 7.4}

{\it (continuation) Show that the set of $\alpha \in F^*$ such that $x^n = \alpha$ is solvable is a subgroup with $(q-1)/n$ elements.
}

\begin{proof}
Here $n \mid q-1$.

Let $\varphi = F^* \to F^*$ be the application defined by $\varphi(x) = x^n$. Then $\varphi$ is a homomorphism of groups, and $\ker \varphi$ is the set of solutions of $x^n = 1$.
As $n \mid q-1$, $x^n = 1$ has exactly $n$ solutions (Prop 7.1.1, Corollary2, or Ex 7.3 with $\alpha = 1$). So $\vert \ker \varphi \vert = n$.

Thus $\mathrm{Im}\ \varphi \simeq F^*/\ker \varphi$ is a subgroup with cardinality $\vert F^* \vert / \vert \ker \varphi \vert = (q-1)/n$, and $\mathrm{Im} \ \varphi$ is the set of $\alpha$ such that $x^n = \alpha$ is solvable.

Conclusion: the set of $\alpha \in F^*$ such that $x^n = \alpha$ is solvable is a subgroup with $(q-1)/n$ elements.

\end{proof}

\paragraph{Ex. 7.5}

{\it (continuation) Let $K$ be a field containing $F$ such that $[K:F] = n$. For all $\alpha \in F^*$, show that the equation $x^n = \alpha$ has $n$ solutions in $K$. [Hint: Show that $q^n-1$ is divisible by $n(q-1)$ and use the fact that $\alpha^{q-1} = 1$.]

}

\begin{proof}
As $q\equiv 1 \ [n], \frac{q^n-1}{q-1} = 1+q+\cdots+q^{n-1} \equiv 0 \ [n]$, then $n \mid \frac{q^n-1}{q-1} : $
$$q^n-1 = k n (q-1), k \in \mathbb{N}.$$

Since $\alpha \in F^*$, $\alpha^{q-1} = 1$, thus $$\alpha^{(q^n-1)/n} = (\alpha^{q-1})^k = 1.$$

As $\vert K \vert = q^n$, Prop. 7.1.2 (or the final remark in Ex.7.3) show that there exists $x \in K^*$ such that $x^n = \alpha$. Then, from Ex.7.3, we know that there exist $n$ solutions in $K$.

Conclusion: if $[K:F] = n$, for all $\alpha \in F^*$, the equation $x^n = \alpha$ has $n$ solutions in $K$.

\end{proof}

\paragraph{Ex. 7.6}

{\it Let $K \supset F$ be finite fields with $[K:F] = 3$. Show that if $\alpha \in F$ is not a square in $F$, it is not a square in $K$.

}

\begin{proof}
Let $q = \vert F \vert$. Then $\vert K \vert = q^3.$

If the characteristic of $F$ is 2, then $q = 2^k$ for some integer $k\geq 1$, and for all $x \in F$, $x = x^q = \left (x^{2^{k-1}}\right)^2$. Therefore all elements in $F$ (or $K$) are squares.
We can now suppose that the characteristic of $F$ is not 2, so that $1 \neq -1$ in $F$.

As $\alpha$ is not a square in $F$, $\alpha^{(q-1)/2} \ne 1$ (Prop. 7.1.2). From  $0 = \alpha^{q-1} - 1 = (\alpha^{(q-1)/2}-1)(\alpha^{(q-1)/2}+1)$, we deduce that $\alpha^{(q-1)/2} = -1$. Then 
$$\alpha^{(q^3-1)/2} = (\alpha^{(q-1)/2})^{q^2+q+1} = (-1)^{q^2+q+1} = -1,$$
since $q^2+q+1$ is always odd.

$\alpha^{(q^3-1)/2} \neq 1$ : this implies (Prop. 7.1.2) that $\alpha$ is not a square in $K$.
\end{proof}

\paragraph{Ex. 7.7}

{\it  Generalize Exercise 6 by showing that if $\alpha$ is not a square in $F$, it is not a square in any extension of odd degree and is a square in every extension of even degree.

}

\begin{proof}
Write $n = [K:F]$, and $q = \mathrm{Card}\ F$.

As $\alpha$ is not a square in $F$, the characteristic of $F$ is not 2 (see Ex.7.6), and $\alpha^{(q-1)/2} \ne 1$. Since $\alpha^{q-1} = 1$, $\alpha^{(q-1)/2}= -1$.
$$\alpha^{(q^n-1)/2} =( \alpha^{(q-1)/2})^{1+q+\cdots+q^{n-1}} = (-1)^{1+q+\cdots+q^{n-1}} .$$

$\bullet$ If $n$ is odd, $1 + q +\cdots+q^{n-1} \equiv 1 \pmod 2$, thus $\alpha^{(q^n-1)/2} = -1\ne 1$, and consequently $\alpha$ is not a square in $K$.

$\bullet$ If $n$ is even, as $q$ is odd ($\mathrm{char}(F) \ne 2$), $1 + q +\cdots+q^{n-1} \equiv 0 \pmod 2$, thus $\alpha^{(q^n-1)/2} = 1$, so $\alpha$ is a square in $K$.
\end{proof}

\paragraph{Ex. 7.8}

{\it  In a field with $2^n$ elements, what is the subgroup of squares.
}

\bigskip

Let $F$ be a field with $q = 2^n$ elements. 

\bigskip

{\bf Proof 1}
\begin{proof}
$d = (q-1) \wedge 2 = (2^n - 1) \wedge 2 = 1$, thus each $\alpha \in F^*$ verifies $\alpha^{(q-1)/d} = \alpha^{q-1} = 1$. Theorem 7.1.2 show that $\alpha$ is a square in $F$, of exactly one root.

The subgroup of squares is $F^*$.
\end{proof}

{\bf Proof 2}
\begin{proof}
For all $x \in F$, $x = x^q = \left (x^{2^{n-1}}\right)^2$. So all elements in $F$ are squares. 
\end{proof}

\paragraph{Ex. 7.9}

{\it If $K \supset F$ are finite fields, $\vert F \vert = q, \alpha \in F, q\equiv 1 \pmod n$, and $x^n = \alpha$ is not solvable in $F$, show that $x^n = \alpha$ is not solvable in $K$ if $(n,[K:F]) = 1$.
}

\begin{proof}
Let $k = [K:F]$. From hypothesis, $k\wedge n = 1$, so there exist integers $u,v$ such that $uk+vn = 1$.

As $n \mid q-1$, $n \wedge (q-1) = n$, the hypothesis "$x^n = \alpha$ is not solvable in $F$" implies that $\alpha^{(q-1)/n} \neq 1$ (Prop. 7.1.2).

Write $\omega =\alpha^{(q-1)/n}$, so $\omega \ne 1$ and $\omega^n = 1$.

As $n \mid q-1$, $n \mid q^k - 1$ and
$$\alpha^{(q^k-1)/n} = (\alpha^{(q-1)/n})^{1+q+q^2+\cdots+q^{k-1}} =  \omega^{1+q+q^2+\cdots+q^{k-1}}.$$
Moreover $1+q+\cdots+q^{k-1} \equiv k \pmod n$, and $\omega^n = 1$, so $\alpha^{(q^k-1)/n} = \omega^k$.

If $\omega^k = 1$, then $\omega = \omega^{uk+vn} = (\omega^k)^u (\omega^n)^v = 1$, which is in contradiction with $\omega = \alpha^{(q-1)/n} \ne 1$.

Thus $\alpha^{(q^k-1)/n} = \omega^k \ne 1$. This proves that the equation $x^n = \alpha$ has no solution in $K$.
\end{proof}

\paragraph{Ex. 7.10}

{\it If $K \supset F$ be finite fields and $[K:F] = 2$. For $\beta \in K$, show that $\beta^{1+q} \in F$ and moreover that every element in $F$ is of the form $\beta^{1+q}$ for some $\beta \in K$.
}

\begin{proof}
If $\beta = 0$, $\beta^{1+q} = 0 \in F$, and if $\beta \in K^*$, $\beta^{q^2-1} = 1$, so $\left(\beta ^{1+q}\right)^{q-1} = 1$, thus $ \beta^{1+q} \in F$ (Prop. 7.1.1, Corollary 1).

Let  $g$ be a generator of $K^*$. Then $K^* = \{1,g,g^2,\cdots,g^{q^2-2}\}$.

For every integer $k \in \Z$, 
$$g^k  \in F^* \iff (g^k)^{q-1} = 1 \iff g^{k(q-1)}=1 \iff q^2-1 \mid k(q-1) \iff q+1 \mid k.$$
Thus $F^* = \{1,g^{q+1}, g^{2(q+1)},\cdots, g^{(q-2)(q+1)}\}$. If $\alpha \in F^*$, there exists $i, 0 \leq i \leq q-1,$ such that $\alpha = g^{i(q+1)}$.
If we write $\beta = g^i$, then $\alpha = \beta^{1+q}$ (and for $\alpha = 0$, we take $\beta = 0$).

Conclusion: if $K$ is a quadratic extension of $F$ ($F,K$ finite fields), every element in $F$ is of the form $\beta^{1+q}$ for some $\beta \in K$.
\end{proof}

\paragraph{Ex. 7.11}

{\it With the situation being that of Exercise 10 suppose that $\alpha \in F$ has order $q-1$. Show that there is a $\beta \in K$ with order $q^2-1$ such that $\beta^{1+q} = \alpha$.
}

\bigskip

Write $|a|$ the order of an element $a$ in a group $G$. We recall the following lemma :

\bigskip

{\bf Lemma.} If $|a| = d$, then for all $i \in \Z$, $| a^i | = \frac{d}{d\wedge i}$.
\begin{proof}
Indeed, for all $k \in \Z$,
$$(a^i)^k=e \iff a^{ik}=e \iff d \mid ik \iff \frac{d}{d \wedge i} \mid \frac{i}{d \wedge i}\,k \iff \frac{d}{d \wedge i} \mid k.$$
\end{proof}
\begin{proof}(Ex. 7.11)

Let $\alpha \in F^*$ with $|\alpha| = q-1$, and $g$ a generator of $K^*$, so $|g| = q^2-1$. We know from exercise 7.10 that there exists an integer i such that $\alpha = g^{i(q+1)}$.

Let $h = g^{q+1}$. As $h^{q-1} = 1$, then $h \in F^*$, and since $|g| = q^2-1$, $|h| = q-1$, so $h$ is a generator of $F^*$.

Note that for all $s\in \Z$, $\alpha = g^{(i+s(q-1))(q+1)}$, since $g^{q^2-1} = 1$.

We will show that we can choose $s$ such that $j = i +s(q-1)$ is relatively prime with $q+1$. Then  $j$ is such that $\alpha = g^{j(q+1)} = h^j$.


$i$ is odd : if not, $\alpha$ is an element of the subgroup of squares in $F^*$, so its order divides $(q-1)/2$, in contradiction with $|\alpha| = q-1$.

 $(q-1)\wedge (q+1) \mid 2$. Since $i-1$ is even, there exist integers $s,t$ verifying the B\'{e}zout's equation
$$i-1 = t(q+1)-s(q-1).$$
Then $j = i+s(q-1) = 1 + t(q+1)$ is relatively prime with $q+1$ : $j \wedge (q+1) = 1$.

Moreover, as $\alpha = h^j$, with $|\alpha| = |h| = q-1$, the lemma implies that
$$q-1 = |\alpha| = \frac{q-1}{(q-1)\wedge j},$$
so $(q-1) \wedge j = 1$.
As $(q+1) \wedge j = 1$ and $(q-1) \wedge j = 1$, then $(q^2-1) \wedge j = 1$.

Let $\beta = g^j$. Then $\alpha = \beta^{1+q}$, and using the lemma,
$$|\beta| = |g^j| = \frac{q^2-1}{(q^2-1)\wedge j} = q^2-1.$$
Conclusion:  there exists a $\beta \in K^*$ with order $q^2-1$ such that $\beta^{1+q} = \alpha$.
\end{proof}

\paragraph{Ex. 7.12}

{\it Use Proposition 7.2.1 to show that given a field $k$ and a polynomial $f(x) \in k[x]$ there is a field $K\supset k$ such that $[K:k]$ is finite and $f(x) = a(x-\alpha_1)(x-\alpha_2)\cdots(x-\alpha_n)$ in $K[x]$.
}

\begin{proof}
We show by induction on the degree $n$ of $f$ that for all polynomials $f \in k[x]$ with $\deg(f) = n \geq 1$, there exists a field extension $K$ such that $[K:k]$ is finite, and $f(x)$ splits in linear factors on $K$.

If $n = 1$, $f(x) = ax+b = a(x-\alpha_0)$, where $\alpha_0 = -b/a$ : $K = k$ is suitable.

Suppose that the property is true for all polynomials of degree less than $n$ on an arbitrary field $k$.

Let $f(x) \in k[x], \deg(f) = n$. From proposition 7.2.1. applied to an irreducible factor of $f$, there exists a field $L, [L:k]<\infty$ and $\alpha_1 \in L$ such that $f(\alpha_1) = 0$. Then $f(x) = (x-\alpha_1) g(x), g(x) \in L[x]$. 

Applying the induction hypothesis in the field $L$ on the polynomial $g \in L[x]$ with $\deg(g) = n-1$, we obtain a field $K, [K:L]<\infty$ such that $g(x) = a(x-\alpha_2)\cdots(x-\alpha_n)$ with $\alpha_i \in K$. So  $f(x) = a(x-\alpha_1)(x-\alpha_2)\cdots(x-\alpha_n)$ splits in linear factors in $K$. The induction is done.
\end{proof}

\paragraph{Ex. 7.13}

{\it Apply Exercise 7.12 to $\F_p = \Z/p\Z$ and $f(x) = x^{p^n} -x$ to obtain another proof of Theorem 2.
}

\begin{proof}
Let $f(x) =  x^{p^n} -x$. We know from Ex. 7.12 that there exists a finite extension $K$ of $\F_p$ such that $f$ splits in linear factors on $K$ :
$$f(x) = \prod_{k=1}^{p^n} (x-\alpha_k), \qquad \alpha_1,\ldots, \alpha_{p^n} \in K.$$
The set $k =\{\alpha_1,\cdots,\alpha_{p^n}\} \subset K$ of the roots of $x^{p^n} -x$ is a subfield of $K$. Indeed, if $\alpha, \beta \in k$,
\begin{enumerate}
\item[(a)] $f(1) = 0$, so $1 \in k$
\item[(b)]$(\alpha- \beta)^{p^n} = \alpha^{p^n} - \beta^{p^n} = \alpha - \beta$, so $\alpha - \beta \in k$.
\item[(c)] $(\alpha\beta)^{p^n} = \alpha^{p^n} \beta^{p^n} = \alpha  \beta$, so $\alpha \beta \in k$.
\item[(d)] $(\alpha^{-1})^{p^n} = (\alpha^{p^n})^{-1} = \alpha^{-1}$, so $\alpha^{-1} \in k$ if $\alpha \ne 0$.
\end{enumerate}
As $f'(x) = -1$, $f(x) \wedge f'(x) = 1$. Thus $f$ has no multiple root, and the cardinality of $k$ is $p^n$.

Let $g(x) \in \F_p[x]$ a factor of $f(x)$, irreducible in $\F_p[x]$, with $d =\deg(g)$. As $g \mid f$, $g$ splits in linear factors in $k[x]$. Let $\alpha$ be a root of $g(x)$ in $k$. As $g$ is irreducible on $\F_p$, 
$d = \deg(g) = [\F_p[\alpha] : \F_p]$. Moreover $ n = [k : \F_p] = [k : \F_p[\alpha]]\,[\F_p[\alpha] : \F_p]$, so $d \mid n$.

Conversely, suppose that $g$ is any irreducible polynomial in $\F_p[x]$, with $d = \deg(g) \mid n$. Then $K_0 = \F_p[x]/ \langle g\rangle$  contains a root $\alpha$ of $g$, and $[K_0:\F_p] = \deg(g) = d$, so $\alpha^{p^d} = \alpha$.

As $d\mid n$ , then $p^d-1 \mid p^n-1$ and $x^{p^d -1}-1 \mid x^{p^n-1}-1$ (Lemma 2,3 in section 1), thus 
$$x^{p^d} - x \mid x^{p^n}-x.$$
$f(\alpha) = \alpha^{p^n} - \alpha = 0$ and $g$ is the minimal polynomial of $\alpha$, so $g \mid f$.

Since $f$ has no multiple roots in $K$, $g^2 \nmid f$.

Conclusion : 
$$ x^{p^n}-x = \prod_{d\mid n} F_d(x),$$
where $F_d(x)$ is the product of the monic irreducible polynomial of degree $d$.
\end{proof}

\paragraph{Ex. 7.14}

{\it  Let $F$ be a field with $q$ elements and $n$ a positive integer. Show that there exist irreducible polynomials in $F[x]$ of degree $n$.
}

\begin{proof}
Let $F = \F_q$ a field with $q = p^m$ elements, and $n$ a positive integer.

By Theorem 2 Corollary 3, there exists an irreducible polynomial $f(x) \in \F_p[x]$ of degree $nm$. Let $g$ be an irreducible factor of $f$ in $\mathbb{F}_q[x]$, and let $\alpha$ be a root of $g$ in an extension of $\F_q$.

\bigskip

We show that $\F_q \subset \F_p[\alpha]$.

$\F_q$ and $\F_p[\alpha]$ are two subfields of the same finite field $\F_q[\alpha]$. Moreover, $|\F_q| = p^m$, and $|\F_p[\alpha]| = p^{nm} $. As $m\mid nm$, $\F_q \subset \F_p[\alpha]$ .

Indeed, for all  $\gamma \in \F_q[\alpha]$,$$ \gamma \in \F_q \Rightarrow  \gamma^{p^m} =  \gamma \Rightarrow  \gamma^{p^{mn}} =  \gamma \Rightarrow  \gamma \in \F_p[\alpha].$$
So $\F_q \subset \F_p[\alpha]$.

\bigskip

We show that $\F_q[\alpha] =  \F_p[\alpha]$.

As $\F_p \subset \F_q$, $\F_p[\alpha] \subset \F_q[\alpha]$.

Let $\beta \in \F_q[\alpha]$. Then $\beta = \sum\limits_{i=1}^k a_i\alpha^i$, where $a_i \in \F_q \subset \F_p[\alpha]$, thus $a_i = p_i(\alpha),\ p_i \in \F_p[x]$. Therefore
$$\beta = \sum\limits_{i=1}^k p_i(\alpha) \alpha^i \in \F_p[\alpha].$$
We have proved $\F_q[\alpha] =  \F_p[\alpha]$.

Since  $|\F_p[\alpha]| = p^{nm} $,
$$nm = [\mathbb{F}_p[\alpha] : \mathbb{F}_p] = [\mathbb{F}_q[\alpha]:\mathbb{F}_p] = [\mathbb{F}_q[\alpha] : \mathbb{F}_q]\times [\mathbb{F}_q:\mathbb{F}_p] =  [\mathbb{F}_q[\alpha] : \mathbb{F}_q]\times m .$$
Thus  $[\mathbb{F}_q[\alpha] : \mathbb{F}_q] = n$. Moreover $g$ is the minimal polynomial of $\alpha$ on $\F_q$, thus $\deg(g) = n$.

Conclusion: if $F$ is a field with $q = p^m$ elements, there exist irreducible polynomials in $F[x]$ of degree $n$ for all positive integers $n$.
\end{proof}

\paragraph{Ex. 7.15}

{\it Let $x^n-1 \in F[x]$, where $F$ is a finite field with $q$ elements. Suppose that $(q,n) = 1$. Show that $x^n-1$ splits into linear factors in some extension field and that the least degree of such a field is the smallest integer $f$ such that $q^f \equiv 1 \pmod n$.

}

\begin{proof}
From exercise 7.12, we know that $x^n-1$ splits into linear factors in some extension field $K$, with $[K:F]<\infty$ :
$$u(x)= x^n-1 = (x-\zeta_0)(x-\zeta_1)\cdots(x-\zeta_{n-1}), \qquad \zeta_i \in K.$$
Then $u'(x)\wedge u(x) = nx^{n-1} \wedge (x^n-1) = 1$, since $x(nx^{n-1}) - n(x^n-1) = n$, and $n\neq 0$ in the field $F$, because we know from the hypothesis $q \wedge n=1$ that the characteristic $p$ doesn't divide $n$. So the $n$ roots of $x^n-1$ are distinct.

The set $G = \{x \in K \ \vert \ x^n=1\}$ is a subgroup of $K^*$, thus $G$ is cyclic of order $n$. Let $\zeta$ a generator of $G$. Then
$$x^n-1 = (x-1)(x-\zeta)(x-\zeta^2)\cdots(x-\zeta^{n-1}).$$

Let $p(x)$ be the minimal polynomial of $\zeta$ on $F$, and $f$ the degree of $p$ :
$$f = \deg(p) = [F[\zeta] : F].$$
Thus $\mathrm{Card}\, F[\zeta] = q^f$, and since $\zeta \in F[\zeta]^*$, $\zeta^{q^f - 1} -1 = 0$.
As the order of $\zeta$ in the group $G$ is $n$, $n \mid q^f - 1$, namely $q^f \equiv 1 \pmod n$.

\bigskip

Let $k$ be any positive integer such that $q^k \equiv 1 \pmod n$.

Then $n \mid q^k - 1$, thus $\zeta^{q^k-1} - 1 = 0$, and  $\zeta^{q^k} - \zeta = 0$. Let $L$ be any extension of $K$ such that $x^{q^k} - x$ splits in linear factors in $L$, and let $M=\{\alpha \in L \mid \alpha^{q^k} = \alpha\}$. We know that $M$ is a subfield of $L$ (see Ex. 13), with cardinality $q^k$, so that $[M:F]= k$. As $\zeta^{q^k} - \zeta = 0$, $\zeta $ belongs to $M$. Therefore $F[\zeta] \subset M$, thus $f = [F[\zeta] : F] \leq k = [M:F]$.

So $f = [F[\zeta] : F]$ is the smallest $k \in \N^*$ such that $q^k\equiv 1 \pmod n$.

If $K$ is any extension of $F$ containing the roots of $x^n-1$, then $K \supset F[\zeta]$, where $\zeta$ is a primitive root of unity, so $[K : F] \geq [F[\zeta]:F] = f$.

Conclusion: the minimal degree of a extension $K \supset F$ containing the roots of $x^n - 1$, with $n\wedge q = 1$, is the smallest positive integer $f$ such that $q^f \equiv 1 \pmod n$, the order of $q$ modulo $n$.
\end{proof}

\paragraph{Ex. 7.16}

{\it Calculate the monic irreducible polynomials of degree 4 in $\Z/2\Z[x]$.

}

\begin{proof}
Write $F_d$ the product of irreducible monic polynomials in $\mathbb{F}_2[x]$.

Theorem 2 gives $$x^{16}-x=x^{2^4}-x = \prod\limits_{d\mid 4} F_d(x) = F_1(x) F_2(x) F_4(x)$$ 
and 

$$x^{4}-x=x^{2^2}-x = \prod\limits_{d\mid 2} F_d(x) = F_1(x) F_2(x)$$

so $F_4(x) = \frac{x^{16}-x}{x^4-x} = \frac{x^{15}-1}{x^3-1} = x^{12}+x^9+x^6+x^3+1$,

$F_4(x) = (x^4+x^3+x^2+x+1)(x^4+x+1)(x^4+x^3+1)$.

Among the 16 monic polynomials of degree 4 in $\mathbb{F}_2[x]$, 3 are irreducible :
\begin{align*}
P_1(x) &= x^4+x^3+x^2+x+1,\\
 P_2(x)&=x^4+x+1,\\
  P_3(x)&=x^4+x^3+1.
\end{align*}

With sage : 
\begin{verbatim}
sage: A = PolynomialRing(GF(2),'x')
sage: x = A.gen()
sage: f = (x^16-x)/(x^4-x)
sage: factor(f)
(x^4 + x + 1) * (x^4 + x^3 + 1) * (x^4 + x^3 + x^2 + x + 1)
\end{verbatim}
\end{proof}

\paragraph{Ex. 7.17}

{\it Let $q$ and $p$ be distinct odd primes. Show that the number of monic irreducibles of degree $q$ in $\Z/p\Z$ is $q^{-1}(p^q -p)$.
}

\begin{proof}
From Theorem 2 Corollary 2, we know that the number of irreducible polynomials on $\F_p$ of degree $q$ is given by
$$N_q = \frac{1}{q}\sum_{d\mid q} \mu\left(\frac{q}{d}\right) p^d.$$
As $q$ is prime, $d$ takes the values $1,q$, with $\mu(1) = 1, \mu(q) = -1$, so
$$N_q = \frac{p^q -p}{q}.$$
\end{proof}

\paragraph{Ex. 7.18}

{\it Let $p$ be a prime with $p\equiv 3 \pmod 4$. Show that the residue classes modulo $p$ in $\Z[i]$ form a field with $p^2$ elements.
}

\begin{proof}
If $p$ is a prime rational integer, with $p\equiv 3 \pmod 4$, then $p$ is a prime in $\Z[i]$.

Indeed, $p$ is irreducible : if $p = uv,\ u,v \in \Z[i]$, where $u = c+di,v$ are not units, then $p^2 = N(u)N(v),\ N(u)>1,N(v)>1$, so $p = N(u) = u \overline{u} = c^2+d^2$.

 As $c^2\equiv 0,1 \pmod 4, d^2 \equiv 0,1 \pmod 4$, so $p \equiv 0,1,2 \pmod 4$, which is in contradiction with the hypothesis.
 
 So $p$ is irreducible in $\Z[i]$, and since $\Z[i]$ is a principal ideal domain, $p$ is prime in $\Z[i]$, thus $\Z[i]/( p )$ is a field.
 
 Let $z = a+bi \in \Z[i]$. The Euclidean division of $a,b$ by $p$ gives 
 $$a = qp+r,\ 0\leq r < p, \qquad b = q'p+s, \ 0\leq s < p,$$
 so $$z \equiv r+is \pmod p,\ 0\leq r <p, 0 \leq s < p.$$
 Let's verify that these $p^2$ elements are in different classes of congruences modulo $p$.
 
 If $r+is \equiv r'+is' \pmod p$, then $(r-r')/p + i(s-s')/p \in \Z[i]$, so $r\equiv r',s\equiv s' \pmod p$.
 
 As $r,r',s,s'$ are between $0$ and $p-1$, $r=r', s= s'$.
 
 So the cardinality of the field $\Z[i]/( p )$ is $p^2$.
\end{proof}

\paragraph{Ex. 7.19}

{\it Let $F$ be a finite field with $q$ elements . If $f(x) \in F[x]$ has degree $t$, put $|f| = q^t$. Verify the formal identity $\sum_f |f|^{-s} = (1-q^{1-s})^{-1}$. The sum is over all monic polynomials.

}

\begin{proof}
Let $U$ be the set of monic polynomials in $\mathbb{F}_q[x]$, and $U_t$ the set of monic polynomials of degree $t$, and $s\in \C$. Then $U = \bigcup_{t \in \N} U_t$, so
\begin{align*}
\sum_{f \in U} \vert f \vert^{-s }&= \sum_{t=0}^\infty \sum_{f \in U_t} \vert f \vert^{-s}\\
&=\sum_{t=0}^\infty \frac{1}{q^{ts}} \sum_{f\in U_t} 1.
\end{align*}
As $\sum_{f\in U_t} 1 = \mathrm{Card} \, (U_t) = q^t$, then, for $\mathrm{Re}(s) >1$

\begin{align*}
\sum_{f \in U} \vert f \vert^{-s } &= \sum_{t=0}^\infty \frac{1}{q^{t(s-1)}}\\
&=\frac{1}{1-\frac{1}{q^{s-1}}}\\
&= (1-q^{1-s})^{-1}.
\end{align*}
As $\left | \frac{1}{q^{t(s-1)}} \right | = \frac{1}{q^{t(\mathrm{Re}(s)-1)}}$, the sum is absolutely convergent for $\mathrm{Re}(s)>1$. This justifies the grouping of terms in this sum.

\bigskip

Conclusion:  if $\mathrm{Re}(s)>1$, 
$$\sum_{f \in U} \vert f \vert^{-s } =(1-q^{1-s})^{-1},$$
where $U$ is the set of monic polynomials in  $\mathbb{F}_q[x]$.
\end{proof}

\paragraph{Ex. 7.20}

{\it With the notation of Exercise 19 let $d(f)$ be the number of monic divisors of $f$ and $\sigma(f) = \sum_{g\mid f} |g|$, where the sum is over the monic divisors of $f$. Verify the following identities :
\begin{enumerate}
\item[(a)] $\sum_f d(f) |f|^{-s} = (1 -q^{1-s})^{-2}$.
\item[(b)] $\sum \sigma(f)|f|^{-s} = (1-q^{1-s})^{-1} (1-q^{2-s})^{-1}$.
\end{enumerate}

}

\begin{proof}
(a) With the notation of 7.19, for $s \in \mathbb{C}, \mathrm{Re}(s) >1$, $ \sum_{f\in U} \vert f \vert^{-s}$ is absolutely convergent and

$$(1-q^{1-s})^{-1} = \sum_{f\in U} \vert f \vert^{-s}.$$

Then
\begin{align*}
(1-q^{1-s})^{-2} &= \sum_{f\in U} \vert f \vert^{-s}\sum_{g\in U} \vert g\vert^{-s}\\
&=\sum_{(f,g)\in U^2} \vert fg\vert^{-s}\\
&= \sum_{h\in U} \sum_{g\in U,\, g\mid h} \vert h \vert^{-s},
\end{align*}
indeed, the application
\[\varphi :
 \left \{
\begin{array}{ccc}
U\times U & \to &   \{(h,g)\in U\times U, g\mid h\}  \\
  (f,g) &   \mapsto &(fg,g)    
\end{array}
\right.
\]
is a bijection.
 
So
 \begin{align*}
 (1-q^{1-s})^{-2} &=\sum_{h\in U}  \vert h \vert^{-s} \mathrm{Card}\{g \in U, g \mid h\}\\
 &= \sum_{h\in U} \vert h \vert^{-s} d(h)\\
 &= \sum_{f\in U} d(f) \vert f \vert^{-s}.
 \end{align*}
 
 (b) Similarly,
 \begin{align*}
 (1-q^{1-s})^{-1}  (1-q^{2-s})^{-1} &= \sum_{f\in U} \vert f \vert^{-s} \sum_{g\in U} \vert g \vert^{-s+1}\\
 &=\sum_{(f,g)\in U^2} \vert g \vert\, \vert fg\vert^{-s}\\
 &= \sum_{h\in U} \sum_{g\in U,\, g\mid h} \vert g \vert \, \vert h \vert^{-s}\\
 &=\sum_{h\in U}\vert h \vert ^{-s}  \sum_{g\in U,\, g\mid h} \vert g \vert\\
 &=\sum_{h\in U}  \sigma(h) \vert h \vert ^{-s}\\
  &=\sum_{f\in U}  \sigma(f) \vert f \vert ^{-s}.\\
 \end{align*}
\end{proof}

\paragraph{Ex. 7.21}

{\it Let $F$ be a field with $q=p^n$ elements. For $\alpha \in F$ set $f(x) = (x-\alpha)(x-\alpha^p)(x-\alpha^{p^2})\cdots(x-\alpha^{p^{n-1}})$. Show that $f(x) \in \Z/p\Z[x]$. In particular, $\alpha + \alpha^p+\cdots+\alpha^{p^{n-1}}$ and $\alpha\alpha^p\alpha^{p^2}\cdots\alpha^{p^{n-1}}$ are in $Z/p\Z$.

}

\begin{proof}
Let 
$F : 
\left\{
\begin{array}{ccc}
 \mathbb{F}_q &  \to  &  \mathbb{F}_q \\
 x & \mapsto     &  x^p    
\end{array}
\right. .
$


As the characteristic of $\F_q$ is $p$, $(x+y)^p=x^p+y^p$ et $(xy)^p = x^py^p$, and each homomorphism of field is injective, so $F$ is a field automorphism (Frobenius automorphism).

For every automorphism $H$ in $\F_q$, and every polynomial $p(x) =\sum a_ix^i \in \F_q[x]$, write $(H\cdot p)(x) = \sum_i H(a_i) x^i$. Then for all $(p,q) \in \F_q[x]^2$, $H\cdot (pq) = (H\cdot p)(H\cdot q)$.

With this notation,
$$f(x) =(x-\alpha)(x-F\alpha)(x- F^2\alpha)\cdots (x-F^{n-1}\alpha),$$
$$(F\cdot f)(x) =(x-F\alpha)(x-F^2\alpha)(x- F^3\alpha)\cdots (x-F^{n}\alpha).$$

Since $\alpha \in \mathbb{F}_{p^n}$, $F^n\alpha = \alpha^{p^n} = \alpha$ , thus
$$F\cdot f = f.$$
In other words, if $f(x) =\sum_i a_i x^i$, then for all $i$, $F(a_i) = a_i$, so $a_i^p = a_i$, thus $a_i \in \F_p$, and $f \in \F_p[x]$.
In particular, the coefficients $a_{n-1}= \alpha + \alpha^p+\cdots+\alpha^{p^{n-1}}, a_0=\alpha\alpha^p\alpha^{p^2}\cdots\alpha^{p^{n-1}}$ are in $\F_p$.
\end{proof}

\paragraph{Ex. 7.22}

{\it (continuation) Set $\mathrm{tr}(\alpha) = \alpha+\alpha^p+\cdots+\alpha^{p^{n-1}}$. Prove that
\begin{enumerate}
\item[(a)] $\mathrm{tr}(\alpha) + \mathrm{tr}(\beta) = \mathrm{tr}(\alpha+\beta)$.
\item[(b)] $\mathrm{tr}(a\alpha) = a\, \mathrm{tr}(\alpha)$ for $a \in \Z/p\Z$.
\item[(c)] There is an $\alpha \in F$ such that $\mathrm{tr}(\alpha) \ne 0$.
\end{enumerate}
}

\begin{proof}
Let $F$ the Frobenius automorphism of $\F_q$ introduced in Ex.7.21.

(a),(b) : If $x,y \in \F_q$, and $a \in \F_p$, then $a^p = a$, so $F(x+y) =(x+y)^p = x^p+y^p = F(x)+F(y)$, and $F(ax) = a^p x^p = a x^p =a F(x)$, so $F$ is $\F_p$-linear, and also $tr = I + F + F^2+\cdots+F^{n-1}$. 

(c) The polynomial $p(x) = x+x^p+x^{p^2}+\cdots+x^{p^{n-1}}$ has degree $p^{n-1}$, so $p(x)$ has at most $p^{n-1}$ roots in $\F_q$, and $|\F_q| =p^n >\deg(p) = p^{n-1}$. Therefore there exists in $\F_q$ some element $\alpha$ which is not a root of $p(x)$, and so  $\mathrm{tr}(\alpha) = p(\alpha) \ne 0$.
\end{proof}

\paragraph{Ex. 7.23}

{\it (continuation) For $\alpha \in F$ consider the polynomial $x^p-x-\alpha \in F[x]$. Show that this polynomial is either irreducible or the product of linear factors. Prove that the latter alternative holds iff $\mathrm{tr}(\alpha) = 0$.

}

\begin{proof}
Let $f(x) = x^p -x -\alpha \in F[x]$. There exists an extension $K \supset F$ with finite degree on $F$ which contains a root $\gamma$ of $f$.

As $\gamma^p - \gamma - \alpha = 0$, then for all $i\in \F_p$,
$$(\gamma + i)^p - (\gamma+i) - \alpha = (\gamma^p - \gamma - \alpha) + i^p - i = 0.$$
So $f$ has $n$ distinct roots in $K$ : $\gamma, \gamma+1,\ldots,\gamma+p-1$, and so
$$f(x) = (x-\gamma)(x-\gamma-1)\cdots(x-\gamma-(p-1)).$$
$F[\gamma] $ contains all roots of $f$.

$\bullet$ If $\gamma \in F$, $f(x)$ splits in linear factors in $F$. $f(x)$ is not irreducible, since $\deg(f) = p>1$.

$\bullet$ If $\gamma \not \in F$, we will show that $f$ is irreducible in $F[x]$.

If not, then $f(x) = g(x) h(x)$ is the product of two monic polynomials $g,h \in F[x]$ such that $1\leq \deg(g) \leq p-1$.

The unicity of the decomposition in irreducible factors in $F[\gamma][x]$ shows that
$$g(x) = \prod_{i\in A} (x-\gamma -i),$$
where $A$ is a subset of $\F_p$, with $A \ne \emptyset, A \ne \F_p$.
As $g(x) \in F[x]$, $\sum\limits_{i\in A} (\gamma+i) = k \gamma +l \in \F_p$, where $1 \leq k = |A| \leq p-1$ and $l = \sum\limits_{i\in A} i \in \F_p$.

So $k\gamma \in \F_p$. Since $\gamma \not \in \F_p$, $k$ is not invertible in $\F_p$, in contradiction with $1\leq k \leq p-1$. Consequently,  $f(x)$ is irreducible.

We conclude that $x^p-x-\alpha \in F[x]$ is irreducible iff $\gamma \not \in F$.

\bigskip

Let $F$ be the Frobenius automorphism of $K$ (cf. Ex. 7.21).

$$\alpha = F(\gamma) - \gamma, F(\alpha) = F^2(\gamma) - F(\gamma), \ldots, F^{n-1}(\alpha) = F^n(\gamma) - F^{n-1}(\gamma).$$
The sum of these equalities gives
$$\mathrm{tr}(\alpha) = \alpha+F(\alpha)+\cdots+F^{n-1}(\alpha) = F^n(\gamma) - \gamma = \gamma^{p^n} - \gamma .$$
As the cardinality of $F$ is $q=p^n$,
$$\gamma \in F \iff \gamma^{p^n} - \gamma = 0 \iff \mathrm{tr}(\alpha) = 0.$$

Conclusion : $x^p - x - \alpha$ is irreducible iff $\mathrm{tr}(\alpha) \ne 0$. If $\mathrm{tr}(\alpha)= 0$, $x^p - x - \alpha$ splits in linear factors in $F[x]$.
\end{proof}

\paragraph{Ex. 7.24}

{\it Suppose that $f(x) \in \Z/p\Z[x]$ has the property that $f(x+y) = f(x) + f(y) \in \Z/p\Z[x,y]$. Show that $f(x)$ must be of the form $a_0x +a_1x^p+a_2x^{p^2}+\cdots+a_mx^{p^m}.$
}

\bigskip

{\bf Lemma} {\it If the prime number $p$ divides all binomial coefficients $\binom{n}{1}, \binom{n}{2},\ldots, \binom{n}{n-1}$, then $n$ is a power of $p$.
\begin{proof}
Let $ u(x) = (x+1)^n -x^n - 1 \in \F_p[x]$. Then $u(x) = \sum\limits_{k=1}^{n-1} \binom{n}{i} x^i = 0$.

Write $n = p^a q$, with $p\wedge q = 1$. We must show that $q=1$. Suppose at the contrary that $q>1$. Then 
$$u(x)= 0 = (x+1)^{p^\alpha q} - x^{p^\alpha q} - 1=(x^{p^\alpha} + 1)^q -x^{p^\alpha q} - 1=\sum\limits_{k=1}^{q-1} \binom{q}{k} x^{kp^a}.$$
Therefore the coefficient $\binom{q}{1} = q$ of $x^{p^a}$ is null in $\F_p$, thus $p \mid q$ : this is absurd. Therefore $q=1$ and $n = p^a$.
\end{proof}


\begin{proof}(Ex. 7.24)

Suppose that $f\in \F_p[x]$ verifies the equality $f(x+y) = f(x)+f(y)$ in $\F_p[x,y]$.

Write $f(x) = \sum\limits_{k=1}^d c_i x^i$.
\begin{align*}
0 = f(x+y) - f(x) -f(y)&= \sum_{n=0}^d c_n[(x+y)^n-x ^n - y^n]\\
&=\sum_{n=0}^d \sum_{k=1}^{n-1} c_n\binom{n}{k} x^ky ^{n-k}\\
\end{align*}
Therefore,  for all $n$, for all $k$, $1\leq k \leq n-1$, $c_n \binom{n}{k} = 0$ in $\F_p$.

From the lemma, if $n$ is not a power of $p$, there exists a $k$, $1\leq k \leq n-1$ such that $\binom{n}{k} \not \equiv 0 \pmod p$, thus $c_n = 0$.
If we write $a_k = c_{p^k}$, then $f(x)$ is of the form
$$f(x) = a_0 x+a_1x^p +a_2 x^{p^2}+\cdots+a_m x^{p^m}.$$
\end{proof}

{ \Large \bf Chapter 8} 

\paragraph{Ex. 8.1}

{\it Let $p$ be a prime and $d=(m,p-1)$. Prove that $N(x^m = a) =\sum \chi(a)$, the sum being over all $\chi$ such that $\chi^d = \varepsilon$.
}

\begin{proof}
Let $d = m\wedge (p-1)$. We prove that $N(x^m =a) = N(x^d = a)$ for all $d \in \F_p$.

$\bullet$ If $a = 0$, $0$ is the only root of $x^m -a$ or $x^d -a$, so  $N(x^m = a) = N(x^d = a) = 1$.

$\bullet$ If $a \in \F_p^*$ and $x^m = a$ has a solution, then we know from the demonstration of Proposition 4.2.1 that $N(x^m = a) = d$, and $N(x^d = a) = d \wedge (p-1) = d$, thus $N(x^m = a) = N(x^d =a)$.

$\bullet$ If $a \in \F_p^*$ and $x^m = a$ has no solution, then (Prop. 4.2.1) $a^{(p-1)/d} \ne 1$, thus $x^d = a$ has no solution : $N(x^m = a) = 0 = N(x^d =a)$.

Using Prop. 8.1.5, as $d \mid p-1$, we obtain
$$N(x^m =a) = N(x^d = a) = \sum_{\chi^d = \varepsilon} \chi(a).$$
\end{proof}

\paragraph{Ex. 8.2, false sentence.}

{\it With the notation of Exercise 1 show that $N(x^m = a) = N(x^d = a)$ and conclude that if $d_i = (m_i,p-1)$, then $\sum_i a_ix^{m_i} = b$ and $\sum_i a_i x^{d_i} = b$ have the same number of solutions.
}

\bigskip

This result is false. I give a counterexample with $p=5$ :  $x + x^3 =0 \in \F_5[x]$ has 3 solutions $0,2,-2$.  As $3\wedge (p-1) = 3 \wedge 4 = 1$, the reduced equation is $x + x = 0$, which has an unique solution $0$. The true sentence is :

\bigskip\paragraph{Ex. 8.2}
{\it With the notation of Exercise 1 show that $N(x^m = a) = N(x^d = a)$ and conclude that if $d_i = (m_i,p-1)$, then $\sum_i a_ix_i^{m_i} = b$ and $\sum_i a_i x_i^{d_i} = b$ have the same number of solutions.
}

\begin{proof}
From Ex. 8.1, we know that
$$N(x^m=a) = \sum_{\chi^d=\varepsilon} \chi(a) = N(x^d=a).$$
Using this result, we obtain
\begin{align*}
N\left(\sum\limits_{i=1}^l a_i x_i^{m_i} = b\right) &= \sum\limits_{a_1u_1+\cdots+a_l u_l= b}\, \prod\limits_{i=1}^l N(x^{m_i} = u_i)\\
&= \sum\limits_{a_1u_1+\cdots+a_l u_l= b}\, \prod\limits_{i=1}^l N(x^{d_i} = u_i)\\
&=N\left(\sum\limits_{i=1}^l a_i x_i^{d_i} = b\right).
\end{align*}
\end{proof}

\paragraph{Ex. 8.3}

{\it Let $\chi$ be a non trivial multiplicative character of $\F_p$ and $\rho$ be the character of order 2. Show that $\sum_t\chi(1-t^2) = J(\chi,\rho)$.[Hint: Evaluate $J(\chi,\rho)$ using the relation $N(x^2 = a) = 1 + \rho(a)$.]

}

\begin{proof}
\begin{align*}
J(\chi,\rho) &= \sum\limits_{a+b=1} \chi(a)\rho(b)\\
&= \sum\limits_{a+b=1}\chi(a)(N(x^2=b) -1)\\
&=\sum\limits_{a+b=1}\chi(a)N(x^2=b) -\sum\limits_{a+b=1}\chi(a).\\
\end{align*}
As $\chi \neq \varepsilon$, 
$$\sum\limits_{a+b=1}\chi(a)= \sum\limits_{a\in\mathbb{F}_p}\chi(a)=0.$$
Let $C = \{x^2 \ \vert\  x \in \mathbb{F}^*\}$ the set of squares in $\F_p^*$ , $\overline{C}$ its complementary in  $\mathbb{F}_p^*$ : 
$$\mathbb{F}_p = \{0\} \cup C \cup \overline{C}.$$
Then 
\begin{align*}
J(\chi,\rho) &= \sum\limits_{a+b=1}\chi(a)N(x^2=b)\\
&= \sum\limits_{a+b=1, b=0}\chi(a)N(x^2=b)+\sum\limits_{a+b=1,b\in C}\chi(a)N(x^2=b)+\sum\limits_{a+b=1,b\in\overline{C}}\chi(a)N(x^2=b)\\
&=\chi(1)+ 2 \sum\limits_{b\in C} \chi(1-b),
 \end{align*}
 because $N(x^2=b)=0$ if $x \in \overline{C}$, and $N(x^2=b)=2$ if $x\in C$.
 As each $b\in C$ has two roots, and as the set of roots of two distinct $b$ are disjointed,
 $$J(\chi,\rho) = \chi(1) + \sum\limits_{t \in \mathbb{F}_p^*} \chi(1-t^2) = \sum\limits_{t \in \mathbb{F}_p} \chi(1-t^2).$$
 
 Conclusion : if $\chi$ is a non trivial multiplicative character of $\F_p$ and $\rho$  the character of order 2,
 $$ J(\chi,\rho) = \sum_{t\in\F_p}\chi(1-t^2) .$$
\end{proof}

\paragraph{Ex. 8.4}

{\it Show, if $k \in \F_p, k\neq 0$, that $\sum_t \chi(t(k-t)) = \chi(k^2/2^2)J(\chi,\rho)$.

}

\begin{proof}
We know from Ex. 8.3 that  $J(\chi,\rho) = \sum_t \chi(1-t^2)$, so

\begin{align*}
J(\chi,\rho) &= \sum_{t \in \F_p}\chi(1-t)\chi(1+t)\\
&=\sum_{u \in \F_p} \chi(u) \chi(2-u)\qquad(u = 1-t)\\
&=\chi(2^2) \sum_{u \in \F_p} \chi\left(\frac{u}{2}\right) \chi\left(1-\frac{u}{2}\right)\\
&= \chi(2^2) \sum_{v \in \F_p} \chi(v) \chi(1-v)\qquad(u = 2v)\\
&= \chi(2^2) \chi(k^{-2}) \sum_{v \in \F_p} \chi(kv) \chi(k-kv)\\
&=\chi(2^2/k^2) \sum_{t \in \F_p} \chi(t) \chi(k-t)\qquad(t=kv).
\end{align*}

Conclusion: if $k \in \mathbb{F}^*$, and $\chi$ is a non trivial character, $\rho$ the character of order 2,
$$ \sum_{t\in \F_p}\chi(t(k-t)) = \chi(k^2/2^2) J(\chi,\rho).$$
\end{proof}

\paragraph{Ex. 8.5}

{\it If $\chi^2 \ne \varepsilon$, show that $g(\chi)^2 = \chi(2)^{-2}J(\chi,\rho)g(\chi^2)$. [Hint: Write out $g(\chi)^2$ explicitly and use Exercise 4.]

}

\begin{proof}
Let  $\zeta = e^{2i\pi/p}$.
Using the result of Ex. 8.4, we obtain
\begin{align*}
g(\chi)^2 &= \left(\sum_t \chi(t) \zeta^t \right)\left(\sum_s \chi(s) \zeta^s\right)\\
&=\sum_{s,t} \chi(t)\chi(s) \zeta^{t+s}\\
&=\sum_k\left(\sum_{s+t=k} \chi(t) \chi(s)\right)\zeta^k\\
&=\sum_k\left( \sum_t \chi(t(k-t) \right) \zeta^k\\
&= \chi(-1)\sum_t \chi(t^2) + \sum_{k\neq0} \chi(k^2/2^2) J(\chi,\rho) \zeta^k\\
&= \chi(-1)\sum_t \chi^2(t) + \chi(2)^{-2}  J(\chi,\rho) \sum_{k\neq0} \chi^2(k)  \zeta^k\\
\end{align*}

If $\chi^2 \neq \varepsilon, \sum_t \chi^2(t)=0$, so

$$g(\chi)^2 = \chi(2)^{-2}  J(\chi,\rho) g(\chi^2).$$
\end{proof}

\paragraph{Ex. 8.6}

{\it (continuation) Show that $J(\chi,\chi) = \chi(2)^{-2} J(\chi,\rho)$.

}

\begin{proof}
As $\chi^2 \ne \varepsilon$, Theorem 1 Chapter 8 gives $J(\chi,\chi) = g(\chi)^2/g(\chi^2)$, and Exercise 8.5 gives $g(\chi)^2/g(\chi^2) = \chi(2)^{-2} J(\chi,\rho)$, so
$$J(\chi,\chi) =  \chi(2)^{-2} J(\chi,\rho).$$
\end{proof}

\paragraph{Ex. 8.7}

{\it Suppose that $p\equiv 1 \pmod 4$ and that $\chi$ is a character of order 4. Then $\chi^2 = \rho$ and $J(\chi,\chi) = \chi(-1) J(\chi,\rho)$. [Hint: Evaluate $g(\chi)^4$ in two ways.]

}

\begin{proof}
As $\chi$ is a character of order 4, $\chi^2$ is a character of order 2, and $\rho$ (Legendre's character) is the unique character of order 2, so $\chi^2 = \rho$.

From Prop. 8.3.3 we have
$$g(\chi)^4 = \chi(-1) p J(\chi,\chi) J(\chi,\chi^2)  =  \chi(-1) p J(\chi,\chi) J(\chi,\rho).$$
Squaring the result of Ex. 8.5, we obtain
$$g(\chi)^4 = \chi(2)^{-4} J(\chi,\rho)^2 \left[g(\chi^2)\right]^2.$$
Moreover $\chi(2^4) = \chi^4(2) = \varepsilon(2) = 1$, and $g(\chi^2) = g(\rho) = g$, so $\left[g(\chi^2)\right]^2=g^2 = (-1)^{(p-1)/2} p = p $ (From Prop. 6.3.2 and $p\equiv 1 \pmod 4)$. 

Equating these two results, we obtain 
$$\chi(-1) p J(\chi,\chi) J(\chi,\rho)= J(\chi,\rho)^2 p.$$
As $g(\chi)^4 \ne 0$ since $|g(\chi)|^2 = p$, we have $J(\chi,\rho) \ne 0$, so
$$\chi(-1) J(\chi,\chi) = J(\chi,\rho).$$
$[\chi(-1)]^2 = \chi((-1)^2) = \chi(1) = 1$, so $\chi(-1) = \pm 1$, and $\chi(-1)^{-1} = \chi(-1)$, thus
$$ J(\chi,\chi) = \chi(-1) J(\chi,\rho).$$
\end{proof}

\paragraph{Ex. 8.8}

{\it Generalize Exercise 3 in the following way. Suppose that $p$ is a prime, $\sum_t \chi(1-t^m) = \sum_{\lambda} J(\chi,\lambda)$, where $\lambda$ varies over all characters such that $\lambda^m = \varepsilon$. Conclude that $\left | \sum_t \chi(1-t^m) \right | \leq (m-1)p^{1/2}$.

}

\begin{proof}
For all $y \in \F_p$, write  $A_y =\{x \in \mathbb{F}_p\ \vert \ x^m = y\}$. Then $\vert A_y\vert = N(x^m = y)$.

$\mathbb{F}_p = \coprod\limits_{y \in \mathbb{F}_p} A_y$ is the disjoint union of the $A_y$:

\begin{enumerate}
\item[$\bullet$] if $x \in A_y \cap A_{y'}$, then $x^m = y = y'$. This proves
$$y \ne y' \Rightarrow A_y \cap A_{y'} = \varnothing.$$
\item[$\bullet$] Every $x \in \F_p$ satisfies $x \in A_{x^m}$, thus
$$\bigcup_{y \in \F_p} A_y = \F_p.$$
\end{enumerate}
(Note that some $A_y$ may be empty.)

Therefore
$$\sum_{t\in \mathbb{F}_p} \chi(1-t^m) = \sum_{y\in \mathbb{F}_p} \sum\limits_{t \in A_y} \chi(1-t^m) = \sum_{y\in \mathbb{F}_p} \vert A_y \vert \chi(1-y) =\sum_{y\in \mathbb{F}_p} N(x^m = y) \chi(1-y).$$
Moreover, $N(x^m=y) = \sum\limits_{\lambda^m = \varepsilon} \lambda(y)$ (Prop. 8.1.5), so
\begin{align*}
\sum_{t\in \mathbb{F}_p} \chi(1-t^m) &= \sum_{y\in \mathbb{F}_p}  \sum_{\lambda^m = \varepsilon} \lambda(y)  \chi(1-y)\\
& =  \sum\limits_{\lambda^m = \varepsilon}  \sum\limits_{x+y=1} \chi(x) \lambda(y)\\
& = \sum\limits_{\lambda^m = \varepsilon} J(\chi,\lambda).
\end{align*}
Conclusion : $$\sum\limits_{t\in \mathbb{F}_p} \chi(1-t^m) = \sum\limits_{\lambda^m = \varepsilon} J(\chi,\lambda).$$

We know that there exist $m$ characters whose order divides $m$. We know that $\chi \ne \varepsilon$, $J(\chi,\varepsilon)=0$, and  $\vert J(\chi,\lambda) \vert = \sqrt{p}$ for every $\lambda \ne \varepsilon,\lambda\ne \chi^{-1}$ (Theorem 1 and Corollary).

Moreover, by Theorem 1(c), $|J(\chi,\chi^{-1})|  = 1 \leq \sqrt{p}$, so
$$\left \vert \sum_{t\in \mathbb{F}_p} \chi(1-t^m)\right  \vert \leq \sum_{\lambda^m = \varepsilon, \lambda \neq \varepsilon} \vert J(\chi,\lambda) \vert \leq  (m-1)\sqrt{p}.$$
\end{proof}

\paragraph{Ex. 8.9}

{\it Suppose that $p\equiv 1 \pmod 3$ and that $\chi$ is a character of order 3. Prove (using Exercise 5) that $g(\chi)^3 = p\pi$, where $\pi = \chi(2)J(\chi,\rho)$.
}

\begin{proof}
As $\chi$ is a character of order 3, $\chi^2\neq \varepsilon$. From Exercise 5, we know that
$$g(\chi)^2 = \chi(2)^{-2} J(\chi,\rho) g(\chi^2).$$
So
$$g(\chi)^3 = \chi(2)^{-2} J(\chi,\rho) g(\chi^2)g(\chi).$$
Recall  (¤8.2) that
$$\overline{g(\chi)} = \sum_t \overline{\chi(t)} \zeta^{-t} = \chi(-1) \sum_t \overline{\chi(-t)} \zeta(-t) = \chi(-1) g(\overline{\chi}),$$
Here $\chi(-1) = 1$, because $\chi(-1) = \chi((-1)^3) = \chi^3(-1) = \varepsilon(-1) = 1$.
Hence
$$g(\chi^2) g(\chi) = g(\bar{\chi}) g(\chi) =\overline{g(\chi)} g(\chi) = \vert g(\chi) \vert^2 =p.$$
Moreover $\chi(2)^3 = \chi^3(2) = 1$, so $\chi(2)^{-2} = \chi(2)$.

\medskip

Conclusion : if $\chi$ is a character of order 3,
 $$g(\chi)^3 = p \pi, \ \mathrm{where}\ \pi = \chi(2) J(\chi,\rho).$$
\end{proof}

\paragraph{Ex. 8.10}

{\it (continuation) Show that $\chi \rho$ is a character of order 6 and that $$g(\chi \rho)^6 = (-1)^{(p-1)/2} p \overline{\pi}^4.$$
}

\begin{proof}
$(\chi \rho)^6 = \chi^6 \rho^6 = \varepsilon, (\chi \rho)^2 = \chi^2 \neq \varepsilon, (\chi \rho)^3 = \rho^3 = \rho \neq \varepsilon$, so  $\chi \rho$ is of order 6.

$J(\chi,\rho) g(\chi \rho) = g(\chi) g(\rho)$ since $\chi, \rho, \chi \rho$ are non trivial characters. So
$$g(\chi \rho)^6 = \frac{g(\chi)^6 g(\rho)^6}{J(\chi, \rho)^6}.$$
From Exercise 8.9, $g(\chi)^6 = p^2 \pi^2$. Proposition 6.3.2 gives $g(\rho)^2 = (-1)^{(p-1)/2} p$, so $g(\rho)^6 =  (-1)^{(p-1)/2} p^3$. 
As $\pi = \chi(2) J(\chi,\rho)$,  $J(\chi,\rho)^6 = \chi(2)^{-6} \pi^6 = \pi^6$, since $\chi(2)^3=1$. Therefore
 $$g(\chi \rho)^6 = \frac{p^2 \pi^2 (-1)^{(p-1)/2} p^3}{\pi^6} = (-1)^{(p-1)/2} p^5 \pi^{-4}.$$
 Moreover, $\pi \bar{\pi} = \chi(2) \overline{\chi(2)} J(\chi,\rho) \overline {J(\chi,\rho)} = \vert J(\chi,\rho) \vert^2 =p$ (Theorem 8.1, Corollary), so $ \pi^{-1} = \bar{\pi}/p$. In conclusion,
$$g(\chi \rho)^6 = (-1)^{(p-1)/2} p \bar{\pi}^4.$$
\end{proof}

\paragraph{Ex. 8.11}

{\it Use Gauss' theorem to find the number of solutions to $x^3+y^3 = 1$ in $\F_p$ for $p=13,19,37$, and $97$.
}

\begin{proof}
$\bullet$ $p = 13$.

 $4 \times 13 = 52 =(-5)^2+27\times 1^2$, where $-5 \equiv 1 \pmod 3$, so $A = -5$.

If $p=13$, $N(x^3+y^3=1) = p-2+A = 13 - 2  - 5 = 6$ : the solutions are only the trivial solutions.

\medskip

$\bullet$ $p=19$.

$4 \times 19 = 76 = 7^2+27\times 1^2$, where $7\equiv1 \pmod 3$, so $A = 7$.

If $p = 19$, $N(x^3+y^3 = 1) = 19 -2 + 7 = 24$.

\medskip

$\bullet$ $p=37$.

$4\times 37 = 148 = (-11)^2 + 27 \times 1^2$, where $-11 \equiv 1 \pmod 3$, so $A = -11$.

If $p=37$, $N(x^3+y^3 = 1) = 37 - 2 -11 = 24$.

\medskip

$\bullet$ $p=97$.

$4 \times 97 = 388 = 19^2 + 27 \times 1^2$, where $19 \equiv 1 \pmod 3$, so $A = 19$.

If $p=97$, $N(x^3+y^3=1) = 97 - 2 + 19 = 114$.

(These results were verified on pari/gp).)
\end{proof}

\paragraph{Ex. 8.12}

{\it If $p\equiv 1 \pmod 4$, then we have seen that $p = a^2+b^2$ with $a,b \in \Z$. If we require that $a$ and $b$ are positive, that $a$ be odd, and that $b$ is even, show that $a$ and $b$ are uniquely determined. (Hint: Use the fact that unique factorization holds in $\Z[i]$ and that if $p = a^2+b^2$ then $a+bi$ is a prime in $\Z[i]$.)
}

\begin{proof}
Suppose that $p$ is prime, $p\equiv 1 \pmod 4$, and $p = a^2+b^2 = c^2+d^2$, where $a,b,c,d$ are positive integers, $a,c$ odd, $b,d$ even. We will show that $a=c, b=d$.

As $p = N(a+bi)$, $\pi = a+bi$ is irreducible in $\Z[i]$ : indeed $\pi = uv$ implies that $p=N(\pi) = N(u)N(v)$, so $N(u)= 1$ or $N(v) = 1$, and $u$ or $v$ is an unit.

Since $\Z[i]$ is a principal ideal domain, $\pi$ is a prime in $\Z[i]$.

$(a+bi)(a-bi) = (c+di)(c-di)$, so the prime $\pi$ divides $c+di$, or it divides $c-di$.

As $N(\pi) = N(c+di) = N(c-di)$, the quotient is an unit. Therefore $\pi$ is an associate of $c+di$ or $c-di$. Since the units in $\Z[i]$ are $1,-1,i,-i$,
$$a+bi = \pm (c+di),\ \mathrm{or}\ a+bi = \pm i (c+di), \ \mathrm{or}\ a+bi = \pm (c-di), \ \mathrm{or}\ a+bi = \pm i(c-di).$$
In all cases, $a=\pm c, b = \pm d$, or $a=\pm d, b = \pm c$.
Since $a,b,c,d$ are positive, $a=c, b=d$, or $a=d,b=c$.
As $a,c$ are odds, and $b,d$ even, $a=c, b=d$ : the unicity of the decomposition is proved.
\end{proof}

\paragraph{Ex. 8.13}

{\it If $p\equiv 1 \pmod 3$, we have seen that $4p = A^2+27B^2$, with $A,B \in \Z$. If we require that $A\equiv 1 \pmod 3$, show that $A$ is uniquely determined. (Hint: Use the fact that unique factorization holds in $\Z[\omega]$. This proof is a little trickier than that for Exercise 12.)
}

\begin{proof}
Suppose that $4p = A^2 + 27 B^2 = C^2 + 27 D^2$, where $A \equiv C \equiv 1 \pmod 3$. We will show that $A = C$.

Let $\omega = e^{2i\pi/3} = -1/2 + i \sqrt{3}/2$. Then $i\sqrt{3} = 2 \omega +1$, and for all $x,y$,
$x^3 + 3 y^2 = (x+i\sqrt{3}y)(x-i\sqrt{3}y) = (x + (2\omega+1)y)(x - (2\omega+1)y),$
$$x^2+3y^2 = (x+y+2\omega y)(x-y-2\omega y).$$
With $x =A, y = 3B$, we obtain
$$4p = A^2+27B^2 = (A+3B+6\omega B)(A-3B-6\omega B).$$
Note that $A,B$ are of same parity, since $4p = A^2 + 27 B^2$.

So we can write $p = ((A+3B)/2 + 3\omega B)((A-3B)/2 - 3\omega B)$ :
$$p = \pi \overline{\pi},\ \mathrm{where}\ \pi = \frac{A+3B}{2} + 3\omega B \in \Z[\omega].$$
$\pi$ is a prime in $\Z[\omega]$ : indeed $\pi = uv,\ u,v \in \Z[\omega]$ implies $p = N(\pi) = N(u)N(v)$, then $N(u) = 1$ or $N(v) = 1$, $u$ or $v$ is an unit, so $\pi$ is irreducible in the principal ideal domain $\Z[\omega]$, thus $\pi$ is a prime in $\Z[\omega]$.
$$\pi \overline{\pi} = \left(\frac{A+3B}{2} + 3\omega B\right)\left(\frac{A-3B}{2} - 3\omega B\right)=\left(\frac{C+3D}{2} + 3\omega D\right)\left(\frac{C-3D}{2} - 3\omega D\right).$$
As $\pi$ is a prime, it divides $\frac{C+3D}{2} + 3\omega D$ or its conjugate. Since they have the same norm $p$, they are associated. The units of $\Z[\omega]$ are $\pm1,\pm \omega,\pm \omega^2$, so there exists $12$ cases :
\begin{align*}
\frac{A+3B}{2} + 3\omega B &= \pm \left(\frac{C+3D}{2} + 3\omega D\right),\\
\frac{A+3B}{2} + 3\omega B &= \pm \omega \left(\frac{C+3D}{2} + 3\omega D\right),\\
\frac{A+3B}{2} + 3\omega B &= \pm \omega^2\left(\frac{C+3D}{2} + 3\omega D\right),\\
\frac{A+3B}{2} + 3\omega B &= \pm \left(\frac{C-3D}{2} - 3\omega D\right),\\
\frac{A+3B}{2} + 3\omega B &= \pm \omega \left(\frac{C-3D}{2} - 3\omega D\right),\\
\frac{A+3B}{2} + 3\omega B &= \pm \omega^2\left(\frac{C-3D}{2} - 3\omega D\right).\\
\end{align*}
If we replace $D$ by $-D$, we obtain the 6 last cases from the 6 first cases, so it is sufficient to examine the first 6 cases. Recall that $(1,\omega)$ is a $\Z$-base of $\Z[\omega]$.
\begin{enumerate}
\item[1)] $A + 3B + 6\omega B = C + 3D + 6\omega D$.

Then $B = D$ and $A+3B = C + 3D$, so $A = C$, which is the expected result.
The five other cases are impossible :
\item[2)] $A+3B + 6\omega B = -C-3D -6\omega D$.

Then $B = -D, A = -C$. As $A\equiv C \equiv 1 \pmod 3$, this is impossible.
\item[3)] $A + 3 B + 6\omega B = \omega(C+3D+6\omega D) = \omega(C+3D) + (-1 - \omega) 6 D = -6D + \omega(C-3D)$.

Then $A+3B = - 6 D$, $A \equiv 0 \pmod 3$, this is impossible.
\item[4)] $A+3B + 6 \omega B = -\omega (C+3D+6\omega D) = -\omega (C+3D) + (1+\omega) 6D = 6D + \omega(-C+3D)$.

Then $A+3B = -6D$, $A \equiv 0 \pmod 3$, this is impossible.

\item[5)] $A+3B + 6\omega B = \omega^2(C+D+6\omega D) = (-1-\omega)(C + 3D) + 6D  = -C + 3D + \omega (-C-3D)$.
Then $A+3B = - C +3D$, $A\equiv -C \pmod 3$ , this is impossible.

\item[6)] $A + 3B + 6 \omega B = -\omega^2(C+3D + 6\omega D) = (1 + \omega)(C+3D) - 6D = (C-3D) + \omega(C+3D)$.

Then $6B = C+ 3D$, $C \equiv 0 \pmod 3$, this is impossible.
\end{enumerate}
In conclusion $A = C$.
\end{proof}

\paragraph{Ex. 8.14}

{\it Suppose that $p\equiv 1 \pmod n$ and that $\chi$ is a character of order $n$. Show that $g(\chi)^n \in\Z[\zeta]$, where $\zeta = e^{2\pi i/n}$.
}

\begin{proof}
From Proposition 8.3.3 we know that
$$g(\chi)^n = \chi(-1) p J(\chi,\chi) J(\chi, \chi^2) \cdots J(\chi, \chi^{n-2}).$$
Let $\mathbb{U}_n =\{x \in \C\ | \ x^n = 1\} = \{1, \zeta, \ldots, \zeta^{n-1}\}$, with $\zeta = e^{2\pi i/n}$, the group of $n$-th roots of unity. As the order of $\chi$ is $n$, for all $x \in \F_p^*$, $(\chi(x))^n = \chi^n(x) = \varepsilon(x) = 1$, so $\chi(x) \in \mathbb{U}_n$, and also $\chi^k(x) = (\chi(x))^k$.

Therefore $J(\chi,\chi^k) = \sum\limits_{x+y=1} \chi(x) \chi^k(y) \in \Z[\zeta]$. Moreover $\chi(-1) = \pm 1$, so $\chi(-1)$ and $p$ are in $\Z[\zeta]$. In conclusion $g(\chi)^n \in \Z[\zeta]$.
\end{proof}

\paragraph{Ex. 8.15}

{\it Suppose that $p\equiv 1 \pmod 6$ and let $\chi$ and $\rho$ be characters of order 3 and 2, respectively. Show that the number of solutions to $y^2 = x^3 + D$ in $\F_p$ is $p+\pi+\overline{\pi}$, where $\pi = \chi \rho(D) J(\chi\rho)$. If $\chi(2) = 1$, show that the number of solutions to $y^2 = x^3 + 1$ is $p+A$, where $4p = A^2+27B^2$ and $A\equiv 1 \pmod 3$. Verify this result numerically when $p=31$.
}

\begin{proof}
$x \mapsto -x$ is a bijection between the set of roots of $x^3 = b$ and the set of roots of $(-x)^3 = b$, so $N(x^3 = b) = N((-x)^3 = b) = N(x^3 = -b)$. 

As $\chi$ is a character of order 3, the characters whose order divides 3 are $\varepsilon,\chi,\chi^2$. Using Prop. 8.1.5, we obtain, if $D\ne 0$,
\begin{align*}
N(y^2=x^3+D)  &= \sum\limits_{a+b=D} N(y^2=a) N((-x)^3 = b)\\
&= \sum\limits_{a+b=D} N(y^2=a) N(x^3=b)\\
&= \sum\limits_{a+b=D}(1+\rho(a))(1+\chi(b)+\chi^2(b))\\
&=\sum\limits_{i=0}^1\sum\limits_{j=0}^2\sum\limits_{a+b=D} \rho^i(a) \chi^j(b)\\
&=\sum\limits_{i=0}^1\sum\limits_{j=0}^2\rho(D)^i \chi(D)^j\sum\limits_{a'+b'=1} \rho^i(a') \chi^j(b')\qquad (a = Da', b = D b')\\
&=\sum\limits_{i=0}^1\sum\limits_{j=0}^2\rho(D)^i \chi(D)^j J(\chi^j,\rho^i).\\
\end{align*}
We know (Theorem 1) that  $J(\chi,\varepsilon)=J(\chi^2,\varepsilon)=J(\varepsilon,\rho)=0, J(\varepsilon,\varepsilon)=p$, so
$$N(y^2=x^3+D) = p + \rho(D)\chi(D) J(\chi,\rho) + \rho(D) \chi^2(D) J(\chi^2,\rho).$$
As $\chi^2(D) = \chi^{-1}(D) = \overline{\chi(D)}$, and as $\overline{\rho(D)} = \rho(D)$, then $J(\chi^2,\rho) =  J(\overline{\chi},\overline{\rho}) = \overline{J(\chi,\rho)}$, and
$$N(y^2 = x^3+D) = p +\pi + \bar{\pi},\ \mathrm{where}\ \pi = (\rho \chi)(D) J(\chi,\rho).$$

If  $\chi(2)=1$, then from Exercise 8.6 we have
$$J(\chi,\chi) = \chi(2)^{-2} J(\chi,\rho) = J(\chi,\rho).$$
With $D = 1$ (if $\chi(2)=1$), we obtain
$$N(y^2=x^3+1) = p+\pi +\bar{\pi}, \pi = J(\chi,\rho) = J(\chi,\chi).$$
From Prop. 8.3.4 we know that  $J(\chi,\chi) = a + b\omega, b\equiv0 \pmod 3, a \equiv-1\pmod 3$.

$ \pi+ \overline{\pi} = 2\, \mathrm{Re}\, J(\chi,\chi) =2a-b \equiv 1 \pmod 3$, and $p = |J(\chi,\rho)|^2 = a^2-ab+b^2$, so $4p =(2a-b)^2 + 3b^2$.

Writing $A=2a-b, B=b/3$, we obtain $4p = A^2+27B^2,A\equiv1 \pmod 3$ (the unicity of $A$ if proved in Exercise 8.13).

Conclusion : $N(y^2=x^3+1) = p+A$, where $4p = A^2+27B^2, A\equiv1 \pmod 3$.

\bigskip

If $p=31$, 3 is a primitive element, and $2 = 3^{24} = (3^8)^3$ in $\F_{31}$, therefore $\chi(2)=1$.

$31 = 4 + 27$, $4\times31 = 124 = 4^2+27\times2^2$, and $4\equiv 1 \pmod 3$, so

if $p=31$, $N(y^2 = x^3+1) = 35$.
\end{proof}

\paragraph{Ex. 8.16}

{\it Suppose that $p\equiv 1 \pmod 4$ and that $\chi$ is a character of order 4. Let $N$ be the number of solutions to $x^4+y^4 = 1$ in $\F_p$. Show that $N = p+1-\delta_4(-1) 4 + 2\, \mathrm{Re}\, J(\chi,\chi) + 4 \, \mathrm{Re}\, J(\chi,\rho)$.
}

\begin{proof}
Let $\chi$ be a character of order 4 : such a character exists since $p\equiv 1 \pmod 4$. Then

\begin{align*}
N(x^4+y^4=1) &=\sum\limits_{a+b=1} N(x^4=a)N(y^4=b)\\
&=\sum\limits_{a+b=1} \sum\limits_{i=0}^3 \chi^i(a) \sum\limits_{j=0}^3 \chi^j(b)\\
&= \sum\limits_{i=0}^3 \sum\limits_{j=0}^3\sum\limits_{a+b=1} \chi^i(a)  \chi^j(b)\\
&= \sum\limits_{i=0}^3 \sum\limits_{j=0}^3 J(\chi^i,\chi^j)\\
&= p-\chi(-1)-\chi^2(-1) - \chi^3(-1) \\
&+ J(\chi,\chi) + J(\chi,\chi^2) +J(\chi^2,\chi)\\
&+J(\chi^2,\chi^3)+J(\chi^3,\chi^2)+J(\chi^3,\chi^3),
\end{align*}
since from Theorem 1, we have $J(\varepsilon,\varepsilon) = p, J(\varepsilon,\chi^j) = 0$ for $j = 1,2,3$, and $J(\chi^i, \chi^{4-i}) = -\chi^i(-1)$.

Moreover
 $$-[\chi(-1)+\chi^2(-1)+\chi^3(-1)] = 1-[1+\chi(-1)+\chi^2(-1)+\chi^3(-1)],$$
 
 and
 $$
\left\{
\begin{array}{ccll}
  1+\chi(-1)+\chi^2(-1)+\chi^3(-1) = \frac{1-\chi^4(-1)}{1-\chi(-1)} & = 0&\ \mathrm{if}\ \chi(-1) \neq 1    \\
  &     =4&\ \mathrm{if}\ \chi(-1) =1.
\end{array}
\right.
$$

 Let $g$ a generator of $\F_p^*$. Recall that  $\chi(g) = e^{qi\pi/2}$ with $q$ odd, so $\chi : a = g^k \mapsto e^{i qk\pi /2} = i^{qk}$, thus
 $$\chi(a) = 1 \iff \chi(g^k)=1 \iff  i^{qk} = 1 \iff 4 \mid k \iff a = b^4, b \in \F^*.$$
 $\delta_4$ is defined by  $\delta_4(a) = 1$ if $a$ is a fourth power, 0 if not. Then  
 $$-[\chi(-1)+\chi^2(-1)+\chi^3(-1)] = 1 - \delta_4(-1)4.$$
 Moreover $J(\chi,\chi)+ J(\chi^3,\chi^3) = 2 \,\mathrm{Re}\,( J (\chi,\chi))$, and
 $$J(\chi,\chi^2)+J(\chi^3,\chi^2) + J(\chi^2,\chi)+J(\chi^2,\chi^3) = 2 \,\mathrm{Re}\,(J(\chi,\chi^2)) + 2\,\mathrm{Re}\,(J(\chi^2,\chi))= 4\,\mathrm{Re}\,(J(\chi,\chi^2)).$$
$\chi$ is of order 4, so $\rho = \chi^2$ is the unique character of order 2, the Legendre's character.

In conclusion,
 $$N(x^4+y^4=1) = p+1- \delta_4(-1)4+ 2 \,\mathrm{Re}\,( J (\chi,\chi))+4\,\mathrm{Re}\,(J(\chi,\rho)).$$
\end{proof}

\paragraph{Ex. 8.17}

{\it (continuation) By Exercise 8.7, $J(\chi,\chi) = \chi(-1) J(\chi,\rho)$. Let $\pi = -J(\chi,\rho)$. Show that
\begin{enumerate}
\item[(a)] $N = p-3 - 6\,\mathrm{Re}\, \pi$ if $p\equiv 1 \pmod 8$.
\item[(b)] $N = p+1 - 2 \,\mathrm{Re}\,  \pi$ if $p \equiv 5 \pmod 8$.
\end{enumerate}
}

\begin{proof}
Let $g$ a generator in  $\mathbb{F}_p^*$. As $(g^{(p-1)/2})^2=1$ and $g^{(p-1)/2} \neq 1$, then $g^{(p-1)/2} = -1$.
As in Exercise 8.16, write $\chi(g) = e^{q i\pi/2}$, with $q$ odd.

Then $-1$ is a fourth power in $\F_p^*$ iff (see Exercice 8.16)
\begin{align*}
\delta_4(-1) = 1 &\iff \chi(-1)=1\\
&\iff \chi(g^{(p-1)/2}) = 1\\
&\iff e^{q ((p-1)/2 )i\pi/2} = 1\\
&\iff  4 \mid q(p-1)/2\\
&\iff 4 \mid (p-1)/2\\
&\iff p\equiv 1 \pmod 8.
\end{align*}
By Exercise 8.7, as $\chi$ is a character of order 4,
$$J(\chi,\chi) = \chi(-1) J(\chi,\rho).$$
$\bullet$ If $p\equiv 1 [8],$

$\chi(-1) = 1$, so $J(\chi,\chi) = J(\chi,\rho)$, and $\delta_4(-1) = 1$.
\begin{align*}
N &= p+1 - \delta_4(-1) 4 + 2 \re J(\chi,\chi)  + 4 \re J(\chi,\rho)\\
&=p-3+6 \re J(\chi,\rho)\\
 &= p-3-6\re \pi,\qquad  \ \mathrm{where}\ \pi = -J(\chi,\rho).
\end{align*}
$\bullet$ If $p\equiv 5 [8],$

$\chi(-1) = -1$, so $J(\chi,\chi) = -J(\chi,\rho)$, and $\delta_4(-1) = 0$.
\begin{align*}
N &= p+1 - \delta_4(-1) 4 + 2 \re J(\chi,\chi)  + 4 \re J(\chi,\rho)\\
&= p+1 + 2 \re J(\chi,\rho)\\
 &= p+1 - 2 \re \pi.
\end{align*}
\end{proof}

\paragraph{Ex. 8.18}

{\it (continuation) Let $\pi = a+bi$. One can show (see Chapter 11, Section 5) that $a$ is odd, $b$ is even, and $a\equiv 1 \pmod 4$ if $4\mid b$ and $a\equiv -1 \pmod 4$ if $4\nmid b$. Let $p=A^2+B^2$ and fix $A$ by requiring that $A\equiv 1 \pmod 4$. Then show that
\begin{enumerate}
\item[(a)] $N = p-3 -6A$ if $p\equiv 1 \pmod 8$,
\item[(b)] $N = p+1 + 2A$ if $p \equiv 5 \pmod 8$.
\end{enumerate}
}

\begin{proof}
Recall that $\pi = -J(\chi,\rho) \in \Z[i]$, so $\pi = a + bi,\ a,b \in \Z$.
\begin{enumerate}
\item[1)] We begin by proving that $\pi \equiv 1 \pmod {2+2i}$ (see Chapter 11, Section 5).

For all $ t \in \F_p^*$, $\rho(t) = \pm 1$, so $\rho(t) - 1 \equiv 0 \pmod 2$.

Let's verify that $\chi(t) - 1 \equiv 0 \pmod{1+i}$. $\chi(t) \in \{1,-1,i,-i\}$, so  $\chi(t) - 1 \in \{0,-2, i - 1, - i-1\}$. As $2 = (1-i)(1+i)$ and $i-1 = i(1+i)$, we obtain
$$\forall t \in \F_p^*, \ 1+i \mid \chi(t) - 1.$$
Thus
$$\forall s \in \F_p^*, \forall t \in \F_p^*, \ (\rho(s)-1)(\chi(t) - 1) \equiv 0 \pmod {2+2i}.$$
Moreover, if $s = 0, t= 1$, then $\chi(t) = 1$, and if $s=1,t=0$, then $\rho(s) = 1$, therefore
$$\sum_{s+t=1} (\rho(s)-1)(\chi(t)-1) \equiv 0 \pmod {2+2i}.$$
This gives, when developing this expression, :
$$-\pi - \sum_{t \in \F_p} \chi(t) - \sum_{s\in \F_p} \rho(s) + p \equiv 0 \pmod {2+2i}.$$
As $\sum_t \chi(t) = \sum_s \rho(s) = 0$, we obtain
$$\pi \equiv p \pmod {2+2i}.$$
Finally, $p \equiv 1 \pmod 4$, and $2+2i \mid 4$, since $4 = (1-i)(2+2i)$, thus $p \equiv 1 \pmod {2+2i}$, and
$$\pi \equiv 1 \pmod {2+2i}.$$

\item[2)] By Corollary of Theorem 1, $\mathrm{N}(\pi) = \mathrm{N}(J(\chi,\rho)) = p = a^2+b^2$.

We know that $p\equiv 1 \pmod 4$, $p = a^2 + b^2$ and $a+ib \equiv 1 \pmod {2+2i}$. Then  we prove that $a$ is odd, $b$ is even, and $a\equiv 1 \pmod 4$ if $4\mid b$ and $a\equiv -1 \pmod 4$ if $4\nmid b$. 

$a + bi \equiv 1 \pmod {2+2i}$, thus $a+bi \equiv 1 \pmod 2$, so that $a$ is odd, and $b$ is even.

$\bullet$ If $4 \mid b$, then $2+2i \mid b$.

Therefore $a \equiv 1 \pmod {2+2i}$, and by complex conjugation, $a\equiv 1 \pmod{2-2i}$, so $(2+2i)(2-2i) = 8 \mid (a-1)^2$, thus $4 \mid a-1$.

$\bullet$ If $4 \nmid b$, then $b = 4k+2, k\in \Z$.

Therefore, $1 \equiv a+bi \equiv a +2i \pmod{2+2i}$. As $2i \equiv -2 \pmod{2+2i}$, $a\equiv 3 \equiv -1 \pmod {2+2i}$.
By conjugation, $a \equiv -1 \pmod{2-2i}$. Multiplying these congruences, we obtain $8 \mid (a+1)^2$, thus $a \equiv -1 \pmod 4$.

\item[3)] $\pi = -J(\chi,\rho) = a + bi$ is such that  $a^2+b^2 = p$, $a$ odd, $b$ even and also
$$(4 \mid b \ \mathrm{and} \ a\equiv 1\ [4]) \ \mathrm{or}\ (4\nmid b\ \mathrm{and}\ a\equiv -1 \ [4]).$$
If $p = A^2+B^2$, $A$ odd and $B$ even, then also $p = (-A)^2 + B^2$, and $A \equiv 1 \pmod 4$ or $-A \equiv 1 \pmod 4$. So there exists a decomposition $p = A^2+B^2$ such that $A\equiv1 \pmod 4$, and $A$ is uniquely determined (Ex. 12).

Let's verify that $4 \mid b$ if $p \equiv 1 \pmod 8, 4 \nmid b$ if $p \equiv 5 \pmod 8$.

$p = a^2 +b^2, a = 2a'+1, b = 2b'$, so $p = 4a'^2 + 4a'+1 + 4b'^2 = 8 \frac{a'(a'+1)}{2} + 1 + 4b'^2$.

Hence $4 \mid b \iff 2 \mid b' \iff 8 \mid p-1$.

Therefore if $p\equiv 1 \pmod 8$, $\re \pi = a = A$, and if $p\equiv 5 \pmod 8$, $\re \pi = a = -A$.

In conclusion, by Exercise 8.17 :

 if $p = A^2+B^2, A\equiv1 \pmod 4$, and $N =N(x^4+y^4 = 1)$ in $\F_p$,
\begin{enumerate}
\item[(a)] $N = p-3 -6A$ if $p\equiv 1 \pmod 8$,
\item[(b)] $N = p+1 + 2A$ if $p \equiv 5 \pmod 8$.
\end{enumerate}
\end{enumerate}


Note : if $p\equiv -1 \pmod 4$, then there is no character of order 4 on $\F_p^*$,  and $d = 4 \wedge (p-1) = 4 \wedge (4k+2) = 2$. By Exercise 1, we obtain
$$N(x^4 = a) = \sum_{\chi^2 = 1} \chi(a) = 1 + \rho(a) = N(x^2 = a).$$
\begin{align*}
N(x^4+y^4 = 1) &= \sum_{a+b=1} N(x^4 = a) N(y^4=b)\\
&= \sum_{a+b=1} N(x^2=a) N(y^2=b)\\
&=N(x^2+y^2=1).
\end{align*}
Using Chapter 8, Section 3, we obtain
$$N(x^4+y^4=1) = p+1 \ \mathrm{if} \ p\equiv -1 \pmod 4.$$
\end{proof}

\paragraph{Ex. 8.19}

{\it Find a formula for the number of solutions to $x_1^2+x_2^2+\cdots+x_r^2 = 0$ in $\F_p$.
}

\begin{proof}
Let $\chi$ be the Legendre character. Then
\begin{align*}
N(x_1^2+x_2^2+\cdots+x_r^2=0) &= \sum\limits_{a_1+a_2+\cdots+a_r=0} N(x_1^2=a_1)N(x_2^2=a_2)\cdots N(x_r^2 = a_r)\\
&=\sum\limits_{a_1+a_2+\cdots+a_r=0} (1+\chi(a_1))(1+\chi(a_2)\cdots(1+\chi(a_r))\\
&=p^{r-1} + J_0(\chi,\chi,\cdots,\chi)
\end{align*}
(We used Proposition 8.5.1).
For all $k$, $\chi^{2k} = \varepsilon, \chi^{2k+1} = \chi$.

$\bullet$ If $r$ is odd, $\chi^r \neq \varepsilon$, so $J_0(\chi,\chi,\cdots,\chi) = 0$ (Proposition 8.5.1).
$$N(x_1^2+x_2^2+\cdots+x_r^2=0)  = p^{r-1}.$$

$\bullet$ If $r$ is even, $\chi^r = \varepsilon$, so $J_0(\chi,\chi,\cdots,\chi) = \chi(-1)(p-1)J(\chi,\chi,\cdots, \chi)$, where there are  $r-1$ components in the Jacobi sum ( Proposition 8.5.1).

By Theorem 3, $J(\chi,\chi,\cdots,\chi) g(\chi^{r-1}) = g(\chi)^{r-1}$, and  $g(\chi^{r-1}) = g(\chi)$, so
$$J(\chi,\chi,\cdots,\chi)=g(\chi)^{r-2}.$$
$g(\chi)^2 = \chi(-1) p$, therefore $\chi^{r-2} = \chi(-1)^{(r/2)-1} p^{(r/2)-1} = (-1)^{((p-1)/2) (r/2-1)}p^{(r/2)-1)}$. So
$$N(x_1^2+x_2^2+\cdots+x_r^2=0)  = p^{r-1}+(-1)^{\frac{p-1}{2} \frac{r}{2}} (p-1)p^{\frac{r}{2}-1}.$$
(Verified in C++ with small values of $p$ and $r$.)

Conclusion :
$$
\left\{
\begin{array}{ccll}
 N(x_1^2+x_2^2+\cdots+x_r^2=0)  & =  & p^{r-1}  &\mathrm{if}\ r\ \mathrm{is}\ \mathrm{odd},\\
  &  = &   p^{r-1}+(-1)^{\frac{p-1}{2} \frac{r}{2}} (p-1)p^{\frac{r}{2}-1} &\mathrm{if}\ r\  \mathrm{is}\ \mathrm{even}.
\end{array}
\right.
$$
\end{proof}

\paragraph{Ex. 8.20}

{\it Generalize Proposition 8.6.1 by finding an explicit formula for the number of solutions to $a_1x_1^2+a_2x_2^2+\cdots+a_rx_r^2 = 1$ in $\F_p$.
}

\begin{proof}
Write $\chi$ the Legendre character.
\begin{align*}
N(a_1x_1^2+\cdots+a_r x_r^2=1)&=\sum\limits_{a_1u_1+\cdots+a_r u_r=1}N(x_1^2=u_1)\cdots N(x_r^2=u_r)\\
&=\sum\limits_{a_1u_1+\cdots+a_r u_r=1}(1+\chi(u_1)\cdots(1+\chi(u_r))  \hspace{0.5cm}(v_i = a_i u_i)\\
&=\sum\limits_{v_1+\cdots+v_r=1}(1+\chi(a_1)^{-1}\chi(v_1))\cdots(1+\chi(a_r^{-1})\chi(v_r))\\
&=p^{r-1} + \chi(a_1^{-1})\cdots\chi(a_r^{-1})J(\chi,\chi,\cdots,\chi).
\end{align*}

$\chi(a_i^{-1}) = \overline{\chi(a_i)} = \chi(a_i) = \legendre{a_i}{p}.$

$J(\chi,\chi,\cdots,\chi)$ is computed in Chapter 5 Section 6. We obtain
$$
\left\{
\begin{array}{ccll}
 N(a_1x_1^2+\cdots+a_r x_r^2=1)  & =  & p^{r-1}+\legendre{a_1}{p}\cdots\legendre{a_r}{p}(-1)^{\frac{r-1}{2}\frac{p-1}{2}}p^{\frac{r-1}{2}}&\mathrm{if}\ r\ \mathrm{is}\ \mathrm{odd}, \\
  &  = &p^{r-1}-\legendre{a_1}{p}\cdots\legendre{a_r}{p}(-1)^{\frac{r}{2}\frac{p-1}{2}}p^{\frac{r}{2}-1}&\mathrm{if}\ r\  \mathrm{is}\ \mathrm{even}.
\end{array}
\right.
$$
\end{proof}

\paragraph{Ex. 8.21}

{\it Suppose that $p\equiv 1 \pmod d$, $\zeta = e^{2\pi i/p}$, and consider $\sum_x \zeta^{ax^d}$. Show that $\sum_x \zeta^{ax^d} = \sum_r m(r)\zeta^{ar}$, where $m(r) = N(x^d = r)$.
}

\begin{proof}
Let $A_r = \{x \in \mathbb{F}_p \ \vert \ x^d=r\}$

Then  $\mathbb{F}_p = \coprod_r A_r$, thus

$$\sum\limits_{x\in \mathbb{F}_p} \zeta^{ax^d} = \sum\limits_{r \in \mathbb{F}_p} \sum\limits_{x \in A_r} \zeta^{a x^d} = \sum\limits_{r \in \mathbb{F}_p} \vert A_r\vert \zeta^{ar}= \sum\limits_{r \in \mathbb{F}_p} m(r)\zeta^{ar},$$

where $m(r) = \vert A_r\vert = N(x^d=r)$.
\end{proof}

\paragraph{Ex. 8.22} 
{\it  (continuation) Prove that $\sum_x \zeta^{ax^d} = \sum_\chi g_a(\chi)$, where the sum is over all $\chi$ such that $\chi^d=\varepsilon, \chi \ne \varepsilon$. Assume that $p \nmid a$.
}

\begin{proof}
By Exercise 8.21,
$$S = \sum_{x \in \F_p} \zeta^{ax^d} = \sum_{ r \in \F_p} m(r) \zeta^{ar}.$$

As  $d \mid p-1$, by Proposition 8.1.5,
$$m(r) = N(x^d=r) = \sum_{\chi^d = \varepsilon} \chi(r).$$

Therefore
 $$S = \sum_{r\in \F_p}\, \sum_{\chi^d = \varepsilon} \chi(r) \zeta^{ar} =\sum_{\chi^d = \varepsilon} \sum_{r\in \F_p} \chi(r) \zeta^{ar}.$$

If $\chi = \varepsilon, \sum\limits_{r\in \F_p} \chi(r) \zeta^{ar} = \sum\limits_{r\in \F_p} \zeta^{ar} = 0$, since $a\not \equiv 0 \pmod p$.

By definition $g_a(\chi) =  \sum_r \chi(r) \zeta^{ar}$, so, if $d \mid p-1$, $p \nmid a$,

$$\sum_{x\in \mathbb{F}_p} \zeta^{a x^d} = \sum_{\chi^d=\varepsilon,\, \chi\neq \varepsilon} g_a(\chi).$$
\end{proof}

\paragraph{Ex. 8.23}

{\it Let $f(x_1,x_2,\ldots,x_n) \in \F_p[x_1,x_2,\ldots,x_n]$. Let $N$ be the number of zeros of $f$ in $\F_p$. Show that $N = p^{n-1} + p^{-1} \sum_{a\neq 0} (\sum_{x_1,\ldots,x_n } \zeta^{af(x_1,\ldots,x_n)})$.
}

\begin{proof}
Let $A_r= \{(x_1,x_2,\cdots,x_n)\in\mathbb{F}_p^n \ \vert \ f(x_1,x_2,\cdots,x_n)=r\}$. Then $\mathbb{F}_p^n = \coprod\limits_{r\in\mathbb{F}_p} A_r$, so, for all $a \in \mathbb{F}_p$,
\begin{align*}
\sum\limits_{(x_1,x_2,\cdots,x_n) \in \mathbb{F}_p^n} \zeta^{af(x_1,x_2,\cdots,x_n)} 
&=\sum\limits_{r\in \mathbb{F}_p} \sum\limits_{(x_1,x_2,\cdots,x_n) \in A_r}  \zeta^{ar}\\
&=\sum\limits_{r\in \mathbb{F}_p} \vert A_r\vert \zeta^{ar}.\\
\end{align*}
Let $m(r) = \vert A_r\vert = N(f(x_1,x_2,\cdots,x_n)=r)$. Then
\begin{align*}
\sum\limits_{a\in \mathbb{F}_p} \sum\limits_{(x_1,x_2,\cdots,x_n) \in \mathbb{F}_p^n} \zeta^{af(x_1,x_2,\cdots,x_n)} &=\sum\limits_{a\in \mathbb{F}_p}\sum\limits_{r\in \mathbb{F}_p} m(r) \zeta^{ar}\\
&=\sum\limits_{r\in \mathbb{F}_p} m(r) \sum\limits_{a\in \mathbb{F}_p} \zeta^{ar}
\end{align*}
As $\sum\limits_{a\in \mathbb{F}_p} \zeta^{ar}=0$ if $r\neq 0$, and $\sum\limits_{a\in \mathbb{F}_p} \zeta^{ar} = p$ if $r=0$, we obtain
$$\sum\limits_{a\in \mathbb{F}_p} \sum\limits_{(x_1,x_2,\cdots,x_n) \in \mathbb{F}_p^n} \zeta^{af(x_1,x_2,\cdots,x_n)} = m(0) p = p N.$$

Moreover
$$\sum\limits_{a\in \mathbb{F}_p} \sum\limits_{(x_1,x_2,\cdots,x_n) \in \mathbb{F}_p^n} \zeta^{af(x_1,x_2,\cdots,x_n)} = p^n + \sum\limits_{a\in \mathbb{F}_p^*} \sum\limits_{(x_1,x_2,\cdots,x_n) \in \mathbb{F}_p^n} \zeta^{af(x_1,x_2,\cdots,x_n)},$$

so $$pN = p^n +\sum\limits_{a\in \mathbb{F}_p^*} \sum\limits_{(x_1,x_2,\cdots,x_n) \in \mathbb{F}_p^n} \zeta^{af(x_1,x_2,\cdots,x_n)}.$$  In conclusion,

$$N = p^{n-1}+p^{-1} \sum\limits_{a\in \mathbb{F}_p^*} \sum\limits_{(x_1,x_2,\cdots,x_n) \in \mathbb{F}_p^n} \zeta^{af(x_1,x_2,\cdots,x_n)}.$$
\end{proof}

\paragraph{Ex. 8.24}

{\it (continuation) Let $f(x_1,x_2,\ldots,x_n) = a_1x_1^{m_1}+a_2x_2^{m_2}+\cdots+a_n x_n^{m_n}$. Let $d_i = (m_i,p-1)$. Show that $N = p^{n-1} + p^{-1} \sum_{a\neq 0} \prod_{i=1}^n \sum_{\chi_i} g_{aa_i}(\chi_i)$ where $\chi_i$ runs over all characters such that $\chi_i^{d_i} = \varepsilon$ and $\chi_i \neq \varepsilon$.
}

\begin{proof}
By Exercise 8.2, $$N = N(a_1 x_1^{m_1}+\cdots a_n x_n^{m_n} = 0) = N(a_1 x_1^{d_1}+\cdots a_n x_n^{d_n} = 0),$$where $d_i = m_i \wedge (p-1)$ divides $p-1$.

By Exercise 8.23, 
$$N = p^{n-1}+p^{-1} \sum\limits_{a\in \mathbb{F}_p^*} \sum\limits_{(x_1,x_2,\cdots,x_n) \in \mathbb{F}_p^n} \zeta^{a(a_1 x_1^{d_1}+\cdots a_n x_n^{d_n} )}$$

By Exercise 8.22, since $p \nmid a, p\nmid a_i$,
\begin{align*}
\sum\limits_{(x_1,x_2,\cdots,x_n) \in \mathbb{F}_p^n} \zeta^{a(a_1 x_1^{d_1}+\cdots a_n x_n^{d_n} )} &=\left( \sum\limits_{x_1 \in \mathbb{F}_p} \zeta^{aa_1 x_1^{d_1}}\right)\cdots\left(\sum\limits_{x_n \in \mathbb{F}_p} \zeta^{aa_n x_n^{d_n}}\right)\\
&=\left(\sum\limits_{\chi_1^{d_1} = \varepsilon,\, \chi_1\neq \varepsilon} g_{aa_1}(\chi_1)\right)\cdots\left(\sum\limits_{\chi_n^{d_n}= \varepsilon,\,\chi_n \neq \varepsilon} g_{aa_n}(\chi_n)\right)\\
&=\prod\limits_{i=1}^n \sum\limits_{\chi_i^{d_i}= \varepsilon,\, \chi_i \neq \varepsilon} g_{aa_i}(\chi_i)
\end{align*}
In conclusion,

$$N = p^{n-1}+p^{-1} \sum\limits_{a\in \mathbb{F}_p^*}  \prod\limits_{i=1}^n \sum\limits_{\chi_i^{d_i} = \varepsilon,\chi_i \neq \varepsilon} g_{aa_i}(\chi_i).$$
\end{proof}

\paragraph{Ex. 8.25}

{\it Deduce from Exercise 8.24 that $$|N-p^{n-1}| \leq (p-1)(d_1-1)\cdots(d_n-1)p^{(n/2)-1}.$$
}

\begin{proof}
As $\vert g_{aa_i}(\chi_i)  \vert= \sqrt{p}$,

$$\left\vert\sum\limits_{\chi_i^{d_i} = \varepsilon,\,\chi_i \neq \varepsilon} g_{aa_i}(\chi_i)\right\vert \leq \sqrt{p} \ n_i,$$ where $n_i = \mathrm{Card}\,\{\chi_i \neq \varepsilon \ \vert\  \chi_i^{d_i} = \varepsilon \}$.

As $d_i \mid p-1$, there exists exactly $d_i$ characters of order dividing $d_i$, so $n_i = d_i-1$ :

$$\left | \sum\limits_{\chi_i^{d_i} = \varepsilon,\,\chi_i \neq \varepsilon} g_{aa_i}(\chi_i)\right | \leq \sqrt{p} (d_i-1).$$

By Exercise 8.24,
 $$N = p^{n-1}+p^{-1} \sum\limits_{a\in \mathbb{F}_p^*}  \prod\limits_{i=1}^n \sum\limits_{\chi_i^{d_i} = \varepsilon,\chi_i \neq \varepsilon} g_{aa_i}(\chi_i)$$
 
so $$\vert N - p^{n-1}\vert  \leq p^{-1} (p-1) \sqrt{p}(d_1-1)\cdots \sqrt{p}(d_n-1),$$ that is
 
$$\vert N - p^{n-1}\vert  \leq (p-1)(d_1-1)\cdots(d_n-1) p^{\frac{n}{2}-1}.$$
\end{proof}

\paragraph{Ex. 8.26}

{\it  Let $p$ be a prime, $p \equiv 1 \pmod 4$, $\chi$ a multiplicative character of order $4$ on $\F_p$, and $\rho$ the Legendre symbol. Put $J(\chi,\rho) = a + bi$. Show
\begin{enumerate}
\item[(a)] $N(y^2+x^4 = 1) = p-1 + 2a$.
\item[(b)] $N(y^2 = 1 -x^4) = p + \sum\rho(1-x^4)$.
\item[(c)] $ 2a \equiv -(-1)^{(p-1)/4} \binom{2m}{m}   \pmod p$ where $m = (p-1)/4$.
\item[(d)] Verify (c) for $p=13,17,29$.
\end{enumerate}
}

\begin{proof}
\begin{enumerate}
\item[(a)] By Proposition 8.1.5,
\begin{align*}
N(y^2+x^4=1) &=\sum\limits_{a+b=1} N(y^2=a)N(x^4=b)\\
&=\sum\limits_{a+b=1}(1+\rho(a))(1+\chi(b)+\chi^2(b)+\chi^3(b))\\
&=\sum\limits_{i=0}^1 \sum\limits_{j=0}^3\sum\limits_{a+b=1}\rho^i(a) \chi^j(b)\\
&=\sum\limits_{i=0}^1 \sum\limits_{j=0}^3 J(\rho^i, \chi^j)
\end{align*}

As $J(\varepsilon,\varepsilon)=p$, and $0 = J(\varepsilon,\chi) = J(\varepsilon,\chi^2)=J(\varepsilon,\chi^3) = J(\rho,\varepsilon)$, we obtain

$$N(y^2+x^4=1)  = p + J(\rho,\chi)+J(\rho,\chi^2)+J(\rho,\chi^3).$$

As $J(\rho,\chi^3)=J(\bar{\rho},\bar{\chi}) = \overline{J(\rho,\chi)}$,
and
$J(\rho,\chi^2) = J(\rho,\rho) = J(\rho,\rho^{-1}) = -\rho(-1) = -(-1)^{(p-1)/2} = -1$ (since $p\equiv 1 \pmod 4$).
Moreover, by Exercise 8.7,
$$J(\chi,\rho) = \chi(-1) J(\chi,\chi) = \pm J(\chi,\chi) \in \mathbb{Z}[i] : J(\chi, \rho) = a+bi, (a,b) \in \mathbb{Z}^2.$$

Thus $N(y^2+x^4 = 1) = p + 2\re J(\chi,\rho) + J(\rho,\rho) = p-1+2a$

In conclusion, 
$$N(y^2+x^4 = 1) = p-1+2a,  \ \mathrm{where}\  J(\chi,\rho) = a+bi.$$

\item[(b)]  By Exercise 8.8, 
\begin{align*}
\sum\limits_{t \in \mathbb{F}_p} \rho(1-t^4) &= \sum\limits_{\lambda^4 = \varepsilon} J(\rho,\lambda)\\
&=J(\rho,\varepsilon) + J(\rho,\chi)+ J(\rho,\chi^2)+ J(\rho,\chi^3)\\
&=J(\rho,\rho)+ J(\rho,\chi)+J(\rho,\bar{\chi})\\
&=-1 + 2 \re J(\chi,\rho)\\
&=-1+2a
\end{align*}

So $N(y^2=1-x^4) = p-1+2a = p+\sum\limits_{t\in \mathbb{F}_p} \rho(1-t^4)$.


\item[(c)]  Reducing modulo $p$,and writing $\overline{a} \in \F_p$ the class of $a$, we obtain :
\begin{align*}
2\,\overline{a}  &= 1 + \sum\limits_{t\in \mathbb{F}_p} \rho(1-t^4)\\
&= 1 + \sum\limits_{t\in \mathbb{F}_p}(1-t^4)^{\frac{p-1}{2}}\\
&=1 + \sum\limits_{t\in \mathbb{F}_p} \sum\limits_{k=0}^{(p-1)/2} \binom{\frac{p-1}{2}}{k} (-1)^k t^{4k}\\
&= 1 + \sum\limits_{k=0}^{(p-1)/2}(-1)^k \binom{\frac{p-1}{2}}{k}\sum_{t\in \mathbb{F}_p}  t^{4k}\\
&= 1 + \sum\limits_{k=1}^{(p-1)/2}(-1)^k \binom{\frac{p-1}{2}}{k}\sum\limits_{t\in \mathbb{F}_p^*}   t^{4k}.
\end{align*}

Let $S_k = \sum\limits_{t\in \F_p^*}  t^{4k} ,\  k>0 $, and $g$ a generator of $\F_p^*$ : $t = g^i, 0 \leq i \leq p-2$.

$$S_k = \sum\limits_{i=0}^{p-2} (g^{i})^{4k}=\sum\limits_{i=0}^{p-2} (g^{4k})^i.$$

If $g^{4k} \ne 1$, $S_k \equiv \frac{g^{4k(p-1)}-1}{g^{4k}-1} = 0$, if not $S_k = p-1 = -1$.


For every $k$, $1\leq k \leq (p-1)/2$,
$$g^{4k} = 1\iff p-1 \mid 4k \iff \frac{p-1}{4} \mid k \iff k=\frac{p-1}{4} \ \mathrm{or}\  k=\frac{p-1}{2},$$ therefore

$$2a \equiv 1 - (-1)^{\frac{p-1}{4}}\binom{\frac{p-1}{2}}{\frac{p-1}{4}} - (-1)^{\frac{p-1}{2}}\binom{\frac{p-1}{2}}{\frac{p-1}{2}}\pmod p,$$

$$2a \equiv - (-1)^{\frac{p-1}{4}}\binom{\frac{p-1}{2}}{\frac{p-1}{4}}\ \pmod p.$$


\item[(d)] 

 $\bullet$ If $p=13$, I choose the primitive root $g = \overline{2}$, and $\chi$ the character of order 4 defined by $\chi(g) = i$ (the only other character of order 4 is $\overline{\chi}$).
 
Then $J(\chi,\rho) = -3+2i, a = -3$.

$$N=p-1+2a=6.$$

$2a=-6, -(-1)^{\frac{13-1}{4}} \binom{6}{3} = 20$ et $-6\equiv20\pmod {13}$.

$\bullet$ If $p=17$, $g = 3$,  $\chi(g) = i$, $J(\chi,\rho) = -1+4i, a=-1$.

$$N = p-1+2a = 14.$$

$2a=-2, -(-1)^{\frac{17-1}{4}} \binom{8}{4} =-70 \equiv -2\pmod{17}$.

$\bullet$ Si $p=29$, $g = 3$,  $\chi(g) = i$, $J(\chi,\rho) = 5+2i, a=5$.

$$N = p-1+2a = 38.$$

$2a=10, -(-1)^{\frac{29-1}{4}} \binom{14}{7} =3432 \equiv 10\pmod {29}$ ($3422 = 118\times 29$).

\end{enumerate}

\bigskip

Note : By Prop. 8.3.1 (and Ex. 8.7),  $ p = | J(\chi,\rho) |^2 = |J(\chi,\chi)|^2$, so $a^2 + b^2 = p$ (where $p = 4m+1$).

As $\binom{2m}{m} = 2 \binom{2m-1}{m-1}$ is even, and as $p$ is an odd prime $a \equiv \pm \frac{1}{2} \binom{2m}{m} \pmod p$.

Since $p\geq 5$, $p = a^2+b^2$ implies $|a| < \sqrt{p} < p/2$, thus the least remainder of $\frac{1}{2} \binom{2m}{m}$ is $\pm a$. 

Moreover, from Wilson theorem, we obtain a square root of $-1$ in $\F_p$ ($p = 4 m +1$) :
$$-1 \equiv (p-1)! \equiv \left[ (-1)^{(p-1)/4} \left( \frac{p-1}{2}\right)!\right]^2 = [(2m)!]^2.$$
Since $(\overline{b}\overline{a}^{-1})^2 = -\overline{1}$ in $\F_p$, we obtain $ b \equiv (2m)! a \pmod p$. The conclusion is the proposition of Gauss, which  gives an explicit formula for the solution of $p=a^2+b^2$ :

{\bf Proposition} {\it  Let $p$ a prime of the form $p=4m+1$.

If 
\begin{align*}
a &\equiv \frac{1}{2} \binom{2m}{m} \pmod p,\qquad -\frac{p}{2} < a < \frac{p}{2},\\
b &\equiv (2m)!\, a\ \pmod p,\qquad  -\frac{p}{2} < b < \frac{p}{2},
\end{align*}
then $p = a^2+b^2$.
}
\end{proof}

\paragraph{Ex. 8.27}

{\it Let $p\equiv 1 \pmod 3$, $\chi$ a character of order 3, $\rho$ the Legendre symbol. Show
\begin{enumerate}
\item[(a)] $N(y^2 = 1 - x^3) = p + \sum \rho(1-x^3)$.
\item[(b)] $N(y^2 + x^3 = 1) = p + 2 \re J(\chi,\rho)$.
\item[(c)] $2a-b \equiv - \binom{(p-1)/2}{(p-1)/3} \pmod p$ where $J(\chi,\rho) = a + b \omega$.
\end{enumerate}
}

\begin{proof} 


\begin{enumerate} 

\item[(b)] By Exercise 8.15,

$$N(y^2=x^3+D) = p + 2 \re(\pi), \ \mathrm{where}\ \pi =(\rho \chi)(D) J(\chi,\rho).$$

Moreover, with $D = 1$, we obtain
\begin{align*}
N(y^2+x^3=1) &= N(y^2+(-x)^3=1) \\
 &= p+  2 \re J(\chi,\rho).
\end{align*}

$J(\chi,\rho) = \sum\limits_{a+b=1} \chi(a) \rho(b)$, where $\rho(b)\in \{-1,1\}, \chi(a) \in \{1,\omega,\omega^2\}$, so $J(\chi,\rho) \in \mathbb{Z}[\omega]$.

$J(\chi,\rho) = a + b\omega, a\in \mathbb{Z}, b \in \mathbb{Z}$ and $2 \re(J(\chi,\rho)) = 2a-b$.

\item[(a)] By Exercise 8.8,
\begin{align*}
\sum\limits_x\rho(1-x^3) &= \sum\limits_{\lambda^3=\varepsilon} J(\rho,\lambda)\\
&=J(\rho,\varepsilon) + J(\rho,\chi) + J(\rho,\chi^2)\\
&= 2 \re J(\chi,\rho).
\end{align*}
So
\begin{align*}
N(y^2=1-x^3) &= p + \sum_x \rho(1-x^3)\\
&= p+2a-b.
\end{align*}


\item[(c)]Reducing modulo $p$, we obtain  in $\F_p$ :
\begin{align*}
2\overline{a}  -\overline {b} &= \sum\limits_{x} \rho(1-x^3)\\
&=  \sum\limits_{t \in \F_p} (1-t^3)^{\frac{p-1}{2}}\\
&= \sum\limits_{t \in \F_p} \sum\limits_{k=0}^{(p-1)/2} \binom{\frac{p-1}{2}}{k} (-1)^k t^{3k}\\
&=  \sum\limits_{k=0}^{(p-1)/2}(-1)^k \binom{\frac{p-1}{2}}{k}\sum\limits_{t \in \F_p}   t^{3k}\\
&= \sum\limits_{k=1}^{(p-1)/2}(-1)^k \binom{\frac{p-1}{2}}{k}\sum\limits_{t \in \F_p}   t^{3k}. 
\end{align*}

Let $S_k = \sum\limits_{t \in \F_p}   t^{3k} \ (0<k\leq\frac{p-1}{2}) $, and $g$ a primitive root in $\F_p$ : $t = g^i, 0 \leq i \leq p-2$.

$S_k = \sum\limits_{i=0}^{p-2} (g^{i})^{3k}=\sum\limits_{i=0}^{p-2} (g^{3k})^i$

If $g^{3k} \ne 1 , S_k = \frac{g^{3k(p-1)}-1}{g^{3k}-1} = 0$, if not $S_k = p-1 = -1$.

$$g^{3k} = 1 \iff p-1 \mid 3k \iff \frac{p-1}{3} \mid k \iff k=\frac{p-1}{3}$$ and $\frac{p-1}{3}$ even, so

$$2a-b \equiv -\binom{\frac{p-1}{2}}{\frac{p-1}{3}}\pmod p \ \ (\mathrm{where}\ J(\chi,\rho) = a+b \omega).$$
\end{enumerate}
\end{proof}

\paragraph{Ex. 8.28}

{\it Let $p\equiv 3 \pmod 4$ and $\chi$ the quadratic character defined on $\Z/p\Z$. Show
\begin{enumerate}
\item[(a)] $\sum_{x=1}^{p-1} x\chi(x) = 2 \sum_{x=1}^{(p-1)/2} x\chi(x) - p \sum_{x=1}^{(p-1)/2} \chi(x)$.
\item[(b)] $\sum_{x=1}^{p-1} x\chi(x) =  4 \chi(2)\sum_{x=1}^{(p-1)/2} x \chi(x) - p \chi(2) \sum_{x=1}^{(p-1)/2}\chi(x)$.
\item[(c)] If $p\equiv 3 \pmod 8$ then $\sum_{x=1}^{p-1} x\chi(x)/p =- \frac{1}{3} \sum_{x=1}^{(p-1)/2} \chi(x)$.
\item[(d)] If $p \equiv 7 \pmod 8$ then $\sum_{x=1}^{p-1} x\chi(x)/p =- \sum_{x=1}^{(p-1)/2} \chi(x)$.
\end{enumerate}
}
Note : I added two minus signs in (c) and (d) to write a true sentence. See the verification below.
\begin{proof}
\begin{enumerate}
\item[(a)]  $\chi(-1) = (-1)^{(p-1)/2}=-1$, and $\chi(p-x) = \chi(-x) = \chi(-1) \chi(x) = -\chi(x)$, thus
\begin{align*}
\sum_{x=1}^{p-1} x \chi(x) &= \sum_{x=1}^{(p-1)/2} x \chi(x)+ \sum_{x=(p-1)/2+1}^{p-1} x \chi(x)\\
&=\sum_{x=1}^{(p-1)/2} x \chi(x) + \sum_{y=1}^{(p-1)/2} (p-y) \chi(p-y)\qquad (x = p-y)\\
&=\sum_{x=1}^{(p-1)/2} x \chi(x) -\left [ p\sum_{x=1}^{(p-1)/2}  \chi(x) - \sum_{x=1}^{(p-1)/2} x \chi(x)\right ]\\
&= 2 \sum_{x=1}^{(p-1)/2} x \chi(x) - p \sum_{x=1}^{(p-1)/2}  \chi(x).
\end{align*}


\item[(b)] If we separate even and odd indices, we obtain, as $p$ is odd :
\begin{align*}
\sum_{x=1}^{p-1} x \chi(x)  &=\sum_{k=1}^{(p-1)/2} 2k \chi(2k) + \sum_{k=0}^{(p-1)/2 - 1} (2k+1) \chi(2k+1)\\
&= \sum_{x=1}^{(p-1)/2} 2x \chi(2x)+\sum_{x=1}^{(p-1)/2}(p-2x) \chi(p-2x)\qquad (2k+1 = p - 2x) \\
&=2 \chi(2)\sum_{x=1}^{(p-1)/2} x \chi(x) - \chi(2) \sum_{x=1}^{(p-1)/2} (p-2x) \chi(x)\\
&=4 \chi(2) \sum_{x=1}^{(p-1)/2} x \chi(x) - p\chi(2) \sum_{x=1}^{(p-1)/2}  \chi(x).
\end{align*}

\item[(c)] Let $S= \sum\limits_{x=1}^{(p-1)/2}  \chi(x), T = \sum\limits_{x=1}^{(p-1)/2} x \chi(x)$. Then
\begin{align*}
(a) \sum_{x=1}^{p-1} x \chi(x) &= 2 T - p S,\\
(b) \sum_{x=1}^{p-1} x \chi(x) &= 4 \chi(2) T - p \chi(2) S.
\end{align*}

Subtracting theses equalities, we obtain

\begin{align*}
(4\chi(2)-2 )T &= p(\chi(2) - 1) S,\\
\frac{T}{p} &= \frac{\chi(2) - 1}{4\chi(2) - 2} S.
\end{align*}

$\chi(2) = (-1)^{(p^2-1)/8} = -1 $ if $p\equiv 3\pmod 8$, so  $\frac{\chi(2) - 1}{4\chi(2) - 2} = \frac{1}{3}$, and $T/p = (1/3) S$.
$$\sum_{x=1}^{ p-1} x \chi(x) /p = 2T/p - S =  -(1/3)S,$$
$$\sum_{x=1}^{ p-1} x \chi(x) /p = -\frac{1}{3} \sum\limits_{x=1}^{(p-1)/2}  \chi(x).$$

\item[(d)] 
$\chi(2) = (-1)^{(p^2-1)/8}  = 1$ if $p\equiv 7\pmod 8$ : $\frac{\chi(2) - 1}{4\chi(2) - 2} = 0$.
$$\sum_{x=1}^{p-1} x \chi(x) /p = - \sum_{x=1}^{(p-1)/2}  \chi(x).$$
Verification : with $p=7$, the squares are $1,4,2=9$, so
\begin{align*}
\sum_{x=1}^{p-1} x \chi(x) /p  &= \frac{1}{7} \left[ \legendre{1}{7} + 2 \legendre{2}{7} + 3 \legendre{3}{7} + 4 \legendre{4}{7} + 5 \legendre{5}{7} + 6 \legendre{6}{7} \right]\\
&= \frac{1}{7} \left( 1 + 2 -3+4-5-6\right) = -1\\
\sum_{x=1}^{(p-1)/2}  \chi(x)&=\legendre{1}{7} + \legendre{2}{7} + \legendre{3}{7}\\
&= 1 + 1 - 1 = 1.
\end{align*}
With $p = 3$,
\begin{align*}
\sum_{x=1}^{p-1} x \chi(x) /p  &= \frac{1}{3} \left[ \legendre{1}{3} + 2 \legendre{2}{3} \right]\\
&= \frac{1}{3} \left( 1  - 2 \right)= -\frac{1}{3}\\
\sum_{x=1}^{(p-1)/2}  \chi(x)&=\legendre{1}{3}\\
&= 1 
\end{align*}
This confirms the misprints  in the initial sentence.
\end{enumerate}
\end{proof}


\end{document}
