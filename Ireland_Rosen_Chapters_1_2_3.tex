%&LaTeX
\documentclass[11pt,a4paper]{article}
\usepackage[frenchb,english]{babel}
\usepackage[applemac]{inputenc}
\usepackage[OT1]{fontenc}
\usepackage[]{graphicx}
\usepackage{amsmath}
\usepackage{amsfonts}
\usepackage{amsthm}
\usepackage{amssymb}
\usepackage{yfonts}
%\input{8bitdefs}

% marges
\topmargin 10pt
\headsep 10pt
\headheight 10pt
\marginparwidth 30pt
\oddsidemargin 40pt
\evensidemargin 40pt
\footskip 30pt
\textheight 670pt
\textwidth 420pt

\def\imp{\Rightarrow}
\def\gcro{\mbox{[\hspace{-.15em}[}}% intervalles d'entiers 
\def\dcro{\mbox{]\hspace{-.15em}]}}

\newcommand{\D}{\mathrm{d}}
\newcommand{\Q}{\mathbb{Q}}
\newcommand{\Z}{\mathbb{Z}}
\newcommand{\N}{\mathbb{N}}
\newcommand{\R}{\mathbb{R}}
\newcommand{\C}{\mathbb{C}}
\newcommand{\F}{\mathbb{F}}
\newcommand{\ord}{\mathrm{ord}}
\newcommand{\legendre}[2]{\genfrac{(}{)}{}{}{#1}{#2}}



\title{Solutions to Ireland, Rosen ``A Classical Introduction to Modern Number Theory''}
\author{Richard Ganaye}
\begin{document}

\maketitle

{\Large \bf Chapter 1}


\paragraph {  Ex 1.1} 

{\it Let $a$ and $b$ be nonzero integers. We can find nonzero integers $q$ and $r$ such that $a = qb + r$ where $0 \leq r < b$. Prove that $(a, b) = (b, r)$.}

\begin{proof}
Notation : if $a,b$ are integers in  $\mathbb{Z}$, $a \wedge b$ is the non negative greatest common divisor of  $a,b$, the generator in  $\mathbb{N} = \{0,1,2,\ldots\} $ of the ideal $(a,b) = a\mathbb{Z}+ b \mathbb{Z}$.


Let $d \in \mathbb{Z}$.

$\bullet$ If $d \mid a, d \mid b$, then  $d \mid a-qb = r$, so $d \mid b, d \mid r$.

$\bullet$ If $d \mid b, d \mid r$, then $d \mid qb+r = a$, so $d \mid a, d \mid b$.

$$ \forall d \in \mathbb{Z}, \  ( d \mid b, d \mid r)  \iff (d \mid a, d \mid b).$$

If $a = bq+r$, the set of common divisors of $a,b$ is equal to the set of common divisors of $b,r$.

As $a \wedge b$ is the smallest positive element of this set, so is  $b \wedge r$, we conclude that $a \wedge b = b \wedge r$.
\end{proof}

\paragraph { Ex 1.2} 

{\it If $r\neq 0$, we can find $q_1$ and $r_1$ such that $b = q_1r+r_1$, with $0 \leq r_1 < r$. Show that $(a,b) = (r,r_1)$. This process can be repeated. Show that it must end in finitely many steps. Show that the last nonzero remainder must equal $(a,b)$. The process looks like
\begin{align*}
a &= b q + r,  &0 \leq r <b\\
b &= q_1 r + r_1, &0 \leq r_1 <r\\
r &= q_2 r_1 + r_2,  &0\leq r_2 <r_1\\
&\vdots\\
r_{k-1} &= q_{k+1} r_k + r_{k+1}, &0 \leq r_{k+1} <r_k\\
r_k &= q_{k+2} r_{k+1}\\
\end{align*}

Then $r_{k+1} = (a,b)$. This process of finding $(a,b)$ is known as the Euclidian algorithm.

\begin{proof}
The Euclidian division of $b$ by $r$ gives $b = q_1 r + r_1, 0\leq r_1 < r$. The result of exercise 1.1 applied to the couple $(b,r)$ shows that
$$b \wedge r = r \wedge r_1.$$
Let $N \in \mathbb{N}$. While the remainders $r_i, i \leq N$, are not equal to 0, we can define the sequences $(q_i), (r_i)$ by

$$r_{-1} = a, r_0 = b, \qquad r_{i-1} = q_{i+1} r_i  + r_{i+1},\  0\leq r_{i+1} < r_i \quad  0\leq i \leq N$$.

If no  $r_i, i\in \mathbb{N}$, is equal to 0, we can continue this construction indefinitely. So we obtain a strictly decreasing sequence $(r_i)_{i \in \mathbb{N}}$ of positive numbers : it is impossible. Therefore, there exists an index $k$ such as $r_{k+2} = 0$, this is the end of the algorithm.

\begin{align*}
a &= b q + r,  &0 \leq r <b\\
b &= q_1 r + r_1, &0 \leq r_1 <r\\
r &= q_2 r_1 + r_2,  &0\leq r_2 <r_1\\
&\vdots\\
r_{k-1} &= q_{k+1} r_k + r_{k+1}, &0 \leq r_{k+1} <r_k\\
r_k &= q_{k+2} r_{k+1}, &r_{k+2} = 0\\
\end{align*}

From exercise 1,  $r_{i-1} \wedge r_{i} = r_{i} \wedge r_{i+1}, 0 \leq i \leq k$,  so

$$a \wedge b = b \wedge r = \cdots = r_k \wedge r_{k+1} = r_{k+1} \wedge r_{k+2} =  r_{k+1} \wedge 0 = r_{k+1}.$$

The last non zero remainder is the gcd of $a,b$. 
\end{proof}

\paragraph { Ex 1.3}
{\it Calculate $(187, 221)$, $(6188, 4709)$, $(314, 159)$.}
\begin{proof}
With direct instructions in Python, we obtain :

\begin{verbatim}
>>> a, b = 187,  221
>>> print("q = ",a//b); a, b = b, a%b; print(a,b)
q =  0
221 187
>>> print("q = ",a // b); a, b = b, a%b; print(a,b)
q =  1
187 34
>>> print("q = ",a // b); a, b = b, a%b; print(a,b)
q =  5
34 17
>>> print("q = ",a // b); a,b = b, a%b; print(a,b)
q =  2 
17 0
\end{verbatim}
This gives the equalities
\begin{align*}
187 &= 0 \times 221 + 187\\
221 &= 1\times187 + 34\\
187 &= 5 \times 34 + 17\\
34 &= 2 \times 17 + 0
\end{align*}

So $187 \wedge  221  = 17$.

With the same instructions, we obtain 
\begin{align*}
 6188 &= 1 \times 4709 + 1479\\
4709 &= 3 \times 1479 + 272\\
1479 &= 5 \times 272 + 119\\
272&=2\times119 + 34\\
119 &= 3 \times 34+ 17\\
34 &= 2 \times 17 + 0\
\end{align*}
$6188 \wedge 4709 = 17$.

Finally
\begin{align*}
 314 &=1 \times159+155\\
159 &= 1 \times 155 + 4\\
155 &=38\times 4 + 3\\
4&= 1\times 3 + 1\\
3 &=3 \times 1+0\\
\end{align*}

$314 \wedge 159 = 1$.

The Python script which gives the gcd is very concise :
\begin{verbatim}
def gcd(a,b):
    a, b = abs(a), abs(b)
    while b != 0:
        a, b = b, a % b 
    return a
\end{verbatim}
\end{proof}

\paragraph {  Ex 1.4} 

{\it Let $d = (a, b)$. Show how one can use the Euclidean algorithm to find numbers $m$ and $n$ such that $am + bn = d$.(Hint: In Exercise 2 we have that $d = r_{k+1}$. Express $r_{k+1}$ in terms of $r_k$ and $r_{k+1}$, then in terms of $r_{k-1}$ and $r_{k-2}$, etc.).
}
\begin{proof}
With a slight modification of the notations of exercise 2, we note the Euclid's algorithm under the form

$$r_0 = a, r_1 = b, \qquad r_i = r_{i+1} q_{i+1} + r_{i+2} , \quad 0 <  r_{i+2} < r_{i+1},\ 0 \leq i < k, \quad r_k =q_{k+1}r_{k+1},\ r_{k+2} = 0$$

We show by induction on $i$ ($i\leq k + 1$) the proposition
$$P(i)  :  \exists (m_i, n_i) \in \Z \times \Z,\  r_i = am_i +b n_i.$$

$\bullet$ $r_0 = a = 1.a + 0 .b$. Define $m_0 = 1, n_0 = 0$. We obtain $r_0 = a m_0 +b n_0$,  then $P(0)$ is true.

$r_1 = b = 0.a + 1 .b$. Define $m_1 = 0, n_1 = 1$. We obtain $r_1 = a m_1 +b n_1$,  then $P(1)$ is true.

$\bullet$ Suppose for $0\leq i<k$ the induction hypothesis $P(i)$ et $P(i+1)$ :
\begin{align*}
r_i &= am_i +b n_i ,\quad &m_i,n_i \in \mathbb{Z},\\
r_{i+1} &= am_{i+1} + b n_{i+1},\quad  &m_{i+1},n_{i+1} \in \mathbb{Z}.
\end{align*}
Then  $r_{i+2} = r_i - r_{i+1} q_{i+1} = a(m_i - q_{i+1} m_{i+1}) + b (n_i - q_{i+1} n_{i+1})$.

If we define $m_{i+1} = m_i - q_{i+1} m_{i+1}, n_{i+1} = n_i - q_{i+1} n_{i+1}$, we obtain  $r_{i+2} = am_{i+2} + b n_{i+2} ,\ m_{i+2}, n_{i+2} \in \Z$, so $P(i+2)$.

$\bullet$ The conclusion is that  $P(i)$ is true for all $i, 0 \leq i \leq k+1$, in particular $r_{k+1} = a m_{k+1} + b n_{k+1}$, that is 
$$a\wedge b = d = am+b n,$$
where $m = m_{k+1}, n = n_{k+1} \in \Z$.
\end{proof}

\paragraph {  Ex 1.5} 
{\it Find $m$ and $n$ for the pairs $a$ and $b$ given in Ex 1.3}
\begin{proof}
From exercises 1.3,  1.4, we know that the sequences $(r_i),(m_i),(n_i)$ are given by
\begin{align*}
r_0 &= a ,r_1 = b\\
m_0 &= 1, m_1 = 0\\
n_0 &= 0,n_1 = 1\\
\end{align*}
and for all $i <k$, 
\begin{align*}
r_{i+2}&= r_{i} -q_{i+1} r_{i+1}\\
m_{i+2}&= m_{i} -q_{i+1} m_{i+1}\\
n_{i+2}&= n_{i} -q_{i+1} n_{i+1}\\
\end{align*}
and for all $i$ 
$$r_{i} = m_i a +n_i b.$$
This gives the direct instructions in Python :
\begin{verbatim}
>>> a,b = 187, 221
>>> r0,r1,m0,m1,n0,n1 = a,b,1,0,0,1  
>>> q = r0//r1; 
>>> q = r0//r1; r0,r1,m0,m1,n0,n1 = r1, r0 -q*r1,m1, m0 -q*m1, n1, n0 - q*n1
>>> print(r0,r1,m0,m1,n0,n1)
221 187 0 1 1 0
>>> q = r0//r1; r0,r1,m0,m1,n0,n1 = r1, r0 -q*r1,m1, m0 -q*m1, n1, n0 - q*n1
>>> print(r0,r1,m0,m1,n0,n1)
187 34 1 -1 0 1
>>> q = r0//r1; r0,r1,m0,m1,n0,n1 = r1, r0 -q*r1,m1, m0 -q*m1, n1, n0 - q*n1
>>> print(r0,r1,m0,m1,n0,n1)
34 17 -1 6 1 -5
>>> q = r0//r1; r0,r1,m0,m1,n0,n1 = r1, r0 -q*r1,m1, m0 -q*m1, n1, n0 - q*n1
>>> print(r0,r1,m0,m1,n0,n1)
17 0 6 -13 -5 11
\end{verbatim}
So
 $$17 = 187 \wedge 221 = 6 \times 187 - 5 \times 221.$$
Similarly
$$ 17 = 6188 \wedge 4709 = 121 \times 6188 - 159 \times 4709.$$
$$1 = 314 \wedge 159 = -40 \times 314 + 79 \times159.$$

We obtain the same results with the following Python script :
\begin{verbatim}
def bezout(a,b):
    """input  : entiers a,b
        output : tuple (x,y,d),
        (x,y) solution de ax+by = d, d = pgcd(a,b)
    """
    (r0,r1)=(a,b)
    (u0,v0) = (1,0)
    (u1,v1) = (0,1)
    while r1 != 0:
        q = r0 // r1
        (r2,u2,v2) = (r0 - q*r1,u0 - q*u1,v0 - q*v1)
        (r0,r1) = (r1,r2)
        (u0,u1) = (u1,u2)
        (v0,v1) = (v1,v2)
    return (u0,v0,r0)
\end{verbatim}
\end{proof}

\paragraph {  Ex 1.6} 
{\it Let $a, b, c \in \Z$. Show that the equation $ax + by = c$ has solutions in integers iff $(a, b) |c$.
 }
 \begin{proof}
 Let $d = a \wedge b$. 
 
 $\bullet$ If $a x + by = c, x,y \in \Z$, as $d \mid a, d \mid b$, $d \mid ax + by = c$.
 
 $\bullet$ Reciprocaly, if $d \mid c$, then $c = d c',\ c' \in \Z$. 
 
 From Prop. 1.3.2., $d\Z = a \Z + b\Z$, so $d = a u + b v,\ u,v \in \Z$, and $c = dc' = a (c'u) + b (c'v) =a x + b y$, where $x = c' u, y = c' v$ are integers.
 
 Conclusion : 
 
 $$\exists (x,y) \in \Z \times \Z, \ ax + by = c \iff a\wedge b \mid c.$$
\end{proof}

\paragraph {  Ex 1.7} 
{\it Let $d = (a, b)$ and $a = da'$ and $b = db'$. Show that $(a', b') = 1$.}

\begin{proof}
Suppose $d \neq 0$ (if $d = 0$, then $a= b = 0$, and $a', b'$ are any numbers in $\Z$ and the result may be false, so we must suppose $d \neq 0$).

As $d = am + b n,\ m,n \in \Z$, $d = d(a'm + b' n)$, so $1 = a' m +b' n$, which proves $a' \wedge b' = 1$.

conclusion : if $d = a \wedge b \neq 0$, and $a = da', b = db'$, then $a'\wedge b' = 1$.

\end{proof}
\paragraph{Ex. 1.8}

{\it Let $x_0$ and $y_0$ be a solution to $ax + by = c$. Show that all solutions have the form $x = x_0 + t(b/d)$, $y = y_0 - t(a/d)$, where $d = (a, b)$ and $t \in \Z$.
}

\begin{proof}
Suppose $a\neq 0, b \neq 0$.

Let $x_0$ and $y_0$ be a solution to $ax + by = c$.

If $(x,y)$ is any solution of the same equation,
\begin{align*}
ax + by &= c\\
ax_0 +by_0 &= c,\\
\end{align*}
then $$a(x-x_0) = -b(y-y_0),$$
so $$\frac{a}{d}(x-x_0) = -\frac{b}{d} (y-y_0).$$
Let $a' = a/d, b' = b/d$ : from ex. 1.7, we know that $a'\wedge b' = 1$.

As $a'(x-x_0) =- b'(y-y_0)$, $b' \mid a'(x-x_0)$, and $b'\wedge a' = 1$, so (Gauss' Lemma : prop. 1.1.1) $b' \mid x-x_0$.

There exists $t \in \Z$ such that $x-x_0 = tb'$. Then $a'tb' = -b'(y-y_0)$. As $b \neq 0$, $b' \neq 0$, so $a't = -(y-y_0)$ :
\begin{align*}
x = x_0 + t(b/d)\\
y = y_0 - t (a/d)
\end{align*}
Reciprocaly, $a(x_0 +t(b/d)) + b(y_0 - t(a/d)) = ax_0 + by_0 =c$.

Conclusion : if $a\neq 0, b \neq 0$, and $ax_0+by_0 = c$,
$$ax+by = c \iff \exists t \in \Z,\ x = x_0 + t(b/d), y = y_0 - t (a/d).$$
\end{proof}


\paragraph{Ex. 1.9}

{\it Suppose that $u, v \in \Z$ and that $(u, v) = 1$. If $u \mid n$ and $v\mid n$, show that $uv \mid n$. Show that this is false if $(u, v) \ne 1$.
}

\begin{proof}
As $u \mid n$, $n = uq,q \in \Z$, so $v \mid n = uq$, and $v \wedge u = 1$, so (Gauss' lemma : prop. 1.1.1), $v \mid q$ : $q = v l, l \in \Z$, and $n = uv l$ : $uv \mid n$.

If the case $u \wedge v \neq 1$, we give the counterexample $6 \mid 18,9 \mid 18$, but $6 \times 9 \nmid 18$.
\end{proof}

\paragraph{Ex. 1.10}

{\it Suppose that $(u, v) = 1$. Show that $(u+v, u-v)$ is either 1 or 2.
}

\begin{proof} Let $d = (u + v) \wedge (u- v)$. Then $d\mid u+v,d \mid u-v$, so $d \mid 2u = (u+v)+(u-v)$ and $d \mid 2v = (u+v) - (u-v)$. So $d \mid (2u) \wedge (2v) = 2(u\wedge v) = 2$. As $d\geq 0$, $d =1$ or $d = 2$.
\end{proof}

\paragraph{Ex. 1.11}

{\it Show that $(a, a+k)\mid k$.
}

\begin{proof} Let $d = a \wedge (a+k)$. As $d \mid a, d \mid (a+k)$, $d \mid k = (a+k) - a$. 

Conclusion : $a \wedge (a+k) \mid k.$
\end{proof}

 \paragraph{Ex. 1.12}

{\it

Suppose that we take several copies of a regular polygon and try to fit them evenly about a common vertex. Prove that the only possibilities are six equilateral triangles, four squares, and three hexagons.
}

\begin{proof}
Let $n$ be the number of sides of the regular polygon, $m$ the number of sides starting from a summit in the lattice, $\alpha$ the measure of the exterior angle, $\beta$ the measure of the interior angle (in radians) ($\alpha + \beta = \pi$).

Then $\alpha = 2\pi/n, \beta = \pi - 2\pi /n$.

$m\beta = 2\pi, m(\pi - 2 \pi /n) = 2 \pi, m(1-2/n) = 2$, so
\begin{align}
\frac{1}{m} + \frac{1}{n} = \frac{1}{2}, \qquad m>0,n>0.
\end{align}
As this equation is symmetric in $m,n$, we may suppose first $m\leq n$.

 In this case $1/m \geq 1/n$, so $2/n \leq 1/2$ : $n\geq 4$.
 
 If $n>6$, $1/n<1/6,1/m = 1/2 - 1/n > 1/2 -1/6 = 1/3$, so $m<3,m\leq 2$ : $m=1$ or $m=2$.

If $m=1$, $n<0$ : it is impossible. If $m=2$, $1/n = 0$ : also impossible. Therefore $n\leq 6$ : $4\leq n \leq 5$.
If $n = 4,m=4$. if $n = 5$, $n = 10/3$ : impossible. if $n=6$, $m=3$. 
Using the symetry, the set of solutions of (1) is
$$S = \{(3,6),(6,3),(4,4)\},$$
corresponding with the usual lattices composed of equilateral triangles, squares or hexagons.
\end{proof}

\paragraph{Ex. 1.13}

{\it Let $n_1,n_2,\ldots,n_s\in \Z$. Define the greatest common divisor $d$ of $n_1,n_2,\ldots,n_s$ and prove that there exist integers $m_1,m_2,\ldots,m_s$ such that $n_1m_1+n_2m_2 \cdots+n_sm_s = d.$
}

\begin{proof}
Let $n_1,n_2,\ldots,n_s\in \Z$. The ideal of $\Z$, $(n_1,\ldots,n_s) = n_1\Z+\cdots+n_s\Z$ is principal, so there exists an unique $d\in \Z, d\geq 0$ such that
$$n_1\Z+\cdots+n_s\Z = d\, \Z\quad (d\geq 0).$$
We define
\begin{align}
d = \mathrm{gcd}(n_1,\ldots,n_s)  \iff n_1\Z+\cdots+n_s\Z = d\, \Z \ \mathrm{and}\ d\geq 0.
\end{align}
The characterization of the gcd is

$$d = \mathrm{gcd}(n_1,\ldots,n_s)  \iff$$
\begin{align}
(i) &\ d \geq 0\\
(ii)&\ d\mid n_1,\ldots, d \mid n_s\\
(iii)&\  \forall \delta \in \Z,\ (\delta \mid n_1,\ldots,\delta \mid n_s) \Rightarrow \delta \mid d
\end{align}

$(\Rightarrow)$ Indeed, if we suppose (1), then $d\geq 0$, and $n_1 = n_1.1 + n_2.0 + \cdots+n_s.0 \in n_1\Z+\cdots+n_s\Z = d\, \Z$, so $d\mid n_1$. Similarly $d\mid n_i, 1\leq i \leq s$ so (i)(ii) are true. if $\delta \mid n_i, 1 \leq i \leq s$, as $d = n_1m_1+\cdots+n_sm_s, m_1,\ldots,m_s \in \Z$, then $\delta \mid d$.

$(\Leftarrow)$ Suppose that $d$ verify (i)(ii)(iii). From (ii), we see that $n_i \Z \subset d \Z, i= 1,\ldots,s$, so $n_1\Z+\cdots+n_s \Z \subset d\Z$.

As $\Z$ is a principal ring, there exists $\delta\geq 0$ such that $n_1\Z+\cdots+n_s\Z = \delta\, \Z$. $n_i \in n_1\Z+\cdots+n_s\Z$ so $n_i \in \delta \Z, \ i=1,\ldots,s$ : $\delta \mid n_1,\ldots, \delta \mid n_s$. From (iii), we deduce $\delta \mid d$. As $\delta \Z \subset d\Z$, $d \mid \delta$, with $d\geq 0, \delta \geq 0$. Consequently, $d = \delta$ and  $n_1\Z+\cdots+n_s\Z =d\, \Z, d\geq 0$, so $d =\mathrm{gcd}(n_1,\ldots,n_s) $.

At last, as $n_1\Z+\cdots+n_s\Z = d\, \Z$, there exist integers  $m_1,m_2,\ldots,m_s$ such that $n_1m_1+n_2m_2+\cdots+n_sm_s = d$.
\end{proof}

\paragraph{Ex. 1.14} 

{\it Discuss the solvability of $a_1x_1+a_2x_2+\cdots + a_rx_r = c$ in integers. (Hint: Use Exercise 13 to extend the reasoning behind Exercise 6.)
}

\begin{proof}
Let $a_1,a_2,\ldots, a_r \in \Z$.

Note $\mathrm{gcd}(a_1,a_2,\ldots,a_r) = a_1\wedge a_2\wedge \cdots \wedge a_r$. The following result generalizes Ex. 6 :

$$\exists (x_1,x_2,\ldots,x_r)\in \Z^r, \  a_1x_1+a_2x_2+\cdots + a_rx_r = c\iff a_1\wedge a_2\wedge \cdots \wedge a_r \mid c.$$
Let $d = a_1\wedge a_2\wedge \cdots \wedge a_r$.

$\bullet$ If $a_1x_1+a_2x_2+\cdots + a_rx_r = c$, as $d\mid a_1,\ldots,d\mid a_r$, $d \mid a_1x_1+a_2x_2+\cdots + a_rx_r = c$.

$\bullet$ Reciprocaly, if $d \mid c$, then $c = d c', c' \in \Z$.

As $d\Z = a_1\Z+a_2\Z+\cdots+a_r\Z$, so $d = a_1 m_1+a_2m_2+\cdots+a_rm_r,\ m_1,m_2,\ldots,m_r \in \Z$.
$c = dc' = a_1(m_1c')+\cdots a_r(m_rc') = a_1x_1+\cdots + a_r x_r$, where $x_i = m_ic',i=1,2,\ldots,r$.
\end{proof}

\paragraph{Ex. 1.15}

{\it Prove that $a \in \Z$ is the square of another integer iff $\mathrm{ord}_p(a)$ is even for all primes $p$.  Give a generalization.
}

\begin{proof}
 Suppose $a = b^2, b\in \Z$. Then $\mathrm{ord}_p(a) = 2\, \mathrm{ord}_p(b)$ is even for all primes $p$.

 Conversely, suppose that $\mathrm{ord}_p(a)$ is even for all primes $p$. We must also suppose $a> 0$. Let $a =  \prod\limits_p p^{a(p})$ the decomposition of $a$ in primes. As $a(p)$ is even, $a(p) = 2 b(p)$ for an integer $b(p)$ function of the prime $p$. Let $b =  \prod\limits_p p^{b(p)}$. Then $a = b^2$.
 
 With a similar demonstration, we obtain the following generalization for each integer $a \in \Z, a > 0$ :
 
 $a = b^n$ for an integer $b \in \Z$ iff $n \mid \mathrm{ord}_p(a)$ for all primes $p$.
\end{proof}

\paragraph{Ex. 1.16}

{\it If $(u, v) = 1$ and $uv = a^2$, show that both $u$ and $v$ are squares.
}

\begin{proof}
Here $u,v \in \N$, where $\N = \{0,1,2,\ldots\}$.

For all primes $p$ such that $p \mid u$, $\mathrm{ord}_p(u) + \mathrm{ord}_p(v) = 2\  \mathrm{ord}_p(a)$.
As $u\wedge v = 1$ and $p\mid u$, $p \nmid v$, so $\mathrm{ord}_p(v) =0$. Consequently, $\mathrm{ord}_p(u) $ is even for all prime $p$ such that $p \mid u$. From Exercise 1.15, we can conclude that $u$ is a square. Similarly, $v$ is a square.
\end{proof}

\paragraph{Ex. 1.17}

{\it Prove that the square root of 2 is irrational, i.e., that there is no rational number $r = a/b$ such that $r^2 = 2$.
}

\begin{proof}
  Suppose there exists $r \in \Q, r>0$ such that $r^2 = 2$. Then $r = a/b, a \in \N^*, b\in \N^*$. With $d = a\wedge b$, $a = da', b = db', a'\wedge b' = 1$, so $r = a'/b', a' \wedge b' = 1$, so we may suppose $r = a/b , a>0,b>0,a\wedge b = 1$ and $a^2 = 2 b^2$.
  
 $a^2$ is even, then $a$ is even (indeed, if $a$ is odd, $a = 2k+1, k\in \Z$, $a^2 = 4k^2+4k+1 = 2(2k^2+2k) +1$ is odd).
 
 So $a = 2A, A\in \N$, then $4A^2 = 2 b^2$, $2A^2 = b^2$. 
 
 With the same reasoning, $b^2$ is even, then $b$ is even : $b = 2B, B \in \N$. $2\mid a, 2 \mid b$, $2 \mid a\wedge b$,  in contradiction with $a\wedge b = 1$.
 
 Conlcusion  : $\sqrt{2}$ is irrational.
\end{proof}

\paragraph{Ex. 1.18}

{\it Prove that $\sqrt[n]{m}$ is irrational if $m$ is not the $n$-th power of an integer.
}

\begin{proof}
Here $m \in \N$.

Suppose that $r = \sqrt[n]{m} \in \Q$. As $r\geq 0$, $r = a/b, a\geq 0, b >0, a\wedge b = 1$, and $r^n = m$, so $a^n = m b^n$.

For all primes $p$, $n\  \mathrm{ord}_p(a) = \mathrm{ord}_p(m)  + n\  \mathrm{ord}_p(b)$, so $n \mid \mathrm{ord}_p(m) $.

From Ex. 1.15, we conclude that $m$ is a $n$-th power.

Conclusion : if $m \geq 0$ is not the $n$-th power of an integer, $\sqrt[n]{m}$ is irrational.  
\end{proof}

\paragraph{Ex. 1.19}

{\it Define the least common multiple of two integers $a$ and $b$ to be an integer $m$ such that $a \mid m$, $b \mid m$, and $m$ divides every common multiple of $a$ and $b$. Show that such an $m$ exists. It is determined up to sign. We shall denote it by $[a, b]$.
}

\begin{proof}
  As $a\Z \cap b \Z$ is an ideal of $\Z$, and $\Z$ is a principal ideal domain, there exists an unique $m\geq 0$ such that $a\Z \cap b\Z = m \Z$. So by definition,
  $$m = [a,b] \iff a\Z \cap b \Z = m \Z \ \mathrm{and}\ m \geq 0.$$
  We may note also $[a,b] = a \vee b$.
  
  characterization of lcm :
  \begin{align*}
  m = a \vee b &\iff\\
  (i)\ & m \geq 0\\
  (ii)\  &a \mid m, b \mid m\\
  (iii)\  &\forall \mu \in \Z, (a \mid \mu, b \mid \mu) \Rightarrow m \mid \mu
  \end{align*}
  
  $(\Rightarrow)$ By definition, $m\geq 0$. $m \in m\Z = a\Z \cap b \Z$, so $a \mid m$ and $b \mid m$ : (ii) is verified.
  If $\mu \in \Z$ is such that $a\mid \mu, b \mid \mu$, then $\mu \in a \Z \cap b \Z = m \Z$, so $m \mid \mu$ : (iii) is true.
  
  $(\Leftarrow)$ Suppose that $m$ verifies  (i),(ii),(iii). Let $m'$ such that $a\Z \cap b \Z = m' \,\Z, m'\geq 0$. We show that $m=m'$.
  
  As $m' \in a \Z \cap b \Z, a \mid m', b \mid m'$, so from (iii) $m \mid m'$. From (ii), we see that $m \in a \Z \cap b\Z = m' \Z$, so $m' \mid m, m\geq 0, m' \geq 0$. The conclusion is $m=m'$ and $a\Z \cap b \Z = m \Z, m\geq 0$, so $m = a \vee b$.
\end{proof}

\paragraph{Ex. 1.20}

{\it Prove the following:

(a) $\ord_p[a,b] = \max(\ord_p(a), \ord_p(b))$.

(b) $(a,b)[a,b] = ab$.

(c) $(a+b, [a,b] )= (a,b)$.
}

\begin{proof}
(a) Let $a = \varepsilon \prod\limits_p p^{a(p)},b =  \varepsilon'  \prod\limits_p p^{b(p)}, \varepsilon, \varepsilon' = \pm1$, and 
$$m =  \prod\limits_p p^{\max(a(p),b(p))}.$$
Then 

(i) $m \geq 0$.

(ii) As $a(p) \leq \max(a(p),b(p))$, $p^{a(p)} \mid p^{\max(a(p),b(p))}$, so $a \mid m$. Similarly, $b\mid m$.

(iii) If $\mu = \varepsilon'' \prod\limits_p p^{c(p)}$ is a common multiple of $a$ and $b$, then for all primes $p$,  $a(p) \leq c(p), b(p) \leq c(p)$, so $\max(a(p),b(p) \leq c(p)$, so $m \mid \mu$.
$m$ verifies the characterisation of lcm :

$$m = a \vee b =  \prod\limits_p p^{\max(a(p),b(p))}.$$
So $\ord_p[a,b] = \max(\ord_p(a),\ord_p(b))$.

(b) Similarly, we prove that 
$$a \wedge b = \prod\limits_p p^{\min(a(p),b(p))}.$$
As $\max(a,b) + \min(a,b) = a +b$, we obtain
$$(a \vee b)(a \wedge b) = \vert ab \vert.$$

Second proof (without decompositions in primes) :


Let $d = a \wedge b$. If $d = 0$, then $a=b= 0$ and $(a\vee b) (a \wedge b) = ab$.

Suppose now that $d \neq 0$. There exist integers $a',b'$ such that $$a = da',b = d b', a' \wedge b' = 1.$$

Let $m = da'b'$ : 
$a = da' \mid m$ and $b = db' \mid m$. If $\mu$ is a common multiple of $a$ and $b$, then $d \mid \mu$, and $a' \mid \mu/d, b' \mid \mu/d$. As $a'\wedge b' = 1$, $a'b' \mid \mu/d$ (see Ex.1.9). so $m = da'b' \mid \mu$.

$\vert m \vert$ verifies the characterization of lcm (Ex. 1.19), so $a\vee b = \vert m \vert = \vert  da'b'  \vert= \vert ab\vert /d$.
 Conclusion : $ (a\vee b)(a \wedge b) = \vert ab \vert$.
 
 (c) Let $\delta \in \Z$. If $\delta \mid a, \delta \mid b$, then $\delta \mid a+b$ and $\delta \mid a \vee b$.
 
 Conversely, suppose that $\delta \mid a+b, \delta \mid a \vee b$.
 
 Let $a',b' \in \Z$ such that $a = da',b = db', a' \wedge b' = 1$. Then $a \vee b = da'b'$, so 
 \begin{align*}
& \delta \mid d(a'+b'),\\
 &\delta \mid da'b'.
 \end{align*}
 Multiplying the first relation by $b'$ and $a'$, we obtain : 
 $\delta \mid da'b' +db'^2, \delta \mid da'^2 +da'b'$. As $\delta \mid da'b'$, we obtain :
 \begin{align*}
 & \delta \mid d b'^2\\
 &\delta \mid d a'^2
 \end{align*}
 As $a'^2\wedge b'^2 = 1$, $\delta \mid d(a'^2 \wedge b'^2) =d$, so $\delta \mid a, \delta \mid b$.

The set of divisors of $a,b$ is the same that the set of divisors of $a+b,a\vee b$, so
$$(a+b ) \wedge( a \vee b) = a \wedge b.$$
\end{proof}

\paragraph{Ex. 1.21}

{\it Prove that $\ord_p(a+b) \geq \min(\ord_p \,a,\ord_p\, b )$ with equality holding if $\ord_p\, a \neq \ord_p\, b$.
}

\begin{proof}
As $a\wedge b \mid a + b, \ord_p(a\wedge b) \leq \ord_p(a+b)$, so $\min(\ord_p(a),\ord_p(b)) \leq \ord_p(a+b)$.

Suppose $\ord_p(a) \neq \ord_p(b)$,The problem being symmetric in $a,b$, we may suppose $\alpha = \ord_p(a) < \beta = \ord_p(b)$.
So there exist $q,r \in \Z$ such that
\begin{align*}
a &= p^\alpha q, \ p \nmid q\\
b &= p^\beta r, \ p\nmid r \qquad \alpha < \beta
\end{align*}
Then $a+b = p^\alpha(q+ p^{\beta-\alpha} r)$, where $p\nmid q + p^{\beta-\alpha}r $ (as $p \mid p^{\beta - \alpha}$ and $ p \nmid q$).

So $\ord_p(a+b) = \alpha = \min(\ord_p(a), \ord_p(b))$.
\end{proof}

\paragraph{Ex. 1.22}

{\it Almost all the previous exercises remain valid if instead of the ring $\Z$ we consider the ring $k[x]$. Indeed, in most we can consider any Euclidean domain. Convince yourself of this fact. For simplicity we shall continue to work in $\Z$.
}

\begin{proof}
We can adapt all the preceding proofs to the Euclidean domain $k[x]$. The only difference is that the units in $\Z$ are $\pm1$, and the units in $k[x]$ are the elements of $k^*$.
\end{proof}

\paragraph{Ex. 1.23}

{\it Suppose that $a^2 + b^2 = c^2$ with $a, b, c \in \Z$ For example, $3^2 + 4^2 = 5^2$ and $5^2 +
12^2 = 13^2$ . Assume that $(a, b) = (b, c) = (c, a) = 1$. Prove that there exist integers $u$
and $v$ such that $c - b = 2u^2$ and $c + b = 2v^2$ and $(u, v) = 1$ (there is no loss in
generality in assuming that $b$ and $c$ are odd and that $a$ is even). Consequently $a = 2uv$,
$b = v^2 - u^2$, and $c = v^2 + u^2$. Conversely show that if $u$ and $v$ are given, then the
three numbers $a$, $b$, and $c$ given by these formulas satisfy $a^2 + b^2 = c^2$.

}

\begin{proof}Suppose $x^2+y^2 = z^2,\ x,y,z \in \Z$. Let $d = x\wedge y \wedge z$. If $d = 0$, then $x=y=z=0$. If $d \neq 0$, and $a = x/d, b=y/d,c=z/d$, then $a^2+b^2 = c^2$, with $a\wedge b \wedge c = 1$. If a prime $p$ is such that $p \mid a, p\mid b$, then $p \mid c^2$, so $p \mid c$ (as $p$ is a prime). Then $p \mid a\wedge b \wedge c =1$ : this is impossible, so $a\wedge b = 1$, and similarly $a\wedge c = 1, b \wedge c = 1$.

If $a,b$ are odd, then $a^2 \equiv b^2 \equiv 1 \pmod 4$, so $c^2 \equiv 2 \pmod 4$. As the squares modulo 4 are $0,1$, this is impossible. As $a\wedge b = 1$, $a,b$ are not both even, so $a,b$ are not of the same parity. Without loss of generality, we may exchange $a,b$ so that $a$ is even, $b$ is odd, and then $c$ is odd.

$a^2 =c^2 - b^2 = (c-b)(c+b)$, so $$\left(\frac{a}{2}\right)^2 = \left(\frac{c-b}{2}\right) \left(\frac{c+b}{2}\right).$$
where $a/2,(c-b)/2,(c+b)/2$ are integers.

If $d \mid (c-b)/2$ and $d \mid (c+b)/2$, then $d \mid c = (c+b)/2+ (c-b)/2$, and $d \mid b = (c-b)/2 - (c-b)/2$, so $d \mid c\wedge b = 1$. This proves 
$$\left(\frac{c+b}{2}\right) \wedge \left(\frac{c-b}{2}\right) = 1.$$
Using Ex. 1.16, we see that $(c+b)/2$ and $(c-b)/2$ are squares : there exist $u,v$ such that
$$c-b = 2u^2, c+b = 2 v^2,\qquad u\wedge v = 1.$$
$(a/2)^2 = u^2v^2$, and we can choose the signs of $u,v$ such that $a = 2 uv$. Then $b = v^2 - u^2, c = v^2+u^2$.
There exists $\lambda \in \Z$ ($\lambda = d$) such that $x = 2 \lambda uv, y = \lambda (v^2 - u^2), z = \lambda (v^2 +u^2)$.

Conversely, if $\lambda, u, v$ are any integers, $(2\lambda uv)^2 + (\lambda(v^2-u^2)^2 = \lambda^2 (4 u^2v^2 + v^4 + u^4 -2 u^2v^2) = \lambda^2 ( v^4 + u^4 +2 u^2v^2) = (\lambda (u^2+v^2))^2$.

Conclusion : if $x,y,z \in \Z$,
$$
x^2+y^2 = z^2 \iff \exists \lambda \in \Z, \exists (u,v) \in \Z^2, u\wedge v = 1,$$
$$
\left\{
\begin{array}{ccc}
 x  & =   &2 \lambda uv  \\
  y & = &\lambda(v^2 - u^2)     \\
  z&  = & \lambda (v^2+u^2)  
\end{array}
\right.
\quad \mathrm{or}\quad \left\{
\begin{array}{ccc}
 x  & =   &\lambda(v^2 - u^2)  \\
  y & = &   2 \lambda uv  \\
  z&  = & \lambda (v^2+u^2)  
\end{array}
\right.
$$
\end{proof}

\paragraph{Ex. 1.24}

{\it Prove the identities

(a) $x^n - y^n = (x-y) (x^{n-1}+x^{n-2} y + \cdots +y^{n-1})$

(b) For $n$ odd, $x^n + y^n = (x+y) (x^{n-1}- x^{n-2} y + \cdots +y^{n-1})$
}

\begin{proof}
Let $R$ any commutative ring, and $x,y \in R$.

a) Let $$S = \sum_{i=0}^{n-1} x^{n-1-i} y^i.$$ Then
\begin{align*}
xS &= \sum_{i=0}^{n-1} x^{n-i} y^i = x^n + \sum_{i=1}^{n-1} x^{n-i} y^i\\
yS &= \sum_{i=0}^{n-1} x^{n-1-i} y^{i+1} = \sum_{j=1}^{n} x^{n-j} y^j \qquad(j = i+1)\\
&=y^n + \sum_{i=1}^{n-1} x^{n-i} y^i.
\end{align*}
So $xS - y S = x^n-y^n$,
$$x^n - y^n = (x-y) \sum_{i=0}^{n-1} x^{n-1-i} y^i = (x-y)(x^{n-1}+x^{n-2} y + \cdots +x^{n-1-i}y^i+ \cdots + y^{n-1}).$$

b) If we substitute $-y$ by $y$, we obtain
$$x^n - (-1)^n y^n = (x+ y) \sum_{i=0}^{n-1}(-1)^i x^{n-1-i} y^i .$$

If $n$ is odd,
$$x^n  + y^n = (x+ y) \sum_{i=0}^{n-1}(-1)^i x^{n-1-i} y^i  = (x+y)(x^{n-1}-x^{n-2} y + \cdots +(-1)^i x^{n-1-i}y^i+ \cdots + y^{n-1}).$$
\end{proof}

\paragraph{Ex. 1.25}

{\it If $a^n - 1$ is a prime, show that $a = 2$ and that $n$ is a
prime. Primes of the form $2^p - 1$ are called Mersenne primes. For
example, $2^3 - 1 = 7$ and $2^5 - 1 = 31$. It is not known if there are
infinitely many Mersenne primes.
}

\begin{proof}
Suppose $n>1, a\geq 0$, and $a^n - 1$ is a prime. As $0^n -1 = -1,1^n-1 = 0$ are not primes, $a \geq 2$.

As $(a^n-1) = (a-1) (a^{n-1} + \cdots + a^{i} + \cdots  + 1)$, $a-1$ is a factor of the prime $a^n-1$, so $a-1 = 1$ or $a-1 = a^n-1$.

As $a\geq 2$, and $n >1$, $a = a^n$ is impossible, so $a=2$.

If $n\geq 2$ wasn't  prime, then $n = uv, 1 < u <n, 1 < v <n$, and
$$2^n-1 = 2^{uv} - 1 = (2^u -1) (2^{u(v-1)}+ \cdots+2^{ui}+ \cdots+ 1).$$

with $1 = 2^1 - 1<2^u -1 < 2^n-1$. $2^n - 1$ has a non trivial factor : this is impossible, so $n$ is a prime.

Conclusion : if $a^n - 1\ (a \geq 0, n > 1)$ is a prime, then $a = 2$ and $n$ is a prime.
\end{proof}

\paragraph{Ex. 1.26}

{\it If $a^n + 1$ is a prime, show that $a$ is even and that $n$ is a power
of 2. Primes of the form $2^{2^t} + 1$ are called Fermat primes. For
example, $2^{2^1} + 1 = 5$ and $2^{2^2} + 1 = 17$.  It is not known if
there are infinitely many Fermat primes.
}

\begin{proof}
If $a = 1, a^n + 1$ is a prime. Suppose $a >1$, and $n>1$. If $a$ was odd, $a^n+1>2$ is even, so is not a prime. Consequently, if $a^n+1$ is prime, $a>1$, then $a$ is even.

Write $n = 2^t u$, where $u$ is odd.

If $u>1$, then, from Ex. 24(b), we obtain
$$a^n+1 = a^{2^tu}+1 = (a^{2^t} + 1)\sum_{i=0}^{u-1} (-1)^ia^{i2^t}.$$
So $1<a^{2^t} + 1 < a^n+1$, and $a^{2^t} + 1$ is a non trivial factor of $a^n+1$, in contradiction with the hypothesis. 

Conclusion : if $a^n+1$ is a prime ($a>1,n>1$), $a$ is even and $n$ is a power of $2$.
\end{proof}

\paragraph{Ex. 1.27} 

{\it For all odd $n$ show that $8\mid n^2 - 1$. If $3\nmid n$,
show that $6\mid n^2 - 1$.
}

\begin{proof}
As $n$ is odd, write $n=2k+1, n \in \Z$. Then
$$n^2 - 1 = (2k+1)^2-1 = 4k^2+4k = 4k(k+1).$$
As $k$ or $k+1$ is even, $8 \mid n^2 - 1$.

$(n-1)n(n+1) = n (n^2-1)$, product of three consecutive numbers, is a multiple of 3.

As $3 \nmid n$, and 3 is a prime, $3 \wedge n = 1$, so $3 \mid n^2-1$.
$$3 \nmid n \Rightarrow 3 \mid n^2 -1.$$
(This is also a consequence of Fermat' Little Theorem.)

As $n$ is odd, $n^2-1$ is even.
$3 \mid n^2-1, 2 \mid n^2-1$ and $2 \wedge 3 = 1$, so $6 \mid n^2-1$.
\end{proof}

\paragraph{Ex. 1.28}

{\it For all $n$ show that $30 \mid  n^5 - n$ and that $42\mid n^7 - n$.
}

\begin{proof}
If we want to avoid Fermat's Little Theorem (Prop. 3.3.2. Corollary 2 P. 33), note that
\begin{align*}
(n-2)(n-1)n(n+1)(n+2) &= n(n^2-1)(n^2-4)\\
&=n^5-5n^2+4n\\
&=n^5 - n +5 (-n^2+n)
\end{align*}
As the product of 5 consecutive numbers is divisible by 5,
$$5 \mid n^5 - n.$$
Moreover,
\begin{align*}
&2 \mid (n-1)(n+1) = n^2 -  1  \mid n^4 - 1 \mid n^5 -n\\
&3 \mid (n-1)n(n+1) = n(n^2 - 1) \mid n(n^4-1) = n^5 - n
\end{align*}
As $2,3,5$ are distinct primes, $2\times 3 \times 5 = 30 \mid n^5-n$.

Similarly,
\begin{align*}
(n-3)(n-2)(n-1)n(n+1)(n+2)(n+3) &= n(n^2-1)(n^2-4)(n^2-9)\\
&=n(n^4-5n^2+4)(n^2-9)\\
&=n^7-14n^5+49n^3-36n\\
&=n^7-n + 7(-2n^5+7n^3-5n)
\end{align*}
As the product of 7 consecutive numbers is divisible by 7,
$$7 \mid n^7 -n.$$
Moreover
\begin{align*}
&2 \mid (n-1)(n+1) = n^2 -  1 \mid n^6 - 1 \mid n^7 -n\\
&3 \mid (n-1)n(n+1) = n(n^2 - 1) \mid n(n^6-1) = n^7 - n
\end{align*}
As $2,3,7$ are distinct primes $2\times 3 \times 7 = 42 \mid n^7-n$.
\end{proof}

\paragraph{Ex. 1.29}

{\it  Suppose that $a, b, c, d \in \Z$ and that $(a, b) = (c, d) =
1$. 

If $(a/b) + (c/d) =$ an integer, show that $b = \pm d$.
}

\begin{proof}
If $\frac{a}{b}+ \frac{c}{d} = n \in \Z\quad (a\wedge b = c\wedge d = 1)$, then $ad+bc = nbd$, so $d \mid bc, d\wedge c=1$, which implies $d\mid b$. Similarly $b \mid d$. Then $d =\pm b$.
\end{proof}

\paragraph{Ex. 1.30}

{\it
Prove that  $H_n = \frac{1}{2} + \frac{1}{3} + \ldots + \frac{1}{n}$
is not an integer.
}

\begin{proof}
Let $s$ such that $2^{s} \leq n < 2^{s+1}$ ($s = \left \lfloor \frac{\ln n}{\ln 2} \right \rfloor \geq 1$).

$$H_n = \frac{1}{2} + \cdots + \frac{1}{n} = \frac{\sum_{i=2}^n a_i}{n!}, \qquad \mathrm{where}\ a_i = \frac{n!}{i} \in \Z.$$
Let $k = \ord_2(n!)$.
We will show that $\ord_s(a_i)$ is minimal for $i_0 = 2^s$, where $\ord_2(a_{i_0}) = k - s$, and that this minimum is reached only for this index $i_0$.

Indeed, each $i$ such that $2 \leq i \leq n$ can be written with the form $i = 2^tq, 2 \nmid q$. Then $i = 2^t q \leq n < 2^{s+1}$, so $2^t < 2^{s+1}, t < s+1, t \leq s$, which proves
$$\ord_2(a_i) = k - t \geq k - s = \ord_2(a_{i_0}).$$
Moreover, if $\ord_2(a_i) = \ord_2(a_{i_0})$, then $k - t = k - s$, so $s = t$.

$i = 2^s q, 2 \nmid q$. If $q>1$, then $i\geq 2^{s+1}>n$ : it's impossible. So $q=1$ and $i = 2^s = i_0$.

Using Ex 1.21, we see that
$$\ord_2\left(\sum_{i=2}^n a_i \right)= \ord_2(a_{i_0}) = k - s < k = \ord_2(n!).$$
So $$H_n = \frac{2^{k-s} Q}{2^k R} = \frac{Q}{2^s R},$$ where $Q,R$ are odd integers. $H_n$ is a quotient of an odd integer by an even integer : $H_n$ is never an integer.
\end{proof}

\paragraph{Ex. 1.31}

{\it Show that $2$ is divisible by $(1+i)^2$ in $\Z[i]$.
}

\begin{proof}
$(1+i)^2 = 1 +2i - 1 = 2i$, so $2 = -i(1+i)^2$ is divisible by $(1+i)^2$.
(As $i$ is an unit, $2$ and $(1+i)^2$ are associate.)
\end{proof}

\paragraph{Ex. 1.32}

{\it
For $\alpha = a + bi \in \Z[i]$ we defined $\lambda(\alpha) =
a^2 + b^2$. From the properties of $\lambda$ deduce the identity $(a^2+ b^2)(c^2 + d^2) = (ac - bd)^2 + (ad + bc)^2$.
}

\begin{proof}
For all complex numbers $\alpha,\beta$, $\vert \alpha \beta\vert = \vert \alpha\vert \vert \beta\vert$, so
$$\lambda (\alpha \beta) = \lambda(\alpha) \lambda(\beta).$$
If $\alpha = a+bi \in \Z[i), \beta = c + d i \in \Z[i]$, then $\alpha \beta = (ac - bd) + (ad +bc)i$, so
$$(a^2+ b^2)(c^2 + d^2) = (ac - bd)^2 + (ad + bc)^2.$$
\end{proof}

\paragraph{Ex. 1.33}

{\it Show that $\alpha \in \Z[i]$ is a unit iff  $\lambda(\alpha) = 1$. Deduce that 1, -1, i, and - i are the only units in $\Z[i]$.
}

\begin{proof}
Let $\alpha = a+bi \in \Z[i]$.

$\bullet$ If $\lambda(\alpha) = 1$, then $\alpha \overline{\alpha} = 1$, where $\overline{\alpha} = a-bi \in \Z[i]$, so $\alpha$ is an unit.

$\bullet$  Conversely, if $\alpha$ is an unit, there exists $\beta \in \Z[i]$ such that $\alpha \beta = 1$, then $\lambda( \alpha) \lambda(\beta) = 1$, where $\lambda(\alpha), \lambda(\beta)$ are positive integers, hence $\lambda(\alpha) = 1$.

So $\alpha = a + ib$ is an unit of $\Z[i]$ if and only if $a^2+b^2 = 1$. In this case, $\vert a \vert^2\leq 1$, $a \in \{ 0,1,-1\}$. If $a = 0, b = \pm 1$, and if $a = \pm 1, b = 0$, so the only units of $\Z[i]$ are $1,i,-1,-i$.
\end{proof}

\paragraph{Ex. 1.34}
{\it Show that 3 is divisible by $(1 - \omega)^2$ in $\Z[\omega]$.
}

\begin{proof}
As $\omega^3 = 1, \overline{\omega} = \omega^2$, and $1+\omega + \omega^2 = 0$,

$\vert 1-\omega \vert^2 = (1-\omega)(1-\omega^2) = 1 +\omega^3 - \omega -\omega^2 = 3$, so
$$3 = (1-\omega)(1-\omega^2).$$
Consequently, 
$$3 = (1-\omega) (1-\omega^2) =(1+\omega)(1-\omega)^2=  -\omega^2(1-\omega)^2.$$
3  is divisible by $(1 - \omega)^2$ in $\Z[\omega]$ (as $-\omega^2$ is an unit, $3$ and  $(1-\omega)^2$ are associated. $3$ is not irreducible in $\Z[\omega]$).
\end{proof}

\paragraph{Ex. 1.35}

{\it For $\alpha = a + b\omega \in \Z[\omega]$ we defined
$\lambda(\alpha) = a^2 - ab + b^2$. Show that $\alpha$ is a unit iff
$\lambda(\alpha) = 1$. Deduce that $1, -1, \omega, -\omega, \omega^2 ,
and -\omega^2$ are the only units in $\Z[\omega]$.
}

\begin{proof}
If $\alpha = a + b\omega \in \Z[\omega]$, using $1+\omega+ \omega^2=0$ and $\overline{\omega} = \omega^2$, we obtain
\begin{align*}
\alpha \overline{\alpha} &=(a+b \omega)(a+b \omega^2)\\
&= a^2 + b^2 +ab(\omega + \omega^2)\\
&= a^2+b^2-ab\\
&=\lambda(\alpha)
\end{align*}

Consequently, $\lambda$ is a multiplicative function.

$\bullet$ If $\lambda(\alpha) = 1$, then $\alpha \overline{\alpha} = 1$, where $\overline{\alpha} = a+b\omega^2 = (a-b) -b\omega\in \Z[\omega]$, so $\alpha$ is an unit.

$\bullet$  Conversely, if $\alpha$ is an unit, there exists $\beta \in \Z[\omega]$ such that $\alpha \beta = 1$, then $\lambda( \alpha) \lambda(\beta) = 1$, where $\lambda(\alpha), \lambda(\beta)$ are positive integers, so $\lambda(\alpha) = 1$.

\begin{align*}
\lambda(\alpha) = 1 &\iff a^2 -ab +b^2 = 1\\
&\iff (2a-b)^2+3b^2 = 4\\
\end{align*}
$3b^2 \leq 4$, so $b=0$ or $b=\pm1$.

 If $b=0$, then $a = \pm 1$, $\alpha = 1$ or $\alpha = -1$

If $b = 1$, then $(2a-1)^2 = 1, 2a-1 = \pm 1$ : $a=0$ or $a=1$, $\alpha = \omega$ or $\alpha = 1+\omega = -\omega^2$.

If $b=-1$, then $(2a+1)^2 = 1, 2a+1 = \pm 1$ : $a = 0$ or $a=-1$, $\alpha= -\omega$ or $\alpha = -1-\omega = \omega^2$.

So
$$\lambda(\alpha) = 1 \iff \alpha \in \{1, \omega, \omega^2,-1,-\omega,-\omega^2\}.$$

The  set of units of $\Z[\omega]$ is the group of the roots of $x^6-1$.
\end{proof}

\paragraph{Ex. 1.36}

{\it Define $\Z[\sqrt{-2}]$ as the set of all complex numbers of the form $a+b\sqrt{-2}$, where $a,b\in \Z$. Show that $\Z[\sqrt{-2}]$ is a ring. Define $\lambda(\alpha) = a^2+2b^2$ for $\alpha = a+b\sqrt{-2}$. Use $\lambda$ to show that $\Z[\sqrt{-2}]$ is a Euclidean domain.
}

\begin{proof}
Note $\sqrt{-2} = i \sqrt{2}$, and $A = \Z[\sqrt{-2}]$.

Let $\alpha = a + b \sqrt{-2}, \beta =  c + d\sqrt{-2} \in A$ :

$\bullet$ $1 = 1 + 0 \sqrt{-2} \in A$.

$\bullet$ $\alpha - \beta = (a + b \sqrt{-2})- (c + d\sqrt{-2}) = (a-c) + (b-d) \sqrt{-2} \in A$.

$\bullet$  $\alpha \beta= (a + b \sqrt{-2}) (c + d\sqrt{-2}) = (ac -2 bd) + (ad+bc) \sqrt{-2} \in A.$

So $A$ is a subring of $(\mathbb{C},+,\times)$ : $\Z[\sqrt{-2]}$ is a ring.

Let $z = a + b\sqrt{-2}$ any complex number. Let $a_0, b_0 \in \Z$ such that $\vert a-a_0 \vert \leq 1/2, \vert b-b_0 \vert \leq 1/2$ (it suffice to take for $a_0$ the nearest integer of $a$ : $a_0 =\left \lfloor a + \frac{1}{2}\right \rfloor$). Let $z_0 = a_0 + b_0 \sqrt{-2}$.

As $\lambda(z) = z \overline{z} = a^2 + 2 b^2$, then 
$$\lambda(z-z_0) = (a-a_0)^2 + 2(b-b_0)^2 \leq \frac{1}{4} + 2 \times \frac{1}{4} = \frac{3}{4}<1.$$

Conclusion : for any $z \in \mathbb{C}$, there exists $z_0 \in A$ such that $\lambda(z-z_0) < 1$.

Let $(z_1,z_2) \in A \times A, z_2 \ne 0$. We apply the preeceeding result to the complex $z_1/z_2$ : there exists $q \in A$ such that $\left \vert \frac{z_1}{z_2} - q \right \vert \leq 1$. Let $r = z_1 - qz_2$. Then $z_1 = qz_2 +r, \lambda(r) < \lambda(z_2)$.

So $\Z[\sqrt{-2}]$ is a Euclidean domain.
\end{proof}

\paragraph{Ex. 1.37}

{\it Show that the only units in $\Z[\sqrt{-2}]$ are $1$ and $-1$.
}

\begin{proof}
As in Ex. 35, we prove that $\alpha = a + b\sqrt{-2}$ is an unit if and only if $\lambda(\alpha) = 1$, i.e. $a^2+2b^2 = 1$. As $2b^2 \leq 1, b = 0$, and $a^2 = 1$. So the only units are $1$ and $-1$.
\end{proof}

\paragraph{Ex. 1.38}

{\it Suppose that $\pi \in \Z[i]$ and that $\lambda(\pi) = p$ is a prime in $\Z$. Show that $\pi$ is a prime in $\Z[i]$. Show that the corresponding result holds in $\Z[\omega]$ and $\Z[\sqrt{-2}]$.
}

\begin{proof} If $\pi = \alpha \beta$, where $\alpha, \beta \in \Z[i]$, then $p = \lambda(\pi) = \lambda(\alpha) \lambda(\beta)$. As $p$ is a prime in $\Z$, and $\lambda(\alpha) \geq 0,,\lambda(\beta) \geq 0$, $\lambda(\alpha) = 1$ or $\lambda(\beta)=1$, so (Ex.1.33) $\alpha$ or $\beta$ is an unit. Consequently, $\pi$ is irreducible in $\Z[i]$. As $\Z[i]$ is a PID,  $\pi$ is a prime in $\Z[i]$ (Prop. 1.3.2 Corollary 2).

As $\Z[\omega]$ and $\Z[\sqrt{-2}]$ are Euclidean domains, the same result is true in these principal ideals domains.
\end{proof}

\paragraph{Ex. 1.39}

{\it Show that in any integral domain a prime element is irreducible.
}

\begin{proof}
Let $R$ an integral domain, and $\pi$ a prime in $R$.

If $\pi = \alpha \beta, \ \alpha, \beta \in R$, a fortiori $\pi$ divides $\alpha \beta$. As $\pi$ is a prime, $\pi$ divides $\alpha$ or $\beta$, say $\alpha$, so there exists $\xi \in R$ such that $\alpha = \xi \pi$, so $\pi = \xi \pi \beta, \pi(1 - \xi \beta)= 0$. As $A$ is an integral domain, and $\pi \neq 0$ by definition,  $1 = \xi \beta$, so $\beta$ is an unit. If  $\pi = \alpha \beta$, $\alpha$ or $\beta$ is an unit, so $\pi$ is irreducible.
\end{proof}


{\large \bf Chapter 2}


\paragraph { Ex 2.1} 


{\it Show that $k[x]$, with $k$ a finite field, has infinitely many irreducible polynomials.
}

\begin{proof}
Suppose that the set $S$ of irreducible polynomials is finite : $S = \{P_1,P_2,\ldots,P_n\}$. 

Let $Q = P_1P_2\cdots P_n + 1$. As $S$ contains the polynomials $x-a,a \in k$, $\deg(Q) \geq  q=\vert k \vert >1$. Thus $Q$ is divisible by an irreducible polynomial. As $S$ contains all the irreducible polynomials, there exists $i, 1\leq i \leq n$, such that $P_i \mid Q = P_1P_2\cdots P_n + 1$, so $P_i \mid 1$, and $P_i$ is an unit, in contradiction with the irreducibility of $P_i$.

Conclusion : $k[x]$ has  infinitely many irreducible polynomials. As each polynomial has only a finite number of associates, there exists infinitely many monic irreducible polynomials.
\end{proof}

\paragraph{Ex. 2.2.}

{\it Let $p_1,p_2,\ldots,p_t \in \Z$ be primes and consider the set of all rational numbers $r = a/b,\ a,b\in \Z$, such that $\ord_{p_i} a\geq \ord_{p_i}b$ for $i=1,2,\ldots,t$. Show that this set is a ring and that up to taking associates $p_1,p_2,\ldots,p_t$ are the only primes.
}

\begin{proof}
Let $R$ the set of such rationals. Simplifying these fractions, we obtain
$$r \in R \iff \exists p \in \Z, \exists q \in \Z\setminus\{0\}, \ r = \frac{p}{q}, \ q \wedge p_1p_2\cdots p_t = 1.$$

$\bullet$ $1 = 1/1  \in R$.

$\bullet$ if $r,r' \in R$, $r = p/q, r' = p'/q'$, with $q \wedge p_1p_2\cdots p_t=1, q' \wedge p_1p_2\cdots p_t=1$. then $qq'\wedge p_1p_2\cdots p_t = 1$, and $r-r' = \frac{pq'-qp'}{qq'}$, $rr' = \frac{pp'}{qq'}$, so $r-r', rr' \in R$.

Thus $R$ is a subring of $\Q$.

If $r=a/b \in R$ is an unit of $R$, then $b/a \in R$, so $\ord_{p_i}a = \ord_{p_i}(b), \ i= 1,\ldots,t$. After simplification, $r = p/q$, with $p \wedge p_1\cdots p_t=1, q \wedge p_1\cdots p_t = 1$, and such rationals are all units.

$p_i, 1\leq i \leq t$, is a prime : if $p_i \mid rs$ in $R$, where $r=a/b,s = c/d \in R$, then there exists $u = e/f \in R$ such that $rs = p_i u$, with $b,d,e$ relatively prime with $p_1,\ldots,p_t$. Then $ac f= p_i bde$. As $p_i\wedge f=1$, $p_i$ divides $a$ or $c$ in $\Z$, so $p_i$ divides $r$ or $s$ in $R$.

If $r = a/b \in R$, with $b\wedge p_1\cdots p_r=1$, $a  = p_1^{k_1}\cdots p_t^{k_t} v, v \in \Z, k_i\geq 0, i=1,\ldots,t$. So $r = u p_1^{k_1}\cdots p_t^{k_t} $, where $u = v/b$ is an unit.

Let $\pi$ be any prime in $R$. As any element in $R$, $\pi = u p_1^{k_1}\cdots p_t^{k_t}, k_i\geq 0, u = a/b$ an unit. $u^{-1} \pi = p_1^{k_1}\cdots p_t^{k_t}$, so $\pi \mid p_1^{k_1}\cdots p_t^{k_t}$ (in $R$). As $\pi$ is a prime in $R$, $\pi \mid p_i$ for an index $i=1,\ldots,t$. Moreover $p_i \mid \pi$, so $p_i$ and $\pi$ are associate.

Conclusion: the primes in $R$ are the associates of $p_1,\ldots,p_t$.
\end{proof}

\paragraph{Ex. 2.3}

{\it Use the formula for $\phi(n)$ to give a proof that there are infinitely many primes.

[Hint: If $p_1,p_2,\ldots,p_t$ were all the primes, then $\phi(n) = 1$, where $n=p_1p_1\cdots p_t$.]
}

\begin{proof}
Let $\{p_1,\cdots,p_t\}$ the finite set of primes,with $p_1<p_2<\cdots<p_t$, and $n=p_1\cdots p_t$. By d?finition, $\phi(n)$ is the number of integers $k, 1\leq k \leq n$, such that $k\wedge n =1$. From the existence of decomposition in primes, if $k\geq 1$,  $k = p_1^{k_1}\cdots p_t^{k_t}$, where $k_i\geq 0, i=1,\ldots, t$. So $k\wedge n = 1$ if and only if $k=1$. Thus $\phi(n)=1$ The formula for $\phi(n)$ gives
$\phi(n) = (p_1-1)\cdots(p_t-1)=1$. As $p_i\geq 2$, this equation implies that  $p_1 = p_2 = \cdots = p_t=2$, so $t=1$, and the only prime number is 2. But 3 is also a prime number : this is a contradiction. 

Conclusion : there are infinitely many prime numbers.
\end{proof}

\paragraph{Ex. 2.4}

{\it If $a$ is a nonzero integer, then for $n > m$ show that $(a^{2^n} + 1, a^{2^m} + 1) = 1$ or $2$ depending on whether $a$ is odd or even.
}

\begin{proof}
Let $d = a^{2^n} + 1 \wedge a^{2^m} + 1$. Then $d \mid  a^{2^n} + 1, d \mid  a^{2^m} + 1$. So
\begin{align*}
a^{2^n}  &\equiv -1 \pmod d\\
a^{2^m} &\equiv -1 \pmod d
\end{align*}
As $n>m$, $2^{n-m}$ is even, so
$$-1 \equiv a^{2^n} = \left(a^{2^m}\right)^{2^{n-m}} \equiv (-1)^{2^{n-m}} \equiv 1 \pmod d.$$
$-1\equiv 1 \pmod d$, then $d \mid 2\ (d\geq 0)$. Thus $d =1$ or $d = 2$.

If $a$ is even, $a^{2^n} + 1$ is odd, so $d=1$.

If $a$ is odd, both $a^{2^n} + 1, a^{2^m} + 1$ are even, so $d=2$.
\end{proof}

\paragraph{Ex. 2.5}

{\it Use the result of Ex. 2.4 to show that there are infinitely many primes. (This proof is due to G.Polya.)
}

\begin{proof}
Let $F_n = 2^{2^n} + 1, n\in \N$. We know from Ex. 2.4 that $n\ne m \Rightarrow F_n \wedge F_m = 1$.
Define $p_n$ as the least prime divisor of $F_n$. If $n \ne m , F_n \wedge F_m = 1$, so $p_n \ne p_m$. 
The application $\varphi : \N \to \N, n \mapsto p_n$ is injective (one to one), so $\varphi(\N)$ is an infinite set of prime numbers.
\end{proof}

\paragraph{Ex. 2.6}

{\it For a rational number $r$ let $\lfloor r \rfloor$ be the largest integer less than or equal to $r$, e.g., $\left \lfloor \frac{1}{2}\right \rfloor = 0$, $\lfloor 2\rfloor = 2$, and $\left \lfloor 3+\frac{1}{3} \right \rfloor=3$. Prove $\ord_p\,  n! =\left  \lfloor \frac{n}{p} \right \rfloor+\left  \lfloor \frac{n}{p^2} \right \rfloor+\left  \lfloor \frac{n}{p^3} \right \rfloor \cdots$.
}

\begin{proof}
The number $N_k$ of multiples $m$ of $p^k$ which are not multiple of $p^{k+1}$, where $1\leq m \leq n$, is 
$$N_k = \left \lfloor \frac{n}{p^k}\right \rfloor -\left \lfloor \frac{n}{p^{k+1}}\right \rfloor.$$
Each of these numbers brings the contribution $k$ to the sum $\ord_p\,  n! = \sum\limits_{k=1}^n \ord_p\,  k$.
Thus
\begin{align*}
\ord_p\,  n! &= \sum_{k\geq 1} k \left ( \left \lfloor \frac{n}{p^k}\right \rfloor -\left \lfloor \frac{n}{p^{k+1}}\right \rfloor  \right)\\
&=  \sum_{k\geq 1} k  \left \lfloor \frac{n}{p^k}\right \rfloor  - \sum_{k\geq 1}k \left \lfloor \frac{n}{p^{k+1}}\right \rfloor\\
&=  \sum_{k\geq 1} k  \left \lfloor \frac{n}{p^k}\right \rfloor  - \sum_{k\geq 2}(k-1) \left \lfloor \frac{n}{p^{k}}\right \rfloor\\
&= \left \lfloor \frac{n}{p}\right \rfloor  + \sum_{k\geq 2} \left \lfloor \frac{n}{p^{k}}\right \rfloor\\
&= \sum_{k\geq 1} \left \lfloor \frac{n}{p^{k}}\right \rfloor
\end{align*}

Note that $\left \lfloor \frac{n}{p^{k}}\right \rfloor = 0$ if $p^k >n$, so this sum is finite.
\end{proof}

\paragraph{Ex. 2.7}

{\it Deduce from Ex. 2.6 that $\ord_p n! \leq n/(p - 1)$ and that $\sqrt[n]{n!} \leq \prod_{p \leq n} p^{1/(p-1)}$.
}

(The original statement $\prod_{p \mid n} p^{1/(p-1)}$ was modified.)
\begin{proof}
\begin{align*}
\ord_p\, n!  &= \sum_{k\geq 1} \left \lfloor \frac{n}{p^{k}}\right \rfloor \leq  \sum_{k\geq 1}  \frac{n}{p^{k}} = \frac{n}{p} \frac{1}{1 - \frac{1}{p}} = \frac{n}{p-1}
\end{align*}

The decomposition of $n!$ in prime factors is

$n! = p_1^{\alpha_1}p_2^{\alpha_2}\cdots p_k^{\alpha_k}$ 
where $\alpha_i = \ord_{p_i}\, n! \leq \frac{n}{p_i-1}$, and $p_i \leq n, \ i=1,2,\cdots,k$.

Then
\begin{align*}
n! &\leq p_1^{\frac{n}{p_1-1}}p_2^{\frac{n}{p_2-1}}\cdots p_k^{\frac{n}{p_n-1}}\\
\sqrt[n]{n!} &\leq p_1^{\frac{1}{p_1-1}}p_2^{\frac{1}{p_2-1}}\cdots p_k^{\frac{1}{p_n-1}}\\
&\leq \prod_{p\leq n} p^{\frac{1}{p-1}}
\end{align*}
(the values of $p$ in this product describe all prime numbers $p\leq n$.)
\end{proof}

\paragraph{Ex. 2.8}

{\it Use Exercise 7 to show that there are infinitely many primes.
}

\begin{proof}
If the set $\mathbb{P}$ of prime numbers was finite, we obtain from Ex.2.7, for all $n \geq 2$:
$$\sqrt[n]{n!} \leq C = \prod_{p\in \mathbb{P}} p^{\frac{1}{p-1}},$$
where $C$ is an absolute constant.

Yet  $\lim\limits_{n\to \infty} \sqrt[n]{n!}  = + \infty$. Indeed
\begin{align*}
\ln(\sqrt[n]{n!}) &= \frac{1}{n}(\ln 1 + \ln 2 + \cdots + \ln n)\\
\end{align*}
As $\ln$ is an increasing fonction, 
\begin{align*}
\int_{i-1}^i \ln t\, \D t &\leq \ln i, \ i=2,3,\ldots,n
\end{align*}
So
\begin{align*}
\int_1^n \ln t \, \D t &= \sum_{i=2}^n \int_{i-1}^i \ln t \, \D t \leq \sum_{i=2}^n \ln i =  \sum_{i=1}^n \ln i
\end{align*}
Thus
\begin{align*}
\ln(\sqrt[n]{n!}) &\geq  \frac{1}{n} \int_1^n \ln t\,  \D t\ = \frac{1}{n}(n \ln n - n +1) = \ln n - 1 + \frac{1}{n}
\end{align*}
As $\lim\limits_{n\to \infty}  \ln n - 1 + \frac{1}{n} = + \infty$, $\lim\limits_{n\to \infty} \ln(\sqrt[n]{n!}) = + \infty$, so $\lim\limits_{n\to \infty} \sqrt[n]{n!} = + \infty$.

So there exists $n$ such that $\sqrt[n]{n!} \geq C$ : this is a contradiction. $\mathbb{P}$ is an infinite set.
\end{proof}

\paragraph{Ex. 2.9}

{\it A function on the integers is said to be multiplicative if $f(ab) =f(a)f(b)$. whenever $(a, b) = 1$. Show that a multiplicative function is completely determined by its value on prime powers.
}

\begin{proof}
Let the decomposition of $n$ in prime factors be $n =p_1^{k_1}\cdots p_t^{k_t}, p_1< \cdots<p_t$. As $p_i^{k_i} \wedge p_j^{k_j}=1$ for $i\ne j,\ i,j = 1,\ldots,t$,
$$f(n) =f(p_1^{k_1}\cdots p_t^{k_t})=f(p_1^{k_1})\cdots f(p_t^{k_t})$$
(by induction on the number of prime factors.)

So $f(n)$ is completely determined by its value on prime powers.
\end{proof}

\paragraph{Ex. 2.10}

{\it If $f(n)$ is a multiplicative function, show that the function $g(n) = \sum_{d \mid n} f(d)$ is also multiplicative.
}

\begin{proof}
If $n \wedge m =1$,
\begin{align*}
g(nm)&= \sum_{\delta \mid nm} f(\delta)\\
&= \sum_{d\mid n, d'\mid m }f(dd')\\
\end{align*}
Actually, if $d\mid n, d'\mid m$, so $\delta = dd' \mid nm$, and conversely, if $\delta \mid nm$, as $n\wedge m = 1$, there exist $d,d'$ such that $d\mid n,d'\mid m$, and $\delta = dd'$.

If $d \mid n, d' \mid m$, with $n\wedge m = 1$, then $d \wedge d' = 1$, so
\begin{align*}
g(nm) &=\sum_{d\mid n } \sum_{d'\mid m} f(d) f(d')\\
&= \sum_{d\mid n}f(d) \sum_{d'\mid m} f(d')\\
&=g(n) g(m)
\end{align*}
$g$ is a multiplicative function.
\end{proof}

\paragraph{Ex. 2.11}

{\it Show that $\phi(n) = n \sum_{d\mid n}\mu(d)/d$ by first proving that $\mu(d)/d$ is multiplicative and then using Ex. 2.9 and 2.10.
}

\begin{proof}
Let's verify that $\mu$ is a multiplicative function.

If $n\wedge m = 1$, then $ n = p_1^{a_1}\cdots p_l^{a_l}, m= q_1^{b_1}\cdots q_r^{b_r}$, where $p_1,\ldots,p_l, q_1,\ldots q_r$ are distinct primes. Then the decomposition in prime factors of $nm$ is  $nm = p_1^{a_1}\cdots p_l^{a_l} q_1^{b_1}\cdots q_r^{b_r}$. If one of the $a_i$ or one of the $b_j$ is greater than 1, then $\mu(nm) = 0 = \mu(n)\mu(m)$. Otherwise, $n = p_1\cdots p_l, m = q_1\cdots q_r, nm = p_1\cdots p_l q_1\cdots q_r$, and $\mu(nm) = (-1)^{l+r} = (-1)^l(-1)^r = \mu(n)\mu(m)$. So
$$\frac{\mu(nm)}{nm} = \frac{\mu(n)}{n} \frac{\mu(m)}{m}.$$
that is, $n \mapsto \frac{\mu(n)}{n}$ is a multiplicative function.

From Ex.2.10, $n \mapsto \sum_{d\mid n} \frac{\mu(d)}{d}$ is also a multiplicative function, and so is $\psi$, where $\psi$ is defined by
$$\psi(n) = n \sum_{d\mid n} \frac{\mu(d)}{d}.$$
To verify the equality $\phi = \psi$, it is sufficient from Ex. 2.9 to verify $\phi(p^k) = \psi(p^k)$ for all prime powers $p^k, k\geq 1$ ($\phi(1) = \psi(1) = 1$).
\begin{align*}
\psi(p^k) &= p^k \sum_{d\mid p^k } \frac{\mu (p^k)}{p^k}\\
&=p^k \left(\frac{\mu(1)}{1}+\frac{\mu(p)}{p}\right)
\end{align*}
(The other terms are null.)

So $$\psi(p^k) = p^k\left(1 - \frac{1}{p} \right) = p^k - p^{k-1} = \phi(p^k).$$
Thus $\phi = \psi$ : for all $n\geq 1$,
$$\phi(n) =  n \sum_{d\mid n} \frac{\mu(d)}{d}.$$
\end{proof}

\paragraph{Ex. 2.12}

{\it Find formulas for $\sum_{d\mid n} \mu(d)\phi(d)$, $\sum_{d \mid n} \mu(d)^2\phi(d)^2$, and $\sum_{d\mid n} \mu(d)/\phi(d)$.
}

\begin{proof}
As $\mu,\phi$ are multiplicative, so are $\mu \phi, \mu^2\phi^2 ,\mu/\phi$. We deduce from Ex. 2.10 that the three following fonctions $F,G,H$ are multiplicative, defined by
$$F(n) = \sum_{d\mid n} \mu(d)\phi(d), G(n) = \sum_{d \mid n} \mu(d)^2\phi(d)^2, H(n) = \sum_{d\mid n} \mu(d)/\phi(d),$$so it is sufficient to compute their values on prime powers $p^k, k\geq 1$.

\begin{align*}
F(p^k) &= \sum_{i=0}^k \mu(p^i)\phi(p^i)\\
&= \phi(1) - \phi(p)  = 1 - (p-1) = 2-p
\end{align*}
So $F(n) = \prod_{p\mid n} (2-p)$.

Similarly,
\begin{align*}
G(p^k) &= \sum_{i=0}^k \mu(p^i)^2\phi(p^i)^2\\
&= \phi(1)^2 + \phi(p)^2 = 1 + (p-1)^2 = p^2-2p +2
\end{align*}
\begin{align*}
H(p^k) &=  \sum_{i=0}^k \mu(p^i)/\phi(p^i)\\
&= 1/\phi(1) -1/\phi(p) = 1 - 1/(p-1) = (p-2)/(p-1)
\end{align*}
\end{proof}

\paragraph{Ex. 2.13}

{\it Let $\sigma_k(n) = \sum_{d\mid n} d^k$ . Show that $\sigma_k(n)$ is multiplicative and find a formula for it.
}

\begin{proof}
As $n \mapsto n^k$ is multiplicative, then so is $\sigma_k$ (Ex. 2.10).

$\bullet$ Suppose that $k \ne 0$.

If $n = p^\alpha$ is a prime power ($\alpha \geq 1$),
\begin{align*}
\sigma_k(p^\alpha) &= \sum_{i=0}^{\alpha} p^{ik}\\
&=\frac{ p^{(\alpha+1)k} -  1}{p^k-1}
\end{align*}


$\bullet$ if $k=0$, $\sigma_0(n)$ is the number of divisors of $n$.
\begin{align*}
\sigma_0(p^\alpha) &= \sum_{i=0}^{\alpha} 1\\
&=\alpha+1
\end{align*}
Conclusion : if $n = p_1^{\alpha_1}\cdots p_t^{\alpha_t}$ is the decomposition of $n$ in prime factors, then
\begin{align*}
\sigma_0(n) &= (\alpha_1+1)\cdots(\alpha_t +1),\\
\sigma_k(n) &= \prod_{i=0}^t \frac{p_i^{(\alpha_i+1)k}- 1}{p_i^k - 1}\ (k\ne 0).\\
\end{align*}
\end{proof}

\paragraph{Ex. 2.14}

{\it If $f(n)$ is multiplicative, show that $h(n) = \sum_{d \mid n} \mu(n/d)f(d)$ is also multiplicative.
}

\begin{proof}
We show first that the Dirichlet product $f \circ g$ of two multiplicative functions $f,g$ is multiplicative.
Suppose that $n\wedge m = 1$. If $d\mid n, d'\mid m$, so $\delta = dd' \mid nm$, and conversely, if $\delta \mid nm$, as $n\wedge m = 1$, there exist $d,d'$ such that $d\mid n,d'\mid m$, and $\delta = dd'$. Thus
\begin{align*}
(f\circ g)(nm)&= \sum_{\delta \mid nm} f(\delta) g\left(\frac{m}{\delta}\right)\\
&=\sum_{d\mid n,d'\mid m} f(dd')g\left(\frac{nm}{dd'}\right)\\
&=\sum_{d\mid n} \sum_{d'\mid m} f(d)f(d') g\left(\frac{n}{d}\right)g\left(\frac{m}{d'}\right)\\
&=\sum_{d\mid n} f(d)g\left(\frac{n}{d}\right)\sum_{d'\mid m} f(d') g\left(\frac{m}{d'}\right)\\
&=(f\circ g)(n)(f\circ g)(m)
\end{align*}
Applying this result with $g = \mu$, we obtain that $n\mapsto h(n) = \sum_{d \mid n} \mu(n/d)f(d)$ is multiplicative, if $f$ is multiplicative.
\end{proof}

\paragraph{Ex. 2.15}

{\it 
Show that
\begin{enumerate}
  \item[(a)] $\sum_{d\mid n} \mu(n/d)\nu(d) = 1$ for all n.
  \item[(b)] $\sum_{d\mid n} \mu(n/d) \sigma(d) = n$ for all n.
\end{enumerate}
}

\begin{proof}
Here $\nu = \sigma_0, \sigma = \sigma_1$.
\begin{enumerate}
\item[(a)] From the M?bius Inversion Theorem, as $\nu(n) = \sum_{d\mid n }1 =\sum_{d\mid n } I(d)$, where $I(n) = 1$ for all $n\geq 1$,
$$1 = I(n) =  \sum_{d\mid n} \mu(n/d) \nu(d).$$
\item[(b)]  From the same theorem, as $\sigma(n) = \sum_{d\mid n }d = \sum_{d\mid n } \mathrm{Id}(d)$, where  $\mathrm{Id}(n) = n$ for all $n\geq 1$,
$$ n= \mathrm{Id}(n) =  \sum_{d\mid n} \mu(n/d) \sigma(d).$$ 
\end{enumerate}
\end{proof}

\paragraph{Ex. 2.16}

{\it Show that $\nu(n)$ is odd iff $n$ is a square.
}

\begin{proof}
$\bullet$ If $n=a^2$ is a square, where $a = p_1^{k_1}\cdots p_t^{k_t}$, then $\nu(n) = (2k_1+1)\cdots(2k_t+1)$ is odd.

$\bullet$ Conversely, if $n = q_1^{l_1}\cdots q_r^{l_r}$ is odd, then $(l_1 +1)\cdots(l_r+1)$ is odd. So each $l_i+1$ is odd, and then $l_i$ is even, for $i=1,2,\ldots,r$ : $n$ is a square.
\end{proof}

\paragraph{Ex. 2.17}

{\it Show that $\sigma(n)$ is odd iff $n$ is a square or twice a square.
}

\begin{proof}
$\bullet$ Note that for all $r\geq 0$, $\sigma(2^r) = 1+2+2^2+\cdots+2^r = 2^{r+1} - 1$ is always odd.

If $p \ne 2$, $\sigma(p^{2k}) = 1 + p + p^2+\cdots+p^{2k}$ is a sum of $2k+1$ odd numbers, so is odd.

So if $n = a^2$, or $n = 2a^2 ,a \in \Z$, $\sigma(n)$ is odd.

$\bullet$ Conversely, suppose that $\sigma(n)$ is odd, where $n = p_1^{k_1} p_2^{k_2}\cdots p_t^{k_t}$, with $p_1=2 < p_2 < \cdots < p_t$. 
Then $$\sigma(n) = (2^{k_1+1} - 1) \frac{p_2^{k_2+1} - 1}{p_2-1} \cdots  \frac{p_t^{k_t+1} - 1}{p_t-1}$$
is odd. Then each $\frac{p_i^{k_i+1} - 1}{p_i-1} = 1+p_i + \cdots+p_i^{k_i}\ (i=2,\cdots,t)$ is odd. As each $p_i^j, j=0,\ldots,k_i$ is odd, the number of terms $k_i+1$ is odd, so $k_i$ is even ($i = 2,\ldots,t$). Thus $n$ is a square, or twice a square.
\end{proof}

\paragraph{Ex. 2.18}

{\it Prove that $\phi(n)\phi(m) = \phi((n, m))\phi([n, m])$.
}

\begin{proof}
Let $p_1,\cdots,p_r$ the common prime factors of $n$ and $m$.
\begin{align*}
n &= p_1^{\alpha_1}\cdots p_r^{\alpha_r} q_1^{\lambda_1}\cdots q_s^{\lambda_s}\\
m&= p_1^{\beta_1}\cdots p_r^{\beta_r} s_1^{\mu_1}\cdots s_t^{\mu_t}\\
\end{align*}
where $\alpha_i, \beta_i, \lambda_j,\mu_k \in \N^*, \ 1\leq i \leq r, 1\leq j \leq s, 1\leq k \leq t$ (the formula $\phi(p^\alpha) = p^\alpha - p^{\alpha - 1}$ is not valid if $\alpha = 0$).
Then 
\begin{align*}
n\wedge m &= p_1^{\gamma_1}\cdots p_r^{\gamma_r}\\
n \vee m &= p_1^{\delta_1}\cdots p_r^{\delta_r} q_1^{\lambda_1}\cdots q_s^{\lambda_s}s_1^{\mu_1}\cdots s_t^{\mu_t},\\
\end{align*}
where $\gamma_i = \min(\alpha_i,\beta_i) , \delta_i = \max(\alpha_i,\beta_i)\ (\gamma_i \geq 1, \delta_i \geq 1), 1\leq i \leq r$.
Then
\begin{align*}
\phi(n\wedge m) &=  \prod_{i=1}^r (p_i^{\gamma_i} - p_i^{\gamma_i - 1})\\
\phi(n\vee m)      &=  \prod_{i=1}^r(p_i^{\delta_i} - p_i^{\delta_i - 1}) \prod_{i=1}^s (q_i^{\lambda_i} - q_i^{\lambda_i - 1}) \prod_{i=1}^t (s_i^{\mu_i} - s_i^{\mu_i - 1})
\end{align*}
As $\alpha_i + \beta_i = \min(\alpha_i,\beta_i) +  \max(\alpha_i,\beta_i) = \gamma_i + \delta_i, 1\leq i \leq r$, then
\begin{align*}
\phi(n)\phi(m) &= \prod_{i=1}^r(p_i^{\alpha_i} - p_i^{\alpha_i - 1}) \prod_{i=1}^s(q_i^{\lambda_i} - q_i^{\lambda_i - 1}) \prod_{i=1}^r(p_i^{\beta_i} - p_i^{\beta_i - 1}) \prod_{i=1}^t(s_i^{\mu_i} - s_i^{\mu_i - 1})\\
&=\prod_{i=1}^r \left [ p_i^{\alpha_i + \beta_i} \left( 1- \frac{1}{p_i}\right)^2 \right]  \prod_{i=1}^s(q_i^{\lambda_i} - q_i^{\lambda_i - 1})  \prod_{i=1}^t(s_i^{\mu_i}- s_i^{\mu_i - 1})\\
&=\prod_{i=1}^r \left [ p_i^{\gamma_i+\delta_i} \left( 1- \frac{1}{p_i}\right)^2 \right]  \prod_{i=1}^s(q_i^{\lambda_i} - q_i^{\lambda_i - 1})  \prod_{i=1}^t(s_i^{\mu_i}- s_i^{\mu_i - 1})\\
&=\prod_{i=1}^r (p_i^{\gamma_i} - p_i^{\gamma_i - 1}) \prod_{i=1}^r(p_i^{\delta_i} - p_i^{\delta_i - 1})  \prod_{i=1}^s(q_i^{\lambda_i} - q_i^{\lambda_i - 1})  \prod_{i=1}^t(s_i^{\mu_i}- s_i^{\mu_i - 1})\\
&= \phi(n\wedge m) \phi(n\vee m)
\end{align*}
\end{proof}

\paragraph{Ex. 2.19}

{\it Prove that $\phi(nm)\phi((n, m)) = (n, m)\phi(n)\phi(m)$.
}

\begin{proof}
With the  notations of Ex. 2.18,
\begin{align*}
\phi(nm) &= \prod_{i=1}^r p_i^{\alpha_i + \beta_i} \left( 1- \frac{1}{p_i}\right) \prod_{i=1}^s q_i^{\lambda_i} \left( 1- \frac{1}{q_i}\right) \prod_{i=1}^t s_i^{\mu_i} \left( 1- \frac{1}{s_i}\right)\\
\phi(n\wedge m) &= \prod_{i=1}^r p_i^{\gamma_i} \left( 1- \frac{1}{p_i}\right)
\end{align*}
so
\begin{align*}
(n\wedge m) \phi(n)\phi(m) &= \prod_{i=1}^r p_i^{\gamma_i} \prod_{i=1}^r \left [ p_i^{\alpha_i + \beta_i} \left( 1- \frac{1}{p_i}\right)^2 \right] \prod_{i=1}^s q_i^{\lambda_i} \left( 1- \frac{1}{q_i}\right) \prod_{i=1}^t s_i^{\mu_i} \left( 1- \frac{1}{s_i}\right)\\
&=  \prod_{i=1}^r \left [ p_i^{\alpha_i + \beta_i + \gamma_i} \left( 1- \frac{1}{p_i}\right)^2 \right] \prod_{i=1}^s q_i^{\lambda_i} \left( 1- \frac{1}{q_i}\right) \prod_{i=1}^t s_i^{\mu_i} \left( 1- \frac{1}{s_i}\right)\\
&=\phi(nm)\phi(n\wedge m)
\end{align*}
Conclusion : 
$$(n\wedge m) \phi(n)\phi(m) = \phi(nm)\phi(n\wedge m).$$
\end{proof}

\paragraph{Ex. 2.20}

{\it Prove that $\prod_{d \mid n} d = n^{\nu(n)/2}$.
}

\begin{proof}
Let $$n = p_1^{\alpha_1}\cdots p_k^{\alpha_k}$$ the decomposition of $n$ in prime factors.
\begin{align*}
\left( \prod_{d \mid n}d \right)^2 &=  \prod_{d \mid n} d \  \prod_{d \mid n} d \\
&= \prod_{d \mid n} d  \  \prod_{\delta \mid n} \frac{n}{\delta} \qquad (\delta = n/d)\\
&=n^{\nu(n)}  \prod_{d \mid n} d  \  \prod_{d \mid n} \frac{1}{d} \\
&=n^{\nu(n)}
\end{align*}
Conclusion :
$$\prod_{d \mid n} d = n^{\frac{\nu(n)}{2}}.$$
\end{proof}

\paragraph{Ex. 2.21}

{\it Define $\land(n) = \log p$ if $n$ is a power of $p$ and zero otherwise. Prove that $\sum_{d \mid n} \mu(n/d)\log d = \land (n)$. [Hint: First calculate $\sum_{d \mid n} \land(d)$ and then apply the M?bius inversion formula.]
}

\begin{proof}
$$
\left\{
\begin{array}{cccl}
  \land(n)& =  & \log p & \mathrm{if}\  n =p^\alpha,\ \alpha \in \N^*  \\
  &  = &   0 & \mathrm{otherwise }.
\end{array}
\right.
$$
Let $n = p_1^{\alpha_1}\cdots p_t^{\alpha_t}$ the decomposition of $n$ in prime factors. As $\land(d) = 0$ for all divisors of $n$, except for $d = p_j^i, i>0, j=1,\ldots t$,
\begin{align*}
\sum_{d \mid n} \land(d)&= \sum_{i=1}^{\alpha_1} \land(p_1^{i}) + \cdots+ \sum_{i=1}^{\alpha_t} \land(p_t^{i})\\ 
&= \alpha_1 \log p_1+\cdots + \alpha_t \log p_t\\
&= \log n
\end{align*}
By M?bius Inversion Theorem,
$$\land(n) = \sum_{d \mid n} \mu\left (\frac{n}{d}\right ) \log d.$$
\end{proof}

\paragraph{Ex. 2.22}

{\it Show that the sum of all the integers $t$ such that $1 \leq t \leq n$ and $(t, n) = 1$ is $\frac{1}{2} n \phi(n)$.
}

\begin{proof}
Suppose $n >1$ (the formula is false if $n=1$).

Let $S = \sum\limits_{1\leq t \leq  n, \ t\wedge n =1} t = \sum\limits_{1\leq t \leq  n-1, \ t\wedge n =1} t$.

Using the symmetry  $t \mapsto n-t$, as $t\wedge n = 1 \iff (n-t) \wedge n = 1$, we obtain
\begin{align*}
2S &=  \sum_{1\leq t \leq  n-1, \ t\wedge n =1} t +  \sum_{1\leq t \leq  n-1, \ t\wedge n =1} t\\
&=  \sum_{1\leq t \leq   n-1, \ t\wedge n =1} t + \sum_{1\leq s \leq  n-1, \ (n-s) \wedge n =1}  n-s \qquad (s = n-t)\\
&=  \sum_{1\leq t \leq   n-1, \ t\wedge n =1} t + \sum_{1\leq t \leq  n-1, \ (n-t) \wedge n =1}  n-t \\
&=  \sum_{1\leq t \leq   n-1, \ t\wedge n =1} t + \sum_{1\leq t \leq  n-1,\  t \wedge n =1}  n-t \\
&= \sum_{1\leq t \leq  n-1,\  t \wedge n =1}  n\\
&= n \ \mathrm{Card} \{t \in \N \ \vert \  1\leq t \leq n-1, t\wedge n = 1\}\\
&= n \phi(n)
\end{align*} 

Conclusion : 
$$\forall n \in \N^*, \ \sum_{1\leq t \leq  n, \ t\wedge n =1} t = \frac{1}{2} n \phi(n).$$
(See another interesting proof in Adam Michalik's paper.)

\end{proof}


\paragraph{Ex. 2.23}

{\it Let $f(x) \in \Z[x]$ and let $\psi(n)$ be the number of $f(j), j = 1,2, \ldots, n$, such that $(f(j), n) = 1$. Show that $\psi(n)$ is multiplicative and that $\psi(p^t) = p^{t-1} \psi(p)$. Conclude that $\psi(n) = n \prod_{p|n} \psi(p)/p$.
}

\begin{proof}
My interpretation of this statement is that $\psi(n)$ is the number of $j, j=1,2, \ldots, n$, such that $(f(j), n) = 1$ (if $f$ is not one to one, we may obtain a different value).

Let $A_n = \{j \in \Z , 1\leq j \leq n \ \vert  \ f(j) \wedge n = 1\}$ : then $\psi(n) = \vert A_n \vert$.
If $f(x) = \sum_{k=0}^d a_k x^k$, note $f_n(x) \in (\Z/n\Z)[x]$ the polynomial $f_n(x) = \sum_{k=0}^n [a_k]_n x^k$ (here, we represent the class of $j \in \Z$ in $\Z/n\Z$ by $[j]_n$ ).
We can write without inconvenient $f=f_n$.

Let $B_n = \{a \in \Z/n\Z \ \vert \ f(a) \in (\Z/n\Z)^*\}$, where $(\Z/n\Z)^*$ is the group of invertible elements of $\Z/n\Z$. 

Then $u : A_n \to B_n, j \mapsto [j]_n$ is a bijection. 

Indeed $u$ is well defined : if $j \in A_n, f(j) \wedge n = 1$ , so $f([j]_n) = [f(j)]_n \in (\Z/n\Z)^*$. 

$u$ is injective : $[j]_n = [k]_n$ with $ 1 \leq j \leq n, 1\leq k \leq n$ implies $j=k$.

$u$ is surjective : if $a \in \Z/n/Z$ verifies $f(a) \in (\Z/n\Z)^*$, let $j$ the unique representative of $a$ such that $1\leq j \leq n$. Then $f(j) \wedge n =1$, so $u(j) = a$.

Thus $$\psi(n) = \vert B_n \vert,\ \mathrm{where} \ B_n = \{a \in \Z/n\Z \ \vert \ f(a) \in (\Z/n\Z)^*\}.$$

Suppose $n\wedge m = 1$. Let 
$$
\varphi : 
\left\{
\begin{array}{ccc}
  B_{nm}& \to   & B_n\times B_m  \\
  {[}j{]}_{nm}& \mapsto  & ({[}j{]}_{n},{[}j{]}_{m})
\end{array}
\right.
$$

$\bullet$ $\varphi$ is well defined : $[j]_{nm} = [k]_{nm} \Rightarrow j\equiv k \pmod {nm} \Rightarrow (j\equiv k \pmod {n}, j\equiv k \pmod {m}) \Rightarrow ([j]_n,[j]_m) =( [k]_n,[k]_m)$.

$\bullet$ $\varphi$ is injective : if $\varphi([j]_{nm}) = \varphi([k]_{nm})$, then $[j]_n = [k]_n, [j]_m=[k]_m$, so $n\mid j-k, m\mid j-k$. As $n\wedge m = 1, nm \mid j-k$ so $[j]_{nm} = [k]_{nm}$.

$\bullet$ $\varphi$ is surjective : if $ (a,b) \in B_n\times B_m$, there exist $ j,k \in \Z, 1\leq j \leq n, 1 \leq j \leq m$, such that $a = [j]_n, b = [k]_n$. From the Chinese Remainder Theorem, there exists $i \in \Z, 1\leq i \leq n$, such that $i \equiv j \pmod n, i \equiv k \pmod m$. Then $\varphi([i]_{nm}) = ([i]_n,[i]_m) = ( [j]_n, [k]_m) = (a,b)$.

Finally, $\psi(nm) = \vert B_{nm} \vert = \vert B_n \vert\,  \vert B_m \vert = \psi(n) \psi(m)$, if $n \wedge m =1$ : $\psi$ is a multiplicative function.

The interval $I = [1,p^t]$ is the disjoint reunion of the $p^{t-1}$ intervals $I_k = [kp+1, (k+1)p]$ for $k = 0,1,\cdots, p^{t-1} - 1$, so $\psi(p^t) = \sum\limits_{k=0}^{p^{t-1}-1} \mathrm{Card}\,C_k$, where $C_k =  \{j \in I_k \vert \  f(j) \wedge p^t = 1\} =  \{j \in I_k \vert \  f(j) \wedge p= 1\}$.

As $ f(j) \wedge p = 1 \iff f(j - kp) \wedge p = 1$, the application $v : C_k \to C_0, j \mapsto j -kp$ is well defined and is bijective, so $\vert C_k \vert = \vert C_0 \vert = \psi(p)$. Thus $\psi(p^t) = p^{t-1}\, \mathrm{Card}\, I_0 = p^{t-1} \psi(p)$ :
$$\psi(p^t) = p^{t-1} \psi(p).$$
If $n = \prod_{p \mid n}  p^{t(p)}$, then 
\begin{align*}
\psi(n) &= \prod_{p \mid n} \psi(p^{t(p)})\\
&=  \prod_{p \mid n} p^{t(p)-1} \psi(p)\\
&= n \prod_{p \mid n} \frac{\psi(p)}{p}
\end{align*}
\end{proof}

\paragraph{Ex. 2.24}

{\it Supply the details to the proof of Theorem 3.
}

As Adam Michalik, I suppose that there is a misprint, we must prove Theorem 4 :

{\it Let $k$ a finite field with $q$ elements.

$\sum q^{-\deg p(x)}$ diverges, where the sum is over all monic irreducible $p(x)$ in $k[x]$.
}
\begin{proof}
Notations :

${\cal P}$ : set of all monic polynomials $p$ in $k[x]$.

${\cal P}_n$ :  set of all monic polynomials $p$ in $k[x]$ with $\deg(p) \leq n$.

${\cal M}$ : set of all monic irreducible polynomials $p$ in $k[x]$.

${\cal M}$ : set of all monic irreducible polynomials $p$ in $k[x]$ with $\deg(p) \leq n$.

We must prove that $\sum\limits_{p \in {\cal M}} q^{-\deg p(x)}$ diverges.

$\bullet$ $\sum\limits_{p \in {\cal P}} q^{-\deg p(x)}$ diverges : 
\begin{align*}
\sum_{f\in{\cal P}_n} \frac{1}{q^{\deg f}} &= \sum_{d=0}^n\  \sum_{\deg(f) = d} \frac{1}{q^d}\\
&=\sum_{d=0}^n \frac{1}{q^d}\, \mathrm{Card}\, \{f \in {\cal P}\  \vert \ \deg(f) = d\}\\
&= \sum_{d=0}^n \frac{1}{q^d}\,  q^d = n+1.
\end{align*}
So $\sum\limits_{ f \in \cal P} q^{-\deg f}$ diverges.

$\bullet$ $\sum\limits_{ f \in \cal P} q^{-2\deg f}$ converges :
\begin{align*}
\sum_{f \in{\cal P}_n} q^{-2\deg(f)} &= \sum_{d=0}^n\ \sum_{\deg(f) = d} \frac{1}{q^{2d}}\\
&=\sum_{d=0}^n \frac{1}{q^{2d}} \, \mathrm{Card} \{f \in {\cal P} \ \vert \ \deg(f) = d\}\\
&= \sum_{d=0}^n \frac{1}{q^{d}}\\
&\leq \frac{1}{1-\frac{1}{q}}
\end{align*}
As any finite subset of ${\cal P}$ is included in some ${\cal P}_n$, $\sum\limits_{ f \in \cal P} q^{-2\deg f}$ converges.

$\bullet$ $\sum\limits_{p \in {\cal M}} q^{-\deg p(x)}$ diverges :

Let ${\cal M}_n = \{p_1,p_2,\ldots,p_{l(n)}\}$ the set of all monic irreducible polynomials such that $\deg p_i \leq n$. Let
$$\lambda(n) = \prod_{i = 1}^{l(n)} \frac{1}{1-\frac{1}{q^{\deg(p_i)}}}.$$
For simplicity, we write $l = l(n)$ for a fixed $n \in \N$. Then
\begin{align*}
\lambda(n) &= \prod_{i=1}^l \sum_{a_i=0}^{\infty} \frac{1}{q^{a_i \deg p_i}}\\
&= \left ( 1+\frac{1}{q^{\deg p_1}}+\frac{1}{q^{\deg p_1^2}}+\cdots\right) \times \cdots \times\left ( 1+\frac{1}{q^{\deg p_l}}+\frac{1}{q^{\deg p_l^2}}+\cdots\right)\\
&= \sum_{(a_1,\cdots,a_j) \in \N^l} \frac{1}{q^{\deg(p_1^{a_1}\cdots p_l^{a_l})}}\\
\end{align*}
Since the monic prime factors of  any polynomial $ p \in {\cal P}_n$ are in ${\cal P}_n$, the decomposition of $p$ is $p =p_1^{a_1}\cdots p_l^{a_l}$, so
$$\lambda(n) \geq \sum_{p \in {\cal P}_n} \frac{1}{q^{\deg p}} = n+1.$$
So $\lim\limits_{n\to \infty} \lambda(n) = \infty$ : this is another proof that there exist infinitely many monic irreducible polynomials in $k[x]$ (cf Ex. 2.1).
\begin{align*}
\log \lambda(n) &= - \sum_{i=1}^{l(n)} \log \left( 1 - \frac{1}{q^{\deg p_i}} \right)\\
&=\sum_{i=1}^{l(n)} \sum_{m=1}^{\infty} \frac{1}{m q^{m \deg p_i}}\\
&=\frac{1}{q^{\deg p_1}} + \cdots + \frac{1}{q^{\deg p_{l(n)}}} + \sum_{i=1}^{l(n)} \sum_{m=2}^{\infty} \frac{1}{m q^{m \deg p_i}}\\
\end{align*}
Yet
\begin{align*}
\sum_{m=2}^{\infty} \frac{1}{m q^{m \deg p_i}} &\leq \sum_{m=2}^{\infty} \frac{1}{q^{m \deg p_i}}\\
&= \frac{1}{q^{2\deg p_i}} \frac{1}{1 - \frac{1}{q^{deg p_i}}}\\
&= \frac{1}{q^{2\deg p_i} - q^{\deg p_i}}\leq \frac{2}{q^{2\deg p_i}}\\
\end{align*}
(the last inequality is equivalent to $2\leq q^{\deg p_i}$). So
$$\log \lambda(n) \leq \frac{1}{q^{\deg p_1}} + \cdots +\frac{1}{q^{\deg p_{l(n)}}}+ 2 \left ( \frac{1}{q^{2 \deg p_1}} + \cdots + \frac{1}{q^{2 \deg p_{l(n)}}} \right ).$$
As $\frac{1}{q^{2 \deg p_1}} + \cdots + \frac{1}{q^{2 \deg p_{l(n)}}}$ is less than the constant $\sum\limits_{ f \in \cal P} q^{-2\deg f}$,
 if $\sum\limits_{p \in {\cal M}} q^{-\deg p(x)}$ converges, then $\log \lambda(n) \leq C$, 
 where $C$ is a constant, so $\lambda(n) \leq e^C$ for all $n \in \N$, in contradiction with $\lim\limits_{n\to \infty} \lambda(n) = \infty$.
 
Conclusion :  $\sum\limits_{p \in {\cal M}} q^{-\deg p(x)}$ diverges.
\end{proof}

\paragraph{Ex. 2.25}

{\it Consider the function $\zeta(s) = \sum_{n=1}^\infty 1/n^s$. $\zeta$ is called the Riemann zeta function. It converges for $s > 1$. Prove the formal identity (Euler's identity) $$\zeta(s) = \prod_p (1-1/p^s)^{-1}.$$
}

\begin{proof}
We prove this equality, not only formally, but for all complex value $s$ such that $\mathrm{Re}(s)>1$.

Let $s \in \C$ and $f(n) = \frac{1}{n^s},\ n\in \N^*$. 

$f$ is completely multiplicative : $f(mn) = f(m) f(n)$ for $m,n \in \N^*$. 

Moreover  $\sum_{n=1}^\infty f(n)$ is absolutely convergent for $\mathrm{Re}(s) > 1$. Indeed, if $s = u+iv, u,v \in \R$, $\vert f(n) \vert = \vert n^{-s}\vert = \vert e^{-s \log(n)} \vert = \vert e^{-u \log(n)} e^{-iv \log(n)} \vert = e^{-u \log(n)} = \frac{1}{n^u}$, so $\sum\limits_{n=1}^\infty \vert f(n) \vert= 1/n^u$ converges if $u=\mathrm{Re}(s) >1$.

With these properties of $f$ ($f$ multiplicative and $\sum_{n=1}^\infty f(n)$  absolutely convergent), we will show that 
$$\sum_{n=1}^\infty f(n) = \prod_p (1 +f(p)+f(p^2)+\cdots).$$
Let $S^* = \sum\limits_{n = 1}^\infty \vert f(n) \vert < \infty$, and $S = \sum\limits_{n = 1}^\infty  f(n) \in \C$. For each prime number $p$, $\sum\limits_{k=1}^\infty \vert f(p^k) \vert$ converges (this sum is less than  $S^*$), so $\sum\limits_{k=0}^\infty f(p^k)$ converges absolutely. Thus, for $x \in \R$, the two finite products
$$P(x) = \prod_{p\leq x }\sum_{k=0}^\infty f(p^k), \qquad P^*(x) =  \prod_{p\leq x} \sum_{k=0}^\infty \vert f(p^k) \vert$$
are well defined.

If $p, q$ are two prime numbers, as $\sum_{i=0}^\infty f(p^i), \sum_{j=0}^\infty f(q^j)$ are absolutely convergent, $(f(p^i) f(q^j))_{(i,j) \in \N^2}$ is sommable, so the sum of these elements can be arranged in any order :
$$ \sum_{i=0}^\infty f(p^i)  \sum_{k=0}^\infty f(q^k) = \sum_{(i,j) \in \N^2} f(p^i) f(q^j) =  \sum_{(i,j) \in \N^2} f(p^iq^j).$$
 If $p_1,\cdots,p_t$ are all the prime $p\leq x$, repeating $t$ times these products, we obtain
\begin{align*}
P(x) & = \prod_{p\leq x }\sum_{k=0}^\infty f(p^k)\\
&= \sum_{i_1=0}^\infty f(p_1^{i_1})\cdots  \sum_{i_t=0}^\infty f(p_t^{i_t})\\
&= \sum_{(i_1,\ldots,i_k)\in \N^k} f(p_1^{i_1} \cdots p_t^{i_t})\\
&=\sum_{n\in \Delta} f(n),
\end{align*}
where $\Delta$ is the set of integers $n \in \N^*$ whose prime factors are not greater than $x$. Let $\overline{\Delta} = \N^* \setminus \Delta$ : this is the set of numbers $n \in \N^*$ such that at least a prime factor is greater than $x$. So
$$P(x) = \sum_{n\in \Delta} f(n)  =  S - \sum_{n \in \overline{\Delta}} f(n).$$
Then
$$\vert P(x) - S \vert \leq \sum_{n\in \overline{\Delta}} \vert f(n) \vert \leq \sum_{n\geq x} \vert f(n) \vert.$$
So $\lim\limits_{x \to + \infty} P(x) = S$, that is
$$\prod_p \sum_{k=0}^\infty f(p^k) = \sum_{n=1}^\infty f(n).$$
Finally, 
\begin{align*}
\sum_{n=1}^\infty \frac{1}{n^s} &=  \prod_p \left ( 1 + \frac{1}{p^s}+ \cdots + \frac{1}{p^{ks}}+ \cdots\right )\\
&=  \prod_p (1-1/p^s)^{-1}
\end{align*}
\end{proof}

\paragraph{Ex. 2.26}

{\it
Verify the formal identities:
\begin{enumerate}
\item[(a)] $\zeta(s)^{-1} = \sum \mu(n)/n^s$
\item[(b)] $\zeta(s)^{2} = \sum \nu(n)/n^s$
\item[(c)]  $\zeta(s)\zeta(s-1) = \sum \sigma(n)/n^s$
\end{enumerate}
}

\begin{proof}
Without any consideration of convergence :
\begin{enumerate}
\item[(a)]
\begin{align*}
\zeta(s) \sum_{m=1}^\infty \frac{\mu(m)}{m^s} &= \sum_{n=1}^\infty \frac{1}{n^s} \sum_{m=1}^\infty \frac{\mu(m)}{m^s}\\
&=\sum_{n,m\geq 1} \frac{\mu(m)}{n^s m^s}\\
&=\sum_{u=1}^\infty \sum_{m\mid u} \mu(m) \frac{1}{u^s}\qquad (u = nm)\\
&=\sum_{u=1}^\infty \frac{1}{u^s} \sum_{m\mid u} \mu(m) \\
&=1
\end{align*}
Indeed, $\sum_{m\mid u} \mu(m) = 1$ if $u=1$, $0$ otherwise. So 
$$\zeta(s)^{-1} = \sum_{n\in \N^*} \mu(n)/n^s.$$
\item[(b)]
\begin{align*}
\zeta(s)^2 &= \sum_{n=1}^\infty \frac{1}{n^s}\sum_{m=1}^\infty \frac{1}{m^s}\\
&= \sum_{n,m\geq 1} \frac{1}{(nm)^s}\\
&=\sum_{u\geq 1} \sum_{n\mid u} \frac{1}{u^s}\\
&=\sum_{u\geq 1} \frac{1}{u^s} \sum_{n\mid u} 1\\
&= \sum_{u\geq 1} \frac{1}{u^s} \nu(u)
\end{align*}
So $$\zeta(s)^2 = \sum_{n=1}^\infty \frac{\nu(n)}{n^s}.$$
\item[(c)] For $\mathrm{Re}(s)>2$, 
\begin{align*}
\zeta(s) \zeta(s-1) &= \sum_{n\geq 1} \frac{1}{n^s}\sum_{m \geq 1} \frac{1}{m^{s-1}}\\
&=\sum_{m,n\geq 1}  \frac{m}{(nm)^s}\\
&=\sum_{u \geq 1} \left ( \sum_{m \mid u} m\right) \frac{1}{u^s}\\
&= \sum_{u\geq 1} \frac{\sigma(u)}{u^s}
\end{align*}
So
$$\zeta(s)\zeta(s-1) = \sum_{n\geq 1} \frac{\sigma(n)}{n^s}.$$
\end{enumerate}
\end{proof}

\paragraph{Ex. 2.27}

{\it Show that $\sum 1/n$, the sum being over square free integers, diverges. Conclude that $\prod_{p < N} (1+1/p) \to \infty$ as $N \to \infty$. Since $e^x > 1 + x$, conclude that $\sum_{p < N} 1/p \to \infty$.
(This proof is due to I.Niven.)
}

\begin{proof}
Let $S \subset \N^*$ the set of square free integers.

Let $N \in \N^*$. Every integer $n, \, 1\leq n \leq N$ can be written as $n = a b^2$, where $a,b$ are integers and $a$ is square free. Then $1\leq a \leq N$, and $1\leq b \leq \sqrt{N}$, so
$$\sum_{n\leq N} \frac{1}{n} \leq \sum_{a \in S, a\leq N}\  \sum_{1\leq b \leq \sqrt{N}} \frac{1}{ab^2} \leq  \sum_{a \in S, a\leq N}\ \frac{1}{a} \, \sum_{b=1}^\infty  \frac{1}{b^2} = \frac{\pi^2}{6} \sum_{a \in S, a\leq N}\ \frac{1}{a}.$$
So $$\sum_{a \in S, a\leq N} \frac{1}{a}  \geq \frac{6}{\pi^2} \sum_{n\leq N} \frac{1}{n}.$$
As $\sum_{n=1}^\infty \frac{1}{n}$ diverges, $\lim\limits_{N \to \infty} \sum\limits_{a \in S, a\leq N} \frac{1}{a} = +\infty$, so the family $\left(\frac{1}{a}\right)_{a\in S}$ of the inverse of square free integers is not summable.

Let $S_N = \prod_{p<N}(1+1/p)$ , and $p_1,p_2,\ldots, p_l\ (l = l(N))$ all prime integers less than $N$. Then
\begin{align*}
S_N &= \left(1+\frac{1}{p_1}\right) \cdots \left(1+\frac{1}{p_l}\right)\\
&=\sum_{(\varepsilon_1,\cdots,\varepsilon_l) \in \{0,1\}^l } \frac{1}{p_1^{\varepsilon_1} \cdots p_l^{\varepsilon_l}}
\end{align*}
We prove this last formula  by induction. This is true for $l=1$ : $\sum_{\varepsilon \in \{0,1\}} 1/p_1^\varepsilon = 1 + 1/p_1$.

If it is true for the integer $l$, then 
\begin{align*}
\left(1+\frac{1}{p_1}\right) \cdots \left(1+\frac{1}{p_l}\right)\left(1+\frac{1}{p_{l+1}}\right) &= \sum_{(\varepsilon_1,\ldots,\varepsilon_l) \in \{0,1\}^l } \frac{1}{p_1^{\varepsilon_1} \cdots p_l^{\varepsilon_l}} \left(1+\frac{1}{p_{l+1}}\right)\\
&=\sum_{(\varepsilon_1,\ldots,\varepsilon_l) \in \{0,1\}^l } \frac{1}{p_1^{\varepsilon_1} \cdots p_l^{\varepsilon_l}} + \sum_{(\varepsilon_1,\ldots,\varepsilon_l) \in \{0,1\}^l } \frac{1}{p_1^{\varepsilon_1} \cdots p_l^{\varepsilon_l}p_{l+1}}\\
&=\sum_{(\varepsilon_1,\ldots,\varepsilon_l,\varepsilon_{l+1}) \in \{0,1\}^{l+1} } \frac{1}{p_1^{\varepsilon_1} \cdots p_l^{\varepsilon_l}p_{l+1}^{\varepsilon_{l+1}}} 
\end{align*}
So it is true for all $l$. 

Thus $S_N = \sum_{n\in \Delta} \frac{1}{n}$, where $\Delta$ is the set of square free integers whose prime factors are less than $N$.

As $\sum 1/n$, the sum being over square free integers, diverges, $\lim\limits_{N\to \infty} S_N = + \infty$ :
$$\lim_{N \to \infty} \prod_{p<N} \left(1+\frac{1}{p}\right) = +\infty.$$
 $e^x \geq 1+x, x \geq \log (1+x)$ for $x>0$, so
$$\log S_N = \sum_{k=1}^{l(N)} \log\left(1+\frac{1}{p_k}\right) \leq \sum_{k=1}^{l(N)} \frac{1}{p_k}.$$
$\lim\limits_{N\to \infty} \log S_N = +\infty$ and $\lim\limits_{N\to \infty} l(N) = +\infty$, so
$$\lim_{N\to \infty} \sum_{p<N} \frac{1}{p} = +\infty.$$
\end{proof}


{\large \bf Chapter 3}

{\it \paragraph{Ex. 3.1}
Show that there are infinitely many primes congruent to $-1$ modulo $6$.
}

\begin{proof}
Let $n$ any integer such that $n\geq 3$, and $N = n! -1 =   2 \times 3 \times\cdots\times n - 1 >1$. 

Then $N \equiv -1 \pmod 6$. As $6k +2, 6k +3, 6k +4$ are composite for all integers $k$, every prime factor of $N$ is congruent to $1$ or $-1$ modulo $6$.  If every prime factor of $N$ was congruent to 1, then $N \equiv 1 \pmod 6$ : this is a contradiction because $-1 \not \equiv 1 \pmod 6$.  So there exists a prime factor $p$ of $N$ such that $p\equiv -1 \pmod 6$.

If $p\leq n$, then $p \mid n!$, and $p \mid N = n!-1$, so $p \mid 1$. As $p$ is prime, this is a contradiction, so $p>n$. 

Conclusion :

 for any integer $n$, there exists a prime $p >n$ such that $p \equiv -1 \pmod 6$ : there are infinitely many primes congruent to $-1$ modulo $6$.
\end{proof}

\paragraph{Ex. 3.2}

{\it Construct addition and multiplication tables for $\Z/5\Z, \Z/8\Z$, and $\Z/10\Z$.
}

\begin{proof}
More a latex exercise than a mathematical one.

\medskip

$\Z/5\Z$ :

\medskip

$
\begin{array}{c|ccccc}
+ & 0 & 1 & 2 & 3 & 4\\ \hline
0 & 0 & 1 & 2 & 3 & 4\\
1 & 1 & 2 & 3 & 4 & 0\\
2 & 2 & 3 & 4 & 0 & 1\\
3 & 3 & 4 & 0 & 1 & 2\\
4 & 4 & 0 & 1 & 2 & 3\\
\end{array}
$
\qquad
$
\begin{array}{c|ccccc}
\times & 0 & 1 & 2 & 3 & 4\\ \hline
0 & 0 & 0 & 0 & 0 & 0\\
1 & 0 & 1 & 2 & 3 & 4\\
2 & 0 & 2 & 4 & 1 & 3\\
3 & 0 & 3 & 1 & 4 & 2\\
4 & 0 & 4 & 3 & 2 & 1\\
\end{array}
$

\bigskip

$\Z/8\Z$:

\medskip

$
\begin{array}{c|cccccccc}
+ & 0 & 1 & 2 & 3 & 4 & 5 & 6 & 7\\ \hline
0 & 0 & 1 & 2 & 3 & 4 & 5 & 6 & 7\\
1 & 1 & 2 & 3 & 4 & 5 & 6 & 7 & 0\\
2 & 2 & 3 & 4 & 5 & 6 & 7 & 0 & 1\\
3 & 3 & 4 & 5 & 6 & 7 & 0 & 1 & 2\\
4 & 4 & 5 & 6 & 7 & 0 & 1 & 2 & 3\\
5 & 5 & 6 & 7 & 0 & 1 & 2 & 3 & 4\\
6 & 6 & 7 & 0 & 1 & 2 & 3 & 4 & 5\\
7 & 7 & 0 & 1 & 2 & 3 & 4 & 5 & 6\\
\end{array}
$
\qquad
$
\begin{array}{c|cccccccc}
\times & 0 & 1 & 2 & 3 & 4 & 5 & 6 & 7\\ \hline
0 & 0 & 0 & 0 & 0 & 0 & 0 & 0 & 0\\
1 & 0 & 1 & 2 & 3 & 4 & 5 & 6 & 7\\
2 & 0 & 2 & 4 & 6 & 0 & 2 & 4 & 6\\
3 & 0 & 3 & 6 & 1 & 4 & 7 & 2 & 5\\
4 & 0 & 4 & 0 & 4 & 0 & 4 & 0 & 4\\
5 & 0 & 5 & 2 & 7 & 4 & 1 & 6 & 3\\
6 & 0 & 6 & 4 & 2 & 0 & 6 & 4 & 2\\
7 & 0 & 7 & 6 & 5 & 4 & 3 & 2 & 1\\
\end{array}
$
\bigskip

$\Z/10\Z$ : 

\medskip

$
\begin{array}{c|cccccccccc}
+ & 0 & 1 & 2 & 3 & 4 & 5 & 6 & 7 & 8 & 9\\ \hline
0 & 0 & 1 & 2 & 3 & 4 & 5 & 6 & 7 & 8 & 9\\
1 & 1 & 2 & 3 & 4 & 5 & 6 & 7 & 8 & 9 & 0\\
2 & 2 & 3 & 4 & 5 & 6 & 7 & 8 & 9 & 0 & 1\\
3 & 3 & 4 & 5 & 6 & 7 & 8 & 9 & 0 & 1 & 2\\
4 & 4 & 5 & 6 & 7 & 8 & 9 & 0 & 1 & 2 & 3\\
5 & 5 & 6 & 7 & 8 & 9 & 0 & 1 & 2 & 3 & 4\\
6 & 6 & 7 & 8 & 9 & 0 & 1 & 2 & 3 & 4 & 5\\
7 & 7 & 8 & 9 & 0 & 1 & 2 & 3 & 4 & 5 & 6\\
8 & 8 & 9 & 0 & 1 & 2 & 3 & 4 & 5 & 6 & 7\\
9 & 9 & 0 & 1 & 2 & 3 & 4 & 5 & 6 & 7 & 8\\
\end{array}
$
\qquad
$
\begin{array}{c|cccccccccc}
\times & 0 & 1 & 2 & 3 & 4 & 5 & 6 & 7 & 8 & 9\\ \hline
0 & 0 & 0 & 0 & 0 & 0 & 0 & 0 & 0 & 0 & 0\\
1 & 0 & 1 & 2 & 3 & 4 & 5 & 6 & 7 & 8 & 9\\
2 & 0 & 2 & 4 & 6 & 8 & 0 & 2 & 4 & 6 & 8\\
3 & 0 & 3 & 6 & 9 & 2 & 5 & 8 & 1 & 4 & 7\\
4 & 0 & 4 & 8 & 2 & 6 & 0 & 4 & 8 & 2 & 6\\
5 & 0 & 5 & 0 & 5 & 0 & 5 & 0 & 5 & 0 & 5\\
6 & 0 & 6 & 2 & 8 & 4 & 0 & 6 & 2 & 8 & 4\\
7 & 0 & 7 & 4 & 1 & 8 & 5 & 2 & 9 & 6 & 3\\
8 & 0 & 8 & 6 & 4 & 2 & 0 & 8 & 6 & 4 & 2\\
9 & 0 & 9 & 8 & 7 & 6 & 5 & 4 & 3 & 2 & 1\\
\end{array}
$

\bigskip

Python code to generate the latex code to create such an array :
\begin{verbatim}
n= 10
print('$')
ligne = '\\begin{array}{c|'+ n*'c'+'}'
print(ligne)
ligne='\\times'
for j in range(n):
    ligne += ' & ' + str(j)
ligne += '\\'
ligne += '\\'
ligne += ' \\hline'
print(ligne)
for i in range(n):
    ligne = str(i)
    for j in range(n):
        ligne +=' & '+ str((i * j) % n)
    ligne += '\\'
    ligne += '\\'
    print(ligne)
print('\\end{array}')
print('$')
\end{verbatim}
\end{proof}

{\it \paragraph{Ex. 3.3}
Let $abc$ be the decimal representation for an integer between $1$ and $1000$. Show that $abc$ is divisible by $3$ iff $a + b + c$ is divisible by $3$. Show that the same result is true if we replace $3$ by $9$. Show that $abc$ is divisible by $11$ iff $a - b + c$ is divisible by $11$.  Generalize to any number written in decimal notation.
}

\begin{proof}
Let $n = \overline{abc}$ the decimal representation of $n$. 

As $10 \equiv 1 \pmod 3$, $10^3 \equiv 10^2 \equiv 10 \equiv 1 \pmod 3$, so
\begin{align*}
3 \mid n &\iff 10^3 a + 10^2 b + c \equiv 0 \pmod 3\\ 
&\iff a+b+c \equiv 0 \pmod 3
& 3 \mid a+b+ c
\end{align*}
As $10 \equiv 1 \pmod 9$ the same demonstration is true for the result
$$9 \mid n \iff 9 \mid a+b+c.$$
Similarly, $10 \equiv -1 \pmod {11}$, and $10^2 \equiv 1, 10^3 \equiv -1$, so
\begin{align*}
11 \mid n &\iff 10^3 a + 10^2 b + c \equiv 0 \pmod 11\\ 
&\iff a-b+c \equiv 0 \pmod 3
\end{align*}
More generally, let $n = \overline{a_la_{l-1}\cdots a_0}$ is the decimal representation of $n$. 

$10^n \equiv 1 \pmod 9$, so
\begin{align*}
9 \mid n &\iff \sum_{k=0}^l a_k 10^k \equiv 0 \pmod 9\\
&\iff \sum_{k=0}^l a_k \equiv 0 \pmod 9\\
&\iff 9 \mid a_0+a_1+\cdots +a_n
\end{align*}
\end{proof}

\paragraph{Ex. 3.4}

{\it Show that the equation $3x^2 + 2 = y^2$ has no solution in integers.
}

\begin{proof}
If $3x^2+2 = y^2$, then  $\overline{y}^2 = \overline{2}$ in $\Z/3\Z$.


As $\{-1,0,1\}$ is a complete set of residues modulo $3$, the squares in $\Z/3\Z$ are $\overline{0} = \overline{0}^2$ and  $\overline{1} = \overline{1}^2 = (\overline{-1})^2$, so $\overline{2}$ is not a square in $\Z/3\Z$ : $\overline{y}^2 = \overline{2}$ is impossible in $\Z/3\Z$.

Thus $3x^2+2 = y^2$ has no solution in integers.
\end{proof}

 \paragraph{Ex. 3.5}

{\it Show that the equation $7x^2 + 2 = y^3$ has no solution in integers.
}

\begin{proof}
If $7x^2 + 2 = y^3,\ x,y \in \Z$, then $y^3 \equiv 2 \pmod 7$ (so $y \not \equiv 0 \pmod 7$)

From Fermat's Little Theorem, $y^6 \equiv 1 \pmod 7$, so $2^2 \equiv y^6 \equiv 1 \pmod 7$, which implies $7 \mid 2^2-1 = 3$ : this is a contradiction. Thus the equation $7x^2 + 2 = y^3$ has no solution in integers.
\end{proof}

\paragraph{Ex. 3.6}

{\it Let an integer $n > 0$ be given. A set of integers $a_1, \ldots, a_{\phi(n)}$ is called a reduced residue system modulo $n$ if they are pairwise incongruent modulo $n$ and $(a_i, n) = 1$ for all $i$. If $(a, n) = 1$, prove that $aa_1 , aa_2, \ldots, aa_{\phi(n)}$ is again a reduced residue system modulo $n$.
}

\begin{proof}
Let  $a_1, \ldots, a_{\phi(n)}$ a reduced residue system modulo $n$.

$\bullet$ As $a\wedge n = 1$ and $a_i \wedge n=1, \ i=1,2,\ldots,\phi(n)$, then $aa_i \wedge n=1$.

$\bullet$ As $a\wedge n = 1$, there exists $a' \in \Z$ such that $aa'\equiv 1 \pmod n$. then
$$a a_i \equiv a a_j \Rightarrow a'aa_i \equiv a'au_j \pmod n \Rightarrow a_i \equiv a_j \pmod n.$$
So $i\neq j \Rightarrow a_i \not \equiv a_j \Rightarrow a a_i \not \equiv a a_j$ :

$aa_1, \ldots, aa_{\phi(n)}$ a reduced residue system modulo $n$.

\medskip

Note that $\{a_1,a_2,\ldots, a_{\phi(n)}\}$ is a reduced residue system modulo $n$  if and only if $\{\overline{a_1},\overline{a_2},\ldots,\overline{a_{\phi(n)}}\} = U(\Z/n\Z)$.
\end{proof}

\paragraph{Ex. 3.7}

{\it Use Ex. 2.6 to give another proof of Euler's theorem, $a^{\phi(n)} \equiv 1 \pmod n$ for $(a, n) = 1$.
}

\begin{proof}
The proof is more clear if we stay in $\Z/n\Z$.

Let $P = \prod\limits_{x \in U(\Z/n\Z)} x$

(if $\{a_1,\ldots,a_{\phi(n)}\}$ is a reduced residue system modulo $n$, then $\overline{P} = \prod\limits_{i=1}^{\phi(n)} a_{i}$.)

Let $a \in \Z$ such that $a \wedge n = 1$, then $b = \overline{a} \in U(\Z/n\Z)$, and
$$
\psi \ 
\left\{
\begin{array}{ccc}
 U(\Z/n\Z) &  \to  &   U(\Z/n\Z)  \\
  x &\mapsto   &bx   
\end{array}
\right.
$$

$\bullet$ $\psi(x) = \psi(x') \Rightarrow bx = bx' \Rightarrow b^{-1} b x = b^{-1} a x' \Rightarrow x = x'$ so $\psi$ is injective.

$\bullet$ Let $y \in U(\Z/n\Z)$. If $x = b^{-1}y$, then $\psi(x) = b b^{-1} y = y$, so $\psi$ is surjective.

$\psi$ is a bijection, so

$$\prod_{x \in U(\Z/n\Z) } bx = \prod_{x \in U(\Z/n\Z) } x,$$
that is
$$b^{\phi(n)} \prod_{x \in U(\Z/n\Z) } x = \prod_{x \in U(\Z/n\Z) } x.$$
As $\prod\limits_{x \in U(\Z/n\Z) } x$ is invertible, 
$$b^{\phi(n)} = 1.$$
That is $\overline{a}^{\phi(n)} = 1$ : for all $a\in \Z$, if $a\wedge n = 1$, then $a^{\phi(n)} \equiv 1 \pmod n$. 
\end{proof}

\paragraph{Ex. 3.8}

{\it Let $p$ be an odd prime. If $k \in \{1, 2, \ldots, p - 1\}$, show that there is a unique $b_k$ in this set such that $kb_k \equiv 1 \pmod p$. Show that $k \ne b_k$ unless $k = 1$ or $k = p - 1$.
}

\begin{proof}
$\bullet$ existence. 

As $p$ is prime and $1\leq k \leq p-1$, $k \wedge p= 1$, so there exist $\lambda_k,\mu_k\in \Z $ such that $\lambda_k p + \mu_k k = 1$.
Let $b_k \in \{0,1,\ldots,p-1\}$ such that $b_k \equiv \mu_k \pmod p$. Then  $k b_k \equiv 1$, and $b_k \not \equiv 0 \pmod p$, so $1\leq b_k \leq p-1$.

$\bullet$ unicity. If $kb_k \equiv kb'_k \pmod p$, where $b_k,b'_k \in \{1,2,\ldots,p-1\}$, then $p \mid k(b'_k-b_k)$, and $p \wedge k = 1$, so $p \mid b'_k-b_k$. $b'_k \equiv b_k$, and $b_k,b'_k \in \{1,2,\ldots,p-1\}$, so $b_k = b'_k$.

If $p$ is a prime number, and $k \in \{1, 2, \ldots, p - 1\}$, there is a unique $b_k$ in $\{1,2,\ldots,p-1\}$ such that $kb_k \equiv 1 \pmod p$.

\medskip

If $k = b_k$, then $k^2 \equiv 1 \pmod p$, so $p \mid (k-1)(k+1)$, and $p$ is a prime, thus $p \mid k-1$ or $p \mid k+1$, that is $k \equiv \pm 1 \pmod p$. 
As $1 \leq k \leq p-1$, $k = 1$ or $k = p-1$ (and $1^2 \equiv (p-1)^2 \equiv 1 \pmod p$).
\end{proof}

\paragraph{Ex. 3.9}

{\it Use Ex. 3.8 to prove that $(p - 1)! \equiv -1 \pmod p$. (misprint corrected)
}

\begin{proof}
Each element $k$ in the product $p!$ can be associated with its inverse $b_k$ modulo k, except $1$ and $p-1$, which are their own inverse, so
$$p! \equiv 1\times (p-1) \equiv -1 \pmod p.$$
\end{proof}

\paragraph{Ex. 3.10}

{\it If $n$ is not a prime, show that $(n - 1)! \equiv 0 \pmod n$, except when $n = 4$.
}

\begin{proof}

Suppose that $n >1$ is not a prime. Then $n = uv$, where $2 \leq u \leq v \leq n-1$.

$\bullet$ If $u \neq v$, then $n = uv \mid (n-1)! = 1\times 2 \times\cdots \times u \times\cdots \times v \times \cdots \times (n-1)$ (even if $u\wedge v \neq 1$ !).

$\bullet$ If $u=v$, $n = u^2$ is a square.

If $u$ is not prime, $u =st,\ 2\leq s \leq t \leq u-1 \leq n-1$, and $n = u' v'$, where $u' =s,v' =st^2$ verify  $2 \leq u' < v' \leq n-1$. As in the first case, $n = u'v' \mid (n-1)!$.  

If $u = p$ is a prime, then $n =p^2$.

In the case $p = 2$, $n = 4$ and $n=4  \nmid (n-1)! = 6$. In the other case $p >2$, and $(n-1)! = (p^2 - 1)!$ contains the factors $p < 2p < p^2$, so $p^2 \mid (p^2-1)!, n \mid (n-1)!$.

Conclusion : if $n$ is not a prime, $(n - 1)! \equiv 0 \pmod n$, except when $n=4$.

\end{proof}

\paragraph{Ex. 3.11}

{\it Let $a_1, \ldots, a_{\phi(n)}$ be a reduced residue system modulo $n$ and let $N$ be the number of solutions to $x^2 \equiv 1 \pmod n$. Prove that $a_1 \cdots a_{\phi(n)} \equiv (-1)^{N/2} \pmod n$.
}

\begin{proof}
If $n=2$, then $N=1$ and the result is false. So we suppose $n>2$.

Let $H$ the subset of $\Z/n\Z$ of all $x \in \Z/n\Z$ such that $x^2=1$ :
$$H = \{x \in \Z/n\Z\ \vert \ x^2 = 1\}$$
(here $1 = \overline{1}$).

$H \subset U(\Z/n\Z)$, and $x \in H, y \in H \Rightarrow x^2 = y^2=1 \Rightarrow (xy^{-1})^2 = 1 \Rightarrow xy^{-1} \in H$, so $H$ is a subgroup of $(U(\Z/n\Z),\times)$, and $N = \mathrm{Card}\, H$.

Each $x \in U(\Z/n\Z)$ such that $x \not \in H$ can be paired with its inverse $x^{-1}$, and $xx^{-1} = 1$, so
$$P := \prod_{x \in U(\Z/n\Z)} x = \prod_{x\in H} \, x.$$
If $x \in H, -x \in H$.

$\bullet$ If $n$ is odd, each $x = \overline{a}\in H (a\in \Z, 1\leq a \leq n-1)$ satisfies $-x \neq x$ : otherwise $2 a \equiv 0 \pmod n, 2a = k n, k \in \Z$ . As $0<2a = kn  < 2n$, then $k=1$, and $n = 2a$ is even, which is in contradiction with the hypothesis.

So each $x \in H$ can be paired with $-x$ in the product $P$, and $x(-x) = -1$, so
$$P = \prod_{x\in H} \, x = (-1)^{N/2}.$$

$\bullet$ If $n$ is even, if $x = \overline{a}\in H\ (a\in \Z, 1\leq a \leq n-1)$ satisfies $x = -x$, then $0 < a = k \frac{n}{2} < n$, so $a = \frac{n}{2}$, and $x = \overline{ \left(\frac{n}{2}\right)}$ is the only element in $\Z/n\Z$ such that $x = -x$. $\overline{2}x = \overline{0}$, and $x \in H$, so $\overline{2} x^2 = \overline{0}, \overline{2} = \overline{0}$ : since $n>2$, this is impossible, so $x \neq -x$ for all $x \in H$, and $\prod_{x\in H} \, x = (-1)^{N/2}$.

Conclusion : if $n>2$, $$\prod_{x \in U(\Z/n\Z)} x = (-1)^{N/2}.$$

If $a_1, \ldots, a_{\phi(n)}$ is a reduced residue system modulo $n$, then $\overline{ a_1 \cdots a_{\phi(n)}}  = P = \prod_{x \in U(\Z/n\Z)} x = (-1)^{N/2}$, so 
$$a_1 \cdots a_{\phi(n)} \equiv (-1)^{N/2}.$$
\end{proof}

\paragraph{Ex. 3.12}

{\it Let ${p \choose k} = \frac{p!}{k!(p-k)!}$ be a binomial coefficient, and suppose $p$ is prime. If $1 \leq k \leq p-1$, show that $p$ divides ${p \choose k}$. Deduce $(a+b)^p \equiv a^p + b^p \pmod p$.
}

\begin{proof}
$p \mid p! = k!(p-k)! {p \choose k}$.

If $1\leq k \leq p-1$, then for each $i, 1\leq i \leq k$, $1\leq i < p$, so $i\wedge p=1$. Thus $ \left(\prod_{i=1}^k i \right) \wedge p = 1, k!\wedge p = 1$. Similarly, $p-k<p$, so $\left( \prod_{i=1}^{p-k} i \right)\wedge p = 1, (p-k)!\wedge p = 1$. Thus $p \wedge k!(p-k)! = 1$, and $p \mid p! = k!(p-k)! {p \choose k}$, so $p \mid {p \choose k}$.

Finally, from binomial formula
\begin{align*}
(a+b)^p &= a^p + \sum_{k=1}^{p-1} {p \choose k} a^k b^{n-k} + b^p\\
&\equiv a^p + b^p \pmod p
\end{align*}
\end{proof}

\paragraph{Ex. 3.13}

{\it Use Ex. 3.12 to give another proof of Fermat's theorem, $a^{p-1} \equiv 1 \pmod p$ if $p$ does not divide $a$.
}

\begin{proof}
If we make the induction hypothesis 
$${\cal P}(k) \iff \forall (a_1,a_2,\ldots,a_k) \in \Z^k,\  (a_1+a_2+\cdots+a_k)^p \equiv a_1^p+a_2^p+\cdots+a_k^p$$
(which is true for $k=1,k=2$)
then, from induction hypothesis and the case $k=2$ already proved in Ex 3.12,
\begin{align*}
(a_1+a_2+\cdots+a_k+a_{k+1})^p &= ((a_1+a_2+\cdots+a_k) +a_{k+1})^p\\
&\equiv (a_1+a_2+\cdots+a_k)^p + a_{k+1}^p \pmod p\\
&\equiv a_1^p+a_2^p+\cdots+a_k^p+ a_{k+1}^p \pmod p\\
\end{align*}
so ${\cal P}(k) \Rightarrow {\cal P}(k+1)$ :

$$\forall k \in \N^*, \forall (a_1,a_2,\ldots,a_k) \in \Z^k,\  (a_1+a_2+\cdots+a_k)^p \equiv a_1^p+a_2^p+\cdots+a_k^p.$$
If we apply this result to the particular case $a_1=a_2=\cdots = a_k = 1$, we obtain
$$\forall k \in \N^*,\ k^p \equiv k \pmod p.$$
and $(-k)^p \equiv - k^p \equiv -k \pmod p$ (even if $p=2$), and $0^p =0$, so
$$\forall k \in \Z,\ k^p \equiv k \pmod p.$$
If $p \nmid a, a \in \Z$, then $p\wedge a = 1$, and $p \mid a^p-a = a(a^{p-1} - 1)$, so $p \mid a^{p-1} - 1, a^{p-1} \equiv 1 \pmod p$ : this is another proof of Fermat's theorem.
\end{proof}

\paragraph{Ex. 3.14}

{\it Let $p$ and $q$ be distinct odd primes such that $p - 1$ divides $q -1$. If $(n, pq) = 1$, show that $n^{q - 1} \equiv 1 \pmod {pq}$.
}

\begin{proof}
As $n \wedge pq = 1, n\wedge p=1, n \wedge q = 1$, so from Fermat's Little Theorem
$$n^{q-1} \equiv 1 \pmod q,\qquad n^{p-1} \equiv 1 \pmod p.$$
$p-1 \mid q-1$, so there exists $k \in \Z$ such that $q-1 = k(p-1)$.
Thus
$$n^{q-1} = (n^{p-1})^k \equiv 1 \pmod p.$$
$p \mid n^{q-1} - 1, q \mid n^{q-1} - 1$, and $p\wedge q = 1$, so $pq \mid n^{q-1} - 1$ :
$$n^{q-1} \equiv 1 \pmod{pq}.$$
\end{proof}

\paragraph{Ex. 3.15}

{\it For any prime $p$ show that the numerator of $1+ \frac{1}{2} + \frac{1}{3} + \ldots + \frac{1}{p-1}$ is divisible by $p$.
}

\begin{proof}
As the result is false for $p=2$, we must suppose $p>2$, so $p$ is odd.

$1+ \frac{1}{2} + \frac{1}{3} + \ldots + \frac{1}{p-1} = \frac{N}{D}$, where
$$ N = (p-1)! + \frac{(p-1)!}{2}+ \cdots+\frac{(p-1)!}{p-1}, \qquad D= (p-1)!.$$
From Wilson's theorem, $(p-1)! \equiv -1 \pmod p$, so in the field $\Z/p\Z$,
$$ \overline{N} = (-\overline{1})(\overline{1}^{-1} + \overline{2}^{-1}+\cdots+\overline{p-1}^{-1}).$$
As the application $\varphi : (\Z/p\Z)^* \to (\Z/p\Z)^*, x \mapsto x^{-1}$ is bijective (it's an involution), 
$$\overline{1} + \overline{2}^{-1}+\cdots+\overline{p-1}^{-1} = \overline{1} + \overline{2}+\cdots+\overline{p-1} = \overline{p} \times \overline{\left(  \frac{p-1}{2}\right)} = \overline{0}.$$
So $p \mid N $, and $p \wedge (p-1)! = 1$, that is $p\wedge D = 1$. Thus $p$ divides the numerator of the reduced fraction of $N/D$.
\end{proof}

\paragraph{Ex. 3.16}

{\it Use the proof of the Chinese Remainder Theorem to solve the system $x \equiv 1 \pmod 7$, $x \equiv 4 \pmod 9$, $x \equiv 3 \pmod 5$.
}

\begin{proof} Let $m_1 = 7,m_2=9,m_3 = 5, m = m_1m_2m_3 = 315, n_1 = m/m_1= m_2m_3=45, n_2 = m_1m_3 = 35, n_3= m_1m_2 = 63$.

If $r_1 = 13, s_1 = -2$, then $r_1m_1+s_1n_1 = 13 m_1 -2m_2m_3 = 13\times 7 -2 \times 45 = 1$,

so $e_1 = s_1n_1 = -2\times 45 = -90$ verifies $$e_1 = -90,\qquad e_1\equiv 1 \pmod 7, e_1 \equiv 0 \pmod 9, e_1 \equiv 0 \pmod 5.$$

If $r_2 = 4,s_2 = -1$, then $r_2m_2+s_2n_2 = 4\times9 - 1\times 35 = 1$,

so $ e_2 = s_2n_2 = -35$ verifies $$e_2 = -35, \qquad e_2 \equiv 0 \pmod 7, e_2 \equiv 1 \pmod 9, e_2 \equiv 0 \pmod 5.$$

If $r_3 = -25,s_3 = 2$, then $r_3m_3 +s_3n_3 = -25\times 5 + 2\times 63 = 1$,

so  $ e_3 = s_3 n _3 = 2\times 63 = 126$ verifies $$e_3 = 126, \qquad e_3\equiv 0\pmod 7, e_3 \equiv 0 \pmod 9, e_3 \equiv 1 \pmod 5.$$

Let $x_0 = e_1 + 4 e_2 + 3 e_3 = 148 $ : then 
$$x_0 = 148,\qquad x_0 \equiv 1 \pmod 7, x_0 \equiv 4 \pmod 9, x_0 \equiv 3 \pmod 5.$$
If $x\in \Z$ is any solution of the system, then $7 \mid x-x_0, 9 \mid x-x_0, 5 \mid x-x_0$, with $7\wedge 9 = 7 \wedge 5 = 9 \wedge 5 = 1$, so $m = 315 \mid x - x_0$ : 
$$x = 148 + k\, 315, k \in \Z,$$
and all these integers are solutions of the system.
\end{proof}

\paragraph{Ex. 3.17}

{\it Let $f(x) \in \Z[x]$ and $n = p_1^{a_1} \cdots p_t^{a_t}$. Show that $f(x) \equiv 0 \pmod n$ has a solution iff $f(x) \equiv 0 \pmod {p_i^{a_i}}$ has a solution for $i = 1, \ldots, t$.
}

\begin{proof}
If $x$ is such that $f(x) \equiv 0 \pmod n$, as $p_i^{\alpha_i} \mid n$, $f(x) \equiv 0 \pmod{p_i^{a_i}}$.

Conversely, let $x_1,x_2,\ldots,x_t$ such that
\begin{align*}
f(x_1)&\equiv 0 \pmod{p_1^{a_1}}\\
&\cdots\\
f(x_t)&\equiv 0 \pmod{p_t^{a_t}}
\end{align*}
As $p_i^{a_i} \wedge p_j^{a_j}=1$ if $i\ne j$, the Chinese Remainder Theorem gives an integer $x$ such that $x\equiv x_i \pmod {p_i^{a_i}},\ i=1,2,\ldots,t$. As $f(x)\in \Z[x]$, $f(x) \equiv f(x_i) \equiv 0 \pmod {p_i^{a_i}}$. So $p_i^{a_i} \mid f(x),\ i=1,2,\ldots,t$, where $p_i^{a_i} \wedge p_j^{a_j}=1$ if $i\ne j$, then $n = p_1^{a_1} \cdots p_t^{a_t} \mid f(x)$ : $x$ is a somution of $f(x) \equiv 0 \pmod n$.

Conclusion : $f(x) \equiv 0 \pmod n$ has a solution iff $f(x) \equiv 0 \pmod {p_i^{a_i}}$ has a solution for $i = 1, \ldots, t$.
\end{proof}

\paragraph{Ex. 3.18}

{\it For $f \in \Z[x]$, let $N$ be the number of solutions to $f(x) \equiv 0 \pmod n$ and $N_i$ be the number of solutions to $f(x) \equiv 0 \pmod {p_i^{a_i}}$. Prove that $N = N_1N_2\cdots N_t$.
}

\begin{proof}
Note $[x]_n$ the class of $x$ modulo $n$. Let $S$ the set of solutions in $\Z/n\Z$ of $f(\overline{x}) = 0$, and $S_i$ the set of solutions in $\Z/p^{a_i}\Z$ of $f(\overline{x}) = 0$.

(We designate with the same letter the polynomial $f$ in $\Z[x]$ or its reduction in $\Z/n\Z[x]$.)

Let
$$\varphi :
\left\{
\begin{array}{ccc}
  S&  \to  & S_1\times S_2\times \cdots\times S_t  \\
  {[}x{]}_n& \mapsto   & ({[}x{]}_{p_1^{a_1}}, {[}x{]}_{p_2^{a_2}},\ldots, {[}x{]}_{p_t^{a_t}})    
\end{array}
\right.
$$

$\bullet$ $\varphi$ is well defined : if $x \equiv x' \pmod n$, then $x \equiv x' \pmod {p_i^{a_i}},\ i=1,2,\cdots,t$, so $({[}x{]}_{p_1^{a_1}}, {[}x{]}_{p_2^{a_2}},\ldots, {[}x{]}_{p_t^{a_t}}) =({[}x'{]}_{p_1^{a_1}}, {[}x'{]}_{p_2^{a_2}},\ldots, {[}x'{]}_{p_t^{a_t}})$.  Moreover, we proved in Ex 3.17 that $[x]_n \in S \Rightarrow [x]_{p_i^{a_i}} \in S_i$.

$\bullet$ $\varphi$ is injective : if $({[}x{]}_{p_1^{a_1}}, {[}x{]}_{p_2^{a_2}},\ldots, {[}x{]}_{p_t^{a_t}}) =({[}x'{]}_{p_1^{a_1}}, {[}x'{]}_{p_2^{a_2}},\ldots, {[}x'{]}_{p_t^{a_t}})$, then $p_i^{a_i} \mid x'-x,\ i=1,2,\ldots,t$, so $n \mid x'-x$ and $[x]_n = [x']_n$.

$\bullet$ $\varphi$ is surjective : if $y=({[}x_1{]}_{p_1^{a_1}}, {[}x_2{]}_{p_2^{a_2}},\ldots, {[}x_t{]}_{p_t^{a_t}})$ is any element of $S_1\times S_2\times \cdots\times S_t$, there exists from Chinese remainder theorem $x \in \Z$ such that $x\equiv x_i \pmod {p_i^{a_i}}$. Then $\varphi([x]_n) = y$ (see Ex. 3.17).

In conclusion, a $\varphi$ is bijective,  $N = \vert S \vert = \vert S_1\times S_2\times \cdots\times S_t \vert= N_1N_2\cdots N_t$.
\end{proof}

\paragraph{Ex. 3.19}

{\it If $p$ is an odd prime, show that $1$ and $-1$ are the only solutions of $x^2 \equiv 1 \pmod {p^a}$.
}

\begin{proof}
$$x^2-1\pmod{p^a} \iff p^a \mid (x-1)(x+1).$$
Let $d = (x-1)\wedge(x+1)$ : $d = 1$ or $d=2$.

$\bullet$ If $d=1$, then $x$ is even (if not, $x-1$ and $x+1$ are even, and $2\mid d$).
As $p^a \mid (x-1)(x+1)$ and $(x-1) \wedge (x+1) = 1$, then $p^a \mid  x-1$, or $p^a \mid x +1$, that is $$x\equiv \pm 1 \pmod {p^a}.$$

$\bullet$ If $d=2$, then $x$ is odd, and
$$p^a \mid 4 \frac{x-1}{2}\frac{x+1}{2}.$$
As $p$ is an odd prime, $p\wedge 4 = 1$, so $p \mid  \frac{x-1}{2}\frac{x+1}{2}$, where  $\frac{x-1}{2}\wedge \frac{x+1}{2} = 1$, hence $p^a \mid  \frac{x-1}{2} \mid x-1$ or $p^a \mid  \frac{x+1}{2} \mid x+1$ : $$x\equiv \pm 1 \pmod {p^a}.$$

$\{-\overline{1},\overline{1}\}$ is the set of roots of $x^2-\overline{1}$ in $\Z/p^a\Z$.
\end{proof}

\paragraph{Ex. 3.20}

{\it Show that $x^2 \equiv 1 \pmod {2^b}$ has one solution if $b = 1$, two solutions if $b = 2$, and four solutions if $b \geq 3$.
}

\begin{proof} Consider the equation $x^2 \equiv 1\pmod{2^b}$.

$\bullet$ If $b=1$, $x^2 \equiv 1\pmod{2} \iff 2 \mid (x-1)(x+1) \iff x\equiv 1 \pmod 2$ : one solution.

$\bullet$ If $b=2$, as $0^2 \equiv 2^2 \equiv 0 \pmod 4$, $x^2 \equiv 1\pmod{4} \iff x\equiv \pm 1 \pmod 4$ : two solutions.

$\bullet$ Suppose $b\geq 3$. The equation has 4 solutions $1,-1,1+2^{b-1}, -1+2^{b-1}$.

Indeed, $(\pm1)^2 \equiv 1 \pmod {2^b}$, and
$$(1+2^{b-1})^2 = 1 + 2.2^{b-1} + 2^{2b-2} = 1+2^b(1+2^{b-2}) \equiv 1 \pmod{2^b},$$ and similarly $(-1+2^{b-1})^2 \equiv 1 \pmod{2^b}$.

These solutions are incongruent modulo $2^b$ : 

$1 \not \equiv -1 \pmod {2^b}$ and $1+2^{b-1}\not \equiv -1+2^{b-1}$ (if not, $2^b \mid 2$, so $b\leq 1$).

$1 + 2^{b-1} \equiv -1 \pmod {2^b} \iff 2^b \mid 2 +2^{b-1} = 2(1 + 2^{b-2})$ : so  $2 \mid 2^{b-1} \mid (1+ 2^{b-2})$, this is impossible because $1+ 2^{b-2}$ is odd ($b\geq 3$). With the same argument, $-1 + 2^{b-1} \not \equiv 1 \pmod{2^b}$. $1 + 2^{b-1} \equiv 1  \pmod{2^b}$ implies $2^{b} \mid 2^{b-1}$, so $2\mid 1$ : this is a contradiction, so $1 + 2^{b-1} \not \equiv 1  \pmod{2^b}$, and also $-1 + 2^{b-1} \not \equiv -1  \pmod{2^b}$. There exist at least 4 solutions.

We show that these are the only solutions : 
$$ \forall x \in \Z,\ x^2 \equiv 1 \pmod{2^b} \Rightarrow x \equiv \pm 1 \pmod{2^{b-1}}.$$
Indeed, if $x^2 \equiv 1 \pmod{2^b} $, $2^b \mid (x-1)(x+1)$, where $d =(x-1) \wedge (x+1) = 2$. 

As in Ex.3.19, if $d=1$, then $2^b \mid x-1$ or $2^b \mid x+1$, a fortiori $x \equiv \pm 1 \pmod{2^{b-1}}$.

If $d=2$, then $x$ is odd, and $2^b \mid 4\frac{x-1}{2}\frac{x+1}{2}$,  so $2^{b-2} \mid \frac{x-1}{2}\frac{x+1}{2}$, with $\frac{x-1}{2} \wedge \frac{x+1}{2}=1$, so $2^{b-2} \mid \frac{x-1}{2}$ or $2^{b-2} \mid \frac{x+1}{2}$, that is $2^{b-1} \mid x-1$ or $2^{b-1} \mid x+1$ : $x \equiv \pm 1 \pmod{2^{b-1}}$.

(Alternatively, we can prove this implication by induction.)

Hence every solution of $x^2 \equiv 1\pmod{2^b}, b\geq 3$ is such that $x = \pm1+k2^{b-1}, k \in \Z$ : there exit only four such value in the interval $[0,2^b[$, namely $1,-1+2^{b-1},1+2^{b-1},-1+2^b$.

Conclusion : if $b\geq 3$, the roots of $x^2-1$ in $\Z/2^b\Z$ are $ \overline{1}, -\overline{1}, \overline{1} + \overline{2}^{b-1},-\overline{1} + \overline{2}^{b-1}$.
\end{proof}

\paragraph{Ex. 3.21}

{\it \
Use Ex. 18-20 to find the number of solutions to $x^2 \equiv 1 \pmod n$.
}

\begin{proof}
Let $n = 2^{a_0}p_1^{a_1}\cdots p_k^{a_k}$ the decomposition in prime factors of $n>1$ ($p_0=2<p_1<\cdots<p_k, a_0\geq 0, a_i>0, 1\leq i \leq k$). Let $N$ the number of solutions of $x^2 \equiv 1 \pmod n$, and $N_i$ the number of solutions of $x^2 \equiv 1 \pmod {p^i}, i = 0,1,\ldots k$. From Ex.3.18, we know that $N = N_0N_1\cdots N_k$, where (Ex. 3.19), $N_i = 2, i=1,2,\ldots,k$, and (Ex.3.20), $N_0=1$ if $a_0=1$ (or $a_0 = 0$), $N_0 = 2$ if $a_0 = 2$, $N_0=4$ if $a_0\geq 3$.

Conclusion : the number of solutions of $x^2 \equiv 1 \pmod n$,  where $n = 2^{a_0}p_1^{a_1}\cdots p_k^{a_k}$, is
\begin{align*}
N &= 2^k\quad \mathrm{if}\ a_0=0\ \mathrm{or}\ a_0 = 1\\
N &= 2^{k+1} \quad \mathrm{if}\ a_0=2\\
N &= 2^{k+2} \quad \mathrm{if}\ a_0\geq 3
\end{align*}
\end{proof}

\paragraph{Ex. 3.22}

{\it Formulate and prove the Chinese Remainder Theorem in a principal ideal domain.
}

{\bf Proposition}. Let $R$ a principal ideal domain, and $m_1,\ldots,m_t \in R$. Suppose that $(m_i,m_j) = 1$ for $i\neq j$ (that is $(m_i) + (m_j) = (1), m_iR +n_i R = R$). Let $b_1,\ldots,b_t \in R$ and consider the system of congruences:
$$x \equiv b_1 \pmod {m_1}, x \equiv b_2 \pmod {m_2},\ldots, x \equiv b_t \pmod {m_t}.$$
This system has solutions and any two solutions differ by a multiple of $m_1m_2\cdots m_t$.
\begin{proof} 
Let $m = m_1m_2\cdots m_t$, and $n_i =m/m_i,\ i=1,2,\ldots,t$. 

As $(m_1,m_i)=(1)$, we can find $u_i,v_i \in R$ such that $m_1u_i+m_iv_i= 1, \ i=2,\ldots,t$.

So $1 = \prod\limits_{i=2}^t(m_1u_i+m_iv_i ) = m_1 u +(m_2\cdots m_t)v$ for some elements $u,v \in R$, thus $(m_1,n_1) = (m_1,m_2m_3\cdots m_t) = (1)$, and similarly $(m_i,n_i)=1$. So there are $r_i,s_i \in R$ such that $r_im_i+s_in_i = 1$. Let $e_i = s_in_i$. Then $e_i \equiv 1 \pmod{m_i}$ and $e_i \equiv 0 \pmod{m_j}$ for $j\neq i$.

Set $x_0 =\sum_{i=1}^t b_ie_i$. Then we have $x_0 \equiv b_i e_i \equiv b_i \pmod{m_i}$ and so $x_0$ is a solution.

Suppose that $x_1$ is another solution. Then $x_1 -x_0 \equiv 0 \pmod{m_i}$ for $i=1,2,\ldots,t$, in other words $m_1,m_2,\ldots,m_t$ divide $x_1-x_0$, with $(m_i,m_j)=1$ : from lemma 2 generalized to principal rings, $m$ divides $x_1-x_0$.
\end{proof}

\bigskip

This result can be generalized to any commutative ring, not necessarily a PID (see S.LANG, Algebra):

{\bf Proposition}. Let $A$ a commutative ring. Let ${\textfrak a}_1,\ldots, {\textfrak a}_n$ be ideals of $A$ such that ${\textfrak a}_i+{\textfrak a}_j = A$ for all $i\neq j$. Given elements $x_1,\ldots,x_n \in A$, there exists $x\in A$ such that $x\equiv x_i \pmod {{\textfrak a}_i}$ for all $i$.

\paragraph{Ex. 3.23}

{\it Extend the notion of congruence to the ring $\Z[i]$ and prove that $a + bi$ is always congruent to $0$ or $1$ modulo $1 + i$.
}

\begin{proof}
If $a,b,c$ are in $\Z[i]$ we say that $a\equiv b \pmod c$ if there exists $q \in \Z[i]$ such that $a-b = qc$.

As $i\equiv -1 \pmod{1+i}, a+bi \equiv a-b \pmod {1+i}$.

$(1-i)(1+i) = 2$, so $2\equiv 0 \pmod{1+i}$.

If $a-b$ is even, $a-b =2k,k\in \Z \subset \Z[i]$, so $a-b \equiv 0 \pmod{1+i}$.

If $a-b$ is odd, $a-b = 2k+1, k \in \Z$, so $a-b \equiv 1 \pmod {1+i}$.

Conclusion : for all $z \in \Z[i]$, $z\equiv0,1 \pmod{1+i}$.
\end{proof}

\paragraph{Ex. 3.24}

{\it Extend the notion of congruence to the ring $\Z[\omega]$ and prove that $a + b\omega$ is always congruent to $-1$, $0$ or $1$ modulo $1 - \omega$.
}

\begin{proof}
Same definition of congrence in $\Z[\omega]$ as in Ex. 3.23.

$\omega \equiv 1 \pmod{1-\omega}$, so $a+b\omega \equiv a+b \pmod{-\omega}$.

$0 = 1-\omega^3 = (1- \omega)(1+\omega+\omega^2)$, with $1-\omega \neq 0$, so  $1+\omega+\omega^2 = 0$. Hence $3 \equiv 0 \pmod{1-\omega}$.

$a+b \equiv 0,1,-1 \pmod 3$, so $a+b \equiv 0,1,-1 \pmod {1-\omega}$

For all $z \in \Z[\omega],\  z \equiv 0,1,-1 \pmod{1-\omega}$.
\end{proof}

\paragraph{Ex. 3.25} 

{\it Let $\lambda = 1 -\omega \in \Z[\omega]$. If $\alpha \in \Z[\omega]$ and $\alpha \equiv 1 \pmod \lambda$, prove that $\alpha^3 \equiv 1 \pmod 9$.
}

\begin{proof} 
$\alpha \equiv 1 \pmod \lambda$, so $\alpha = 1 + \beta \lambda, \beta \in \Z[\omega]$.

$\overline{\lambda} = 1 - \omega^2 = (1-\omega)(1+\omega) = -\omega^2(1-\omega) = -\omega^2 \lambda$ (so $\overline{\lambda}$ and $\lambda$ are associate).
\begin{align*}
\alpha^3-1&=(\alpha-1)(\alpha-\omega)(\alpha-\omega^2)\\
&=(\alpha-1)(\alpha-1+\lambda)(\alpha-1+\overline{\lambda})\\
&=(\alpha-1)(\alpha-1+\lambda)(\alpha-1-\omega^2\lambda)\\
&=\beta \lambda (\beta\lambda + \lambda) (\beta \lambda - \omega^2\lambda)\\
&=\lambda^3 \beta (\beta+1)(\beta-\omega^2)\\
\end{align*}

Moreover, 
\begin{align*}
\beta(\beta+1)(\beta-\omega^2) &\equiv \beta(\beta+1)(\beta-1) \pmod \lambda\\
&\equiv 0 \pmod \lambda
\end{align*}
since $\beta \equiv 0,1,-1 \pmod \lambda$ (see Ex. 3.24).

So $\lambda^4 \mid \alpha^3 - 1$.

As $\lambda \overline{\lambda} = (1-\omega)(1-\omega^2) = 1 - \omega - \omega^2+ \omega^3 = 3$, then $\lambda \overline{\lambda} = -\omega^2 \lambda^2 = 3$, so $\lambda^2$ and $3$ are associate : $\lambda^2 = -\omega \lambda^2$. So $9 =(-\omega^2 \lambda^2)^2 =  \omega \lambda^4$, so $9 \mid \omega^2 9 = \lambda^4 \mid \alpha^3-1$.

For all $\alpha \in \Z[\omega]$, 
$$\alpha \equiv 1 \pmod \lambda \Rightarrow \alpha^3 \equiv 1 \pmod 9.$$
\end{proof}

\paragraph{Ex. 3.26}

{\it Use Ex. 25 to show that $\xi, \eta, \zeta$ are not zero and $\xi^3 + \eta^3 + \zeta^3 = 0$, then $\lambda$ divides at least one of the elements $\xi, \eta, \zeta$.
}

\begin{proof}
Let $\xi, \eta, \zeta \in \Z[\omega] \setminus \{0\}$ such that $\xi^3 + \eta^3 + \zeta^3 = 0$.

With a reductio ad absurdum, suppose that $\lambda \nmid \xi, \lambda \nmid \eta, \lambda \nmid \zeta$.

From Ex. 3.24, 
$$\xi \equiv \pm 1 \pmod \lambda, \eta \equiv \pm 1 \pmod \lambda,\zeta \equiv \pm 1 \pmod \lambda,$$
and from Ex.3.25,
$$\xi^3 \equiv \pm 1 \pmod 9, \eta^3 \equiv \pm 1 \pmod 9,\zeta^3 \equiv \pm 1 \pmod 9,$$
As $\pm1\pm1\pm1 \not \equiv 0 \pmod 9$, this is a contradiction.

Conclusion : if $\xi, \eta, \zeta$ are not zero and $\xi^3 + \eta^3 + \zeta^3 = 0$, then $\lambda$ divides at least one of the elements $\xi, \eta, \zeta$.

(consequence : if $x^3+y^3+z^3 = 0,\ x,y,z \in \Z$, then $3\mid xyz$ : this is the first case of Fermat's theorem for the exponant 3.)
\end{proof}


\end{document}
